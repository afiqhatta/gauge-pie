\documentclass[11pt, oneside]{article}   	% use "amsart" instead of "article" for AMSLaTeX format
\usepackage[margin = 1.1in]{geometry}            		% See geometry.pdf to learn the layout options. There are lots.
\geometry{letterpaper}                   		% ... or a4paper or a5paper or ... 
\usepackage[parfill]{parskip}    		% Activate to begin paragraphs with an empty line rather than an indent
\usepackage{graphicx}				% Use pdf, png, jpg, or eps§ with pdflatex; use eps in DVI mode
								% TeX will automatically convert eps --> pdf in pdflatex	
\usepackage{adjustbox}	
\usepackage[section]{placeins}


%% LaTeX Preamble - Common packages
\usepackage[utf8]{inputenc}
\usepackage[english]{babel}

\title{Foo}
\author{Your name} 
\begin{document} 
\maketitle
\tableofcontents


\pagebreak

\section{My section} 
\subsection{Subsection} 

% Autocompile  - <space> ll

% Contents  - <space> lt

% Errors  - write some wrong stuff then press 

% Get snippets  - In command mode, ':UltiSnipsEdit'. Get most of it from Giles Castell (link in the README.md)

% Inline math - 'mk'
$$

% Display math  - 'dm'  
\[

\] 

% FRACTIONS!!!! - '//' + all intuitive sorts of ways to trigger fractions (try type 2 `p/)
\[
\frac{}{}
\] 

% Greek Letters  - ' ` + corresponding Roman alphabet 
\[ \alpha \delta \delta \delta \mu \nu \nu \beta \]

% Environment  - 'beg' then type in your environment which you want 
\begin{equation*}
	 a + b  = c
\end{equation*}

% Type == for aligned equals 
\begin{align*}
	a &= jn \\
	b &= efefefef \\
\end{align*}

% Figures - type 'fig' then <tab>
\begin{figure}[htpb]
	\centering
	\includegraphics[width=0.8\linewidth]{name.ext}
	\caption{Name}%
	\label{fig:name}
\end{figure}

% Matrices  - pmat or bmat
$ \begin{pmatrix} 0 & 0  \end{pmatrix} $

% For physicists  - bra - ket notation. Type | 1 > 
$ \ket{ 1 } \bra{ 2 } $

% Vectors  - letter followed by ' ., ' 
$   \vec{1} \vec{g}   \vec{a}  .,\vec{1} $

% Overline  - 'bar'
$\overline{ h nu}$

% Subscripts and superscipts. For subscripts, press the number followed by the character. Type 'td' for superscripts
$a_1 a_2  d_2 f_3 h_5 h^{ 12}$

% Tables  - type 'table' then <tab> 
 \begin{table}[htpb]
	\centering
	\caption{caption}
	\label{tab:label}
	\begin{tabular}{c}
	
	\end{tabular}
\end{table}
\end{document} 

% Brackets - type (), lrb, lra, lr[ for round, curly angle and square brackets 

