\documentclass[11pt, oneside]{article}   	% use "amsart" instead of "article" for AMSLaTeX format 
\usepackage[margin = 1.1in]{geometry}            		% See geometry.pdf to learn the layout options. There are lots.
\geometry{letterpaper}                   		% ... or a4paper or a5paper or ... 
\usepackage[parfill]{parskip}    		% Activate to begin paragraphs with an empty line rather than an indent
\usepackage{graphicx}				% Use pdf, png, jpg, or eps§ with pdflatex; use eps in DVI mode
							% TeX will automatically convert eps --> pdf in pdflatex	
\usepackage{adjustbox}	
\usepackage[section]{placeins}


%% LaTeX Preamble - Common packages
\usepackage[utf8]{inputenc}
\usepackage[english]{babel}
\usepackage{textcomp} % provide lots of new symbols
\usepackage{graphicx}  % Add graphics capabilities
\usepackage{flafter}  % Don't place floats before their definition
\usepackage{amsmath,amssymb}  % Better maths support & more symbols
\usepackage[backend=biber]{biblatex}
\usepackage{amsthm}
\usepackage{bm}  % Define \bm{} to use bold math fontsx
\usepackage[pdftex,bookmarks,colorlinks,breaklinks]{hyperref}  % PDF hyperlinks, with coloured links
\usepackage{memhfixc}  % remove conflict between the memoir class & hyperref
\usepackage{mathtools}
\usepackage[T1]{fontenc}
\usepackage[scaled]{beramono}
\usepackage{listings}
\usepackage{physics}
\usepackage{tensor}
\usepackage{simplewick} 
\usepackage{tikz} 
\usepackage{import}
\usepackage{xifthen}
\usepackage{pdfpages}
\usepackage{transparent}
\usepackage{pgfplots}
\usepackage[compat=1.1.0]{tikz-feynman}
\usepackage{subfiles}
\usepackage{simpler-wick}
\usepackage{slashed}

%% Commands for typesetting theorems, claims and other things.
\newtheoremstyle{slanted}
{1em}%   Space above
{.8em}%   Space below
{}%  Body font
{}%          Indent amount (empty = no indent, \parindent = para indent)
{\bfseries}% Thm head font
{.}%         Punctuation after thm head
{0.5em}%     Space after thm head: " " = normal interword space;
{}%         \newline = linebreak
{}%          Thm head spec (can be left empty, meaning `normal')

%% Commands for typesetting theorems, claims and other things. 

\theoremstyle{slanted}
\newtheorem{theorem}{Theorem}
\newtheorem*{thm}{Theorem}
\newtheorem*{claim}{Claim}
\newtheorem{example}{Example}
\newtheorem*{defn}{Definition}

\newcommand{\Lagr}{\mathcal{L}} 
\newcommand{\vc}[1]{\mathbf{#1}}
\newcommand{\pdrv}[2]{\frac{\partial{#1}}{\partial{#2}}}
\newcommand{\thrint}[1]{\int d^3 \vc{x} \left( {#1} \right)}

%% QFT specific macros 
\newcommand{\intp}{ \int \frac{ d^3 p }{ (2 \pi)^3 } \, }
\newcommand{\ann}[1]{a_{ \mathbf{ #1 }}}
\newcommand{\crea}[1]{a^\dagger_{ \mathbf{ #1 }}}
\newcommand{\ve}[1]{ \mathbf{ #1 } } 
\newcommand{\mode}[ 1]{ e^{ i \mathbf{ #1 } \cdot \mathbf{x} }}
\newcommand{\nmode}[1]{ e^{  - i \mathbf{ #1 } \cdot \mathbf{x} }}
\newcommand{\freq}[1]{\omega_\mathbf{ #1} } 
\newcommand{\scal}[1]{\phi ( \mathbf{ #1 })} 
\newcommand{\mom}[1]{ \pi (\mathbf{ #1 })} 
\newcommand{\arr}{\rightarrow} 

\newcommand{\incfig}[1]{%
\def\svgwidth{\columnwidth}
\resizebox{0.75\textwidth}{!}{\input{./figures/#1.pdf_tex}}
}

\newcommand{\anop}[2]{ #1_\mathbf{#2}}
\newcommand{\crop}[2]{#1_\mathbf{#2}^\dagger}

\usepackage{helvet} 

%tikz decoration commands 
\usetikzlibrary{decorations.pathmorphing}


\title{Strange metals} 
\author{Afiq Hatta} 
\begin{document} 
\maketitle

The structure of the talk. 
- Theory of metals 
- Theory of superconductors 
- Something goes wrong - high temperature superconductors come about how do we understand this? 
- A hint of what's going on - linear resistance above the critical temperature - this is a hint of 
some deeper structure. 


\section{Introduction}
Now initially when I started making this talk, I knew I wanted to do something
on the topic of strange metals. Initially, I started by you know, taking a paper, 
trying to reproduce and explain the derivations and so on. But a lot of people in this room know me, 
and I'm sure that everyone agrees that I'm a political guy. 
I'm a big fan of how mathematics 
can be used for the betterment of society, how people should use it, 
and I care about wider applications. So I'm going to use that as a starting point from this talk.
I'm a big believer in using the human cause for motivating mathematics, 
so this is how I'll approach this. 

As I'm sure everyone in this room is aware, the Earth is basically frying out. 
We're in a climate crisis. 
And, in a world in which we're trying to make technology more efficient, 
superconductors are a type of material which everyone should care about. 
What are superconductors? Basically, they're materials that allow electricity to pass through 
them with zero resistance when they're cold enough. 
Now, does anyone know the useful things we can do with?
Well since superconductors allow electricity to flow 
through them very effectively, we can 
create very powerful magnetic fields. 

\subsection{Why are superconductors useful?} 
First and foremost, being able to 
transmit electricity with zero resistance is less 
wasteful, so it's better for the environment. 
Well, if we have zero resistance, the powerful electric current 
can generate a really, really strong magnetic field. 
We call this temperature the 
'critical temperature'. Now, why exactly this occurs is still quite magical 
and mysterious. 


\subsection{The problem with superconductors} 
This sounds great, and you might ask for example, 
why doesn't everyone use superconductors? But the issue with this 
is that superconductors have to be very cold to work. 
Cooling these objects is often very expensive and difficult, 
and the expense and energy consumption of cooling  
Now, we don't understand superconductors very well - 
and recently we've discovered some superconductors which 
behave in that way at higher temperatures than expected.

So know, everyone's like 'Oh shit' how do we 
solve this problem, and what's really going on?
So, I'll be taking us on a journey through 
what we know about metals, all the way up to how 
black hole physics can help us understand what's going on. 

High temperature superconductors are materials 
which can behave like superconductors above $ -  200^{ \circ}C $. 
Currently, we can cool some superconductors we know with liquid helium 
and liquid nitrogen, but not dry ice! Cooling is expensive and 
inefficient, and currently this is the 
main reason why we don't use it. 

\section{What we thought we knew!}
\subsection{Fermi liquid Theory} 
Our current and most ubiquitous model for 
metals is something called Fermi liquid theory, 
which was put forward by the eminent Soviet 
physicist Lev Landau.  
And up until the 1950's, this theory 
has actually worked out really well for us 
in modelling the phases of matter we can find metals in.

Metals are weird and hard to understand. 
Roughly speaking, some available tools we know about 
to model metals are a kind of particle called electrons. 
Electrons are a kind of particle called fermions, 
and one of the things that make fermions special is that 
they obey a rule called the Pauli-exclusion principle. 
Those familiar with quantum mechanics might recognise the term 
'Fermi-Dirac statistics'.

Now, what does the Pauli-exclusion principle say? 
It says that we can't have two electrons in the 
same place at the same time. 
This is a condition in which rules out 
a lot of models which we can have on metals. 

This predicts the emergence of something called quasi-particles, 
which are blob like objects which run through the material
and behave like particles in their own right. 
Now, why do we use the term 'quasi'? 
Well this is a way of using the English language which I 
don't like, but since every one does it I'll talk about it. 
'Quasi' is used to describe this because the mechanism 
is somewhat describing 'emergent behaviour'. 

This is important because Fermi liquid theory predicts that 
these quasi-particles, in a rather surprising way, don't
really interact at all with each other. 
Now, even though there's 
repulsion from the original electrons, since 
these quasi-particles emerge, it's shown that 
it conducts electricity. 

This is good  - so we've come up with a model, it predicts 
the existence of quasi-particles in metals, and
then it says well, these quasi-particles don't cause many problems for us 
in terms of information. 


How does this work? 

In particular, we understand the story of how 'conventional superconductors' 
work. I'll describe this phenomena here. 

\subsection{The theory of superconductors up until (relatively recently)}
The theory of superconductivity up until now 
relied on a fairly simple mechanism. 
It relies on a concept called 'Cooper pairs', where 
when an electron moves forward, it drags with it
other surrounding positively charged atoms. 
Then, what happens is that this creates a small pocket of 
positive charge which attracts an electron behind it. 
This allows the two electrons to shoot basically shoot past the metal 
at high speeds. 

\section{What goes wrong?}
So, we've got this very nice model of metals, which worked 
quite well for until the 1980's. 
In 1986 however, George Bednorz and Alex Muller 
discovered a new kind of material. 
These materials are metals which exhibit superconductivity 
at slightly higher temperatures. 

\subsection{The weird phenomena which puts everything into question} 
'Strange metals' come into the picture because they are a phase 
of matter where resistance 
varies with temperature in ways which we do not expect. 
At temperatures slightly higher than the super-conducting phase for 
example, we see that resistance varies linearly with 
temperature. 

In particular, we have that resistance increases linearly in temperature, 
which is weird - usually it increases quadratically, something predicted 
by Fermi liquid theory. 
We have that 
\[
 \rho \sim T  
\] 
(Insert graph here)

In addition, we find that just above the 
critical temperature Fermi liquid 
theory breaks down. 
Even though the metal conducts heat and 
electricity just above this critical temperature, 
from some clever experiments we know, the 
metal doesn't appear to 
Basically, these quasi-particles 
that I was talking about earlier, break down. 
All we have left is kind of a melted soup. 
For physicists in the audience, this is deeply related 
to the idea of large scale entanglement across the whole metal. 

What happens is that 
just above the critical temperature, 
the electrons start to become a soup and there
are no clear particle like excitations. 
In fact, we know this because we can place an electron inside this 
soup, and it basically disappears into this soup 
without a trace. 
Now, this is significantly different behaviour 
than what we get from 
our previous model where an electron basically retains 
its sense of self. 
Another way of viewing this is 
that these electrons dissipate their energy really quickly and 
like a person getting lost in a crowd, just kind of vanish. 

This article will cover some of the modelling 
that has been done with these kinds of materials.
Currently, materials we know which exhibit this kind of 
behaviour are high temperature cuprates, ruthenium oxides
and iron pnictides. 


\subsection{What is AdS / CFT?}
Explanations for this behaviour. 
To go any further, I'm going to relate this problem 
to how physicists have been trying to unify the 
theory of gravity and quantum mechanics. 
Basically, we have theories of how things 
work on a small scale, and this roughly speaking 
concerns quantum mechanics. 

And, we also have a really good theory of 
how things work on a large scale. In particular, 
the theory of general relativity, which is 
our current best theory of gravity. 

Now, since this is a mixed audience, I'm going to try and
sum this up by saying that this is an insanely hard problem to solve - the problem of 'quantum gravity'. 
It's a problem that physicists have been trying to 
tackle since the theories of gravity and quantum mechanics have been developed. 

Imagine trying to watch a movie
with 3d glasses. What you might see, that if 
you don't wear the glasses, is a weird looking image. 
It's in 2d, and obviously is on the screen. 
But, when we put on our 3d goggles, we see that the image 'pops' 
out from the screen. Now, this is interesting. 
We've specified information on the boundary of 
some space, and then 'popped' it out into 
another dimension. 

Now, this thing that we've popped out allows 
us to learn about things on the screen 
in a different way. 
Often times, it's often easier to work 
with this popped out hologram than it is 
to work with the soup on this boundary. 

This is what we call the AdS/CFT correspondence. 
It's a relationship between what we know about gravitational systems, 
and soups of particles which are on the 
boundary. 

https://arxiv.org/pdf/1310.4319.pdf


\subsection{The SYK Model} 

What are the links between the SYK model and 
the correlation functions, for example?
https://arxiv.org/pdf/1807.03334.pdf



\subsection{The status quo - Fermi liquid theory for metals}

Crucially, we predict that resistance varies with the square of temperature. 
However, this is not something we observe in real life. 

\subsection{The Phase diagram of strange metals} 

\subsection{What has failed before} 
Other theories fail to not make quasi-particles. 
Quasi-particles act like small particles through the metal, 
which gives resistance. 

\subsection{The SYK model} 
Random pairwise movements. 
 - Further work by Xue-Yang Song on strongly correlated metal built 
 from SYK models.

\subsection{Quantum entanglement from inside and outside a black hole horizon}
Links since we have similar equilibration time. 

\section{AdS / CFT} 
d spacetime quantum system with no quasi-particles dual
with d + 1 dimensions. 
Quantum entanglement leads to an emergent spatial direction.

\section{References} 
https://arxiv.org/pdf/1807.03334.pdf

\end{document} 
