\subsection{A Second Look at Universality} 
We can understand the phenomena of universality 
much better now using the language of renormalisation. 
Let's begin by starting at one point in theory space, for example 
in the above case where $ \zeta = 1 $. 
We then start flowing to larger and larger $ \zeta $. 
Where can we land? 
We can't have cycles in theory space! We may be able to have 
shapes in theory space. 
Check out the C-theorem for more information about this. 
\begin{enumerate}
	\item We could have that our couplings like $ \mu^ 2 ( \zeta ) $. or $ g ( \zeta) `$, 
		could flow in some direction off to infinity. 
	\item More interestingly, we could have the coupling constants 
		converging towards a fixed point. At a fixed point, 
		the theory doesn't change with scale which implies that our correlation 
		length  $ \xi $ needs to be $ \xi = 0 $ or  $ \xi = \infty$, 
		since $ \xi $ depends on scale. 
\end{enumerate}

\subsubsection{The Ising model} 
Let's think about the case when $ T > T_c$, with $ 
T \to \infty$. We expect that at this temperature, 
spins change so rapidly that we don't expect the things to 
be correlated. Hence, we have that $ \xi \to 0$. 

On the other hand, when we're in the cold regime
with  $ T < T_ c $, and that  $ T \to 0 $, we also 
have that nothing happens so our correlation length 
$ \xi \to 0 $. 

Our most interesting case is when  $ T = T_c $, 
In this case, we have that  $ \xi \to \infty$, 
and we have fluctuations on every possible length scale. 
Colloquially, this looks like a fractal. 

