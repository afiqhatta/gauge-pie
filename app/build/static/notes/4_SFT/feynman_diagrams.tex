\subsection{Feynman Diagrams} 

Previously we were calculating an expansion \[
	\log \left< e ^{ - F _ I [ \phi _ k ^ - , \phi _ k ^ + ] } \right>_ + 
\] This yields a sum of terms
which look like $ g _ 0 ^ p ( \phi ^ - ) ^ a ( \phi ^ + ) ^ b  $.
We would like an efficient way of computing this 
object once it's averaged in the Gaussian ensemble.
In an almost perfect analogy with quantum field theory, we 
construct a diagrammatic way to calculate
these expansions. These diagrams are called Feynman diagrams.  
We'll outline the rules to construct them below 
 \begin{enumerate}
	 \item Each $ \phi _{ \vec{k} } ^ - $ is represented by 
		 an external solid line. Contributions from 
		 these external lines can be written as 
		 just a factor of $ \phi ^ - $ integrated 
		 over position space. 
	\item Each $ \phi_{ \vec{k} } ^ + $ is represented by 
		an internal dotted line. These lines each contribute 
		to adding a propagator $ \left< \phi_{ \vec{k} '  } ^ + \phi _{ \vec{k} } ^ +  \right>$ 
		in the integral expression. 
	\item Dotted lines are always connected at both ends. The fact that we 
		used Wick's theorem to give us all possible pairwise contractions 
		is the reason for the requirement that all the dotted lines need to be 
		paired up an connected in all possible different ways. 
	\item Each vertex joins 4 lines with $ g_0 $ . When we have a vertex, we need to 
		impose momentum conservation b y
	\item Each line has some momentum $ \vec{k} $, 
		conserved at vertices, which comes from the delta function 
		we get by integrating with respect to the spatial variables first. 
	\item Momenta which are not determined explicitly (for example internal 
		loops), are integrated over with a factor $ \int  d^ d q / ( 2 \pi ) ^ d $. 
	\item Divide by symmetry factors. 
\end{enumerate}
Now, once we've drawn these diagrams, we are in shape to write 
down the integrals that they correspond to. 
At lowest order we have something called a tree diagram 
where 4 momenta go into a single vertex. These are represented 
by $ g_0 \int  d ^ 4 x \, ( \phi ^ - ) ^ 4  $. 
\begin{equation*} 
	\feynmandiagram [small] {
		a [particle = {\(k _ 1 \)}] -- [fermion] e; 
		b [particle = {\(k _ 2 \)}]  -- [fermion] e; 
		 c [particle = {\(k _ 3 \)}]  -- [fermion] e; 
		 d [particle = {\(k _ 4 \)}]  -- [fermion] e; 
};
	= g \int d ^ d x \, ( \phi ^ - ) ^ 4 
\end{equation*}
We then have something called a loop diagram, which 
contributes an overall constant since we integrate over 
high modes. 
The interesting connection is when we have a 
loop attached to two external vertices. This gives us an 
extra term 
\begin{equation*}
	\feynmandiagram [small] {
		a [ particle =\( k _ 1 \) ] -- [fermion] b 
		-- [ghost, half left, momentum=\( q \) ] c 
		-- [ghost, half left] b,
		d [ particle =\( k _ 2 \) ] -- [fermion] b 
	}; =   6 g_0 \int _{ \Lambda / \zeta } ^ \Lambda 
\frac{ d ^ d q }{ q ^ 2 + \mu _ 0 ^ 2 } \int d ^ d x ( \phi ^ - ) ^ 2
\end{equation*}
This term corresponds 
to the contribution which we 
looked at earlier; that of $ \phi_{ \vec{k} _ 1 }^ -  \phi_{ \vec{k} _ 2 } ^ - \phi _{ \vec{k} _ 3 } ^ + \phi _{ \vec{k} _ 4}^ + $ This diagram is qualitatively new because it 
contains a loop which of undetermined momentum which we need to 
integrate over, as opposed to the \textit{tree diagrams} 
which have no internal loops.
You can convince yourself that these are the only possible diagram 
one can get at first order. 
Now, if we move up to second order, we need to consider diagrams 
which contain two vertices, but whose ends are connected. 
We shall consider a \textbf{connected} diagram first, 
whose contribution looks like 

\begin{equation}
	\feynmandiagram[small]{ 
		a[particle = \( k_1 \) ] -- [fermion] b, 
		c[particle = \( k_1 \) ] -- [fermion] b,
		b -- [ghost, half left] d 
		-- [ghost, half left] b , 
		e[particle = \(k_4 \)] -- [fermion] d, 
		f[particle = \(k_3 \)] -- [fermion] d,
	}; 
\end{equation}

Now, this diagram is from second order and corresponds to the term 
given by 
\[
 36 g_0 ^ 2 \int_ 0 ^{ \Lambda / \zeta } \prod_{ i =1 } ^ 4 
 \left[  \frac{ d ^ d k_ i }{ ( 2 \pi ) ^ d } \phi ^ -_{ \vec{k} _ i}  \right] 
 f ( \vec{k} _ 1 + \vec{k} _ 2 ) ( 2 \pi ) ^ d \delta ^ d ( \sum_ i \vec{k} _ i  )
\]  where 
\[
	f ( \vec{k} _ 1  + \vec{f} _ 2 ) = \int _{ \Lambda / \zeta } ^ \Lambda 
	\frac{ d ^ d q}{ ( 2 \pi ) ^ d } \, \frac{1 }{ q ^ 2 + \mu _ 0 ^ 2 } \frac{1}{
	( \vec{k} _ 1 + \vec{k} _ 2 + q ) ^ 2 } 
\] Let's break this down. Our factor of $ f $ comes 
from the fact that, after we impose momentum conservation, 
we still have an undetermined momentum  $ q $ to integrate over. 
Within this integral, we have the two internal line 
propagators which contribute $ \frac{1}{q ^ 2 + \mu _ 0 ^ 2} $ and 
$ \frac{1}{ ( \vec{k} _ 1 + \vec{k} _ 2 + \vec{q} ) ^ 2 + \mu _ 0 ^  2}$. 
Now, this is the only diagram which contributes 
to this order in the interaction. One may ask, what happens to 
the terms like 
\begin{equation*}
	\feynmandiagram[small]{ 
		a  -- [ghost, half left] c -- [ghost, half left] a ,
		b -- [fermion]a;
		d -- [fermion]a; 

		h -- [fermion] e, 
		g -- [fermion] e,  
		f -- [fermion] e, 
		i -- [fermion] e, 
	}; + 
	\feynmandiagram[small]{ 
		a -- [fermion] d; 
		b -- [fermion] d; 
		c -- [fermion] d; 
		e -- [fermion] d; 

		f -- [fermion] k; 
		g -- [fermion] k; 
		h -- [fermion] k; 
		i -- [fermion] k; 
	}; 
\end{equation*}

The key thing to observe here 
is that these diagrams are disconnected, and 
in fact get cancelled out by terms in the expansion of 
$ \left< F _ I  \right> ^ 2 $. 
Multiples of diagrams get cancelled out by cumulant terms! 
So, in our expansion for $ F _ I $ we need only consider 
the sum of connected diagrams.
Finally, we don't consider 
diagrams which look like
\begin{equation*}
	\feynmandiagram[small]{ 
		h -- [fermion] a , 
		a -- [ghost, half left] c 
		-- [ghost, half left] a, 
		a -- [ghost] d, 
		d -- [fermion] f, 
		d -- [fermion] e, 
		d -- [fermion] g,
	}; = 0 
\end{equation*}
This is because we need to impose momentum conservation at each vertex, 
but as you can see here there's a lone 
external propagator meeting a bunch of internal ones, 
so momentums cannot match here. 
This means that diagrams of this form go to zero. 

However, there are other interesting cases to second order 
that we need to think about. For example, 
consider the diagram 
\begin{equation*}
	\feynmandiagram[small]{ 
		a[particle = \( k \) ] -- [fermion] c, 
		b[particle = \( -k \) ]  -- [fermion] c, 
		c -- [ghost, half left, momentum = \( q_1 \) ] d -- 
		[ghost, half left] c, 
		d -- [ghost, half left] e -- 
		[ghost, half left,  momentum = \( q_2 \) ] d, 
	}; = \int \frac{ d ^ d k }{ ( 2\pi ) ^ d } \phi_{ \vec{k} } \phi _{ - \vec{k} } 
	\int^ \Lambda _{ \Lambda / \zeta } \frac{d ^ d q_1 	 d ^ d q_2 }{ ( 2 \pi ) ^{ 2 d } } \frac{1}{q_ 1 ^ 2 
	+ \mu _ 0 ^ 2 } \frac{1}{q_ 2 ^ 2 + \mu _ 0 ^ 2 }
\end{equation*}
The above diagram can be written in simplified fashion as 
\[
	\dots = \int \frac{ d ^ d k }{ ( 2 \pi ) ^ d } \frac{1}{2 } \phi _{ \vec{k} } \phi _{ - \vec{k} } 
	C ( \Lambda ) 
\] 
The key point that we made here is that our function 
which comes from the interaction terms doesn't depend on the momentum 
$ \vec{k} $  in this case, but we've still encoded the 
ultraviolet cutoff $ \Lambda $. If we factor in 
this contribution to our free energy, then 
we've induced another contribution 
in how $ \mu ^ 2 $ changes: 
\[
 \mu _ 0 ^ 2 \to \mu _ 0 ^ 2 + 12 g_0 \int 
 _{ \Lambda / \zeta } ^ \Lambda \frac{ d ^ d q }{ ( 2 \pi ) ^ d } \frac{1}{q ^ 2 + \mu _ 0 ^ 2  } 
 + g_0 ^ 2 C ( \Lambda ) 
\] 
We can also denote general functions which \textbf{do} depend on 
$ \vec{k} $. For example, consider the diagram 
\begin{equation*}
	\feynmandiagram{
		a[particle = \( k \) ] -- [fermion] b, 
		b -- [ghost, half left, momentum =\( q_ 1 \) ] c -- [ghost, half left, 
		momentum = \( k - q_1 - q_2 \) ] b, 
		b -- [ghost, momentum =\( q_2 \) ] c, 
		c -- [fermion] d[particle  = \( k \) ], 
	};  =  \int \frac{d ^ d k }{ ( 2 \pi ) ^ d } \phi_{ \vec{k} } \phi _{ - \vec{k} } 
	\int \frac{ d ^ d q_ 1 d ^ d q_ 2 }{ ( 2 \pi ) ^{ 2d } } \frac{1}{q_1 ^ 2 + 
	\mu _ 0 ^ 2} \frac{1}{q_ 2 ^ 2 + \mu _ 0 ^ 2 } \frac{1}{( k - q_1 - q_2 ) ^ 2 + \mu _ 0 ^ 2 }
\end{equation*} 
This is qualitatively different from 
the previous case because now we 
can write our second order contribution as 
\[
	\dots = g_0 ^ 2 \int \frac{ d ^ d k }{ ( 2 \pi ) ^ d } \frac{1}{2 } A ( k , \Lambda ) 
	\phi ^ - _{ \vec{k} } \phi ^{ -  }_{ - \vec{k}}  
\] Where $ A ( k , \Lambda ) $ actually now depends on $ k$. 
The upshot of this is that, when we return back into  \textbf{real space}, 
we have that since we expand $ A ( k ) = A ( 0 )  + \frac{1}{2 } k^ 2 A '' ( 0 , \Lambda ) $
(ignoring odd powers of $k $ since they ultimately vanish), we get that 
now our \textbf{gradient term } picks up an extra contribution now. 

 
Now, we can write things as general functions of $ \vec{k}  $ and $\Lambda $
when dealing with more complicated terms. 
This means that a diagram of the below form gives us a term like 
\[
	g_0 ^ 2 \int \frac{ d ^ d k }{ ( 2 \pi ) ^ d } \frac{1}{2 } A( \Lambda , k) \phi_{\vec{k} } ^ - 
	\phi _{  - \vec{k} } ^ -  
\] We can now Taylor expand out our general function 
$ A ( k, \Lambda ) $ as 
 \[
	 A ( k , \Lambda ) \simeq A ( 0 ) + \frac{1}{2 } k ^ 2 A'' ( 0 ) + \dots
 \] Now, the first term here
 gives us a correction to $\mu ^ 2  $, and the second term 
 gives us a correction to $ ( \nabla \phi ) ^ 2 $. 
 So, in our full effective field theory we have that 
 \[
	 F ' [ \phi ] = \int d^ d x \, \frac{1}{2 } \gamma ' ( \nabla \phi ) ^ 2, 
	 \quad \gamma ' =  1- 2 g_0 ^ 2 A'' ( 0 , \Lambda ) 
 \] Hence, our rescaled momenta is given by 
 \[
	 \phi_{ \vec{k} } ' = \frac{ \zeta ^{  - \frac{ d + 2 }{ 2 } } }{1 - g_0 ^ 2 A'' ( 0 
	 , \Lambda ) } \phi_{ \frac{ \vec{k} }{ \zeta } } ^ - 
 \]
 \subsection{$ \beta$ - functions}
 Our new cutoff in RG flow is given by the relation 
\[
 \Lambda ' = \frac{ \Lambda }{ \zeta  } = \Lambda e^{ - s }
\] We define our beta functions to be the 
solutions to the differential equation 
\[
	\frac{ d g_ n }{ ds } = `\beta_ n ( g _ n ) , \quad \begin{cases}
		\beta_ n > 0 & \text{ in the IR } \\
		\beta _ n < 0 & \text{ otherwise }
	\end{cases}
\] Recall, that our leading order 
contributions for our coupling constants are given by 
\[
 \mu ^ 2 ( \zeta ) = \zeta ^ 2 ( \mu _ 0 ^ 2  + ag_0 ), \quad
 g ( \zeta ) = \zeta ^{ 4 - d  } ( g_0 - b g_0^ 2 ) 
\] where our constants are given by 
\[
	a = 12 \int_{ \Lambda / \zeta } ^ \Lambda \frac{  d^ d q }{ ( 2 \pi ) ^ d }
	\frac{1}{q ^ 2 + \mu _ 0 ^ 2}, \text{ and } , \quad
	b = 36 \int _{ \Lambda / \zeta } ^  \Lambda \frac{ d ^ d q }{ 
	( 2 \pi ) ^ d  } \frac{1}{( q ^ 2 + \mu _ 0 ^ 2 ) ^ 2 }
\] We differentiate these things with resepct to $ \zeta $. 
\[
	\int _{ \Lambda e ^{ - s} } ^ \Lambda dq \, f ( q) 
	\simeq \left[  \Lambda - \Lambda e^{ - s}  \right] f ( \Lambda ) 
	\simeq \Lambda f ( \Lambda ) s \implies 
	\frac{ d }{ ds } \int _{ \Lambda e ^{ - s} } ^ \Lambda 
	dq \, f ( q ) = \Lambda f ( \Lambda ) 
\] In four dimensions, we integrate this 
thing in terms of spherical polars so that 
we have that our coupling constants are solved by 
\[
 \frac{d \mu ^ 2 }{ ds } = 2 \mu ^ 2 + \frac{ 3g }{ 2 \pi ^ 2 } 
 \frac{ \Lambda ^ 4 }{ \Lambda ^ 2 + \mu ^ 2  } \text{ and } \quad
 \frac{ dg }{ ds } = \frac{ - 9 }{ 2 \pi ^ 2 } \frac{ \Lambda ^ 4 }{
 ( \Lambda ^ 2 + \mu ^ 2 ) } g ^ 2  
\] This gives us differential equations that 
we can solve. 

\subsection{$ \epsilon $ Expansions} 
We learned that our $ \phi ^ 4 $ interaction 
is irrelevant for $ d \geq 4 $ and relavant otherwise. 
Now, lets employ a strange trick and perturb our
dimensions, for example by setting 
\[
 d  = 4 - \epsilon  
\] Returning to our previous beta functions 
which we calculated in four dimensions, generalising 
these to beta functions in $ d $ dimensions gives us 
new differential equations 
\begin{align*}
	\frac{ d \mu ^ 2 }{ ds } &=  2 \mu ^ 2 + 
	\frac{ 12 \Omega_{ d- 1 } }{ ( 2 \pi ) ^ d } \frac{ \Lambda ^ 4  }{ 
	\Lambda ^ 2 + \mu ^ 2  }  \tilde{ g } + \dots  \\
	\frac{ d \tilde{ g }  }{ds } = 
	\epsilon \tilde{ g } - \frac{ 36 \Omega_{ d- 1 } }{ ( 2 \pi ) ^ d }
	\frac{ \Lambda ^ 4 }{ ( \Lambda ^ 2 + \mu ^ 2 ) ^ 2 } \tilde{ g } ^ 2 + \dots 
\end{align*}
We use the approximation $ \Omega_ 3  = 2 \pi ^ 2 + O ( \epsilon ) $ 
to give approximate equations 
 \begin{align*}
	 \frac{ d \mu ^ 2 }{ds } \simeq & 2 \mu ^ 2 + \frac{3}{2 \pi ^ 2 } \frac{
	 \Lambda ^ 4 }{ \Lambda ^ 2 + \mu ^ 2 } \tilde{ g } \\
	 \frac{ \tilde{g }  }{ds  } \simeq & 
	 \epsilon g  - \frac{ 9 }{ 2 \pi ^ 2 } \frac{ \Lambda ^ 4 }{
	 ( \Lambda ^ 2 + \mu ^ 2 ) ^ 2 } \tilde{g } ^ 2  
\end{align*}
From these things, we have the Gaussian 
fixed point we derived before, but 
we also have a new fixed point given by 
\[
 \mu _{ * } ^ 2 = - \frac{3 }{ 4 \pi ^ 2 } \frac{ \Lambda ^ 4 }{ 
 \Lambda ^ 2 + \mu _ * ^ 2 } \tilde{ g } _ * , \quad 
 \tilde{ g } _ * = \frac{ 2 \pi ^ 2 }{ 9 } \frac{ ( \Lambda ^ 2 + \mu _ * ^ 2 ) ^ 2 }{
 \epsilon } 
\] To leading order in $\epsilon $, our solutions for these 
fixed points are 
\[
 \mu _ * ^ 2 =  - \frac{1}{6 } \Lambda  ^2 \epsilon , \quad 
 \tilde{ g } _ *  = \frac{ 2 \pi ^ 2  }{ 9 }  \epsilon 
\] 
\subsubsection{Wilson-Fisher Fixed Point} 
We now work perturbatively 
to understand flows near this fixed point. 
We write out 
\[
 \mu ^ 2 = \mu ^ 2 _ * , \quad 
 \tilde{ g } = \tilde{ g } _ * + \delta \tilde{ g }    
\]  We use 
the matrix method to expand this out 
as 
\[
 \frac{d }{ ds } \begin{pmatrix} \delta \mu ^ 2 \\ 
 d \tilde{ g } \end{pmatrix}  = \begin{pmatrix}  
 2 - \frac{\epsilon}{3  } & \frac{ 3 }{ 2 \pi ^ 2 } \Lambda ^ 2 
 ( 1 + \frac{\epsilon }{ 6 } )  \\ 
 0 & - \epsilon  \\ \end{pmatrix} \begin{pmatrix} \delta \mu ^ 2 \\
 \delta \tilde{ g } \end{pmatrix} 
\] This gives us that 
our scaling dimension coefficients 
are, to first order in $ \epsilon $, 
that 
\[
	\Delta _ t = 2 - \frac{ \epsilon }{ 3 } + \mathcal{ O } ( \epsilon ^2 ) , 
	\quad \Delta _g = - \epsilon + \mathcal{ O  } ( \epsilon ^ 2 )  
\] 

\subsubsection{Revisiting critical exponents} 
Expanding out our critical exponents, 
we have that 
\begin{align*}
	t \to \zeta ^{ \Delta _ t  }  t & = e^{ s \Delta _ t } t  \\
	\xi \sim t ^{ - \nu } & \nu = \frac{1 }{ \Delta _ t } = \frac{1}{2 } 
	+ \frac{ \epsilon }{ 12 } 
\end{align*}
From our old scaling relations, we have that 
\[
 c \sim t ^{ - \frac{ \epsilon }{ 6}  }, \quad 
 \Delta _{ \phi } \sim \frac{ d - 2}{2 }  = 1 - \frac{ \epsilon   }{ 2 }, 
 \quad \beta = \frac{1}{2 } - \frac{ \epsilon }{ 6 }, \quad 
 \gamma =  1 + \frac{ \epsilon }{ 6 } , \quad
 \delta = 3 + \epsilon 
\] We've made a small perturbation, set this perturbation 
to be one, themn recovered our exponents. 
Now, in the case when $ d =  2 $, there's 
no price to be paid for adding $ \phi $ since 
$ \phi $ has zero dimension. 
Our free energy looks like 
\[
 F [ \phi ] = \int d ^ d x \left[  
 \frac{1}{2 } ( \nabla \phi ) ^ 2  + g_{ 2 ( n + 1 )} \phi ^{ 2 ( n + 1 ) }\right] 
\]
This gives us something called conformal symmetry. 
This means that our theory is scale invariant, 
under transformations 
\[
 \vec{x} \to \lambda \vec{x}
\] We can enhance this type of symmetry 
from 
\[
	\vec{X} \to \tilde{\vec{X} } ( \vec{x} )  , \quad 
	\frac{\partial \tilde{ x } ^ i  }{\partial  x_ k }
	\frac{\partial  \tilde{ x } _ j  }{\partial  x _ l } \delta _{ ij }
	= \phi ( \vec{x} ) \delta _{ kl }
\]  This motivates the idea of 
conformal transformations 
\[
	\tilde{ x } ^ i = \frac{x ^ i - ( \vec{x} \cdot  \vec{x} ) a ^ i }{
	1 - 2 ( \vec{x} \cdot  \vec{a} ) + ( \vec{a} \cdot  \vec{a} ) ( \vec{x} \cdot  \vec{x} )  } 
\] 

