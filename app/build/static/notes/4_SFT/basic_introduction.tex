\subsection{Basic Introduction} 
Consider the free energy 
\[
	F [ \phi ] = \int d^ d x\,  \left[  \frac{1}{2 } ( \nabla \phi ) ^ 2 + \frac{1}{2 } \mu ^ 2 \phi ^ 2 + g \phi ^ 4 + \dots  \right] 
\] Recall that when we're writing this theory, 
we only used the few basic assumptions such as $ \mathbb{ Z} _ 2 $ symmetry, 
analyticity, and so on. 
So in essence, we could have included terms like 
$ g _{ 126} \phi ^{ 126 } $,  $ ( \nabla ^ 2 \phi ) ^ 2 $ and so on. 
These terms have coupling constants in front of them, 
such as $ \mu ^ 2 = T - T _ X $, which we call Wilson coefficients. 
We can also rescale our coupling coefficients so that we 
absorb $ \beta $ in our free energy, so we can write 
\[
	\mathcal{Z  } = \int \mathcal{ D } \phi \, e ^{  - F [ \phi ] }
\]  The final piece of the theory is that we also 
specificed some cutoff $ \Lambda $ such that 
\[
 \phi _{ \vec{k} } =0 , \quad  |  \vec{k} |  \geq \Lambda, \Lambda \sim \frac{\pi}{ a }
\] The crux of the renormalisation group 
is what happens when we ask what happens when we change $ \Lambda $? 
Since what we wrote down the most general theory to begin with, 
when we change  $ \Lambda  $ we expect the same form of theory 
but with different coupling constants. 
When we change the cutoff we call this a 'flow'. 

