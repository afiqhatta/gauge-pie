\section{Example Sheet 3} 


\subsection{Question 1}
For looking at the stability 
of the Wilson Fisher fixed point, it's 
a coincidence that the eigenvector associated with one of the 
eigenvalues is $ \begin{pmatrix}  1 & 0  \end{pmatrix} $, which 
corresponds to the perturbation in the $ \delta \mu ^ 2 $ direction. 

\subsection{Question 3}
Look at long range behaviour 
to think about positive definiteness and such.
For example, take a specific field configuration (take $ \phi _ 1 \to \infty$, 
and all else zero). 

Looking at positions of fixed points, 
be careful about where they're placed. 
Make sure we don't have inaccurate labelling of positions! 

Where are the relavant operators? 
They are the eigenvalues corresponding in a certain direction. 

\subsection{Question 4}
The sine-gordon model. 
When we do rescaling, 
our general goal is to map 
\[
 \int \mathcal{ D } \phi ^ - \mathcal{ D  } \phi ^ + e ^{ - F _ 0 \left[  \phi  \right]  }. 
\] 

\subsection{Question 5}
When we compute correlation functions just go straight 
for the 
