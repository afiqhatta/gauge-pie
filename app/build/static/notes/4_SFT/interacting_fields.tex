\subsection{Renormalisation with Interacting Fields} 

\subsubsection{A quick review of finding the effective free energy} 

In the case of a 'free theory',
our energy is written in terms of quadratic terms in 
$ \phi $. Free energy cases are important, because when we split
Fourier modes, we saw earlier that we could split up 
the free energy into a high frequency part and a low frequency part. 
\[
	F_0 [ \phi ] = F_0 \left[  \phi ^ +  \right]  + F_ 0 \left[  \phi ^ -  \right] 
\] However, much like we do in quantum field theory,
these cases, while easy, are not very interesting. In this article, 
we'll explore how to use Wick's theorem to look at the 
behaviour of energies that look like
\[
	F [ \phi ] = \int d^ d x \frac{1}{2 } \left(  \nabla \phi  \right) ^ 2 
	 + \frac{1}{2 } \mu _ 0 ^ 2 \phi ^ 2 + g \phi ^ 4 
\] We've added a $ \phi ^ 4 $ term on the end here. 
This is completely similar to $ \phi ^ 4 $ theory 
in quantum field theory. Now, we ask what happens when we split 
this up into higher and low Fourier modes. In real space, 
we expect that our free energy splits into free terms and 
an interaction term. 
\[
	F [ \phi ] = F_0 [ \phi ^ + ] + F _ 0 [ \phi ^ - ] 
	+ F _ I  [ \phi ^ + , \phi ^ - ] 
\] The presence of the interaction term mean that there
will be some mixing of high and low frequency modes. 
Note the odd notation here. Since we have that 
$ \phi = \phi ( \phi ^ + , \phi ^ - ) $, we write 
 \[
	 F[ \phi ] = F_0 [ \phi ^ + ] + F_0 [ \phi ^ - ] 
	 + F _ I ( \phi ), \quad F _ I ( \phi ) = \int d^ d x \, g \phi ^ 4 
\] So far we've looked at how things look 
in $ \phi $ space, but let's have a quick look as well 
about how things evolve from the perspective of Fourier 
space variables $ \phi_{ \vec{k}}$. Starting 
from our partition function 
\[
	\mathcal{ Z } = \int \mathcal{ D } \phi e ^{ - F [ \phi ] }  
\] We rewrite our measure in terms of a product of Fourier 
coefficients. In this product, we separate high and low frequency 
terms, and 'integrate' out our high frequency terms. This 
then prompts us to write our partition function as 
\[
 \mathcal{ Z } = \int \prod_{ 0 < k < \Lambda ' } d \phi _{ \vec{k} } ^ - d \phi_{ - \vec{k}  } ^ + e ^{ - F_0 [ \phi_{ \vec{k} }^ 
  - ] } \int \prod_{ \Lambda ' < k < \Lambda } dd \phi_{ \vec{k} } ^ + \phi _{ - \vec{k} } ^ + 
  e^{  - F_0 [ \phi _{ \vec{k} } ^ +] } e^{  - F_ I [ \phi _{ \vec{k} } ^ + , \phi _{ \vec{k} } ^ - ]  }
\] This means we define an 'effective' free energy 
by writing our whole sum as just an integral 
over the lower Fourier modes. We rewrite the 
above term as 
\[
  \mathcal{ Z } = \int \prod_{ 0 < k < \Lambda } d \phi _{ \vec{k} } ^ -  d \phi _{  - \vec{k} } ^ - 
  e ^{ - F ' [ \phi _{ \vec{k} }  ] }
\] Comparing coefficients, our 
new effective free energy from the above expression is given by 
\[
	e ^{  - F ' [ \phi_{ \vec{k} } ^ -  } = e^{  - F_0 \left[  \phi _{ \vec{k} } ^ -  \right]  }
	\int \prod_{ \Lambda ' < k \Lambda } d \phi _{ \vec{k}  } ^ + d_{ \phi _{ - \vec{k} } } ^ + 
	e ^{  - F_{I } [ \phi_{ \vec{k} } ^ + , \phi _{ \vec{k} } ^ - ] }
\] We can write this 
as more familiarly as a functional integral over 
just the positive Fourier modes, so that 
\[
	e ^{  - F ' [ \phi _{ -\vec{k} } ^ - ] } = 
	e^{  -F_0 [ \phi _{ \vec{k} } ^ - ]  } \int \mathcal{ D } \phi_{ \vec{k} } ^ + 
	e^{  - F_0 [ \phi _{ \vec{k} } ^ + ]} e^{ i F_I }
\] Let's stare hard at the nature of the term on 
the right hand side. We're integrating $ F _ I $ over high frequency 
Fourier modes but with a pre-faction of $ e ^{  - F _0 }$. 
This precisely is an expression for the expectation of 
$ F _ I $ but over high frequency Fourier modes instead. 
So, we write that 
\[
 	e ^{  - F ' [ \phi _{ -\vec{k} } ^ - ] } = 
	e^{  -F_0 [ \phi _{ \vec{k} } ^ - ]  }  \left< e ^{ - F_I [ \phi_{ \vec{k} } ^ + , 
	\phi _{ \vec{k} } ^ - ] } \right>_+ 
\] 
\subsubsection{Logarithmic expansion of the Interaction Term} 
Now, a quantity immediately 
of interest is what the average value $ \left< F _ I  \right> $ is, 
and how it can give us an expression for 
our effective free energy $ F ' $. 
Now, there's not much we can do about finding 
this average value other than to attempt to 
Taylor expand the equation above 
upon taking logs. 
\[
	F ' [ \phi_{ \vec{k} }  - ] = F_0 [ \phi _{ \vec{k} } ^ - ]  - \log \left< e^{  - F_I } \right>
\] employing the Taylor expansion for $ \log$, 
something special happens because we get 
an expression for the cumulant of $ F _ I $; 
 \[
	 F ' = F _ 0 + \left< F _ I  \right> + \frac{1}{2 } \left( \left< F _ I  \right> ^ 2 - 
	 \left< F _ I  ^ 2  \right> \right) 
\]

\subsubsection{Computing moments of the Interaction Free energy} 
Now, the name of the game is to actually compute 
quantities like $ \left< F _ I  \right>$
Let's do this explicitly for an interaction 
term which has its analogue in $ \phi ^ 4 $ theory 
in quantum field theory. We set 
\[
	F_ I = g \int d^ d x \, \phi ^ 4 ( x ) 
\] where $ g $ is our small perturbation order term. 
The first thing to do is to expand out 
$ F _ I $ in terms of Fourier modes. 
This gives a product of a bunch of Fourier modes
which we need to sort through.
\begin{align*}
	F_ I [ \phi ] &=  \int d ^ d x \int \prod^ 4 _{ i = 1 }
	\frac{ d ^ d k _ i }{ ( 2 \pi ) ^ d } 
	e ^{ - i \vec{k}_ i \cdot  \vec{x} } \phi _{ \vec{k} _ i } \\
		      &=  \int \prod_{ i = 1 } ^ 4 \frac{d ^ d k _ i }{ ( 2 
		      \pi ) ^ d } \,  
		      \phi _{ \vec{k} _ i } ( 2 \pi ) ^ d 
		      \delta ( \sum _ i \vec{k} _ i ) \\ 
\end{align*}
In this step, we mutliplied out the Fourier modes and then 
moved the $ \int d ^ d x $ inside to get the delta function. 
This delta function has the physical interpretation of 
imposing momentum conservation. Now, 
let's do what we did with RG and split our Fourier modes 
$ \phi_{ \vec{k} _ i}  $ into high frequency and 
low frequency parts. 
Our product looks like 
\[
 \prod_{ k = 1 } ^ 4 \phi_{ \vec{k} _ i } = 
 ( \phi _{ \vec{k}_ 1  }^ + + \phi _{ \vec{k} _ 1 } ^ - ) 
 \dots ( \phi _{ \vec{k} _ 4 } ^ + + \phi_{ \vec{k} _ 4 } ^ - ) 
\] Now, 
let's think about what kind of terms we get from 
this product. We count these by considering how many 
low frequency versus high frequency modes there are 
in the expansion. 

\begin{enumerate}
	\item We have terms whose all four factors 
		are low frequency modes. There is only 
		one term of this type, and it's 
		$ \phi _{ \vec{k} _ 1 } ^ - \phi _{ \vec{k} _  2  } ^ - 
		\phi _{ \vec{k} _ 3 } ^ - \phi _{ \vec{k} _ 4 } ^ - $.
		This means that when we compute our 
		average value $ \left< F _ I  \right>_ + $, 
		since this an average over $ \phi ^ + $, 
		this term just factors out of our intergal. 
		We then observe that this is just, in real space, 
		our original interaction term but
		this time with just low frequency modes. 
		Thus, we have a  $ \int d ^ 4 x \, \phi ^ - $ 
		contribution. 
	\item If we have just one factor of $ \phi _{ \vec{k} } ^ +  $
		with the rest of the terms being 
		$ \phi _{ \vec{k} } ^ - $ (we have 
		4 of these in total), then, since we are 
		averaging over an odd number of $ \phi _{ \vec{k} } ^ + $, 
		this vanishes since our Gaussian integral is an 
		even function. Terms like this 
		look like $ \phi_{ \vec{k} _ 1 } ^ - 
		\phi _{ \vec{k} _ 2 } ^ - \phi _{ \vec{k} _ 3 } ^ - 
		\phi _{ \vec{k} _ 4 } ^ + $. When we take the 
		average, we're including a factor 
		of 
		\[
		\left< \phi _{ \vec{k}} ^ +  \right>_ + 
		 = \int \mathcal{ D } \phi_{ \vec{k}} ^ + e ^{ - 
		 F _ 0 [ \phi _{ k } ^ +  ] } \phi_{ \vec{k} } ^ + 
		\] This goes to zero since we have an 
		odd term and an even term integrated together. 
	\item Our only interesting term are terms 
		which look like $ \phi _{ \vec{k} _ 1} ^ - 
		\phi _{ \vec{k} _ 2}  ^- \phi _{ \vec{k} _ 3 } ^ + 
		\phi _{ \vec{k} _ 4} ^ +  $. This contribution is 
		non-trivial and we'll do some work to compute this 
		explicitly in the following section. There 
		are 6 of these terms in the expansion. 
	\item By the same logic with just one $ \phi _{ \vec{k} } ^ + $
		terms that contain things like $
		\phi _{ \vec{k} _ 1 } ^ + \phi _{ \vec{k} _ 2 }^ + 
		\phi _{ \vec{k} _ 3} ^ + \phi _{ \vec{k} _ 4 } ^ -  $
		vanish since we have an odd number of terms 
	\item Terms with just $ \phi _{ \vec{k} } ^ + $ are integrated 
		out and contribute a constant. Let's 
		go through why this 
		is the case, slowly. Recall that our quantity 
		of interest to compute as 
		\[
		 F ' [ \phi ^ -  ] = F_ 0 [ \phi ^ - ] 
		   - \log \left< e ^{ - F _ I } \right>_ +  = 
		   F_ 0 [ \phi ^ - ] + \left< F _ I  \right>_ +  
		\] Now, since this term \textbf{doesn't } have 
		any dependence on $ \phi _ k ^ -   $,
		we don't care about it because our 
		effective free energy is only a function 
		of  $ \phi _ k ^ - $. Hence, this 
		term is only an extraneous constant addition to 
		our effective free energy, and hence 
		doesn't contribute. 
\end{enumerate}
 

Thus we focus on the only interesting term, with two 
$ \phi _ \vec{k} ^ + $  factors. This term looks like 
\[
 \left< \phi_{ \vec{k} } ^ + \phi_{ \vec{k} ' } ^ +  \right> _ + 
\] How do we calculate this? Well, forgetting about 
factors of $ \beta $, we showed earlier that the 
correlation function in \textbf{position space} looks 
like
\[
	\left< \phi ( \vec{x} ) \phi (  \vec{y} )   \right> = 
	\int \frac{ d ^ d k }{ ( 2 \pi ) ^ d } 
	\frac{ e ^{  - i \vec{k} \cdot  ( \vec{x} - \vec{y} ) }}{k ^ 2 + \mu _ 0 ^ 2 }
\] Now, one can convince themselves quite easily 
that when we Fourier expand out, then wrangle out $ \vec{x} $ and $ \vec{y}$
by integrating over them, to find that 
\[
 \left< \phi_{ \vec{k} } ^ + \phi_{ \vec{k} ' } ^ +  \right> 
 = \frac{ ( 2 \pi ) ^ d  \delta ^ d ( \vec{k} + \vec{k} ' ) }{
 k ^ 2 + \mu _ 0 ^ 2 }
\] 
Hence, all together, the only term which contributes 
to the correction $ \left< F_I  \right> _ + $ is 
\begin{align*}
	\left< F _ I  \right> _ + &= 6  
	\int \prod_{ i = 1 } ^ 4 \frac{ d ^ d k_ i }{ ( 2 \pi ) ^ d } 
	\left< \phi _{ \vec{k} _ 1 } ^ + \phi _{ \vec{k} _ 2 } ^ +  \right> _ + 
	\phi _{ \vec{k} _ 3 } ^ - \phi _{ \vec{k} _ 4 } ^ - 
	( 2 \pi ) ^ d \delta ^ d ( \sum _ i \vec{k} _ i  ) \\
				  &= 6 \int \prod_{ i = 1 } ^ 4 \frac{ d ^ d k _ i }{ ( 2 \pi ) ^ d } \frac{ ( 2 \pi ) ^{ 2d } \delta ^ d ( 
				  \vec{k} _ 1 + \vec{k} _ 2 + \vec{k} _ 3 + \vec{k} _ 4 ) \delta ^ d ( \vec{k} _ 1 + \vec{k} _ 2 ) \phi _{ \vec{k} _4 } \phi _{\vec{k} _ 4 } }{ k_  1 ^ 2 + \mu _ 0 ^ 2 } \\
	 &=  6\int d ^ d k_ 1 d ^ d k_ 2 d ^ d k _ 3 
	 \frac{1}{( 2 \pi ) ^ {2d} } \frac{ \delta ( \vec{k} _ 3 + \vec{k} _ 4 ) 
	\phi _{ \vec{k} _ 3 } \phi _{ \vec{k} _ 4 } }{k_ 1 ^ 2 + \mu _ 0 ^ 2}\\
				  &= 6 \int \frac{d ^ d k _3 }{ ( 2 \pi ) ^ d }
				  \phi _{ \vec{k} _ 3 }^ -  \phi _{  - \vec{k} _ 3 } ^ - \int \frac{ d ^ d k _ 1 }{ ( 2 \pi ) ^ d } \frac{1}{k _ 1 ^ 2 + 
				  \mu _ 0 ^ 2 } \\
				  &= 6 \int_ 0 ^{ \Lambda / \zeta } \frac{d ^ d k }{ ( 2 \pi ) ^ d } \phi_{ \vec{k} } \phi _{  - \vec{k} } \int_{ \Lambda / \zeta } ^ \Lambda   \frac{d ^ d q }{ ( 
				  2 \pi ) ^ d } \frac{ 1 }{ q ^ 2 + \mu _ 0 ^ 2 }
\end{align*}
In the last step, we've taken into account the 
ranges of integration for the high and low frequency modes.
Now, we can look at the impact this on our 
effective free energy. 

This means in our effective free energy, 
\begin{align*}
	F ' [ \phi ] &=  F_ 0 [ \phi _ - ] + \left<  F _ I \right>_+ \\
		     & = \int _ 0 ^{ \Lambda / \zeta } \frac{d ^ d k }{ ( 2 \pi ) ^ d } \frac{1}{2 } 
		     \left(  k ^ 2 + \mu _ 0 ^ 2  \right)  \phi_{ \vec{k} } \phi _{  - \vec{k} } + 6 \int ^{ \Lambda / \zeta } _ 0  \frac{d ^ d k }{ ( 2 \pi ) ^ d }
		     \phi_{ \vec{k} } \phi _{  - \vec{k} } 
		     \int _{ \Lambda / \zeta } ^ \Lambda 
		     \frac{ d ^ d q }{ ( 2 \pi ) ^ d } \frac{1}{ q ^ 2 + \mu_ 0 ^ 2 } 
\end{align*}

This is equivalent however to remapping $ \mu _ 0 ^ 2 $ to 
itself plus some correction term that 
we get from the interacting theory which incorporates high 
energies. 
\[
 \mu _ 0 ^ 2 \to \mu _ 0 ^ 2 + 12 \int_{ \Lambda / \zeta } ^ \Lambda 
 \frac{ d ^ d q }{ ( 2 \pi ) ^ d } \frac{1}{q ^ 2 + \mu _ 0 ^ 2 }
\] 
What's the point in doing all this? Well, 
we saying that when we turn on an interaction 
term in $ \phi ^ 4 $, we're inducing 
a change in how the $ \phi ^ 2 $ coupling term changes.
In particular, we induce a RG flow. The $ \mu ^ 2 $ 
coefficient flows as 
\[
 \mu ^ 2 ( \zeta ) \to \zeta ^ 2 ( \mu _ 0 ^ 2 + 12 \int_{ \Lambda / \zeta } ^ \Lambda 
 \frac{ d ^ d q }{ ( 2 \pi ) ^ d } \frac{1}{q ^ 2 + \mu _ 0 ^ 2 }
\]

\subsubsection{Second order corrections} 

