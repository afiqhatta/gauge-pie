\subsection{Correlation functions} 
In mean field theory, we had no local specific information. 
\[
	\left< \phi ( \vec{x} )  \right> =  \begin{cases}
		0 & T > T_c \\
		\pm m_0 & T < T_c 
	\end{cases}
\] Our next simplest way to think about correlations is
\[
	\left< \phi ( \vec{x} ) \phi ( \vec{y} )  \right>
\] this tells us how fluctations propagate throughout time by encoding spatial
variations at multiple points. A naive way to 
think about this is how waves propagate in the material. 
Moreover, we can think about functions deviating from the mean
as the \textbf{connected correlation function}. 
This is defined as 
\[
	\left< \phi ( \vec{x} ) \phi( \vec{y} )  \right>_c = \left< \phi( \vec{x} ) \phi ( \vec{y} )  \right> - \left< \phi  \right>^2 
\] This is also known as the cumulant in statistics. 
We can compute these correlation functions explicitly. 

If we define 
\[
	F[ \phi ( \vec{x} ) ] = \int d^{  d} x \left[ \frac{\gamma}{2 } ( \nabla \phi)^ 2 + \frac{\mu^2 }{2 } \phi^2 + B ( x) \phi  \right] 
\] with the partition function 
\[
	Z [ B ( x) ] = \int \mathcal{ D } \phi e^{  - \beta  F [ \phi ( \vec{x} ), B ( \vec{x} ) ]}
\] We can take a functional derivative to get
\[
	\frac{1}{\beta } \frac{\delta \log \mathcal{ Z }}{\delta B ( \vec{x} ) } = \frac{1}{\beta \mathcal{ Z }} \frac{ \delta \mathcal{ Z }}{ \delta B ( \vec{x} ) }
\] However, this term, the benefit of doing this is that differentiating by $B ( \vec{v} ) $ pulls out a factor of $ \phi $ in the integral. Hence, the quantity we've calculated is 
\[
	- \frac{1}{\mathcal{ Z } } \int \mathcal{ D } \phi ( \vec{x})  e^{  - \beta F[ \phi , B  ] } =  - \left< \phi ( \vec{x} )  \right> \mid_{ B }
\] So, we've found a neat way to encode the average function in our partition function. 
This is exactly the same as encoding our average magnetisation 
in our partition function using mean field theory earlier. 
We can get even more information from this. We calculate an object called the \textbf{two-point} function. 
In analogy to moment generating functions in statistics, 
differentiating this a second time enables us to get the connected 
correlation function we listed earlier. If we act on what we have 
with $  \frac{ \partial  }{ \beta \partial  }$, the product rule gives us 
\[
	\frac{1}{ \beta ^ 2 } \frac{\delta^ 2   \log \mathcal{  Z} }{\delta B ( \vec{x} ) \delta B ( \vec{y} ) } = \frac{1}{ \beta ^ 2 \mathcal{ Z } } \frac{ \delta ^2 \mathcal{ Z } }{\delta B ( \vec{x} ) \delta B ( \vec{y} ) }  - \frac{1}{ \beta ^ 2 \mathcal{Z } ^ 2 } \frac{ \delta \mathcal{ Z } }{ \delta B ( \vec{x} ) } \frac{ \delta \mathcal{ Z } }{ \delta B ( \vec{y}) } 
\] This was obtained by applying the product rule on $ \frac{1}{ \mathcal{ Z } } $ as 
well as the derivative terms. 
The upshot is that the first term pulls down two factors, $ \phi ( \vec{x} )  $ and $ \phi ( \vec{y} ) $ from our free energy. 
Hence, our first term is 
\[
	 \frac{1}{ Z }\int \mathcal{ D } \phi \, \phi ( \vec{x} ) \phi ( \vec{y} ) e^{  - \beta F } = \left< \phi ( \vec{x} ) \phi ( \vec{y} )  \right>\mid_B
\]  Further more, observe that the second term is just 
$ \left< \phi  \right>^ 2 $. 
Thus, our final correlation function can be written in the compact form, writing our subscript $ B $ to denote evaluation at the field configuration $ B $, 
\[
	\left< \phi ( \vec{x} ) \phi ( \vec{y} )  \right>_c  = \frac{1}{\beta ^ 2 } \frac{ \delta ^ 2 \log \mathcal{ Z } }{\delta B ( \vec{x} ) \delta B ( \vec{y} ) } = \left< \phi ( \vec{x} ) \phi_( \vec{y} )  \right>_B - \left<\phi( \vec{x} ) \right>_B \left< \phi( \vec{y} )  \right>_B 
\] Now, above the critical temperature $ T > T_{  c} $, we have that our average 
magnetisation is given by $ \left< \phi ( \vec{x} )  \right>  = 0 $ when $B = 0$. 
Thus, we can condense what we have even further by 
 \[
	 \frac{1}{ \beta ^ 2 } \frac{ \delta ^ 2 \log \mathcal{ Z }  }{\delta B ( \vec{x} )  \delta B ( \vec{y} ) }\mid_{ B  =0 } \left< \phi ( \vec{x} )\phi ( \vec{y} )   \right>_{ B = 0} 
\] 


