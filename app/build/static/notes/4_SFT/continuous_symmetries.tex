\section{Continuous Symmetries} 

What we learned thus far is that 
phase transitions are characterised by symmetries. 
Up until now for example, in the Ising model, we had $ \mathbb{ Z}_ 2  $  symmetry. 
In this section, we'll start to characterise symmetries 
by their type. We label these symmetries as $ G $ and $ H $. 
\begin{enumerate}
	\item G : a symmetry of the free energy 
	\item H : a symmetry of the ground state. 
\end{enumerate} 
Consider the Ising model. 
When we are above the critical temperature, we're 
at the zero state, so we have that 
\[
 \begin{cases}
	 G & = \mathbb{ Z} _ 2 \\
	 H & = \mathbb{ Z} _ 2 
 \end{cases}
 T > T _ c 
\] On the other hand, when we 
have $ T < T _ c$, we have that 
\[
 \begin{cases}
	 G = \mathbb{ Z} _ 2 \\
	 H = 0 
 \end{cases}
 T <  T _ c 
\]
Symmetries are useful because we can then write down the order parameter
and then write down the free energy. The nice
thing about symmetries is that they're very general  - we don;t have
to worry about coupling constants. 
Many systems are characterised by spacetime symmetries, for example crystals. 
However, the most interesting class of symmetries 
are internal symmetries, which are symmetries of the field themselves. 

\subsection{$ O ( n ) $ modes} 
A classic example are the 
$ O ( n ) $ modes. If we have $ n $ scalar fields, 
represented by a vector 
\[
	\mathbf{ \phi } ( \vec{x} )  = \begin{pmatrix}  
	\phi _ 1 ( \vec{x} ) \\ \vdots \\ \phi _ n \end{pmatrix} 
\] We have a symmetry when we map 
\[
	\phi _ a ( \vec{x} ) \to R \indices{_ a ^ b } \phi _ b ( \vec{x} ) , \quad R ^ T  R = 1 
\]  Our free energy then looks like 
\[
	F [ \phi ( \vec{x} ) ] = 
	\int d ^ d x \left[ 
	\frac{ \gamma }{ 2 } ( \nabla \phi _ j ) \cdot  ( \nabla \phi _ j ) +  \frac{\mu ^ 2 }{ 2}
\phi \cdot  \phi + g ( \phi \cdot  \phi ) ^ 2 + \dots \right] 
\] 
The first model which we will unpack is the XY model, shorthand 
for the $ O ( 2) $ model. 

\subsubsection{XY - Models} 
We can package up the two fields in the XY model 
as $ \psi ( \vec{x} )  = \phi _  1 ( \vec{x} )  + i \phi_ ( \vec{x} ) $, 
using $ U ( 1 ) \sim SO ( 2) $. 
Our free energy which is then generated by this symmetry is 
given by 
 \[
	 F [ \phi ( \vec{x} ) ] = \int  d^ d x \, 
	 \left[  ( \nabla \psi ^ * ) \cdot  ( \nabla \psi ) + \
 \frac{\mu ^ 2 }{ 2 } | \psi | ^ 2 + g | \psi | ^ 2\right] 
\] There are a lot of physical applications of this model. 
Objects in this universality class include magnets and superfluids! 

We can play the same game and look at $ O ( 3 ) $ models, which is called the 
Heisenberg model. 
What's more interesting really than just writing these down is to 
also think about symmetries and critical points. 

\subsection{Goldstone modes} 
The statement is that for every spontaneously broken symmetry, 
we get an exactly massless scalar. This shows up everywhere! 
For example, pions in QCD and phonons in condensed matter theory. 
This is really important because massless particles are important. 
For discrete symmetries, for example $ \mathbb{ Z } _ 2 $, 
we have only a finite number of vacua. For example, in the Ising model, 
we have that 
\[
 \left< \phi _ 0  \right>  = \pm m _ 0 
\] However, if we have a continuous symmetry, 
the story is different. For example, in our XY model, 
we only constrain our magnitude. 
\[
 \left<   \|\phi \| \right> = M _ 0  = \sqrt{ \frac{ - \mu ^ 2 }{ 4 g }}  
\] Our ground states are some $ n  $- sphere. 
If we generalise this to the $ O ( n ) $ model, 
our generalised vacua is $ S ^{ n - 1 }$. 
How much energy does it cost to 'move around'? 
Our polynomial terms don't depend on phase. 
Well, our gradient terms do contribute, bu t
over long enough distances these don't contribute anything. 

So, at low energies we expect to have our Goldstone bosons 
dominating the low energy physics. 

\subsubsection{Symmetry and Goldstone Bosons} 
The number of Goldstone bosons will be equal to the 
dimension of the symmetries of the free energy
minus the symmetries of the ground state
\[
 O ( N ) : \left< \phi  \right>  = \begin{pmatrix} M \\ 0 \\ \vdots 0  \end{pmatrix} 
\] We have quotiented our our manifold by 
\[
	S ^{ N - 1 }  = O ( N ) / O ( N - 1) 
\] 
Let's apply this to our XY model. 
If we write out our symmetry 
\[
	\psi ( \vec{x} )  = \left(  M_ 0 + \tilde{ M } ( \vec{x} )   \right)  e^{ i \theta ( \vec{x} ) }
\] Our free energy shifts like
\[
 F [ M , \theta ]  = \int d ^ d x \, \left\{  
 \frac{\gamma }{ 2 } ( \nabla  \tilde{ M } ) ^ 2 + | \mu ^ 2 | \tilde{M} ^ 2 + 
 g \tilde{ M } ^ 4 + \dots + \frac{\gamma }{ 2 } M _0 ^ 2 ( \nabla \theta ) ^ 2 + 
 \gamma M _ 0 \tilde{ M } ( \nabla \theta ) ^ 2 + \dots \right\} 
\] We don't require any tuning here. Our Goldstone modes exhibit 
what we call shift symmetry. 

\subsubsection{Critical Exponents} 
Let's look at the particularly interesting 
case of $ d = 3 $. Our critical exponents look like
\begin{table}[htpb]
	\centering
	\caption{caption}
	\label{tab:label}
	\begin{tabular}{c c c }
		exponent & $ \mu $  & $ \nu $ \\
		MF & 0 & $ \frac{1}{2 }$ \\
                ising & 0.0363 & 0.6300 \\
		N = 2 &  a & b  \\ 
	\end{tabular}
\end{table}
This means our heat capacity has a cusp, 
which has been observed for superfluids. 
We have that our tiny different in the critical exponent is actually very important. 
This is known as the lambda transition. 
\[
	c \sim | T - T _ c |^{ - \alpha } , \quad \alpha   = 2 - 3 \nu , \quad \alpha _{ O ( 2) } = - 0.16 
\] 
\subsubsection{RG in the $ O ( n ) $ model} 
If we return back to our naive dimensional analysis 
\[
	[ \phi ] = \frac{ d - 2 }{ 2 }, \quad [ \mu _ 0 ^ 2 ] = 2 , \quad [ g _ 0 ] = 4 - d 
\] Now however, when 
we do our interaction term, we need to take into 
account that we are already implicitly pairing things 
together. 
\[
	F _ I [ \phi ] \sim d ^ d x \, g \phi ^ 4  = \int d ^ d x \, g ( \phi \cdot  \phi ) ( \phi \cdot  \phi ) 
\] (Insert diagram here) 
So this term contributes a term in the free energy which is $ \sim g_ 0 \delta _{ ab } \delta _{ cd } $. 

To order $ g _ 0 $, we also get new contributions that look like 
\[
 \sim \phi _ a ^ - \phi _ a ^ - \left< \phi_ b ^ + \phi _ B ^ +   \right>, \quad 
 \sim ..
\] 
\subsubsection{Epsilon expansion} 
When we go to the epsilon expansion 
\[
	\frac{ d \mu ^ 2 }{ ds } = 2 \mu ^ 2 + \frac{N + 2 }{ 2 \pi ^ 2 } \frac{ \Lambda ^ 4 }{ 
	( \Lambda ^ 2 + \mu ^ 2 ) } \tilde{ g } ^ 2, \quad 
	\frac{ \tilde{ g }  }{ ds }  = \epsilon g  - \frac{ N +  8 }{  2 \pi ^ 2 }  \frac{ \Lambda ^ 4 }{ 
	\Lambda ^ 2+ \mu ^ 2 } \tilde{ g } ^ 2, \quad \tilde{ g }  = \Lambda ^{ - \epsilon }  
\] This changes our Wilson Fisher fixed point. 
