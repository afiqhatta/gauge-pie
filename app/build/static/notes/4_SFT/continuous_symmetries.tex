\section{Continuous Symmetries} 

What we learned thus far is that 
phase transitions are characterised by symmetries. 
Up until now for example, in the Ising model, we had $ \mathbb{ Z}_ 2  $  symmetry. 
In this section, we'll start to characterise symmetries 
by their type. We label these symmetries as $ G $ and $ H $. 
\begin{enumerate}
	\item G : a symmetry of the free energy 
	\item H : a symmetry of the ground state. 
\end{enumerate} 
Consider the Ising model. 
When we are above the critical temperature, we're 
at the zero state, so we have that 
\[
 \begin{cases}
	 G & = \mathbb{ Z} _ 2 \\
	 H & = \mathbb{ Z} _ 2 
 \end{cases}
 T > T _ c 
\] On the other hand, when we 
have $ T < T _ c$, we have that 
\[
 \begin{cases}
	 G = \mathbb{ Z} _ 2 \\
	 H = 0 
 \end{cases}
 T <  T _ c 
\]
Symmetries are useful because we can then write down the order parameter
and then write down the free energy. The nice
thing about symmetries is that they're very general  - we don;t have
to worry about coupling constants. 
Many systems are characterised by spacetime symmetries, for example crystals. 
However, the most interesting class of symmetries 
are internal symmetries, which are symmetries of the field themselves. 

\subsection{$ O ( n ) $ modes} 
A classic example are the 
$ O ( n ) $ modes. If we have $ n $ scalar fields, 
represented by a vector 
\[
	\mathbf{ \phi } ( \vec{x} )  = \begin{pmatrix}  
	\phi _ 1 ( \vec{x} ) \\ \vdots \\ \phi _ n \end{pmatrix} 
\] We have a symmetry when we map 
\[
	\phi _ a ( \vec{x} ) \to R \indices{_ a ^ b } \phi _ b ( \vec{x} ) , \quad R ^ T  R = 1 
\]  Our free energy then looks like 
\[
	F [ \phi ( \vec{x} ) ] = 
	\int d ^ d x \left[ 
	\frac{ \gamma }{ 2 } ( \nabla \phi _ j ) \cdot  ( \nabla \phi _ j ) +  \frac{\mu ^ 2 }{ 2}
\phi \cdot  \phi + g ( \phi \cdot  \phi ) ^ 2 + \dots \right] 
\] 
The first model which we will unpack is the XY model, shorthand 
for the $ O ( 2) $ model. 

\subsubsection{XY - Models} 
We can package up the two fields in the XY model 
as $ \psi ( \vec{x} )  = \phi _  1 ( \vec{x} )  + i \phi_ ( \vec{x} ) $, 
using $ U ( 1 ) \sim SO ( 2) $. 
Our free energy which is then generated by this symmetry is 
given by 
 \[
	 F [ \phi ( \vec{x} ) ] = \int  d^ d x \, 
	 \left[  ( \nabla \psi ^ * ) \cdot  ( \nabla \psi ) + \
 \frac{\mu ^ 2 }{ 2 } | \psi | ^ 2 + g | \psi | ^ 2\right] 
\] There are a lot of physical applications of this model. 
Objects in this universality class include magnets and superfluids! 

We can play the same game and look at $ O ( 3 ) $ models, which is called the 
Heisenberg model. 
What's more interesting really than just writing these down is to 
also think about symmetries and critical points. 

\subsection{Goldstone modes} 
The statement is that for every spontaneously broken symmetry, 
we get an exactly massless scalar. This shows up everywhere! 
For example, pions in QCD and phonons in condensed matter theory. 
This is really important because massless particles are important. 
For discrete symmetries, for example $ \mathbb{ Z } _ 2 $, 
we have only a finite number of vacua. For example, in the Ising model, 
we have that 
\[
 \left< \phi _ 0  \right>  = \pm m _ 0 
\] However, if we have a continuous symmetry, 
the story is different. For example, in our XY model, 
we only constrain our magnitude. 
\[
 \left<   \|\phi \| \right> = M _ 0  = \sqrt{ \frac{ - \mu ^ 2 }{ 4 g }}  
\] Our ground states are some $ n  $- sphere. 
If we generalise this to the $ O ( n ) $ model, 
our generalised vacua is $ S ^{ n - 1 }$. 
How much energy does it cost to 'move around'? 
Our polynomial terms don't depend on phase. 
Well, our gradient terms do contribute, bu t
over long enough distances these don't contribute anything. 

So, at low energies we expect to have our Goldstone bosons 
dominating the low energy physics. 

\subsubsection{Symmetry and Goldstone Bosons} 
The number of Goldstone bosons will be equal to the 
dimension of the symmetries of the free energy
minus the symmetries of the ground state
\[
 O ( N ) : \left< \phi  \right>  = \begin{pmatrix} M \\ 0 \\ \vdots 0  \end{pmatrix} 
\] We have quotiented our our manifold by 
\[
	S ^{ N - 1 }  = O ( N ) / O ( N - 1) 
\] 
Let's apply this to our XY model. 
If we write out our symmetry 
\[
	\psi ( \vec{x} )  = \left(  M_ 0 + \tilde{ M } ( \vec{x} )   \right)  e^{ i \theta ( \vec{x} ) }
\] Our free energy shifts like
\[
 F [ M , \theta ]  = \int d ^ d x \, \left\{  
 \frac{\gamma }{ 2 } ( \nabla  \tilde{ M } ) ^ 2 + | \mu ^ 2 | \tilde{M} ^ 2 + 
 g \tilde{ M } ^ 4 + \dots + \frac{\gamma }{ 2 } M _0 ^ 2 ( \nabla \theta ) ^ 2 + 
 \gamma M _ 0 \tilde{ M } ( \nabla \theta ) ^ 2 + \dots \right\} 
\] We don't require any tuning here. Our Goldstone modes exhibit 
what we call shift symmetry. 

\subsubsection{Critical Exponents} 
Let's look at the particularly interesting 
case of $ d = 3 $. Our critical exponents look like
\begin{table}[htpb]
	\centering
	\caption{caption}
	\label{tab:label}
	\begin{tabular}{c c c }
		exponent & $ \mu $  & $ \nu $ \\
		MF & 0 & $ \frac{1}{2 }$ \\
                ising & 0.0363 & 0.6300 \\
		N = 2 &  a & b  \\ 
	\end{tabular}
\end{table}
This means our heat capacity has a cusp, 
which has been observed for superfluids. 
We have that our tiny different in the critical exponent is actually very important. 
This is known as the lambda transition. 
\[
	c \sim | T - T _ c |^{ - \alpha } , \quad \alpha   = 2 - 3 \nu , \quad \alpha _{ O ( 2) } = - 0.16 
\] 
\subsubsection{RG in the $ O ( n ) $ model} 
If we return back to our naive dimensional analysis 
\[
	[ \phi ] = \frac{ d - 2 }{ 2 }, \quad [ \mu _ 0 ^ 2 ] = 2 , \quad [ g _ 0 ] = 4 - d 
\] Now however, when 
we do our interaction term, we need to take into 
account that we are already implicitly pairing things 
together. 
\[
	F _ I [ \phi ] \sim d ^ d x \, g \phi ^ 4  = \int d ^ d x \, g ( \phi \cdot  \phi ) ( \phi \cdot  \phi ) 
\] (Insert diagram here) 
So this term contributes a term in the free energy which is $ \sim g_ 0 \delta _{ ab } \delta _{ cd } $. 

To order $ g _ 0 $, we also get new contributions that look like 
\[
 \sim \phi _ a ^ - \phi _ a ^ - \left< \phi_ b ^ + \phi _ B ^ +   \right>, \quad 
 \sim ..
\] 
\subsubsection{Epsilon expansion} 

When we go to the epsilon expansion 
\[
	\frac{ d \mu ^ 2 }{ ds } = 2 \mu ^ 2 + \frac{N + 2 }{ 2 \pi ^ 2 } \frac{ \Lambda ^ 4 }{ 
	( \Lambda ^ 2 + \mu ^ 2 ) } \tilde{ g } ^ 2, \quad 
	\frac{ \tilde{ g }  }{ ds }  = \epsilon g  - \frac{ N +  8 }{  2 \pi ^ 2 }  \frac{ \Lambda ^ 4 }{ 
	\Lambda ^ 2+ \mu ^ 2 } \tilde{ g } ^ 2, \quad \tilde{ g }  = \Lambda ^{ - \epsilon }  
\] This changes our Wilson Fisher fixed point. We see that 
we have different factors of $ N $ here. 
As $ N  $ grows, the effective contribution in the loops grows. 
From here, we can calculate the critical exponents gained from 
these equations. 
\begin{align*}
	\alpha &=  \frac{4 - N }{ 2 ( N + 8 ) } \epsilon  \\
	\beta &=  \frac{1}{2 } - \frac{ 3 }{ 2 ( N + 8 ) } \epsilon  \\
	\gamma &=  1 + \frac{ N + 2 }{ 2 ( N + 8  } \epsilon \\
	\delta &=  3  + \epsilon  
\end{align*}
These are interesting, if we look at $ \beta $ as 
$ N \to \infty $ we loose dependence on $\epsilon$. 

\subsection{Goldstone Bosons in $ d = 2 $} 
Before, when we did symmetry breaking, we got an associated Goldstone boson. 
When we go into our ordered phase, a direction was picked out in the $ O ( n )  $ vector. 
This gives rise to a submanifold in our vacuum (an $ n -1 $ dimensional sphere). 
These bosons are massless - costs very little energy. 
Hence, we expect them to be relevant as, and so appear at all distance scales. 
They show up everywhere. 
At $ d =2 $ something special happens. 
In our previous derivation, we didn't have much talk of dimension itself. 
However, it turns out that at $ d = 2$, independent of the symmetry breaking patterns, 
the underlying physics seems to care about the number of dimensions. 
For the Ising model, we had $ \mathbb{ Z} _ 2 $ discrete symmetries, not continuous. 
But, we showed that at $ d = 1 $, due to domain walls, we had no ordered phased 
since they dominated the physics (so $ d = 1 $ was the critical dimension). 
However, in $ d = 2 $, the vacua are not contnuously connected, so 
we had to use a domain wall. Domain walls cost us free energy. Fluctuations 
along the manifold in n - 1 mexican hat might break the ordered phase. 

In the $ X Y $ model, when we start, our expectation value of 
Goldstone modes $ \left< \mathcal{ O } ( \vec{x} )  \right>  = 0$. 
We can calculate our fluctuation specifically about the vacuum at $ \theta = 0 $ 
as 
\[
	\left< \left[  \mathcal{ ( \vec{x} ) }  - \mathcal{ O } ( 0 )  \right]  ^ 2  \right> 
	= 2 \left< \mathcal{ O } ^ 2 ( \vec{x} )  \right> - 2 \left< \mathcal{ O } ( \vec{x} ) \mathcal{ O } ( 0 )  \right>
\] 
We can calculate this two point function explicitly as 
\[
	\left< \mathcal{ O } ( \vec{x} ) \mathcal{ O } ( 0 )  \right> 
	= \frac{1}{\gamma M _ 0 ^ 2 } \int _ 0 ^{ \Lambda } \frac{d ^ d k }{ ( 2 \pi ) ^ d } \frac{ e ^{ - i \vec{k} \cdot  \vec{x}}}{k ^ 2}
	\sim \begin{cases}
		\Lambda ^{ d - 2 } - r ^{ 2 - d } & d > 2 \\
		\log ( \Lambda r ) & d =2 \\
		r - \Kambda ^{ - 1 } & d = 1 
	\end{cases}
\] Something weird is going on. If we look at the fluctuations, we see that 
as soon as we hit $ d = 2 $ the fluctuations start to become 
arbitrarily large at large distances! 
This is the manifestation of something known as the Mermin-Wagner theorem, 
which states that a continuous symmetry cannot be 
spontaneously broken in $ d = 2  $. These Goldstone modes 
start to contribute some fishy business - they 
give arbitrarily large fluctuations. 

\subsubsection{$ \sigma $ models} 
In the 60's and 70's people were trying to understand pions in the context of $QCD$. 
People started noticing that these are very similar 
to Goldstone bosons - they're now called pseudo-Goldstone bosons. 
In the ordered phase, we have that 
\[
 \left \| \phi \|  \right> \neq 0, \implies \text{Vacuum Manifold } \sim S^{ N - 1} 
\] We have another type 
of mode, which are called longitudinal modes. These 
are gapped since the second derivative is non-zero. 

We have that $ \phi \cdot  \phi  = M_0 ^ 2 $. We can 
rescale $ \phi $ such that $ \vec{n} \cdot  \vec{n} = 1 $. 
In terms of the free energy, 
we have that 
\[
	F [ n ] = \int d ^ d x \frac{1}{2 e ^ 2 } ( \nabla n ) \cdot  ( \nabla n )  , \quad e ^ 2 &=  \frac{1}{\gamma M_0 ^ 2 } \\
\]  There's no potential term 
since in the vacua its set to zero since 
$V ( \phi ) = \frac{\lambda }{ 2 } ( \phi \cdot  \phi - M_0 ^ 2 ) $. 
What does the path integral look like? 
The path integral is now 
\[
	\mathcal{ Z } = \int \mathcal{ D } \vec{n} \delta ( \vec{n} \cdot  \vec{n} -1 ) \exp \left( 
	- \frac{1}{2 e ^ 2 } \int d ^ d x \, ( \nabla \vec{n} ) \cdot  ( \nabla \vec{n} ) \right) 
\] From dimensional analysis, we have that $ \left[  e ^ 2  \right]   = 2 - d $. 
Now, if $ e $ is large, the exponential term is surpressed, 
so we have less fluctuations. 
We write 
\[
	\vec{n} ( \vec{x} ) = \left(  \pi ( \vec{x} ) , \sigma ( \vec{x} )  \right) 
\] where we have that $ \phi  ( \vec{x} )  $ is $ n - 1  $ dimensional. 
Our constraint gives us that 
\[
	\sigma ^ 2 ( \vec{x} )  = 1 - \pi ( \vec{x} ) \cdot \pi ( \vec{x} ) 
\] Plugging this expression in, we get that 
the free energy as a function of $ \pi $ and $ \sigma $ takes the 
form 
\[
	F \left[  \phi ( \vec{x} )  \right]  = \int d ^ d x \, 
	\frac{1}{2 e ^ 2 } \left[  ( \nabla \vec{n} ) ^ 2 + ( \nabla \sigma ) ^ 2  \right] 
\] Substituting our constraint, give us 
\[
 \dots = \int d ^ d x \, \frac{1}{2 e ^ 2 } \left[  
 ( \nabla \pi ) ^ 2 + \frac{( \phi \cdot  \nabla \pi) ^ 2 }{ 1- \pi \cdot  \pi }\right] 
\] When we expand, each interaction comes with an $ e $ term 
which tells us that $ e  $ actually acts like a coupling constant. 

\subsubsection{Background Field Method} 
We want to understand the fate of these Goldstone bosons. 
The thing to do is to perform standard RG calculations.
We could expand the second term, but this is hard. 
Polyakov however, devised a clever method to do this. 
What we do is set 
\[
	n ^ a ( \vec{x} ) \to \tilde{ n } ^ a ( \vec{x} ) \text{long wavelength}  ,\quad \tilde{ \vec{n}   } \cdot \tilde{ \vec{n}} = 1  
\] We introduce the "frame fields" 
which is a basis of $ N - 1$ unit vectors which we call 
\[
	e ^ a_ \alpha ( \vec{x} ) , \quad a  = 1, \dots N, \alpha = 1, \dots N - 1
\] We also impose the constraint that these are orthogonal to 
$ \tilde{ \vec{n} }  ^ a  $, such that $ \tilde{ \vec{n} } ^ a e ^ a _ \alpha = 0 , \forall a $. 
Also, we have that $ e ^ a _ \alpha e ^ a _{ \beta }  = \delta _{ \alpha \beta } $. 

Now we can introduce short wavelength modes 
which we call $ \chi_{ \alpha } ( x ) $, 
and hence write 
\[
	n ^ a ( \vec{x} )  = \tilde{ n } ^ a ( 1 - \chi ^ 2( x )  )^{ \frac{1}{2 } }  + 
	\int _{ n = 1 } ^{ N - 1 } \chi _{ \alpha } ( x) e ^ a _ \alpha ( x)  \implies n ^ 2 =1
\] 

\subsubsection{Path Integral} 
We have that 
\[
	\mathcal{ Z } = \int \mathcal{ D } \tilde{ n } \delta ( \tilde{ n } ^ 2-  1 ) \exp 
	\left(   - \frac{1}{2 e ^ 2 } \int d ^ d x \, ( \nabla \tilde{ n } ) ^ 2   \right) 
	\int \mathcal{ D } \chi \exp \left(  -\frac{1}{ 2 e ^ 2 } \int d ^ d x \, ( \nabla \chi ) ^ 2  \right)  
	e ^{  - F _ I [ \tilde{ n } , \chi ]  }
\] 
Our interaction term is given by 
\[
 F _ I \left[  \tilde{ n } ^ a , \chi _{ \alpha }   \right]  
  = \frac{1}{ 2 e  ^2 } \int d ^ d x \, \left[  
  - \chi ^ 2 ( \nabla \tilde{ n } ) ^ 2 + \chi _ \alpha \chi _ \beta 
  \nabla e ^ a _ \alpha \nabla e ^ a _ \beta + 2 \nabla \tilde{ n } ^ a \nabla ( \chi _ \alpha e ^ a _ \alpha )    \right]  
\]  We want to focus on $ \left< e ^{ - F _ I }  \right> \simeq 1 - \left< F _ I  \right> + \dots $. 
Taking the average, our first term is 
\[
 \left< F _ I  \right>  = \frac{1}{2 e ^ 2 } \int d ^ d  x \, 
 ( - \delta _{ \alpha \beta } ( \nabla \tilde{ n } ^ a ) ^ 2 + \nabla e ^ a _ \alpha \nabla e ^ \alpha _ \beta ) \left< \chi _ \alpha
 \chi _ \beta \right> 
\] We can evaluate the two point function in the integral 
\[
\left< \chi _ \alpha \chi _ \beta   \right>  = e^ 2 \delta _{ \alpha \beta } I _ d  
\] where 
\[
	I _ d = \frac{ \Omega_{ d - 1} }{ ( 2 \pi ) ^ d } \Lambda ^{ d - 2 } \begin{cases}
		\zeta - 1  & d = 1 \\ \log ( \zeta ) & d =  2\\ 1 - 1\zeta ^{2 - d  } & d \geq  3
	\end{cases} 
\] Using the relation 
\[
\tilde{ n } ^ a \tilde{ n } ^ b + e ^ a _ \alpha e ^ b _ \alpha  = \delta ^{ ab }    \implies 
\nabla e ^ a _ \alpha \nabla e ^ a _ \alpha = \nabla \tilde{ n } ^ a \nabla \tilde{ n } ^ a + \dots   
\] 
Finally we have that 
\[
	\left< F _ I  \right>  =( 2 - N ) I _ d \int d ^ d x  \frac{1}{2 } ( \nabla \tilde{ n } ) ^ 2  
\] This gives us a correction 
\[
	\frac{1}{( e ' ) ^ 2 } = \frac{1}{e _ 0 ^ 2 }+ ( 2 - N ) I _{ d }
\] when $ N = 2$ no change. 
We hence have that 
\[
	\frac{1}{e ^ 2 ( \zeta ) } = \zeta ^{ d - 2} \left[  \frac{1}{e _ 0 ^ 2 } + ( 2- N ) I _ d  \right] 
\] As we go to IR, interacting thepry of Goldstones become more strongly coupled. 
This is why we don't get Goldstones in $ d = 2$. if we did, we would get
arbitrarily large fluctuations which destroy ordered. states. 
