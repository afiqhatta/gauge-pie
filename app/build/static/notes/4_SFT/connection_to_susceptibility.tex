\subsection{Connection to Susceptibility} 
Let's apply these ideas to what we know about magnetic susceptiblity. 
Our equation that we've always used is 
\[
 \chi = \frac{\partial  m}{\partial   B } 
\] Now, we can promote this to a localised 
version by setting 
\[
	\chi ( \vec{x}, \vec{y} ) = \frac{\partial  \left< \phi ( \vec{x} )  \right> }{\partial B ( \vec{y} ) } = \beta ( \left< \phi ( \vec{x} ) \phi ( \vec{y} )  \right> \sim \int \mathcal{ D } \phi \frac{\partial \phi }{\partial  B } e^{  - \beta F [ \phi ] }  
\] From this local 
definition, we can define a global magnetic 
susceptibilty as 
\[
	\chi = \int d^ d x \chi ( \vec{x}, 0 ) = \beta \int d^ d x \left< \phi ( \vec{x} ) \phi ( \vec{y} )  \right>
\]
Recall our correlation function 
\[
	\left< \phi ( \vec{x} ) \phi ( \vec{y} )  \right> = \begin{cases}
		\frac{1}{r^{ d - 2}  } & r \ll \zeta \\
		e^{ - \frac{r}{2 } } / r ^{ \frac{ d -1 }{ 2 } } & r \gg \zeta
	\end{cases}
\] This implies that our critical exponents satisfy $ \mu ^ 2 \sim | T - T_{ c } | $, 
and $ \zeta \sim \frac{1}{ | T - T_{ c } | ^{ \frac{1}{ 2 }  } } $. 
Thus at low temperatures, our correlation length diverges, so 
our system because highly correlated. 

