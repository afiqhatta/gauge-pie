%! TeX program = lualatex
\documentclass[11pt, oneside]{article}   	% use "amsart" instead of "article" for AMSLaTeX format
\usepackage[margin = 1.1in]{geometry}            		% See geometry.pdf to learn the layout options. There are lots.
\geometry{letterpaper}                   		% ... or a4paper or a5paper or ... 
\usepackage[parfill]{parskip}    		% Activate to begin paragraphs with an empty line rather than an indent
\usepackage{graphicx}				% Use pdf, png, jpg, or eps§ with pdflatex; use eps in DVI mode
						% TeX will automatically convert eps --> pdf in pdflatex	
\usepackage{adjustbox}	
\usepackage[section]{placeins}

%% LaTeX Preamble - Common packages
\usepackage[utf8]{inputenc}
\usepackage[english]{babel}
\usepackage{textcomp} % provide lots of new symbols
\usepackage{graphicx}  % Add graphics capabilities
\usepackage{flafter}  % Don't place floats before their definition
\usepackage{amsmath,amssymb}  % Better maths support & more symbols
\usepackage[backend=biber]{biblatex}
\usepackage{amsthm}
\usepackage{bm}  % Define \bm{} to use bold math fontsx
\usepackage[pdftex,bookmarks,colorlinks,breaklinks]{hyperref}  % PDF hyperlinks, with coloured links
\usepackage{memhfixc}  % remove conflict between the memoir class & hyperref
\usepackage{mathtools}
\usepackage[T1]{fontenc}
\usepackage[scaled]{beramono}
\usepackage{listings}
\usepackage{physics}
\usepackage{tensor}
\usepackage{tikz}
\usepackage{pgfplots} 
\usepackage{subfiles} 
\usepackage{helvet}
\usepackage{float}

\usepackage{tikz-feynman}
\tikzfeynmanset{compat=1.1.0}


%% Commands for typesetting theorems, claims and other things. 
\newtheorem{theorem}{Theorem}
\newtheorem*{thm}{Theorem}
\newtheorem*{claim}{Claim}
\newtheorem*{example}{Example}
\newtheorem*{defn}{Definition}

\newcommand{\Lagr}{\mathcal{L}}
\newcommand{\vc}[1]{\mathbf{#1}}
\newcommand{\pdrv}[2]{\frac{\partial{#1}}{\partial{#2}}}
\newcommand{\thrint}[1]{\int d^3 \vc{x} \left( {#1} \right)}
\newcommand{\ve}[1]{ \mathbf{#1} } 
\newcommand{\spint}{ \int d^d x \, } 
\newcommand{\pdif}[2] { \frac{ \partial #1 }{ \partial #2 } } 
\newcommand{ \R}{ \mathbb{ R} } 
\renewcommand\vec{\mathbf}


\title{Part III Statistical Field Theory}
\author{Notes taken by Afiq Hatta, based on Part III lectures by Matthew McCullough}
\begin{document}
\maketitle
\tableofcontents

\pagebreak 

\subfile{useful_resources.tex}
\subfile{introduction.tex}
\subfile{why_statistical_field_theory.tex}
\subfile{phase_transitions_and_critical_exponents.tex}
\subfile{universality.tex}
\subfile{explaining_universality_with_the_renormalization_group_flow.tex}
\subfile{the_ising_model_and_critical_exponents.tex}
\subfile{the_problems_with_just_mean-field_theory.tex}
\subfile{setting_up_the_ising_model.tex}
\subfile{reformulating_the_above_in_terms_of_free_energy.tex}
\subfile{using_the_landau_approach.tex}
\subfile{a_brief_look_at_universality.tex}
\subfile{the_landau-ginzburg_approach.tex}
\subfile{domain_walls.tex}
\subfile{evaluating_path_integrals_and_gaussian_path_integrals_.tex}
\subfile{correlation_functions.tex}
\subfile{computing_gaussian_path_integrals.tex}
\subfile{green's_functions.tex}
\subfile{connection_to_susceptibility.tex}
\subfile{critical_exponents.tex}
\subfile{the_renormalisation_group.tex}
\subfile{basic_introduction.tex}
\subfile{flowing_to_the_'ir'_(long_distances).tex}
\subfile{a_second_look_at_universality.tex}
\subfile{classifying_fixed_points.tex}
\subfile{scaling.tex}
\subfile{relation_to_critical_exponents.tex}
\subfile{relevant,_irrelevant_or_marginal.tex}
\subfile{the_gaussian_fixed_point.tex}
\subfile{interacting_fields.tex}
\subfile{applying_wicks_free_energy.tex}
\subfile{feynman_diagrams.tex}
\subfile{continuous_symmetries.tex}
\subfile{example_sheet_1.tex}
\subfile{question_1.tex}
\subfile{question_2.tex}
\subfile{question_3.tex}
\subfile{question_4.tex}
\subfile{question_5.tex}
\subfile{question_5.tex}
\subfile{question_7.tex}
\subfile{example_sheet_2.tex} 
\end{document} 
