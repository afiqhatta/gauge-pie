\subsection{Domain Walls}
However, we recognise that this obeys the same pattern as the mean field approximation! This is reassuring. However, we can go into a bit more detail, and explore field configurations in one dimension which smoothly move between $ \sqrt{ \frac{  - \alpha_2 }{ \alpha_4 } }$ and $  - \sqrt{ \frac{  - \alpha_2 }{ \alpha_4 } } $ (we can't have say, a step function because that means our gradient would be infinite). One can verify that a solution with this asymptotic property of converging to these stable state is given by 
\[ 
m = m_0 \tanh \left( \frac{ x - x_0}{ w }  \right), \quad W = \sqrt{ \frac{ 2 \gamma }{ \alpha_2 } } 
\] This is called a domain wall. 
We need to verify that this is a solution of the equation of motion.
We find that 
\[	\frac{d m }{dx } = \frac{m^{0 }}{W} \sech^2 \left( \frac{x - X}{W} \right), \quad \frac{d^2 m }{dx^ 2 } =  - \frac{2m_0 }{W^ 2 } \tanh \left( \frac{x - X}{W} \right) 	+ \frac{2 m_0 }{W^ 2} tanh^2 \left( \frac{x - X }{W} \right)   	\] 
However, we can just wrangle this into the form 
\[
\gamma \frac{d^ 2 m }{dx^ 2 } = - \frac{2 \gamma }{W^ 2 }m  +  \frac{ 2 \gamma }{W^ 2 m_{0 }^2 }m ^ 3 
\] If we substitute in our asymptotic values $ \pm m_{0 }$, as well as our characteristic width $W$, then 
we recover that indeed 
\[
\gamma \frac{d^ m }{dx^ 2 } = \alpha_2 m + \alpha_4 m^{3}
\] 
Now, we saw that the function $m $ has \textbf{ characteristic length} $ W$. Beyond this length 
we basically consider the function $ m $ to be constant. Let's try 
plug this function back into our expression for our free energy $ F ( m ( \vec{x}) ) $. 
Let's consider how the derivative terms scale first. Since the function is constant outside of our characteristic length, 
we need only to integrate in this range. Our free energy derivative term is 
\[
F_{deriv } ( m ) = \int d^{d} x \frac{1}{2 } \gamma ( \nabla m )^ 2 \sim L^{d -1 } \int dx \gamma \left( \frac{ d m  }{dx} \right)^2  
\] Let's take a look at what we've done here. We're approximating the 3d integral 
into an 'radial' component, 
and we've factored out a characteristic length scale of $L^{d - 1}$, much like 
when we do a spherical intgral in $ d$ dimensions, we take out a factor of 
$ S_{d -1 } $, the $ d - 1$ dimensional hypersphere, and leave in the 
radial integral. 
If we take just the constant term of $\frac{d m  }{ dx }$, then this integral 
becomes 
\[
 L^{d - 1} \gamma \int dx \frac{m_0^2 }{W^ 2}  \sim  L^{d - 1} \gamma \frac{m_{0}^2 }{W} \simeq L^{d -1 }\sqrt{\frac{ - \gamma \alpha_2^3 }{\alpha_4^3 }}  
\] We can also check that 
this is the characteristic length scale of other terms in the free energy. 

We can similarly do the polynomial functions in the free energy. 
We have, for the term with the $\alpha_2 $ coefficent, that 
\[
	F ( m ( \vec{v}) )  = \int d^{d }x \frac{1}{2 } \alpha_2( T ) m^ 2 \sim L^{d - 1} \int dx \frac{1}{2} \alpha_2 m^ 2
\] But again, we repeat the previous argument, 
and integrate just over the characteristic length scale $W$. 
Once again, we just take the constant terms, so that this 
is approximated by 
\begin{align*}
	L^{d - 1} W \alpha_2 m_{0}^2 &= L^{d - 1}\alpha_2W \sqrt{\frac{-\alpha_2}{\alpha_4}}  \\
	&= L^{d -1} W \frac{ - \alpha_2^2 }{\alpha_4 } \sqrt{\frac{ - 2 \gamma }{\alpha_4}}  \\
	& \sim \sqrt{ \frac{ - \gamma \alpha_2^3 }{\alpha_4 }} 
\end{align*}
Similarly, we have that our quartic term scales in this way. 
Hence, this free energy of a domain wall is well defined. 
The free energy of a domain wall $F_{DW}$, compared with the free energy of a ground state scales with a power law for characteristic length, 
\[ 
F_{ DW} \sim L^{ d - 1} \sqrt{ \frac{  - \gamma \alpha_2^3 }{ \alpha_4} } 
\] 
There's a special value of $d = 1$ since our domain wall has no length scale. 
This is what we call a critical dimension. 
Now, we can look at the probability of a domain wall occuring. 
As usual, the probability to have a certain field configuration is 
given by 
\[
	e^{  - \beta F ( m ) }
\] Hence, the probabilty of us having a domain wall at a given point $ x $ is 
\[
	\mathcal{  P}(\text{ domain wall at } x = X )  = \frac{e^{- \beta F_{wall}}}{\mathcal{ Z}}
\] Now, the first thing to notice is that when $ d= 1$, our exponential factor 
is constant since we have no length dependence. 
Overall, this constant is really small. Now, let's imagine 
we're in a section of  $ \R$ between $ \frac{L}{ 2} < x < \frac{L}{2}$. 
Then, we can calculate the probabability of us having a domain wall in this box. 
This means we'd like to calculate 
\[
	\int_{L / 2 }^{ L / 2 } dx \, \mathcal{ P }( \text{ domain wall at } X = x ) 
\] Naively, this looks like it'll just be $ L $ times the probability of a domain wall at a given point. 
However, we need to remember that domain walls have a characteristic length scale  $W$. 
This means we need to multiply by a factor of $ \frac{1}{ W} $ in the measure. 
So, we have that 
\[	\mathcal{ P }( \text{there's a domain wall between } \frac{L}{2 }< x < \frac{L}{2} )  = \frac{L}{W} \frac{e^{ - \beta F_{ wall}}}{ \mathcal{ Z }} 
\] What's the effect of having a domain wall in the range? 
Well, suppose that, in this one dimensional line, that we have a 
fixed magnetisation at the boundary, so suppose that $ m (  - \frac{L}{2 }) =  - m_{0 } $.
Then, the effect of a domain wall would be to flip this to $ m ( \frac{L}{2 })  = m_{ 0 } $ 
at the other end of the line. 

Now we can ask, if we have the condition that say $ m ( L / 2 ) = m_0 $, 
then what's the total probability of the magnetisation staying the same, 
or switching, by the time we get to the other side? 
The simplest way to find out what's happening is to consider 
the probability that we have $ N $ domain walls appearing in the system. 
Let's write down the expression first, then explain why this is. 
Our expression 
\[
	\mathcal{ P }\left( \text{ n domain walls}  \right)  = \frac{e^{ - n \beta F_{wall}}}{\mathcal{Z} W^n } \int_{L / 2}^{ L / 2 } dx_1 \int_{x_1}^{ L / 2 } dx_2 \dots \int_{ x_{n-1}}^{ L / 2} dx_n  
\] To see why this is, observe that, without loss of generality, we can place a domain wall anywhere
in the range, as the first domain wall to be placed. This is represented in the first integral. 
After that, we can place a wall any point after that wall, assuming 
its placed at the point $ x_{1}$. We repeat, to place a wall after that, and so on, 
until we've placed $n $ walls. 
Our probability density given by the exponential $ e^{ - \beta F_{wall}}$ is constant, so we
pull $ n $ of these factors out. 

An elegant argument to evaluate this integral is to observe that 
this term is symmetric amongst permutation of the variables $ x_1, x_2, \dots x_n $. 
So, we're essentially multiplying the first integral $ n $ times, 
but to take into account the permutation symmetry, we need to 
divide by $ n ! $. 
Thus 
\[
	\mathcal{ P }\left( \text{n domain walls} \right) = \frac{1}{n! \mathcal{ Z}} \left( \frac{L e^{ - \beta F_{wall}}}{ W} \right)^{n }   
\] Now that we've done this, calculating the probability of whether 
magnetisation will flip by the time it reaches the other side 
is simply a matter of looking at whether we have an odd or even 
number of domain walls. If we have an even number, 
our magnetisation flips an even number of times and thus is not changed. Hence, 
\[
	\mathcal{ P }( m_0 \to  m_0 )  = \sum_{n \text{ even }} \mathcal{P }\left( \text{n walls} \right) \frac{1}{\mathcal{Z}} \cosh \left( \frac{L e^{ - \beta F_{wall}}}{ W } \right)  
\] Similarly, our probability of switching sign is 
\[
	\mathcal{ P }( m_{0} \to   - m_0 ) = \frac{1}{\mathcal{ Z}} \sinh \left( \frac{L e^{ - \beta F_{wall}}}{ W} \right) 
\] What's the point of all this? 
Well, since our dimension was at $d = 1$, our probability diverges and $ L \to \infty $.
Thus, the probabilities $ \mathcal{ P }( m_0 \to  m _0 ) $ and $ \mathcal{P }( m_0 \to - m_0 ) $ 
are indistiuishable. 
We have a disordered state in one dimension, and hence mean field theory 
here breaks down. However, when 
$ d = 2$, we have that $ F \sim L $, so as we take the limit
as  $ L \to  \infty $, the terms don't diverge and we still retain an memory of 
the boundary conditions initially at $ x =  - \frac{L}{2}$. 


\pagebreak


