\documentclass[11pt, oneside]{article}   	% use "amsart" instead of "article" for AMSLaTeX format
\usepackage[margin = 1.1in]{geometry}            		% See geometry.pdf to learn the layout options. There are lots.
\geometry{letterpaper}                   		% ... or a4paper or a5paper or ... 
\usepackage[parfill]{parskip}    		% Activate to begin paragraphs with an empty line rather than an indent
\usepackage{graphicx}				% Use pdf, png, jpg, or eps§ with pdflatex; use eps in DVI mode
								% TeX will automatically convert eps --> pdf in pdflatex	
\usepackage{adjustbox}	
\usepackage[section]{placeins}


%% LaTeX Preamble - Common packages
\usepackage[utf8]{inputenc}
\usepackage[english]{babel}
\usepackage{textcomp} % provide lots of new symbols
\usepackage{graphicx}  % Add graphics capabilities
\usepackage{flafter}  % Don't place floats before their definition
\usepackage{amsmath,amssymb}  % Better maths support & more symbols
\usepackage[backend=biber]{biblatex}
\usepackage{amsthm}
\usepackage{bm}  % Define \bm{} to use bold math fontsx
\usepackage[pdftex,bookmarks,colorlinks,breaklinks]{hyperref}  % PDF hyperlinks, with coloured links
\usepackage{memhfixc}  % remove conflict between the memoir class & hyperref
\usepackage{mathtools}
\usepackage[T1]{fontenc}
\usepackage[scaled]{beramono}
\usepackage{listings}
\usepackage{physics}
\usepackage{tensor}
\usepackage{simplewick} 
\usepackage{tikz} 
\usepackage{import}
\usepackage{xifthen}
\usepackage{pdfpages}
\usepackage{transparent}
\usepackage{pgfplots}
\usepackage[compat=1.1.0]{tikz-feynman}
\usepackage{subfiles}
\usepackage{simpler-wick}
\usepackage{slashed}
\usepackage{fancyhdr}
\usepackage{enumitem}

\pagestyle{fancy}
\fancyhf{}
\rhead{Notes by Afiq Hatta}
\lhead{String Theory}
\rfoot{Page \thepage}

%% Commands for typesetting theorems, claims and other things.
\newtheoremstyle{slanted}
{1em}%   Space above
{.8em}%   Space below
{}%  Body font
{}%          Indent amount (empty = no indent, \parindent = para indent)
{\bfseries}% Thm head font
{.}%         Punctuation after thm head
{0.5em}%     Space after thm head: " " = normal interword space;
{}%         \newline = linebreak
{}%          Thm head spec (can be left empty, meaning `normal')

%% Commands for typesetting theorems, claims and other things. 

\theoremstyle{slanted}
\newtheorem{theorem}{Theorem}
\newtheorem*{thm}{Theorem}
\newtheorem*{claim}{Claim}
\newtheorem{example}{Example}
\newtheorem*{defn}{Definition}

\newcommand{\Lagr}{\mathcal{L}} 
\newcommand{\vc}[1]{\mathbf{#1}}
\newcommand{\pdrv}[2]{\frac{\partial{#1}}{\partial{#2}}}
\newcommand{\thrint}[1]{\int d^3 \vc{x} \left( {#1} \right)}

%% QFT specific macros 
\newcommand{\intp}{ \int \frac{ d^3 p }{ (2 \pi)^3 } \, }
\newcommand{\ann}[1]{a_{ \mathbf{ #1 }}}
\newcommand{\crea}[1]{a^\dagger_{ \mathbf{ #1 }}}
\newcommand{\ve}[1]{ \mathbf{ #1 } } 
\newcommand{\mode}[ 1]{ e^{ i \mathbf{ #1 } \cdot \mathbf{x} }}
\newcommand{\nmode}[1]{ e^{  - i \mathbf{ #1 } \cdot \mathbf{x} }}
\newcommand{\freq}[1]{\omega_\mathbf{ #1} } 
\newcommand{\scal}[1]{\phi ( \mathbf{ #1 })} 
\newcommand{\mom}[1]{ \pi (\mathbf{ #1 })} 
\newcommand{\arr}{\rightarrow} 

\newcommand{\incfig}[1]{%
\def\svgwidth{\columnwidth}
\resizebox{0.75\textwidth}{!}{\input{./figures/#1.pdf_tex}}
}

\newcommand{\anop}[2]{ #1_\mathbf{#2}}
\newcommand{\crop}[2]{#1_\mathbf{#2}^\dagger}

\usepackage{helvet} 

%tikz decoration commands 
\usetikzlibrary{decorations.pathmorphing}


\title{String Theory Lecture Notes}
\author{Lectured by Dr. Reid-Edwards} 
\begin{document} 
\maketitle
\tableofcontents

\pagebreak 

\section{Introduction}%
\label{sec:introduction}

If you'd like to read about string theory, 
check out 
\begin{itemize}
	\item String Theory, Vol 1, Polchinksi, CUP 
	\item Superstrings, Vol 1, Green et al, CUP
	\item A String Theory Primer, Schomerus, CUP
	\item David Tong's Notes, at arXiv: 09080333
	\item 'Why String Theory' Conlon, CRC  
\end{itemize}

What is string theory? 
We don't really know what string theory is. 
The question itself requires a little more 
fleshing out to make sense of it. String theory 
is a work in progress. A final version of string 
theory would be a theory which we understand physically 
and mathematically. 
It's a work in progress - the final form will 
be quite different from what we have now. 

What do we know? 
Well, in some sense, string theory 
is an attempt to answer the question of how we quantise the 
gravitational field. A theory of \textbf{quantum gravity}. 
There are however, a number of obstacles.
In particular, naive quantisation of the 
Einstein Hilbert action presents a number of problems. 

There are deep conceptual problems associated with this. 
\begin{itemize}
	\item  There's a question about the 
		nature of time in quantum gravity. 
	If we think about time in quantum mechanics, time 
	is treated as a fixed clock in which the Hamiltonian 
	governs the evolution. In GR, space and time are on the same footing. 
	Thus, the descriptions are on a different footing 
	in QM versus GR. This is not neseccarily a technical problem, 
	just something to think about. 
\item How do we quantise things without a pre-existing causal structure? 
	What do we mean by this? We we quantise things in QFT, 
	it's important to know whether two operators 
	are timelike or spacelike separated. We have a notion 
	that for all operators which are spacelike separated, 
	commute. (One should not be able to influence the other). 
	If we try to quantise GR, the metric is the object which 
	we would like to quantise. But, this determines the casual 
	structure. So, it's not immediately obvious 
	what the algebra of operators should look like. 

\item GR has a very big symmetry - diffeomorphism symmetry (symmetry under
	coordinate reparametrisations). This is diffeomorphism 
	invariance. This is something we'll discuss a bit later on. 
	This is considered to be a gauge symmetry. 
\item One thing we'll discuss is that there are no local 
	diffeomorphism invariant observables in GR. It's 
	not even clear what the observables should be. 
\end{itemize}

More importantly, 
there are technical obstacles. The previous issues are hard, 
but can we make some assumptions which help us make progress? 
We can look at perturbation theory; we can take our 
metric an expand it around some classical solution
\[
	g _{ \mu \nu } \left( X  \right)   = \eta _{ \mu \nu } + h _{ \mu \nu } \left( x \right) 
\] for our purposes, this classical solution will be 
Minkowski space-time. 
This means we can use the causal structure of 
the background metric (Minkowski) to 
learn about the causal structure of the perturbation, 
and quantise. 
This immediately neutralises the first two 
conceptual problems. We can call the fluctuations $ h _{ \mu \nu } $ 
as gravitons. 
So, if we take a background static spacetime, add a field, 
then quantise. 

There are some unsatisfactory things about this. 
When we split the metric into two, 
we hide a lot of the deep structure that we want. 
Nonetheless, we can take our Einstein-Hilbert action and expand it out
\[
	S \left[ g  \right]   = \frac{1}{K_ 0 } \int d ^{ D } X \sqrt{  - g }  R ( g ) 
\] Choose a gauge and expand out 
\[
	S \left[  h  \right]   = \frac{1}{K_0 } \int  d ^ D X \left( 
	h _{ \mu \nu } \Box h ^{ \mu \nu } \right) 
\] Since the Ricci scalar contains inverses, 
this expansion goes on forever. This is called 
a non-polynomial action. 
The first quadratic term gives the propagator, 
and the higher order term gives us our interactions. 

The propagator is represented by a wiggly line. 
The interaction term gives us vertices. (Three or four 
wiggly lines coming together) 
These lead to Feynman rules.

However, when we compute loops, 
we get divergences. But, in physical QFT, 
we can absorb these divergences into coupling constants. 
These can be dealt with using standard techniques. From
advanced quantum field theory, this technique 
is called renormalisation. 

Thus, the difficulties run deeper than conceptual ones. 
We simply don't know how to calculate. 
So, string theory 
provides a way to do quantum perturbation theory 
of the gravitational 'field' 
We put field in quotation marks because it's 
not really a field which we'll be dealing with. 

String theory answers some of the questions, 
in the sense that it gives 
us a framework to ask 
meaningful questions in quantum gravity. 
But not all of these questions are questions that 
we can answer. 

The viewpoint that  we're going to 
take for this course will be from 
perturbation theory. But we'll always 
try to understand what this tells us about the 
non-perturbative physics. 

\subsection{Worldsheets and Embeddings}
Let's try 
to put together some sort of language 
to get started. From any popular science book, 
you may find that particles are described 
as vibrating strings. 
The starting point is to consider a 
worldsheet $ \Sigma $ which is a 
2-dimensional surface swept out by a string. 
This is analogous to a 
worldline swept out by a particle.

(Insert diagram of line parametrised by $ \tau $  and pointlike 
object, and diagram of cylinder object with surface called 
$ \Sigma $, with two axes called $ \tau $ and $ \sigma $.   This 
diagram is in 3d Minkwoski space). 
We put coordinates $ \left( \sigma , \tau  \right)  $
on $ \Sigma $, at least locally. 
And, we can define an embedding of the worldsheet $ \Sigma $ 
in the background spacetime, $ \mathcal{ M } $ (Minkowski space),
by the functions $ X ^{ \mu } \left( \sigma, \tau  \right)  $, 
where the $ X ^ \mu $ are coordinates on 
$ \mathcal{ M }  $. So if you like, we have that 
\[
 X : \Sigma \to \mathcal{ M }
\] Why is this an embedding? 
A choice $ \sigma $ and $ \tau $ on $ \Sigma $ gives 
us a location in Minkwoski space. 
There are rules (which we shall investigate), 
for gluing such worldsheets together 
in a way which is consistent with 
the symmetries of the theory. 

So, we can not only describe the embeddings 
of a single string propagating through space-time, 
but multiple strings coming together. 

(Draw a diagram of two tubes merging into one tube, 
then splitting back again into two tubes - this looks like 
a three point vertex)

We shall see that such diagrams like the above 
are in one-to-one 
correspondence with correlation functions in some 
quantum theory. 
It is natural to interpret such diagrams 
as Feynman diagrams in a perturbative expansion 
of some theory about a given vacuum. 

So what we have is a way of understanding 
writing down Feynman rules, and performing 
successively better approximations to 
an exact result in field theory which we don't have. 

So what we have here will turn out to be Feynman rules for a theory 
which we don't yet have. Where do these Feynman rules come from? 
Quantising is tremendously restrictive. 

\subsection*{Summary}
 
\subsubsection*{Worldline actions}
\begin{itemize}[leftmargin=*]
	\item Our action is 
		\[
			S =  - m \int_{ s_1 } ^{ s_2 } ds 
			 = - m \int_{ \tau_1 } ^{ \tau_2 } d \tau \sqrt{ -
			 \eta ^{ \mu \nu } \dot{ x } ^ \mu \dot{ x } ^ \nu   } 
		\] 
	\item We have conjugate momenta with on-shell mass condition  
		\[
		 P^ \mu = -  \frac{m \dot{ x } ^ \mu  }{\sqrt{  - \dot{ x } ^ 2  }  }, 
		 \quad  P ^ 2 + m ^ 2 = 0 
		\]  
	\item It makes more sense to work with Einbeins,
		since we can work in the $ m \to 0 $ limit 
	\[
		S  = \frac{1}{2 } \int d\tau\, \left( e^{ - 1} 
		\eta _{ \mu \nu } \dot{ x } ^ \mu \dot{ x } ^ \nu   - e m ^ 2   \right)  
	\] 
\item Two equations of motion come from the Einbein action 
	\[
		\frac{d }{ d \tau } \left( e ^{ - 1} \dot{ x } ^ \mu   \right)   = 0 , \quad 
		\dot{ x } ^ 2 + e ^ 2 m ^ 2 = 0  
	\]
\item This has symmetries 
	\[
	 \delta x ^ \mu  = \xi \dot{ x } ^ \mu , \quad \delta e  = \frac{ d }{ d \tau } 
	 \left( \xi \dot{ e }   \right) 
	\]
\item In the massless limit, if we replace our Minkowski metric 
	with a general metric, we recover the geodesic equations 
	\[
	 S  = \frac{1}{2 } \int d \tau  e^{ - 1} g_{ \mu \nu } \dot{ x } ^ \mu \dot{  x } ^ \mu , 
	 \quad \ddot{x } ^ \mu + \Gamma ^ \mu _{ \alpha \beta }  \dot{ x } ^ \alpha 
	 \dot{ x } ^ \beta  = 0 
	\]  
\end{itemize}

\subsubsection*{Strings}
\begin{itemize}
	\item A string is two dimensional, 
		embedded with parameters $ \sigma, \tau $
		\[
			X^ \mu  \left( \sigma, \tau  \right)   = X^ \mu 
			\left(  \sigma + 2 \pi n , \tau  \right) , \quad 
			n \in \mathbb{ Z } 
		\]
	\item Our associated action is the Nambu-goto action 
	\[
		S[X] =  - \frac{1}{ 2 \pi \alpha ' }\int d \sigma d \tau 
		\sqrt{  - \det\left( \eta_{ \mu \nu } 
		\partial  _ a X ^ \mu \partial  _ b X ^ \beta \right) } 
	\] 
\item We add an extra degree of freedom $ h _{ ab } $ to introduce 
	the Polyakov action 
	\[
	 S[X, h ] = - \frac{1}{4 \pi \alpha ' } 
	 \int d ^ 2 \sigma \sqrt{  - h }  h ^{ ab } \eta _{ \mu \nu } 
	 \partial  _ a X ^ \mu \partial  _ b X ^ \nu 
	\]  
\end{itemize}

\section*{Example Sheet 1}

 
\end{document} 
