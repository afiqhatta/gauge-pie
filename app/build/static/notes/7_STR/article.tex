\documentclass[11pt, oneside]{article}   	% use "amsart" instead of "article" for AMSLaTeX format
\usepackage[margin = 1.1in]{geometry}            		% See geometry.pdf to learn the layout options. There are lots.
\geometry{letterpaper}                   		% ... or a4paper or a5paper or ... 
\usepackage[parfill]{parskip}    		% Activate to begin paragraphs with an empty line rather than an indent
\usepackage{graphicx}				% Use pdf, png, jpg, or eps§ with pdflatex; use eps in DVI mode
								% TeX will automatically convert eps --> pdf in pdflatex	
\usepackage{adjustbox}	
\usepackage[section]{placeins}


%% LaTeX Preamble - Common packages
\usepackage[utf8]{inputenc}
\usepackage[english]{babel}
\usepackage{textcomp} % provide lots of new symbols
\usepackage{graphicx}  % Add graphics capabilities
\usepackage{flafter}  % Don't place floats before their definition
\usepackage{amsmath,amssymb}  % Better maths support & more symbols
\usepackage[backend=biber]{biblatex}
\usepackage{amsthm}
\usepackage{bm}  % Define \bm{} to use bold math fontsx
\usepackage[pdftex,bookmarks,colorlinks,breaklinks]{hyperref}  % PDF hyperlinks, with coloured links
\usepackage{memhfixc}  % remove conflict between the memoir class & hyperref
\usepackage{mathtools}
\usepackage[T1]{fontenc}
\usepackage[scaled]{beramono}
\usepackage{listings}
\usepackage{physics}
\usepackage{tensor}
\usepackage{simplewick} 
\usepackage{tikz} 
\usepackage{import}
\usepackage{xifthen}
\usepackage{pdfpages}
\usepackage{transparent}
\usepackage{pgfplots}
\usepackage[compat=1.1.0]{tikz-feynman}
\usepackage{subfiles}
\usepackage{simpler-wick}
\usepackage{slashed}
\usepackage{fancyhdr}
\usepackage{enumitem}

\pagestyle{fancy}
\fancyhf{}
\rhead{Notes by Afiq Hatta}
\lhead{String Theory}
\rfoot{Page \thepage}

%% Commands for typesetting theorems, claims and other things.
\newtheoremstyle{slanted}
{1em}%   Space above
{.8em}%   Space below
{}%  Body font
{}%          Indent amount (empty = no indent, \parindent = para indent)
{\bfseries}% Thm head font
{.}%         Punctuation after thm head
{0.5em}%     Space after thm head: " " = normal interword space;
{}%         \newline = linebreak
{}%          Thm head spec (can be left empty, meaning `normal')

%% Commands for typesetting theorems, claims and other things. 

\theoremstyle{slanted}
\newtheorem{theorem}{Theorem}
\newtheorem*{thm}{Theorem}
\newtheorem*{claim}{Claim}
\newtheorem{example}{Example}

\newtheorem*{defn}{Definition}

\newcommand{\Lagr}{\mathcal{L}} 
\newcommand{\vc}[1]{\mathbf{#1}}
\newcommand{\pdrv}[2]{\frac{\partial{#1}}{\partial{#2}}}
\newcommand{\thrint}[1]{\int d^3 \vc{x} \left( {#1} \right)}

%% QFT specific macros 
\newcommand{\intp}{ \int \frac{ d^3 p }{ (2 \pi)^3 } \, }
\newcommand{\ann}[1]{a_{ \mathbf{ #1 }}}
\newcommand{\crea}[1]{a^\dagger_{ \mathbf{ #1 }}}
\newcommand{\ve}[1]{ \mathbf{ #1 } } 
\newcommand{\mode}[ 1]{ e^{ i \mathbf{ #1 } \cdot \mathbf{x} }}
\newcommand{\nmode}[1]{ e^{  - i \mathbf{ #1 } \cdot \mathbf{x} }}
\newcommand{\freq}[1]{\omega_\mathbf{ #1} } 
\newcommand{\scal}[1]{\phi ( \mathbf{ #1 })} 
\newcommand{\mom}[1]{ \pi (\mathbf{ #1 })} 
\newcommand{\arr}{\rightarrow} 

\newcommand{\incfig}[1]{%
\def\svgwidth{\columnwidth}
\resizebox{0.75\textwidth}{!}{\input{./figures/#1.pdf_tex}}
}

\newcommand{\anop}[2]{ #1_\mathbf{#2}}
\newcommand{\crop}[2]{#1_\mathbf{#2}^\dagger}

\usepackage{helvet} 

%tikz decoration commands 
\usetikzlibrary{decorations.pathmorphing}


\title{String Theory Lecture Notes}
\author{Lectured by Dr. Reid-Edwards} 
\begin{document} 
\maketitle
\tableofcontents

\pagebreak 

\section{Introduction}%
\label{sec:introduction}

If you'd like to read about string theory, 
check out 
\begin{itemize}
\item String Theory, Vol 1, Polchinksi, CUP 
\item Superstrings, Vol 1, Green et al, CUP
\item A String Theory Primer, Schomerus, CUP
\item David Tong's Notes, at arXiv: 09080333
\item 'Why String Theory' Conlon, CRC  
\end{itemize}

What is string theory? 
We don't really know what string theory is. 
The question itself requires a little more 
fleshing out to make sense of it. String theory 
is a work in progress. A final version of string 
theory would be a theory which we understand physically 
and mathematically. 
It's a work in progress - the final form will 
be quite different from what we have now. 

What do we know? 
Well, in some sense, string theory 
is an attempt to answer the question of how we quantise the 
gravitational field. A theory of \textbf{quantum gravity}. 
There are however, a number of obstacles.
In particular, naive quantisation of the 
Einstein Hilbert action presents a number of problems. 

There are deep conceptual problems associated with this. 
\begin{itemize}
\item  There's a question about the 
	nature of time in quantum gravity. 
If we think about time in quantum mechanics, time 
is treated as a fixed clock in which the Hamiltonian 
governs the evolution. In GR, space and time are on the same footing. 
Thus, the descriptions are on a different footing 
in QM versus GR. This is not neseccarily a technical problem, 
just something to think about. 
\item How do we quantise things without a pre-existing causal structure? 
What do we mean by this? We we quantise things in QFT, 
it's important to know whether two operators 
are timelike or spacelike separated. We have a notion 
that for all operators which are spacelike separated, 
commute. (One should not be able to influence the other). 
If we try to quantise GR, the metric is the object which 
we would like to quantise. But, this determines the casual 
structure. So, it's not immediately obvious 
what the algebra of operators should look like. 

\item GR has a very big symmetry - diffeomorphism symmetry (symmetry under
coordinate reparametrisations). This is diffeomorphism 
invariance. This is something we'll discuss a bit later on. 
This is considered to be a gauge symmetry. 
\item One thing we'll discuss is that there are no local 
diffeomorphism invariant observables in GR. It's 
not even clear what the observables should be. 
\end{itemize}

More importantly, 
there are technical obstacles. The previous issues are hard, 
but can we make some assumptions which help us make progress? 
We can look at perturbation theory; we can take our 
metric an expand it around some classical solution
\[
g _{ \mu \nu } \left( X  \right)   = \eta _{ \mu \nu } + h _{ \mu \nu } \left( x \right) 
\] for our purposes, this classical solution will be 
Minkowski space-time. 
This means we can use the causal structure of 
the background metric (Minkowski) to 
learn about the causal structure of the perturbation, 
and quantise. 
This immediately neutralises the first two 
conceptual problems. We can call the fluctuations $ h _{ \mu \nu } $ 
as gravitons. 
So, if we take a background static spacetime, add a field, 
then quantise. 

There are some unsatisfactory things about this. 
When we split the metric into two, 
we hide a lot of the deep structure that we want. 
Nonetheless, we can take our Einstein-Hilbert action and expand it out
\[
S \left[ g  \right]   = \frac{1}{K_ 0 } \int d ^{ D } X \sqrt{  - g }  R ( g ) 
\] Choose a gauge and expand out 
\[
S \left[  h  \right]   = \frac{1}{K_0 } \int  d ^ D X \left( 
h _{ \mu \nu } \Box h ^{ \mu \nu } \right) 
\] Since the Ricci scalar contains inverses, 
this expansion goes on forever. This is called 
a non-polynomial action. 
The first quadratic term gives the propagator, 
and the higher order term gives us our interactions. 

The propagator is represented by a wiggly line. 
The interaction term gives us vertices. (Three or four 
wiggly lines coming together) 
These lead to Feynman rules.

However, when we compute loops, 
we get divergences. But, in physical QFT, 
we can absorb these divergences into coupling constants. 
These can be dealt with using standard techniques. From
advanced quantum field theory, this technique 
is called renormalisation. 

Thus, the difficulties run deeper than conceptual ones. 
We simply don't know how to calculate. 
So, string theory 
provides a way to do quantum perturbation theory 
of the gravitational 'field' 
We put field in quotation marks because it's 
not really a field which we'll be dealing with. 

String theory answers some of the questions, 
in the sense that it gives 
us a framework to ask 
meaningful questions in quantum gravity. 
But not all of these questions are questions that 
we can answer. 

The viewpoint that  we're going to 
take for this course will be from 
perturbation theory. But we'll always 
try to understand what this tells us about the 
non-perturbative physics. 

\subsection{Worldsheets and Embeddings}
Let's try 
to put together some sort of language 
to get started. From any popular science book, 
you may find that particles are described 
as vibrating strings. 
The starting point is to consider a 
worldsheet $ \Sigma $ which is a 
2-dimensional surface swept out by a string. 
This is analogous to a 
worldline swept out by a particle.

(Insert diagram of line parametrised by $ \tau $  and pointlike 
object, and diagram of cylinder object with surface called 
$ \Sigma $, with two axes called $ \tau $ and $ \sigma $.   This 
diagram is in 3d Minkwoski space). 
We put coordinates $ \left( \sigma , \tau  \right)  $
on $ \Sigma $, at least locally. 
And, we can define an embedding of the worldsheet $ \Sigma $ 
in the background spacetime, $ \mathcal{ M } $ (Minkowski space),
by the functions $ X ^{ \mu } \left( \sigma, \tau  \right)  $, 
where the $ X ^ \mu $ are coordinates on 
$ \mathcal{ M }  $. So if you like, we have that 
\[
X : \Sigma \to \mathcal{ M }
\] Why is this an embedding? 
A choice $ \sigma $ and $ \tau $ on $ \Sigma $ gives 
us a location in Minkwoski space. 
There are rules (which we shall investigate), 
for gluing such worldsheets together 
in a way which is consistent with 
the symmetries of the theory. 

So, we can not only describe the embeddings 
of a single string propagating through space-time, 
but multiple strings coming together. 

(Draw a diagram of two tubes merging into one tube, 
then splitting back again into two tubes - this looks like 
a three point vertex)

We shall see that such diagrams like the above 
are in one-to-one 
correspondence with correlation functions in some 
quantum theory. 
It is natural to interpret such diagrams 
as Feynman diagrams in a perturbative expansion 
of some theory about a given vacuum. 

So what we have is a way of understanding 
writing down Feynman rules, and performing 
successively better approximations to 
an exact result in field theory which we don't have. 

So what we have here will turn out to be Feynman rules for a theory 
which we don't yet have. Where do these Feynman rules come from? 
Quantising is tremendously restrictive. 


\section{The Classical Particle and String}
In non-relativistic Q, we treat 
time $ \left( t  \right)  $ as a parameter 
and position $ \hat{ X } ^ i $ as an operator. 
Obviously, this kind of restiriction shouldnt 
survive very long in a relativistic string theory. 
So, there are choices to be made here. 
One of those choices is second quantisation.
Second quantisation is when both $ X ^ i $ and $ t $ 
are parameters. Then, 
we quantise fields, for example $ \phi \left( x, t  \right)  $ 
which are the fundamental objects of interest in our 
theory. 
We of course require that the fields 
transform in an appropriate way under field transformations. 
This is what we do in QFT. Most of what we know for example, 
in the standard model, omes from this approach. 

However, there is another way. 
This is \textbf{first quantisation}. 
We elevate $ t $ to be an operator, 
and have something else in the background. 
This is a natural framework for 
describing the relativistic 
embedding for worldline, worldsheet
or worldvolume 
in a spacetime.

And here, $ X ^ \mu  = \left( X ^ i , t  \right)   $ 
is an operator, which is the fundamental 
object which we quantise (our basic variable), 
and we have some other natural parameter 
entering the theory. 
We'll look at concrete examples of what that parameter 
is through this section. 

This other possibility, 
where we think of our fundamental degrees of 
freedom as some object embedded into spacetime, 
is the path we'll take in string theory. 
This is because this approach has been very successful. 

There is string field theory however, which takes the first 
approach, but this leads to most of the results 
of first quantised string theory. 

\subsection{Worldlines and Particles}
Suppose we want to take this approach 
with a particle. 
We consider an embedding 
of a worldline $ \mathcal{ L } $ 
into spacetime $ \mathcal{ M } $. 
We assume zero curvature. 
The basic field is the embedding $ X ^ \mu : 
\mathcal{ L } \to \mathcal{ M } $ 
and an action might be 
\[
S \left[  X  \right]   =  -m \int _{ x_0 } ^{ x_1 } ds  =  -m 
\int_{ \tau_1 }^{ \tau_2 } d \tau \sqrt{  - \eta _{ \mu \nu } \dot{ X  }^ \mu \dot{  X} ^ \nu   } 
\] (Insert diagram of line connecting nodes $ x_ 1 ^ \mu $ and $ x _ 2 ^ \mu $  )
where $ \tau $ (a parameter) is the proper time 
and 
\[
X ^ \mu \left( \tau _ 2  \right)   =  x _ 2 ^ \mu , \quad X ^ \mu \left( \tau _ 1  \right)   = x_1 ^ \mu  
\] are endpoints of the worldline. 
It makes sense 
that our action 
should be proportional to 
the length of the worldline. 
So, a reasonable guess for our action. 
We're taking our space-time metric as $ \left(  -, + , + , +  \right)  $. 
This seems like a reasonable starting point. 

The constant $ m $ has dimensions of mass, 
so a good guess is that this parameter 
is interpreted as the mass. 
We can do some things with this action. 
We can first compute the 
conjugate momentum
to $ X ^ \mu \left( \tau  \right)  $, which is 
\[
P _ \mu \left(\tau   \right)   =  - m \frac{ \dot{ X  }^ \mu   }{ 
\sqrt{  - \dot{ X} ^ 2   }  } , \quad \left( \dot{ X ^ 2 }  = \eta_{ \mu \nu } 
\dot{X } ^ \mu \dot{ X } ^ \nu   \right) 
\] This satisfies $ P ^ 2 + m ^ 2  =  0 $. 
This is what we call an 'on shell' condition. 
There are two symmetries 
associated with this action. 
\begin{itemize}
\item  We have a rigid symmetry, where 
	\[
	 X ^ \mu \left( \tau  \right)  \to \Lambda \indices{ ^ \mu _ \nu } 
	 X^ \nu \left( \tau  \right)  +  a ^ \mu 
 \] where $ \Lambda \indices{ ^ \mu _ \nu }  $ is a Lorentz transformation 
 matrix and $ a ^ \mu $ is a constant displacement. 
\item Also, this action has 
 re parametrisation invariance. In other words, 
 in the physical variable of the $ x $ 's, $ \tau $ is 
 just a parameter which measures the distance along the line. 
 So, we can replace it. 
 If we take 
 \[
	 \tau \to \tau + \xi \left( \tau  \right)  
 \] The embedding $ X ^ \mu $ changes as 
 \[
  X ^ \mu \left( \tau  \right)  \to 
  X ^ \mu \left( \tau + \xi  \right)   = X ^ \mu \left( \tau  \right)  
  + \xi \dot{ X } ^ \mu \left( \tau  \right)  + \dots  
\] To first order, we have that $ \delta X ^ \mu \left( \tau  \right)  
\xi \dot{ X } ^ \mu \left( \tau  \right)   $. 
\end{itemize} 

There's a rewriting 
of this action which makes life a little bit easier. 
Specifically, the action above is 
hard to interpret in the massless case. 
We can rewrite the action 
as \[
X\left[  X , e  \right]    = \frac{1}{2 } \int 
d \tau \, (  e ^{ -1 } \eta _{ \mu \nu } \dot{ X } ^ \mu \dot{ X} ^ \nu   
- e m ^ 2 ) 
\] There are no square roots, and we can take the massless limit. 
We will show that this new action is equivalent to 
the one we wrote down earlier. 
$ e \left( \tau  \right)  $ is some new field on the worldline. 
If you like, you might want to 
think of $ e $ as some one-dimensional metric which 
sets the scale of distances on the line. 

The equations of motion for $ x ^ \mu \left( \tau  \right)  $ 
and $ e \left( \tau  \right)  $ are as follows 
\[
\frac{d }{ d \tau } \left( e ^{ - 1 } \dot{ X }   \right)  = 0 
\] We notice that interestingly, $ e $ does not appear with a time 
derivative. So its equation of 
motion is purely algebraic. If you like, you 
can think of $ e $ as being a lagrange multiplier 
for every single point $ \tau $ on the worldline. 
The $ e $ equation of motion is 
\[
\dot{ X } ^ 2 + e ^ 2 m ^ 2  = 0  
\] $ e \left( \tau  \right)  $ enters 
algebraically 
and it can be thought of 
as a constraint! 
The momentum conjugate to $ X ^ \mu $ 
is 
\[
P _ \mu  = e ^{ - 1 } \dot{ X } ^ \mu  
\] If we combine this with the algebraic constraint
for $ e $, we can combine this with the mass shell 
condition 
to get $ P ^ 2 + m ^ 2  = 0 $. 
So interestingly, this auxillary field $ e $ 
imposes a constraint but is equivalent to 
the space-time energy momentum condition. 

We can write $ e ^{ - 1 }  = \frac{ m }{ | \dot{ X }  |  } $, 
plug this into the action 
to find $ S \left[  X , e  \right]  $ 
subject to the equations of motion 
for $ e \left( \tau  \right)  $ gives 
precisely the action 
\[
S\left[  X  \right]   =  - m \int_{ C } \sqrt{  - \eta _{ \mu \nu } \dot{ X } ^ \mu 
\dot{ X } ^ \nu  }  
\] What guarantees that $ e ^{ - 1 }  $ exists? 
Well, a priori nothing. But, we 
can motivate $ e $ coming from 
the interpretation as being a metric, 
which is invertible. 

With a bit more work, 
we can argue that the $ m $ goes to 
$ 0 $ limit gives 
us the description for null light. 
This action is overall a lot nicer. 

The action $ S \left[  X , e  \right]  $ has the 
symmetries 
\begin{itemize}
\item  Poincare invariance, where $ e $ is 
	invariant. 
\item This also has re-parametrisation 
	invariance, but since $ e $ depends on $ \tau  $, 
	it also has to transform. 
	Infinitesimally, 
	\[
	 \delta X ^ \mu  = \xi \dot{ X } ^ \mu , \quad \delta e  = \frac{ 
	 d }{ d \tau } \left( \xi e  \right)  
	\] provided these variations vanish on the endpoints. 
	$ e $ is not a scalar function on the worldline, 
	but this is the natural way to choose how it 
	transforms so that the action is invariant. 
\end{itemize}
we have a couple of comments. 
The first thing we could to is add curvature 
to our spacetime. 
We could generalise $ \eta _{ \mu \nu } \to g _{ \mu \nu } \left( X \left( \tau  \right)   \right)  $. Then, this becomes 
a highly non-linear model. 

\subsection{Classical Strings}
\subsubsection{Nambu-Goto Action}
The Nambu-Goto action 
is the analog of the action above 
but for a string. 
(Diagram of cylinder with open ends with axes $ \sigma , \tau $ ). 
The fundamental degree of freedom is 
\[
X : \Sigma \to \mathcal{ M } 
\] In this context, we refer to the object which 
the sheet is embedded into as the target space. 
Often, the thing we're embedding into may not 
be space-time for historical reasons. 
The Nambu-Goto action 
is the proposed generalisation 
so that for $ X ^ \mu \left( \sigma , \tau  \right)  $, we have 
\[
S \left[  X  \right] =  - \frac{1}{2 \pi \alpha '  } 
\int d \sigma d \tau \sqrt{  - \det \left( \eta _{ \mu \nu } \partial  _ a X ^ \mu 
\partial  _ b X ^ \nu \right) } 

\] This is an action which 
is proportional to the area 
spread out by the wordsheet. 
$ \alpha  ' $ is a historically labelled 
constant with dimensions of area as measured in 
spacetime. 
One often speaks of the string length $ l _s = 2 \pi \sqrt{ \alpha ' } $. 
We introduce the string tension $ T  = \frac{1}{ 2 \pi \alpha ' } $, 
where we assume throughout that $ \hbar = 1  =c  $. 
These are some characteristic scales in the 
theory. 

The usual sort of issues from Nambu-Goto 
are similar to the issues we faced from 
the original worldline action. 
A much better starting point 
for us is the Polyakov action. 
We place this game of removing the 
square root at the price of introducing an 
extra non-dynamical field. 
\[
S\left[  X, h   \right] =  - \frac{1}{ 4 \pi \alpha ' } 
\int _{ \Sigma } d ^ 2 \sigma \sqrt{  -h }  h ^{ ab } \eta _{ \mu \nu } 
\partial  _ a X ^ \mu \partial  _ b X ^ \nu 
\]  Our entire lecture course 
will start from the 
quantisation of this. 
$ h _{ ab } $ is a metric on  $ \Sigma $ and is 
non-dynamical  - there are no terms 
involving derivatives $ h$ - it is 
merely a constraint field like $ e $ was. 
We will find however that $ h $ plays 
an important role. 

If we remember $ h $ as a metric, 
this is a two dimensional quantum Klein-Gordon 
field which is massless and in $ 2 $ dimensions. 
if we treat this as a two dimensional quantum field theory, 
there are other terms which we may want to add. 

\subsection{Equations of Motion}
In this section, we'll be varying the 
Polyakov action to find our equations 
of motion. The first thing 
we do is to vary the metric. When varying the metric, 
we use an important identity.
Taking the Polyakov action, if we vary it 
with respect to $ h _{ ab } $, we get that 
our change is 
\begin{align*}
	\delta S &=   - \frac{1}{4 \pi \alpha ' } 
	\int d ^ 2 \sigma \, \sqrt{  -h }  h ^{ ab } T _{ ab }  \\ 
	T_{ ab }  &=  \partial  _ a X ^ \mu \partial  _ b X _ \mu 
	- \frac{1}{2 } h _{ ab } h ^{ cd } \partial  _ c X ^ \mu \partial  
	_ d X _ \mu 
\end{align*}
Here, we used the important 
fact that 
\[
 \delta \sqrt{ - h }  =  - \frac{1}{2 } \sqrt{ - h }  h ^{ ab } \delta h _{ ab  }
\] 
Since our action doesn't depend on 
derivatives of our added metric $ h _{ ab } $, 
we then impose that $ T _{ ab }  = 0 $, 
which is our stress-energy tensor condition. 

There are two important remarks to 
make here about $ T _{ ab } $. Since 
we're in two dimensions, we have that 
$ h ^{ ab } T _{ ab }  =0 $ since we can 
diagonalise $ h $ to be Minkowski, and then $ h _{ ab } h ^{ ab }  =2 $. 

We get the equation of motion 
from the Euler-Lagrange equations on 
just $ X ^ \mu $. 
\[
	\partial  _ a \left( 
	\frac{\partial  L }{\partial  \partial  _ a 
X _ \mu }  \right)    =0 \implies = \partial  _ a 
\left( \sqrt{ - h }  h ^{ ab } \partial  _ b X ^ \mu  \right)   =0 
\] 
As we will show, we can use 
diffeomorphisms and Weyl transformations 
so that we can pick $ h_{ ab } $  to be 
the Minkwoski metric. 
In this case, the equation of motion 
above reduces to the wave equation.

\subsubsection{Classical Equivalence of the 
Nambu-Goto action with the Polyakov action}

We show that the Polyakov action 
is equivalent to the Nambu-Goto action. 
To do this, we take the equations 
of motion with the stress tensor and 
back substitute into the action. We use a 
trick and some notation here. We write 
firstly that $ G_{ ab }  = \partial _ a X \cdot  \partial  _ b X $. 
Our stress tensor condition then reads 
\[
	G _{ ab }   =  \frac{1}{2 } h _{ ab} h ^{ cd } G _{ cd }  
\] Now, if we take the determinant 
on both sides, remembering to square the scale factor 
of the right hand side, 
we have that 
\begin{align*}
	G  & = \frac{1}{4 } \left( h ^{ cd } G _{ cd }   \right)  ^ 2
	h \\ 
	\sqrt{ - G }  &=  \pm \frac{1}{2 } 
	\sqrt{ - h }  \left( h ^{ cd } G _{ cd }  \right)   
\end{align*} We took the negative and square root
going into the second line. The Polyakov action is 
\[
 S\left[  X, h  \right]  
  = - \frac{1}{4 \pi \alpha ' } \int d ^ 2 \sigma h ^{ab  }
  \sqrt{-h }  G _{ ab } = - \frac{1}{2 \pi \alpha '  } \int d ^ 2 \sigma \sqrt{ - G }  
\] The final expression 
is the Nambu-Goto action! 

There's a remark to be made 
about which symmetries we have associated with each action. 
We had that the Nambu-Goto action had reparametrisation (or diffeomorphism) invariance, 
but the Polyakov action had Weyl invariance and reparametrisation invariance. 
Namely, these symmetries are represented as follows. 

\begin{itemize}
	\item Our diffeomorphism invariance 
		comes from an infinitesimal change of coordinates 
		\[
		 \sigma ^ a \left( \sigma, \tau  \right)  \to 
		 \sigma ^a \left( \sigma, \tau  \right)  + \epsilon ^ a \left( \sigma, \tau  \right)  
		\] We have that our embedding $ X ^ \mu$ changes 
		the normal way through a Taylor expansion, with 
		\[
		X ^ \mu \to X ^ \mu + \partial _ \nu \epsilon ^ \nu X ^ \mu 
		\] Our metric on the other hand transforms 
		with the Lie derivative of $ h _{ ab } $ 
		with $ \epsilon ^ \nu $ as the generating vector field. 
		We have that 
		\[
		 \delta h _{ ab }  = \nabla _ a \epsilon _ b 
		 + \nabla _ b \epsilon _ a  
		\] We can write this explicitly by expanding the 
		covariant derivative using $ h _{ ab } $ as 
		the metric on our worldsheet.
	\item We also have Weyl invariance 
		which comes from leaving the 
		embedding itself fixed, but 
		changing the metric $ h ^{ ab } $ by 
		a Weyl transformation. 
		So, 
		\[
		 X^ \mu \left( \sigma, \tau  \right)  \to 
		 X ^ \mu \left( \sigma, \tau  \right), 
		 \quad h _{ ab } \to e ^{ 2 \Lambda } h _{ ab } 
		\]  
\end{itemize}
What's going on? Well, since we're adding a field $ h _{ ab } $, 
since it's a 2 by 2 symmetric matrix, we're 
adding 3 degrees of freedom. 
However, our Weyl parametrisation and diffeomorphism invariance 
allows us to fix these degrees of freedom back to zero, 
so we have no net change in the amount of degrees of freedom. 
Now previously, notice that we 
had 

\subsection{Extending the Polyakov action}
There are a couple of things we can do 
to extend our Polyakov action further. 
We list them here. 
\begin{itemize}
	\item Recall that in our action 
		we're always using the Minkowski metric to 
		contract the derivatives of our embedding 
		field $ X ^ \mu $. 
		We could explore changing this to a general 
		metric $ G_{ \mu \nu }\left( X \right) $. 
		We'll look at this more in detail later.
	\item We could add an Einstein-Hilbert term 
		into the mix, which is 
		equal to what we call an Euler characteristic, 
		which is a topologically invariant term. 
		\[
		 S + \int_{ \Sigma } d ^ 2 \sigma 
		 \sqrt{ - h }   R\left( h  \right)   = \chi
		\] 
	\item We could explore adding a cosmological 
		constant term, which looks like 
		\[
		 S + \Lambda \int _{ \Sigma } d ^ 2 \sigma \sqrt{ - h }  
		\] 
	\item We could try include a background field 
		(more detail on this later). 
\end{itemize}

\subsection{Classical solutions}
From our action, we can use 
diffeomorphism invariance to take away two degrees 
of freedom from our metric field $ h _{ ab } $, 
and to set our field to have one scalar degree of freedom: 
\[
	h _{ ab }  = e ^{ 2 \phi  } \begin{pmatrix}  
	- 1 & 0 \\ 0 & 1 \end{pmatrix} 
\]  This on it's own is enough to 
force our action to look like 
\[
 S  = \int _{ \Sigma } d ^ 2 \sigma  - \dot{X } ^ 2 + 
 X^{  ' 2 } 
\] Our scalar factor of $ e ^{ 2 \phi } $ 
cancels out in the end. In fact, we could've used 
our Weyl invariance to take our metric 
$ h _{ ab }  = \eta _{ ab } $ right off the bat.
Alternatively, 
we can go straight to our equation of 
motion and then set our metric to 
be the flat Minkowski metric at a given 
point locally. 
\[
	\partial  _ a \left( \sqrt{ - h }  h ^{ ab } 
	\partial  _ b X ^ \mu \right)   =  \Box X ^ \mu  =0  
\] So, we have the wave equation which govern 
our dynamics. 
This is good because we already know how to deal 
with the wave equation - we split it up 
into left and right moving modes. 

\subsection{Classical Hamiltonian Dynamics of the String}
Today, we're 
going to be interested in the quantisation 
of our closed bosonic string. 
To a first approximation, 
the canonical quantisation theory is 
quite straight forward. 

We're going to continue to
work in what we're going to 
call conformal gauge. 
This is when we take the metric 
on the worldsheet to take 
the form 
\[
h _{ ab }  = e ^{ \Phi } \begin{pmatrix}  
- 1 ^ 0 \\ 0 & 1 \end{pmatrix} 
\] So we have a sort of natural notion of time. 
We can then define the 
canonical momentum field 
conjugate to $  X^ \mu $ 
in the usual way. 
This is just 
\[
P _ \mu \left( \sigma, \tau  \right)   = \frac{\delta S \left[  X  \right]  }{ 
\delta X ^ \mu \left( \sigma, \tau  \right)  }  = \frac{1}{2 \pi \alpha   ' }\dot{ X } _ \mu   
\] We can also do the usual stuff 
and write down the Hamiltonian density. 
Given the Lagrangian density 
$ \mathcal{ L } $, the Hamiltonian 
density is 
\[
\mathcal{ H }  = P _ \mu \dot{ X } ^ \mu  - \mathcal{ L }  = \frac{1}{4 \pi \alpha  ' }
\left( \dot{ X } ^ 2 + X ^{ ' 2 }   \right)  
\]  Recall that the 
dot derivative is the derivative 
with respect to $ \tau $, and the prime is the 
derivative with respect to $ \sigma $. 
It is always useful when looking at 
Hamiltonian dynamics 
to define the Poisson brackets. 
We introduce the bracket as $ \left\{  ,  \right\}  _{ P B } $. 
In particle theory, 
where our coordinates 
$ x ^ \mu \left( \tau  \right)  $ and 
momenta $ p _ \mu \left( \tau  \right)  $ are 
on fundamental variables, it is useful to define the following. 
\[
\left\{  f, g,  \right\}  _{ P B }  = 
\frac{\partial  f }{\partial  X ^ \mu }  \frac{\partial  g }{\partial  p_ \mu }  
- \frac{\partial  f }{\partial  p _ \mu }  \frac{\partial  g }{\partial  x^ \mu } 
\]  So, for example, we have that 
$ \left\{ x ^ \mu , p _ \nu  \right\}   = \delta \indices{ ^ \mu _ \nu }  $. 
The Hamiltonian $ \mathcal{ H } $ plays the 
role as a generator 
of time translations. For example, if we 
have some function of $ x, p $ which is 
\[
\frac{df }{ d \tau }  = \frac{\partial  f }{\partial \tau } + \left\{  f, H  \right\}   
\] which can be shown from the chain rule. 
Here, we're doing something slightly more 
general because $ x , p $ depends 
two parameters, not just the $ \tau $. 
So, if you like, this 
is a field theory generalisation. 
Our field theoretic generalisation 
requires that 
\[
\left\{ X ^ \mu \left( \sigma, \tau  \right) , P _ \nu 
\left( \sigma ' , \tau  \right)  \right\}    = \delta \indices{ ^ \mu _ \nu } 
\delta \left( \sigma - \sigma  '  \right) 
\]   This is a precurcsor 
for equal time commutation relations. 
This is a nice construction! 
If we recall the form of $ X ^ \mu \left( \sigma, \tau  \right)  $ 
to be written in terms of fourier modes $ \alpha^ \mu _ n  $
and $ \overline{ \sigma } ^ \mu _ n $. 
We have the 
mode expansion 
\[
X^ \mu \left( \sigma, \tau  \right)   = x ^ \mu 
+ \sqrt{ \alpha  ' }  p ^ \mu \tau  +  i \sqrt{ \frac{\alpha ' }{ 2 } }  
\sum_{ n \neq 0 } \frac{1}{n } \left( 
\alpha ^ \mu _ n e ^{   -in \left( \tau  - \sigma  \right)  } + 
\overline{ \alpha } ^ \mu _ n e ^{  - in \left( \tau + \sigma  \right)  } \right) , 
\quad \sigma \sim \sigma + 2 \pi 
\]  So imposing 
the natural commutation relations on 
$ X ^ \mu $ and $ P ^ \mu $, gives 
us natural Poisson bracket relations 
for $ \alpha , \alpha ' $. 
This requires 
\[
\left\{  \sigma _ \mu ^ \nu , \sigma _ n ^ \nu  \right\}  _{ P B } 
= - i m \eta ^{ \mu \nu } \delta _{ m + n , 0 }, \quad 
\left\{  \alpha ^ \mu _ m , \overline{ \alpha } ^ \nu _ n  \right\}  _{ P B }  = 0 , \quad 
\left\{  \overline{ \alpha } ^ \mu _ m , \overline{ \alpha } ^ \nu _ n  \right\}  _{ P B } 
=   - i m \eta ^{ \mu \nu } \delta _{ m + n , 0 }
\] The left hand side 
vanishes unless $ n  =- m $. Why are we considering closed strings? 
Apparently, they make life easier later, 
and closed strings give rise to gravity. 
Let's see if this is plausible. 
The relation above is for equal times, 
and the $ X , P $ commutation relation 
is valid for equal times, 
so let's check this holds for $ \tau  = 0$. 
We'll see later on that this 
choice of $ \tau $ is in actual fact not a special case. 
Just for simplicity however, we'll do it this way. 

Our string looks like, at $ \tau  =0 $, 
\[
X ^ \mu \left( \sigma  \right)  = x ^ \mu 
+ i  \sqrt{ \frac{ \alpha ' }{ 2 }}  \sum _{ n \neq  0 } \frac{1}{n } \left( 
\alpha ^ \mu _ n e^{ i n \sigma } + \overline{ \sigma } ^ \mu _ n e ^{  - i n \sigma } \right)
\] Our conjugate momenta is then  
\[
P _ \mu \left( \sigma  \right)   = \frac{p ^ \mu }{ 2 \pi } 
+ \frac{1}{ 2 \pi } \sqrt{  \frac{1}{2 \alpha  ' }}  \sum _{ n \neq  0 } \left( 
\alpha _ n ^ \mu e ^{ i n \sigma } + \overline{ \alpha } _ n ^ \mu e ^{  - i n \sigma } \right)
\] 
Computing the commutation relation from this, 
we have that 
\begin{align*}
\left\{  X ^ \mu \left( \sigma  \right)  , P ^ \nu  \left( \sigma '  \right)   \right\}  _{ P B } &  = \frac{1}{2 \pi } \left\{  x ^ \mu , p^ \nu   \right\}  + 
\frac{i}{4 \pi } \sum _{ m , n \neq 0 } \frac{1}{m } 
\left( \left\{  \alpha _ m ^ \mu , \alpha _ n ^ \nu  \right\}  
e ^{ i \left( m \sigma + n \sigma  '  \right)   } + 
\left\{  \overline{ \alpha } _ m ^ \nu , \overline{ \alpha } _ n ^ \nu  \right\}  
_{ P B } e ^{  - i \left( m \sigma + n \sigma  '  \right)  }\right) \\
&= \frac{\eta ^{ \mu \nu } }{ 2 \pi } + \frac{\eta ^{ \mu \nu } }{ 2 \pi } 
\sum _{n \neq  0 } e ^{  i n \left( \sigma - \sigma '  \right)   }  \\
&=  \frac{\eta ^{ \mu \nu }}{ 2 \pi } \sum _{ n  } e ^{ i n \left( \sigma - \sigma '  \right)  
\end{align*}
But, we have that $ \frac{1}{2 \pi } \sum _{ n } e ^{ i n \left( \sigma - \sigma '  \right)  } $ is just a periodic version of the Dirac delta function. 
Thus, we get 
exactly what we're looking for. 
So, 
\[
\left\{  X ^ \mu \left( \sigma  \right) , P ^ \nu \left( \sigma  '  \right)   \right\} _{ P B }   = \eta ^{\mu\nu 	 } \sigma \left( \sigma  - \sigma '  \right)  
\] 
The other way to go about 
doing this is to find $ \alpha , \overline{\alpha } $ 
in terms 
of integral expressions, and 
then use this to prove the commutation relations 
the other way around. 

\subsection{The Stress Tensor and Wit Algebra}
Introduce the 
worldsheet lightcone coordinates 
\[
\sigma ^{ \pm }  = \tau \pm \sigma 
\] which are a  natural choice of 
coordinates in this gauge to 
help us understand the structure 
of this space a little but more. 

In these coordinates, and this gauge, the 
worldsheet metric 
now looks like 
\[
h  =e ^{ \phi } \begin{pmatrix}   0 & \frac{1}{2 } \\
\frac{1}{2 } & 0 \\ \end{pmatrix} 
\] and $ \partial  _{ \pm }  = \frac{\partial  }{\partial  \sigma ^{ \pm } }   $. 
The action and equations of motion 
becomes 
\[
S =  - \frac{1}{2 \pi \alpha ' } \int_{ \Sigma } d \sigma ^ + d \sigma ^  - 
\partial  _ + X \cdot  \partial  _ - X , \quad \partial  _+ \partial  _ - X ^ \mu  =0 
\] The stress 
tensor $ T _{ ab } $ becomes in these coordinates 
\[
T _{ ++ }   =  - \frac{1}{\alpha ' } \partial _ + X \cdot  \partial _ + X 
, \quad T _{ -- }  = - \frac{1}{ \alpha ' } \partial _ - X \cdot  \partial  _ - X , 
\quad T _{ + - }   = T _{ - +  } =0 
\] So we have two fields on the worldsheet, 
the $ X  $s, and the metric which encodes 
the vanishing stress tensor constraint. 

The constraint is  $ T _{ \pm \pm }  =0 $. 
Let's 
pause a little and think
about what these constraints 
might look like in 
terms of these modes. 
It;s going to be very useful to introduce 
the Fourier modes of $ T _{ \pm \pm } $. 
We define at $ \tau  = 0$, the charges 
\[
L _ n  = - \frac{1}{2 \pi } \int_{ 0 } ^{ 2 \pi } d \sigma T _{ -- } \left( \sigma  \right)  e ^{  - i n \sigma } , \quad 
\overline{ L } _ n  = -\frac{1}{2 \pi } \int _ 0 ^{ 2 \pi } d \sigma 
T _{ ++ } \left( \sigma  \right)  e ^{ i n \sigma }
\] We're just starting out to take $ \tau  = 0 $, 
but we'll see later that this doesn't matter. 
What do these modes look like in terms of $ \alpha $ ? 
We'll choose to explore one of them in detail, 
with the awareness that the other moving mode 
will have similar properties. 

If we differentiate $ X ^ \mu $ as 
\[
\partial  _ - X ^ \mu \left( \sigma , \tau  \right)    
\sqrt{ \frac{ \alpha  ' }{ 2 }  }  \sum _{ n } \alpha _ n e ^{  - i n \sigma ^{  -}  } , \quad \alpha _ 0 ^{ \mu }  = \sqrt{ \frac{ \alpha '  }{ 2 } }  p ^ \mu 
\] We find 
that 
\begin{align*}
L_ n  &=  \frac{1}{2 \pi \alpha  ' } \int _ 0 ^{ 2 \pi } 
d \sigma \partial  _ - X ^ \mu \left( \sigma  \right)  \cdot  \partial  _  - 
X _ \mu \left( \sigma  \right)  \\ 
&=  \frac{1}{4 \pi } \sum _{ m , p } \alpha _ m \cdot  \alpha _ p \int 
_ 0 ^{ 2 \pi } d \sigma e ^{  - i \left( m + p - n  \right)  \sigma } \\
&=  \frac{1}{ 4 \pi } \sum _{ m , p } \alpha_ m \cdot  \alpha _ p 
2 \pi \delta _{ p , n  -m } 
\end{align*}
Similarly, this 
holds for $ \overline{ L } _ n $. 
This gives us the 
relation that 
\[
L _ n  = \frac{1}{2 } \sum _ m \alpha _{ n  - m } \cdot   \alpha _ m , 
\quad \overline{ L } _ n  = \frac{1}{2 } \sum _ m \overline{ \alpha } _{ n - m } \cdot  
\overline{ \alpha } _m 
\]  This constraint can be written 
as $ L _ n  = 0  = \overline{ L } _ n $. 
Using the algebra for the $ \alpha _ n ^ \mu $ $ \left( \overline{ \alpha } _ n ^ \mu  \right)  $, we can compute the algebra for $ L _ n \left( \overline{ L } _ n  \right)  $. 
It's 
not to hard to show that the algebra 
of these objects obey 
the algebra 
\[
\left\{  L _ m , L _ n  \right\}   = - i 
\left( m - n  \right)  L _{ m + n }, \quad 
\left\{  L _ m , \overline{ L } _ n  \right\}   = 0 , 
\quad \left\{  \overline{ L } _ m , \overline{ L } _ n  \right\}   = - i \left( m - n  \right)  \overline{ L } _{ m + n }
\] 
This is often called the Witt algebra. 
We shall see, when we promote this 
to quantum operators, 
we see something very similar, but with an important difference. 
We will see that if we set $ L _ n  = 0  = \overline{ L } _n $, 
at a given  $ \tau $ , then 
the evolution of the system 
preserves $ L _ n   = 0  = \overline{ L } _ n $. 
So what we find, underlying the constraints of 
this theory, that there is this underlying 
infinite dimensional symmetry. 

\subsection{A First Look at the Quantum Theory}
Given the work we've done
on the classical theory, quantising 
this to build a theory on Hilbert space will be a straightforward extension. 

Recall from earlier, that we decided to work with the 
Plyakov action.
We transformed our metric locally due to Weyl rescaling, 
which gives us a two dimensional massless Klein-Gordon theory. 
But, as a result, we 
get constraints from the stress-energy tensor. 

As we did in quantum field theory, 
there are two ways we can proceed in imposing 
the constraints. 

\begin{itemize}
\item We can constrain states then quantise, which 
is an approach that has been reasonably successful
(this is called light-cone quantisation). However, we 
will not be following this path of quantisation. 
\item The second approach is 
to quantise the unconstrained 
theory, then impose constraints 
as a physical condition 
on our Hilbert space. For specifically, recall that the only algebraic 
constraint we got from earlier was our condition on our stress energy tensor, $ T_{ab }  = 0 $. 
Thus, this is the only thing we will impose on our Hilbert space. 


\end{itemize}

\subsubsection{Canonical Quantisation}
We quantise by 
replacing out Poisson brackets 
with commutators. So, functions on phase 
space for example, are replaced by operators. 
We do the transformation 
\[
 \left\{  , \,   \right\}  _{ P B  }  \to  - i \left[  , \,  \right] 
\] So, the structure 
of quantum mechanics is 
similar to Hamiltonian dynamics. 
These give rise to equal time comuttation 
relations: 
\[
	\left[  X ^ \mu \left( \sigma \right) , X ^ \nu \left( \sigma '  \right)    \right]   = 0 , \quad \left[  P _ \mu \left( \sigma  \right)  , 
	P _ \nu \left( \sigma '  \right)   \right]   = 0 , \quad 
	\left[  P _ \mu \left( \sigma  \right)  , 
	X ^ \nu \left(  \sigma '  \right)  \right]   =  - i \delta 
	\indices{ _ \mu ^ \nu  } \delta \left( \sigma - \sigma '  \right)   
\] Now, recall the mode 
expansions we get from expanding 
our position operator $ X ^ \mu $, 
\begin{align*}
	X ^ \mu \left( \sigma , \tau  \right)   &= 
 x ^ \mu +  \alpha' p ^ \mu \tau + i \sqrt{ \frac{ \alpha ' }{ 2 } }  
 \sum _{ n \neq 0   } \frac{1}{n } \left( \alpha _ \nu ^ \mu 
 e ^{  - i n \left( \tau  - \sigma  \right)  } + \overline{ \alpha } _ n ^ \mu 
  e^{  - i n \left( \tau + \sigma  \right) }\right) \\
  P ^ \mu \left( \sigma, \tau  \right)  &=  
  \frac{p ^ \mu }{ 2 \pi } + \frac{1}{2 \pi  } 
  \sqrt{ \frac{1}{2 \alpha  ' }  }  
  \sum _{ n \neq 0 } \left( \alpha _ n ^ \mu e ^{  - i n\left( \tau + \sigma  \right)  } 
  + \overline{ \alpha } ^ \mu _ n e ^{  - in \left( \sigma - \tau  \right)  } \right) 
\end{align*} One can show that the commutation 
relations for the mode operators $ \alpha $ and $ \overline{ \alpha }  $  are 
consistent which changing our Poisson brackets to 
commutators. This gives our relations as 
\[
 \left[  \alpha _ m ^ \mu , \alpha _ n ^ \nu  \right]   =
 m \delta _{ m + n , 0 } \eta ^{ \mu \nu } , \quad 
 \left[  \alpha _ m ^ \mu , \overline{ \alpha } _ n ^ \nu  \right]   = 0 , 
 \quad \left[  \overline{ \alpha }_ m ^ \mu , \overline{ \alpha } _ n ^ \nu  \right] 
  = m \delta_{ m +  n , 0 }\eta ^{ \mu \nu } 
\] For each 
dimension $ \mu $, these are an 
infinite number of ladder operators, where the annihilation  
operators are given by 
$ \alpha _ n $ for $ n > 0 $. However, by looking at the
fact that $  X ^ \mu $ and $ P ^ \mu $ should be real 
and by comparing coefficients, we find the relations 
\[
 \left( \alpha_ n ^ \mu  \right)  ^{ \dagger }  = \alpha _{  -n } ^ \mu 
\] These are 
creation and annihilation ladder 
operators. 
They are creating and annihilating 
different modes on the string. 
It use necessary to 
introduce a vacuum state on $ \Sigma $, 
$ \ket{ 0 } $ such that 
\[
 \alpha _ n ^ \mu \ket{ 0 }  = 0, \quad n \geq 0 
\] It is 
important to note that this is not a vacuum in spacetime, 
it is a vacuum of vibrational modes. 
We recall the 
Fourier modes of $ T _{ ab } $
are $ L _ n $ and $ \overline{ L } _ n $. 
Just as before, 
we can write those modes in terms of $ \alpha $ and $ \overline{ \alpha } $. 

\[
 L _ m  = \frac{1}{2 } \sum_ n \alpha _{ m - n } \cdot  \alpha _ n 
\] 
We call $ L _ m $ Virasoro 
operators. This expression is 
ambiguous for $ m   = 0 $, 
going from the classical to the quantum theory, 
operator ordering starts to matter.
 This is because $ \alpha _ n $ and $ \alpha _{  - n  } $
 do not commute for $ n \neq 0 $, and hence
 we don't know how to order these in our expression for $ L _ 0 $. 
 In this case, we shall take 
 \[
   L _ 0  = \frac{1}{2 } \alpha _ 0 ^ 2 + \sum _{ n > 0  } \alpha 
   _{ -n  } \cdot  \alpha _ n 
 \] 
 We can adopt the usual notion of 
 normal ordering. 
 There are other normal ordering prescriptions 
 we could use.

 \subsubsection{Physics State Conditions}
 Instead of thinking about 
 the stress tensor and $ X  , P $, it's
 more useful to talk about 
 the Virasoro operators $ L _ i $ to 
 impose $ T _{ ab }  =0 $ on the Hilbert space 
 of our theory. 

 Let's first define 
 some notation to make our lives easier. 
 Define 
 \[
  N  = \sum _{ n > 0 } \alpha _{  -n  } \cdot \alpha _ n ,\quad  
  \overline{ N }  = \sum _{ n > 0 } \overline{\alpha _{ - n }}  \cdot  
  \overline{\alpha} _ n 
 \] These operators have the interpretation of 
 being 'weighted number operators' on physical states. 
 Given this, 
 we can write for example, that
 \[
  L_0 = \frac{\alpha ' }{ 4 } p ^2 + N , \quad  
  \overline{ L } _ 0   = \frac{\alpha '  }{ 4 } p ^ 2 + \overline{ N } , 
  \quad \left( \alpha _ 0 ^ \mu  = \sqrt{ \frac{\alpha '  }{ 2 } }  p ^ \mu  \right) 
 \] We require
 that $ T_{ ab   } \ket{ \phi }  = 0 $ 
 for $ \ket{ \phi } $ to be a physical state. 
 This means that 
 \[
  L _ n \ket{ \phi }   = 0, \quad \text{ for } n > 0 
 \] under Hermitian conjugation, this also implies 
 that $ \bra{ \phi } L _{  - n }  = 0  $. 
 We shall also require 
 \[
  L _ 0 \ket{ \phi }  = a \ket{ \phi } , \quad \overline{ L } _ 0 
  \ket{ \phi }  = a \ket{ \phi } 
 \] allowing for the fact that $ L _ 0 $ 
 may be true up to some constant, which is related to 
 the normal ordering. The constant $ a $ reflects 
 the ambiguity in defining $ L_0 $ in the quantum theory. 
We'll come back as to what values $ a $ should take 
depending on our perspective. For 
now, we should take $ a = 1 $. 
This justification is a posteori. 

For convenience, we also define the new operators 
\[
 L _ 0 ^{ \pm }  = L _ 0 \pm \overline{ L } _ 0 
\] Hence, in terms of our newly defined 
operators, our physical 
conditions are 
\[
 \left( L_0 ^{ +  }  - 2 \right)  \ket{ \phi }  =0, \quad 
 L_ 0 ^ - \ket{ \phi }  = 0 , \quad L_ n \ket{ \phi }  = 0 
  = \overline{ L } _ n \ket{ \phi } \text{ for  } n > 0 
\] The $ L _ n $ are difficult to interpret physically, 
but are related to polarisation 
conditions. The first two conditions 
are related to rotational invariance. 
With these physical constraint conditions, we can start to look at the
spectrum of the system.  

\subsection{The Spectrum}

\subsubsection{The Tachyon}
Now it's time to start constructing 
some actual states in our Hilbert space. 
We can construct a space-time momentum 
eigenstate as 
\[
\ket{ k }  = e ^{  i k \cdot  x } \ket{0} 
\] For now, we just work in the 
context where $ x $ is just a position variable 
and not an operator, like in the expression $ e^{ i k \cdot  X } \ket{ 0 } $. 
However, we will see what this looks like 
at a later stage. 
From quantum mechanics, we know 
how $ p_ \mu $ acts  in the position basis. In terms of a position basis in 
the target space, the 
momentum operator is $  -i \frac{\partial   }{\partial  x ^ u }  
 = p _ \mu $, so 
 $ p _ \mu \ket{ k }  = k _ \mu \ket{ k } $. 
 In addition, $ L _ n \ket{ k }  = 0  = \overline{ L } _ n \ket{ k } $
 straightforwardly, although this 
 has yet  to be shown in the lectures and I'm not sure why 
 this is true. 
 We can write $ L_0 ^ -  $ as 
 \[
 L_0 ^  -  = N - \overline{ N } 
 \] So the vanishing condition 
 $ L_0 ^  - \ket{ \phi }  = 0 $ implies
 that $ N  = \overline{ N }$, and  
 suggests a symmetry of 
 right movers versus left movers. 
 This is sometimes called level matching. 
 It is the weighted count difference from 
 either side. 
 If we apply $ N $ or $\overline{ N } $ 
 to any of the states of the form $ \ket{ k } $, 
 we will always find that since $ N $ is the sum 
 of $ \alpha _ n \cdot  \alpha _{ - n } $ , we 
 can always commute the annihilation operator forward, 
 so on the space of states $ \ket{ k } $, 
 we have that the eigenvalues are $ N  = \overline{ N }  = 0  $. 
 We check that $ \left( L _ 0 ^ +   -2  \right)  \ket{ k } =0   $. 
 Thus, we find that 
 \begin{align*}
	 \left( L_0 ^ +  - 2  \right)  \ket{ k} &=  
	 \left( \frac{\alpha  ' }{ 2 } + N + \overline{ N }  - 2  \right)  \ket{ k } \\
						&=  \left( 
						\frac{\alpha ' }{  2 }
					k ^ 2 - 2  \right)  \ket{ k } = 0 
 \end{align*} 
 This gives 
		us the condition on $ k ^ 2  $, as $ k ^ 2   - \frac{4}{\alpha ' }  = 0 $. 
		If we compare this with the energy-momentum condition 
		$ k ^ 2 + M ^ 2   =0  $, 
		with gives 
		\[
		 M ^ 2  =  - \frac{4}{\alpha ' } 
		\]
The state $ \ket{ k } $ has a spacetime interpretation as a tachyon, 
which is a state with negative mass, 
and is therefore unphysical. 
This problem is 
not going to go away. 
We do have the tachyon, and its 
cured by promoting this to the supersymmetric 
string, and adding supersymmetry. 

\subsubsection{The First Excited State}
We'll now look at the next excited state, 
which are the massless states.
Since we have that for a state to be physical, we 
need $ N  = \overline{ N } $, the 
next best thing you can do is to excite 
our state with $ \alpha_{ - 1 }   $ as well 
as $ \overline{ \alpha } _{ - 1 } $. 
Consider states of the form 
\[
 \ket{ \epsilon }  = \epsilon _{ \mu \nu } \alpha _{ -1 } ^ \mu \overline{ \alpha } _{ - 1 } ^ \nu 
 \ket{ k } 
\] One thing to spot here is that by construction 
$ N = \overline{ N }  = 1 $, which 
can be shown by commuting the operators. 
Is this state even physical? 
Let's look at some conditions which we need to view this as a physical state. 
We can look at the energy momentum condition. 
Let's look at the first condition we have to check:  
\begin{align*}
	\left( L_0 ^ +  - 2  \right) \ket{ \epsilon }  &=  \left( 
	\frac{\alpha ' }{ 2} p ^ 2 + N + \overline{ N }  - 2 \right) \ket{ \epsilon}   \\
						       &=  ( \frac{\alpha  ' }{ 2 } k ^ 2 + 2 - 2
						       ) \ket{ \epsilon } \\
						       &=  0  
\end{align*}

This condition implies $ \frac{\alpha ' }{ 2 } k ^ 2  = 0 $, 
so we require that $ k ^ 2  = 0 $ (null). 
Since we have $ \alpha _{ - 1 } $ and $ \overline{ \alpha } _{ - 1 } $
involved, we have another physical condition to check. 
Consider the next condition, which 
gives $ L _ 1 \ket{ \epsilon }   =0 $. 
This condition gives 
\[
	\frac{1}{2 } \left( \sum _ n \alpha _{ 1 - n} \cdot  \alpha _ n  \right)  \epsilon 
	_{ \mu \nu } \alpha _{ - 1 } ^ \mu \overline{ \alpha } _{  - 1  }^{ \nu } \ket{ k } 
	= \frac{1}{2} \epsilon _{ \mu \nu } \overline{ \alpha  }_{ - 1 } 
	^ \nu \left( \alpha _ 0 \cdot  \alpha_ 1 \right) \alpha _{ -  1 } ^ \mu  \ket{ k } 
\] We arrived at the 
equation above by recognising that the only thing which produces a non-trivial 
equation is when $ n  =0$.
Recall that $ \alpha _ 0 ^ \mu  = \sqrt{ \frac{\alpha ' }{ 2 } }  p ^ \mu $. 
We use this fact to commute $ p ^ \mu $ past the 
modes, and then pick up a 
factor of $ k ^ \mu $. 

We require then 
\[
 k _ \rho \alpha _ 1 ^ \rho \alpha _{ - 1 } ^ \mu \ket{ k }  = 0 
\] we can use our commutator 
to show that the above is 
equal to 
\[
 \epsilon _{ \mu \nu } k _ \rho \left( \left[  \alpha _ 1 ^ \rho , \alpha _{ - 1 } ^ \mu  \right]  + 
 \alpha _{ - 1 } ^ \mu \alpha _ 1 ^ \rho  \right)  \ket{ k }  = \epsilon _{ \mu\nu  } 
 \eta ^{ \mu \rho } k _ \rho \ket{ k }  = \epsilon _{ \rho \nu } k ^ \rho \ket{ k }  = 0
\] 
This condition $ L _ 1 \ket{ \epsilon }   = 0 $ requires 
us to impose $ \epsilon _{ \nu \mu } k ^ \mu  = 0 $. 
In other words, we 
can think of this condition as the fact that 
there are no longitudinal polarisations.
Similarly, $ \overline{ L }_ 1 \ket{ \epsilon }  =0 $  requires 
$ \epsilon _{ \mu \nu } k ^ \nu  =0 $. 
So, 
we have three physical state conditions given to us. 
There are no further conditions on this tensor. 

We then have the conditions on $ \ket{ \epsilon } $.
Our first condition is that it is null and massless, 
so $ k ^ 2  = 0$, and no longitudinal polarisations 
$ \epsilon _{ \mu \nu } k ^ \mu  = 0 $ and $ \epsilon _{ \mu \nu  } k ^ \nu  = 0 $, 
thinking of 
$ \epsilon $ as a polarisation tensor. 


We can decompose $ \epsilon _{ \mu \nu } $ into 
symmetric $ h _{ \mu \nu } $, anti-symmetric ($ b _{ \mu \nu }  $  ), 
and trace $ \phi $ parts. 
We first extract the trace bit, 
which we call the Dilaton. 
\[
 \ket{ \phi }  = \phi \alpha _{ -  1 } ^ \mu \overline{ \alpha } _{ - 1 \mu } \ket{ k }  
\]  

We also have two other particles from this 
\begin{align*}
	\ket{ h }  &=  h _{ \mu \nu  } \alpha _{ - 1 } ^ \mu \overline{ \alpha }_{ - 1 } ^ \nu 
	\ket{ k } \quad \text{Graviton}, \quad h _{ \mu \nu }  = h _{ \nu \mu  } \\
	\ket{ b } &=  b _{ \mu \nu } \alpha _{ - 1 } ^ \mu \overline{ \alpha } _{ - 1 }^ \nu 
	\ket{ k }, \quad \text{B-Field}, \quad b _{ \mu \nu }  = - b _{ \nu \mu } \\
\end{align*}

Now let's look at massive states. 
We can now look at states with $ N = \overline{ N }  = 2 $. 
We have 
\[
 A _{ \mu \nu } \alpha _{ - 2 } ^ \mu \overline{ \alpha } _{ - 2 } ^ \nu \ket{ k } 
 + A_{ \mu \nu \lambda } \alpha _{ - 2 } ^ \mu \overline{ \alpha } _{ - 1 } ^ \nu 
 \overline{ \alpha } _{ -  1 } ^ \lambda \ket{ k } + 
 \tilde{ A }_{ \mu \nu \lambda } \overline{\alpha } _{ - 2 } ^ \mu \alpha _{ - 1 }^ \mu  \alpha _{ - 1}  
 ^ \lambda \ket{ k } + A _{ \mu \nu \lambda \rho } \alpha _{ - 1 } ^ \mu \alpha _{ -1 } ^ \nu 
 \overline{ \alpha  } _{ - 1 } ^ \lambda \overline{ \alpha } _{ -1 } ^ \rho \ket{ k } 
\] and $ N $ and $ \overline{ N  } $ count 
number of quanta going around the string. 
There's no profit 
in us solving the mass shell conditions. 
The mass of such states is $ m ^ 2  = \frac{4}{\alpha ' }  $. 
So the string is describing not only the tahcyon 
and massless fields, but an infinite amount of massive 
fields.

For the most part, string theorists 
have concerned themselves with the massless spectrum. 

\subsection{The Big(ish) Picture}
We started with our Polyakov action 
which describes the embedding of our string in our manifold. 
We then used diffeomorphism and Weyl symmetry 
which allowed us to choose $ h _{ ab } $ to be Minkowksi. 
\[
 S  = -\frac{1}{4 \pi \alpha ' } \int_{ \Sigma } d ^2 \sigma \eta _{ \mu \nu } \partial  _ a X ^ \mu 
 \partial  ^ a X ^ \nu 
\] We could deform this 
theory and add a perturbation to the metric. 
This can be achieved by adding a 
small plane wave deformation to the spacetime 
metric.
For example, we might take 
$ \eta _{ \mu \nu } $ and replace it 
by adding some plane wave. 
\[
 \eta _{ \mu \nu } \to \eta _{ \mu \nu } + h _{ \mu \nu  } e ^{ i k \cdot  x } 
\] 
The action changes by 
\[
 \Delta S  =  - \frac{1}{4 \pi \alpha ' } \int_{ \Sigma } d ^ 2 \sigma 
 h _{ \mu \nu } \partial  _ a X ^ \mu \partial  ^ a X ^ \nu e ^{ ikx } 
\] 
For every deformation of the theory, 
there is an associated operator that 
we get from perturbing the action. 
In this case, it's the 
\[
 \mathcal{ O }  = h _{ \mu \nu } \partial  _ a X ^ \mu \partial  ^ a X ^ \nu e ^{ ikx  }
\] This is clearly associated with a 
deformation of the spacetime metric. 

In string theory, (2-dimensional CFTs), 
to each operator $ \mathcal{ O } $ that 
corresponds to a physical deformation 
of the theory, (something that seems reasonable 
is what we mean by a physical deformation)
we have a Hilbert space
This state in the Hilbert space is given 
by applying the operator onto our vacuum, 
then taking the limit as $ \tau $ goes to $  - \infty$. 
In this case, 
\[
 \lim_{ \tau \to  - \infty } \mathcal{ O } \ket{ 0 }  = \ket{ h } 
\] We will 
do this more carefully later on. 
This is called the state-operator correspondence. 
One last comment on this. 

What if we choose to start with a general metric?
For example, we could've started off with 
\[
 S  =  - \frac{1}{4 \pi \alpha '  } \int _{ \Sigma } d ^ 2 \sigma 
 g _{ \mu \nu } \left( x  \right)  \partial  _ a X^ \mu \partial  ^a X ^ \nu 
\]  Thhis is highly non-linear. 
How would we deal with this? 
We could try expand this in terms of a power series 
and then try to deal with the theory perturbatively.
This is now an interacting theory, 
and this is hard to do.

Moreoever, the condition of Weyl invariance 
in the quantum theory constrains what $ g _{ \mu \nu }\left( X  \right)   $ 
we can have. 
Weyl invariance is $ h _{ ab } \to e^{ \omega \left( \omega, \tau  \right)  } h _{ ab } $.
One finds that $ G _{ \mu \nu } $ has to satisfy 
\[
	R _{ \mu \nu } \left( g  \right)  + \mathcal{ O } \left( \alpha '   \right)   =0 
\] Which is the Einstein 
tensor up to string corrections. 
More generally, if we have other 
background fields, 
we find Weyl invariance requires 
the full Einstein equations to be satisfied to 
leading order in $ \alpha ' $.

\subsection{Spurious states and gauge invariance}
Before, when we mentioned $ L _ 0 $ to 
have an ordering ambiguity, 
we put our normal ordering constant  as $ a =1 $. 
We took $ a = 1  $ in the 
conditions $ \left( L_0  - a\right) \ket{ \phi }  = 
\left( \overline{ L } _ 0 - a  \right)  \ket{ \phi }  = 0  $. 
Why did we do this? 
This is so that we can interpret states easily. 
Consider the state 
\[
 \ket{ \chi }  = \sqrt{ \frac{2}{\alpha  ' } }  
 \left( \lambda _ \mu \alpha _{ - 1 } ^ \mu \overline{ L } _{  1 } + \tilde{\lambda }_ \mu 
 \overline{ \alpha } _{ - 1 } ^ \mu L _ 1  \right)  \ket{ k } 
\] Clearly, $ \ket{ \chi } $ is orthogonal 
to all physical states. 
If $ \ket{ \phi } \in \mathcal{ H } $, then $ \bra{ \phi }\ket{ \chi }  = 0  $
because $ L_1 \ket{ \phi }  = \overline{ L } _ 1 \ket{ \phi }   =0 $. 
What conditions do $ \lambda _ \mu , \tilde{\lambda }_ \mu  ,k   $ 
have to satisfy for $ \ket{ \chi }  $ to be physical. 

Is is useful to write $ \ket{ \chi } $ as 
\[
	\ket{ \chi }  = \left(  \lambda _ \mu 
	k _ \nu + \tilde{ \lambda } _ \nu k _ \mu  \right)  \alpha _{ - 1 } ^ \mu 	\overline{ \alpha }_{ - 1 } ^ \nu \ket{ k } 
\] Keeping $ a  $ arbitrary, 
we find that 
\[
	\left( L_0^ +  - 2 a  \right) \ket{ \chi }  = 0 \implies  k ^ 2  = \frac{ 4 \left( 
	a  - 1  \right)  }{ \alpha ' } 
\] We also have that, by symmetry, 
\begin{align*}
	L_ 1 \ket{ \chi } =0 & \text{ if } \left( \lambda \cdot  k  \right)  
	k _ \mu + \tilde{ \lambda } _ \mu k ^ 2  = 0 \\
	\overline{ L } _ 1 \ket{ \chi }  = 0 & \text{ if } \left( 
	\tilde{ \lambda } \cdot  k  \right)  k _ \mu + \lambda _ \mu k ^ 2  = 0 
\end{align*}
Finally, we have that 
\[
	\bra{ \chi }\ket{ \chi }  = \lambda ^ 2 k ^ 2 + 2 \left( 
	\lambda \cdot  k \right)  \left( \tilde{ \lambda } \cdot  k   \right)  + \tilde{ \lambda } ^ 2 k ^ 2  
\] If $ a = 1 $, $ k ^ 2  = 0  $ , $ \lambda \cdot  k , \tilde{ \lambda } \cdot  k  =0  $, 
so $\bra{ \chi }\ket{ \chi } =0 $. 
\subsubsection*{Worldline actions}
\begin{itemize}[leftmargin=*]
	\item Our action is 
		\[
			S =  - m \int_{ s_1 } ^{ s_2 } ds 
			 = - m \int_{ \tau_1 } ^{ \tau_2 } d \tau \sqrt{ -
			 \eta ^{ \mu \nu } \dot{ x } ^ \mu \dot{ x } ^ \nu   } 
		\] 
	\item We have conjugate momenta with on-shell mass condition  
		\[
		 P^ \mu = -  \frac{m \dot{ x } ^ \mu  }{\sqrt{  - \dot{ x } ^ 2  }  }, 
		 \quad  P ^ 2 + m ^ 2 = 0 
		\]  
	\item It makes more sense to work with Einbeins,
		since we can work in the $ m \to 0 $ limit 
	\[
		S  = \frac{1}{2 } \int d\tau\, \left( e^{ - 1} 
		\eta _{ \mu \nu } \dot{ x } ^ \mu \dot{ x } ^ \nu   - e m ^ 2   \right)  
	\] 
\item Two equations of motion come from the Einbein action 
	\[
		\frac{d }{ d \tau } \left( e ^{ - 1} \dot{ x } ^ \mu   \right)   = 0 , \quad 
		\dot{ x } ^ 2 + e ^ 2 m ^ 2 = 0  
	\]
\item This has symmetries 
	\[
	 \delta x ^ \mu  = \xi \dot{ x } ^ \mu , \quad \delta e  = \frac{ d }{ d \tau } 
	 \left( \xi \dot{ e }   \right) 
	\]
\item In the massless limit, if we replace our Minkowski metric 
	with a general metric, we recover the geodesic equations 
	\[
	 S  = \frac{1}{2 } \int d \tau  e^{ - 1} g_{ \mu \nu } \dot{ x } ^ \mu \dot{  x } ^ \mu , 
	 \quad \ddot{x } ^ \mu + \Gamma ^ \mu _{ \alpha \beta }  \dot{ x } ^ \alpha 
	 \dot{ x } ^ \beta  = 0 
	\]  
\end{itemize}


\subsection{Recap}

We have that 
\[
	\delta h_{ ab }  = \left( P _ v  \right)  _{ ab } + 2 \overline{ \omega } h _ ab 
	+ t ^ I P _{ ab I }
\]  where $ \left( P_ v  \right) _{ ab } 	  $
represents our traceless Lie derivative, the second term represents 
Weyl transformations, and $ p _{ ab I }  =\frac{\partial  h _{ ab } }{\partial  m ^ I }  $. 
$ m^ I $ are coordinates on the moduli space 
of Riemann surfaces $ M _ g $, and $ t ^ I \simeq \delta m ^ 2 $ 
are tangent vectors to $ M _ g $. 

\subsection{Moduli space of unpunctured Riemann Surfaces}
Moduli spaces are a space of objects 
like a metric, but we quotient out some symmetry. 
In this case, we quotient out the space of metrics 
by diffeomorphisms and Weyl transformations. 

\begin{defn}{Moduli Space}
The moduli space $ M _ g $ of a genus $ g $ Riemann surface 
$ \Sigma_ g $ is 
\[
 M _ g  = \frac{\left\{  \text{Metrics} \right\} }{\left\{  \text{Diff} \times \text{Weyl}  \right\} }
\] We can view this 
as a slice of physically inequivalent metrics 
on the whole space of $ h _{ ab } $. 	
\end{defn}
Amazingly 
this is a finite dimensional object. 
We won't prove this, but the dimension, over the field of the 
reals, of $ M _ g $ is 
\[
	\text{dim}\left( M _ g  \right)  : = s  = \begin{cases}
		0 \text{ if } & g  = 0 \\
		2 \text{ if } & g  = 1 \\
		6g  - 6 \text{ if } & g \geq 2 
	\end{cases}
\] The first thing might be 
a sphere, the second thing might be a donut, 
and the last thing might include double donuts. 

Let's look at some examples.
\begin{itemize}
	\item For $ g  = 0 $, all metrics on $ g  =0 $ surfaces 
	may be brought to the form $ e ^{ 2 \omega } \begin{pmatrix}  1 & 0 \\ 0 & 1  \end{pmatrix}  $ 
	by diffeomorphisms, so they're trivial 
	up to Weyl transformations. One way 
	to show this is via the Riemann-Roch theorem or 
	the Atiyah-Singer index theorem. 
	Alternatively, we can see this by just looking 
	at our dimension for moduli spaces - we have that $ \dim \left( M_  g  \right)   =0 $, 
	and roughly speaking, forgetting about diffeomorphism transformations, 
	we have that $ \left\{ \text{metric} \right\} \simeq \left\{ \text{Weyl} \right\}  $. 

	
	This agrees with the thinking we did last time about 
	using reparametrization to take $ h _{ ab } $ to be the identity.

	In terms of the interaction picture, we relate and are
	sort of thinking about these things 
	as perhaps tree level contributions from $ g  =0 $, 
	and not want to worry too much about moduli space. 

\item Now what about genus  1? Genus 1 objects are tori, 
	and there's a standard way to construct these objects 
	in the complex plane. We take the complex plane 
	and two lattice vectors $ \lambda_ 1 , \lambda _ 2 \in \mathbb{ C } $, 
	and identify points which differ by these lattice vectors 
	with one another.
	\[
		z \sim z + \lambda _1 m + \lambda _2 n , \quad \lambda _ 1, \lambda _ 2 
		\in \mathbb{ Z}
\] Choosing different $ \lambda _ i $ makes us choose 
tori with different shapes. Notice that $ \lambda _ i $ 
change under diffeomorphisms of the complex plane. Thus, they're
not really a good set of parameters in which to choose 
a tori modulo Weyl transformations and diffeomorphisms. 
But, it can be shown that the ratio 
\[
 \tau  = \frac{\lambda _ 1 }{ \lambda _ 2 } 
\] is invariant under the transformation group $\text{Diff}\times\text{Weyl}$.

\begin{defn}{Complex Structure}
	We call $ \tau $ the complex structure, 
	not to be confused with our time parameter. 
	\[
	 \tau  = \frac{\lambda _ 1  }{ \lambda _ 2 } 
	\] 
\end{defn}
It can be shown that our 
metric on the complex plane can be rewritten as 
$ ds^ 2 = |dz^ 2 + \tau d \overline{ z } | ^ 2 	$. 
with $ \text{Diff} \times \text{Weyl}$ transformations. In fact, 
we can choose as well that  $ \text{Im} \tau > 0 $. 
So, we're restricting to the upper half plane. 
Even \textbf{if} we're just restricting to the upper half plane, 
there is \textbf{still} some extra redundancy to get rid of. 


However, there's a lot of redundancy here. 
We can write the identification
\[
 z \sim z + n_ a \lambda ^ a , \quad n _a   = \left( m , n  \right)  , \quad 
 \lambda ^ a  = \begin{pmatrix}  \lambda _ 1 \\ \lambda _ 2  \end{pmatrix} 
\] If we act with an $SL(2 )  $ transformation
$ n _ a \to U \indices{ ^ b _ a  } n _ b  $ and $ \lambda ^ a \to \left( U ^{ -  1 }  \right)  
\indices{ ^ a _ b  } \lambda ^ b  $, 
this statement is preserved. 
Moreover, if $ U \in SL \left( 2, \mathbb{ Z  }  \right)  $, 
then $ \left( n , m  \right)  $ remain integers, and 
we still have a valid identification. 
Thus, we still have a symmetry 
of the lattice. 

So, the moduli space at $ g  = 1 $ 
can be identified with 
the upper half plane modulo $ SL \left( 2, \mathbb{Z  }   \right)  $, which 
is denoted 
\[
	\frac{\text{UHP}}{SL\left( 2, \mathbb{ Z }   \right)  } = M _ 1 
\] What does this look like? 

\begin{figure}[htpb]
	\centering
	\input{fundamental.pdf_tex}
	\caption{The space of non-physically equivalent $ \tau $ }%
	\label{fig:}
\end{figure}
\end{itemize}
We show this in the diagram. 
There's no analogue for this 
on world lines. 
For higher genus, things are hard. For $ g \geq 2 $, 
it gets hard. But, there are analogues of the $ SL \left( 2, \mathbb{ Z }  \right)  $
(modular group) in all cases. So, 
we can make very strong statements from modular groups 
that order by order, we have UV finiteness. 
This is one of the first times we see that string 
theory is very different from field theory. 
Other data we can introduce, as well as counting 
inequivalent worldsheets and Riemann surfaces, 
are punctures. 

Before we do that, there's 
a subtlety we have to address. 

\subsection{Conformal Killing Vectors}
 There is an overlap $ \text{Diff} \cap \text{Weyl}$, 
 called the conformal Killing group (CKG), where 
 a diffeomorphism is also a Weyl transformation. 
So, you can think of a diffeomorphism
that can be undone by a Weyl transformation. 
In other words, is it possible such that 
\[
	\delta h _{ ab }  = \nabla _a v _ b  +  \nabla _ b v _ a  + 2 \omega h _{ ab }  = 0 
\] So that the diffeomorphism
and Weyl transformation cancel out.
We take the trace to find that 
\[
 \omega   =  - \frac{1}{2 } \nabla _ a v ^ a 
\] We defined 
\[
	\left( P _ v  \right)  _{ ab }  = \nabla _ a v _ b + \nabla _ b v _ a  - 
	h _{ ab } \left( \nabla _ c v ^ c  \right)  
\] Vectors that satisfy 
$ \left( P _ v  \right)  _{ ab }  =0 $ can be what we 
shall call conformal Killing vectors, 
which we shall abbreviate as CKV's. 
This is a sensible name because 
we want Killing vectors up to conformal transformations. 
So the CKG is generated by 
the CKVs. 
For a given genus, the dimension of a conformal 
Killing group 
\[
	\text{dim}_{\mathbb{R }}\left( \text{CKG} \right) : = \kappa  = 
	\begin{cases}
		6, & g  =0: \quad SL\left( 2, \mathbb{ C } \right)  \\
		2, & g  = 1: \quad U \left( 1  \right) \times U \left( 1 \right)    \\
		0, & g \geq 2: \quad  - \\ 
	\end{cases}
\] On the sphere, this is 
the Mobius group. For the torus, it's two 
circles. 
For example, 
\begin{itemize}
	\item $ g  = 0$ the CKG is $ SL \left( 2, \mathbb{ C }  \right)  $. 
		This is the Mobius group. If $ z \in \mathbb{ C } $, 
		then this is the group 
		 \[
			 z \to \frac{az +  b }{ c z + d } , \quad a, b, c, d \in \mathbb{ Z }, 
			 \quad ad  - b c  =  1 
		\] We can specify a particular 
		element of the CKG at genus $ g = 0 $ by 
		describing how three distinct points transform under 
		a given map. 
		We just need three points since 
		we're dealing with complex coordinates, 
		which takes two real numbers each. 
\end{itemize}

\subsection{Moduli Space of Punctured Riemann Surfaces $ M _{ g, n }  $}
Imagine we have a Riemann 
surface of genus $ g $, 
with $ n $ punctures (or marked points). 
What is it's moduli space? 
The naive thing we might expect 
\[
 M _{ g, n  }  = M _ g \otimes \Sigma ^ n 
\] This is the product of the 
non-trivial information we get from the 
locations of the punctures. 
But, we also have to worry about the CKG. 
If we have a $ g = 0  $ surface 
with $ n $ punctures, we 
can use the conformal Killing group to fix 3 of the 
locations of the punctures. In other words, 3 degrees of 
freedom. This leaves only $ n  - 3 $ free punctures. 

Similarly, we can fix the conformal Killing group 
on the torus $ g = 1  $ by fixing the location of one puncture. 
So, we should really take 
\[
 M _{ g, n  }  = M _ g \otimes \Sigma ^{ n  - \kappa / 2 } 
\] and 
\[
 \dim \left( M _{ g, n }  \right)   = 6g  - 6 + 2n 
\] So the space which we 
will be integrating over, which is 
the space of physically inequivalent metrics. 
When the above quantity is negative, 
or correlation function goes to zero. 

\pagebreak 


\section{The Faddeev-Popov Determinant}
We recall 
how we classify small changes from our underlying metric. 
We will need this later. 
Remember, $ \left( P _v  \right) _{ ab } $ is defined 
as the function whose kernel is the conformal 
Killing group, which is the set of diffeomorphisms 
which also happen to be Weyl transformations. 

When we vary $ \delta h _{ab } $, 
we need to vary $ h $ from diffeomorphisms, 
Weyl transformations, and changes in the moduli space.

Our task in this section will be to calculate a Jacobian
factor associated with the path integral. 
As we have seen before, the partition function in 
absence of a source term 
is the path integral over all Riemann surfaces, 
and physically inequivalent metrics. This is given by 
the following path integral, dividing out by diffeomorphisms 
and Weyl transformations 
\[
 \mathcal{ Z} \left[ 0  \right]  
   = \frac{1}{| \text{Diff} \times \text{Weyl}| } \int \mathcal{ D } X \mathcal{ D } h 
   e ^{  - i S \left[  h , X  \right]  } 
\] Introduce the Faddeev-Popov determinant 
$ \Delta _{ FP } $  as the factor which 
gives the following identity relation: 
\[
 1  = \int_{ M _ g } d ^ s t \int _{ \text{Diff} \times \text{Weyl}} \mathcal{ D } \omega  
 \mathcal{ D } v  \, \delta \left[  h - \hat{ h } \right]  \left( \sigma, \tau  \right)  \right) \Delta _{ FP } \left[  h  \right] 
 \prod _{ i , a = 1 ,2 }^ \kappa  \delta\left( v ^ a \left( \tilde{ \sigma } _ i   \right)  \right) 
\] where the integrand is a $ \delta $-functional.
In the factor $ \int _{ \text{Diff} \times \text{Weyl}}$, 
we are integrating over Weyl transformations 
parametrised by the function $ \omega $, and diffeomorphisms 
generated by the function $ v $. 
The factor which is the product of $ \delta \left( v\left( \tilde{\sigma }  _ i  \right)   \right)  $ 
represents where we have punctures on our Riemann surface. 
We define 
\[
 \delta \left[  h - \hat{ h }   \right]  
 = \prod _{ a, b , \sigma , \tau } \delta \left( 
	 h _{ ab } \left( \sigma,  \tau  \right)  - \hat{ h } _{ ab } \left( 
 \sigma, \tau \right)  \right) 
\]  where $ h _{ ab }  - \hat{ h } _{ ab }  = \delta h _{ ab } 
 = \left( P v  \right)  _{ ab } + 2 \overline{ \omega } h _{ ab } + t ^ I  \mu_{ I ab } $ . 
 In this case, a common choice for 
 $ \hat{ h } _{ ab } $ is  $ \eta _{ ab }  $ or 
 some other fixed metric. $ h _{ ab  }  $ is 
 the metric then obtained by applying a transformation. 
 This is somewhat analogous to 
 looking at $ \int d x \delta \left( f \left( x  \right)   \right)  $, 
 where we have to be a bit careful about the 
 vanishing set. 
 We need some way to fix 
 the conformal killing group, which is $ \text{Diff} \cap \text{Weyl}$. 
 This is only worth 
 worrying about in genus 0 and genus 1. 
 To fix the conformal killing group, 
 we ask that the Diffeomorphisms 
 vanish at particular points, 
 which is why we added a factor at the end. 
 We now want a explicit expression for 
 $ \Delta _{ FP } $. 

 Since we have 
 a delta function inside the integral, we can actually take out the 
 Jacobian and evaluate it on the support of 
 the delta functions. 

 \[
 1  = \Delta _{ FP } \left[  \hat{ h }  \right] 
 \int_{ M _ g } d ^ 3 t \int _{ \text{Diff} \times \text{Weyl}} \mathcal{ D } \omega  
 \mathcal{ D } v \delta \left[  h - \hat{ h } \right]  \left( \sigma, \tau  \right)  \right)  \prod _{ i,  a = 1 ,2 }^ \kappa  \delta \left( v ^ a \left( \tilde{ \sigma } _ i   \right)  \right) 
\]

\subsection{Gauge-Fixing the Path Integral}
Let us now try to 
gauge fix the path integral. 
We substitute the identity integral 
into the expression for our partition function. 
If we do this, we find that 
\[
 \mathcal{ Z } \left[  0  \right] 
  = \frac{1}{| \text{Diff} \times \text{Weyl} | }
  \int _{ \text{ Diff } \times \text{Weyl}} 
  D \omega  D v \, \prod_{ a , i } 
  \delta \left( v ^ a \left( \sigma _ i  \right)   \right)  
  \int _{ M _ g } d ^ s t \int \mathcal{ D } X \mathcal{ D } h 
  \delta \left[  h - \hat{ h }  \right]  \Delta _{ FP }
  \left[  \hat{ h }  \right] e ^{ i S \left[  X , h  \right]  } 
\]
Now, rearranging this object, 
we have that 
this is equal to 
\[
 \mathcal{ Z } \left[  0  \right]  
  = \frac{1}{| \text{Diff} \times \text{Weyl} | }
  \int \mathcal{ D } X \int _{ M _ g } d ^ s t 
  \left( \int_{ \text{Diff } \times \text{ Weyl}}  
	  \mathcal{ D } \omega \mathcal{ D } v 
  \prod_{ a , i } \delta \left( 
  v ^ a \left( \tilde{ \sigma }_ i    \right) \right) \right)
  e ^{ i S \left[  \hat{ h } , X  \right]  } \Delta _{ FP } \left[  \hat{h  } \right] 
\] Throughout this, 
we are considering 
over a particular genus. 
In the brackets, 
we have 
\[
 \int _{ \text{Diff} \times \text{Weyl}} 
 \mathcal{ D } \omega \mathcal{ D } v \prod _{ a, i } \delta \left( 
 v ^ a ( \tilde{ \sigma } _ i)   \right)   = \frac{| \text{Diff} \times \text{Weyl} | }{
 | \text{CKG} | }
\] Let's go over this expression. 
It looks like that we are just integrating 
over all of the diffeomorphism and Weyl transformation 
space, but we still have these points 
given by the delta function which fixes 
our conformal killing group, so we have 
to divide out by this. We find perhaps a more respectable starting point. 
This substituting this into the above integral gives 
us the somewhat more reasonable expression 
\[
 \mathcal{ Z } \left[  0  \right]   = 
 \frac{1}{|\text{CKG }|  } \int \mathcal{ D } X \int_{ M _ g }
 d ^ s t \Delta _{ FP } \left[  \hat{ h } \right] e ^{ i S 
 \left[  \hat{ h } , t  \right] }
\] 
Let's find a field theory 
representation for $ \Delta _{ FP } $. 
Now, 
we can invert this expression, and move 
the Faddeev-Popov determinant 
over to the other side to give 
\[
 \Delta _{ FP } ^{ - 1 } \left[  \hat{ h }  \right]  
  = \int _{ M _ g } d ^ s t \int _{ \text{Diff} \times \text{Weyl}} \mathcal{ D } w 
  \mathcal{ D } v \, \delta \left[  \delta h  \right]  
  \prod_{ i , a } \delta \left( v ^ a \left( \tilde{ \sigma }_ i    \right)  \right) 
\] In this 
case, we write, somewhat confusingly, 
that $ \delta h  = h - \hat{h}$, and 
we have that since the difference 
is generated by a diffeomorphism 
and a Weyl transformation, $ \delta h _{ ab }  = \left( P v  \right)  _{ ab } 
+ 2 \overline{ \omega } h _{ ab } + \mu _{ I ab } t ^ I $, 
and $ \overline{ \omega }  = \omega  + $
some other terms. 
We 
introduce what we might think of as auxiliary fields 
$  \beta _{ ab } \left( \sigma, \tau  \right)    $ 
and $ \xi ^ i _ a $ and write these 
$ \delta $ functionals as integrals. 
The idea is that we write 
the deltas as integrals, exactly how 
$\delta \left( x  \right)  = \int d ^ k e ^{ i k \cdot  x } $. 
\[
\Delta ^{ - 1 } _{ FP } \left[  \hat{ h }  \right]  
= \int _{ M _g } d ^ 3 t \int \mathcal{ D } v \mathcal{ D } \omega 
\mathcal{ D } \beta d ^{ 2k } \xi \exp \left( i \left( 
\beta \mid P v + 2 \overline{ \omega } h + t ^ I \mu _ I \right)  + 
i \sum _{ i =  1} ^ k \xi ^ i _ a v ^ a \left( \tilde{ \sigma } _ i   \right) \right) 
\] where we define which is a
result over taking the continuum product 
over $ \sigma , \tau $ and absorbing this 
into the exponential 
\[
\left( \beta \mid P v + 2 \overline{ \omega  } h + t ^ I \mu _ I  \right) 
: = \int _{ \Sigma } d ^2 \sigma \sqrt{ - h }  \beta  ^{ ab } 
\left( \left( P v  \right)  _{ ab } + 2 \overline{ \omega  } h _{ab } + 
t ^ I \mu _{ I ab } \right) 
\] 

\subsubsection{An aside on Grassman 
Variables}

As an aside, we have $ \Delta ^{ - 1 }_{ FP } $, and we 
want $ \Delta _{ FP } $. 
We introduce Grassman variables. These are variables 
$ \theta _ i $ that anti-commute, 
so we have that $ \theta _ 1 \theta _ 2 =  - \theta _ 2 \theta _ 1 $. 
These obey lots of interesting properties, 
for example $ \theta _ i ^ 2  = 0 $. We 
also have the Berezin integration rules, where 
when we integrate over a constant, we get zero, 
and when we integrate over a sole $ \theta $ variable, we get $ 1 $. 
\[
\int d \theta 1  = 0, \quad \int d \theta \theta = 1 
\] 
We have that 
\[
f \left( x, \theta  \right)   = f\left( x  \right) + \theta f ' \left( x  \right) , 
\delta \left( \theta  \right)   = \theta , \quad 
\int d \theta f \left( \theta  \right)   = \frac{\partial  f }{\partial  \theta } 
\] where the differentiation is 
with respect to $ \theta $, and when 
we omit $ \theta $ we mean where $ \theta = 0 $. Let us consider a finite 
dimensional Gaussian integral. 
\[
\frac{1}{\det M }  = \int _ V dz d \overline{ z  } e ^{  - \left( \overline{ z }, 
M z  \right) }
\] Writing the same type of integral with Grassmann variables, 
we have that 
\[
\det M  = \int d \theta d \overline{ \theta } e ^{  - \left( 
\overline{ \theta } , M \theta \right) }
\] See Ryder's QFT book if curious 
This suggests that what we want to do 
is replace everything with Grassman valued fields, 
so that we invert this thing properly. 
We replace all of our fields we integrate over 
in $ \Delta ^{ - 1 } _{ FP } $ Grassman valued 
fields. 
We replace \[
v ^ a \left( \sigma, \tau  \right)  \to c ^ a \left( \sigma, \tau  \right), \quad  
\quad \beta \left( \sigma, \tau  \right)  \to b _{ab } \left( \sigma, \tau  \right) ,\quad 
t ^ I \to \zeta ^ I , \quad \xi ^ i _ a \to \eta ^ i _ a 
\] and we take $ \Delta _{ FP } \left[  \hat{ h }  \right]  $ 
to be given by 
\[
\Delta _{ FP } \left[ \hat{ h }   \right]  
 = \int d ^ s \zeta 
 \mathcal{ D }c \mathcal{ D}c \mathcal{ D } b d ^{ 2k } \eta 
 \exp \left(  i b \mid P c + \zeta ^ I \mu _ I + 
 i \sum _{ i = 1 } ^ \kappa \eta ^ i _ a  c ^ a \left( 
\tilde{ \sigma  } _ i  \right) \right)  
\] We can do the finite dimensional integralw. 
We have done the $ \overline{ \omega } $ integral which gives us 
a delta function which 
constrains $ \beta _{ ab } h ^{ ab }  = 0 $ everywhere. 
When we move to Grassmann valued fields, 
we can take $ b _{ ab } $ to be tracelss also. 
Due to the fact that we're
dealing with Grassman variables, the integral over 
$ \zeta ^ I $ gives us a delta functional, and we have that 
\[
\prod_{ I  = 1 }^ s \delta\left( b\mid \mu _ I  \right)  
= \prod_{ I  = 1 } ^ s \left( b \mid \mu _ I  \right) 
\] On the other hand, we have that the integral involving 
$ \eta _ a ^ i $ gives us 
\[
\prod_{ a = 1 , 2 , i = 1 \dots \kappa }  
\delta \left( c ^ a \left( \tilde{ \sigma } _ i   \right)   \right) 
= \prod _{ i , a } c ^ a \left( \tilde{ \sigma } _ i   \right) 
\] Thus, 
our final result for our determinant is 
\[
\Delta _{ FP } \left[  \hat{ h }  \right]  
= \int \mathcal{D } c \mathcal{ D } b 
e ^{ i S \left[  b, c  \right]  } \prod_{ I  = 1 } ^ s 
\left( b \mid \mu _ I  \right)  \prod_{ i , a } c ^ a 
\left( \tilde{ \sigma } _ i   \right) 
\] In this case, our action for the 
$ b, c  $ Grassmann fields is given by 
\[
S \left[  b, c  \right]  
= \left( b \mid P c  \right)  
= \int d ^ 2 \sigma \sqrt{  -h }  b ^{ ab } \left( P c  \right)  _{ ab } 
= 2 \int _{ \Sigma } d ^ 2 \sigma \sqrt{  - \hat{ h } }  b ^{ ab } 
\left( \nabla _ a c _ b  \right) 
\]

We shall refer to the 
(Grassmann) fields $ b _{ab } $ and $ c ^ a $ as 
Faddeev-Popov ghosts. 
From this, we can 
calculate observables 
from our path integral. 

\subsection{Calculating Observables}
We substitute our expression for 
the Faddeev-Popov determinant into our entire 
path integral. In the previous section, we 
were only working with a particular worldsheet 
with a genus $ g $. Now, we aim to sum
over all genera, and add an exponential 
factor to make sure this sum converges. 

We now piece together everything we did in the 
last section.
Our Faddeev-Popov determinant is given 
by 
\[
\Delta _{ FP }  = \int \mathcal{ D } b \mathcal{ D } c e ^{ i S \left[  b, c  \right]  } 
\prod_ I \left( b \mid \mu _I   \right)  \prod _ a c ^ a \left( 
\tilde{\sigma } ^ i  \right) 
\]  where our action here is defined as 
\[
S \left[  b, c  \right]   = 
\left( b \mid P c  \right)   = \int d ^ 2 \sigma 
\sqrt{ - \hat{ h }  }  b ^{ ab } \left( \nabla _ a c _b  \right)  
\] 


Our final form for $ \mathcal{ Z } \left[  0  \right]  $, 
summing over all possible Riemann surfaces, with some weight, is 
\[
\mathcal{ Z } \left[  0  \right]   = \sum _{ g = 0 } ^ \infty 
e ^{ \lambda \chi }
\frac{1}{|\text{CKG}|} \int _{ M _ g } \int d ^ s t \int \mathcal{ D } X \mathcal{ D } b 
\mathcal{ D } c \, e ^{  i S \left[  X, \hat{ h }, b, c  \right]  } 
\prod_{ I } \left( b \mid \mu _ I  \right)  \prod _{ i , a } c ^ a \left( 
\tilde{ \sigma }_ i   \right) 
\] Here, $ \lambda $ is a constant and $ \chi  = 2g - 2 $ is 
the Euler characteristic of the worldsheet. 
Now we want to think about computing 
observables. We can compute correlation functions 
of observables, by including them 
in our path integral. 
\[
\left< \phi _1 , \quad \phi _ n  \right> 
= \mathcal{ N }\int \mathcal{ D } \phi \, \phi _ 1 \dots \phi _ n e ^{ i S \left[  \phi  \right] }
\] We're summing over connected worldsheets, 
and oriented worldsheets. The orientedness condition is not 
necessarily needed though. 
What might observables look like? They need 
to be invariant under the symmetries. So, we
can build diff invariant observables by taking 
an operator $ \mathcal{ O } \left( \sigma, \tau  \right)  $ 
on the genus $ g $ worldsheet $ \Sigma _ g $ and 
integrating it over $ \Sigma _ g $. For instance, 
\[
\mathcal{ O }  = \int _{ \Sigma _ g } d ^ 2 \sigma \mathcal{ O } \left( \sigma, \tau  \right) 
\] We also need $ \mathcal{ O } $ to be Weyl-invariant, 
which means that when we change our metric 
on our worldsheet and integrate by a Weyl transformation, 
when we scale this metric by a Weyl transformation we get 
the same answer. 

Much like how we deal with expectation values 
of observables in quantum mechanics 
by inserting them into a 
probability distribution, observables 
are the sort of thing you could think about 
inserting into this path integral. 
So, a correlation function of such observables 
would be 
\[
\left< \mathcal{ O }_ 1  \dots 
\mathcal{ O } _ n   \right>  = 
\sum _{ g = 0 } ^ \infty \frac{e ^{ \lambda \chi } }{ |\text{CKG}|}
\int \prod_{ i  =1  }^ n d ^ 2 \sigma _ i \int _{ M _ g } 
d ^ s t \int \mathcal{ D } X \mathcal{ D } b 
\mathcal{ D } c e ^{  i S \left[  X, h  \right]   } 
\prod _ I \left( b \mid \mu _ I  \right)  
\prod_{ i , a } c ^ a \left( \tilde{ \sigma } _ i   \right)  
\mathcal{ O  }_ 1 \left( \sigma _ 1  \right)  \dots 
\mathcal{ O} _ n \left( \sigma _n  \right)  
\] Notice that 
\[
\frac{1}{|\text{CKG}|} \int _{M _ g } d ^ s t \int \prod_{ i = 1 } ^ n 
d ^ 2 \sigma _ i  = \int _{ \mathcal{ M } _ g } d ^ s t \int _{ i  = 1 } 
^{ n - \kappa } d ^ 2 \sigma _ I  = \int _{ M _ {g, n }}
\] This is because we're dividing 
out by the space with some points fixed.
So, we have 
\[
\left< \mathcal{ O } _ 1  \dots \mathcal{ O } _ n  \right> 
= \sum_{ g = 0 } ^ \infty e ^{ \lambda \chi } 
\int _{ M _{g, n } } \int \mathcal{ D } X \mathcal{ D } b \mathcal{ D } c 
e ^{ i S \left[  X, b, c  \right]  } \prod_{ I } \left( b \mid \mu _ I  \right)  
\prod _{ i ,a  } c ^ a \left( \tilde{ \sigma } _ i   \right)  \mathcal{ O } _ 1 
\dots \mathcal{ O } _ n 
\] 

\pagebreak 
\section{Conformal Field Theory}
CFT is a wonderful topic with 
important applications in condensed matter theory, 
string theory and more. If you really want to 
understand quantum field theory with mathematical rigour, 
the best place to start is with low dimensional 
CFT. 

\subsection{Conformal invariance in general dimension}
We're going to highlight 
how interesting things become when our dimension is 
2, the dimension of our worldsheet. In two dimensions, 
we explore transformations which preserve 
angles, which are called conformal transformations. 
This happens when our metric at some 
point is scaled by a particular function. 

\begin{defn}{Conformal transfomation}
A conformal transformation 
is a transformation which leaves the 
angles between any two given 
vectors on our space invariant. 
In other words, we require 
that the metric (which determines the inner product)
only be transformed up to a local scaling. 


More precisely, imagine some coordinate $ x ^ \mu \to x ^{ ' \mu } \left( x  \right)  $ 
such that the metric on our space 
transforms as 
\[
\eta _{ \rho \sigma  } \frac{\partial x ^{ ' \rho } }{\partial  x ^ \mu }  
\frac{\partial x ^{  ' \sigma } }{\partial  x ^ \nu }  =  \Lambda \left( x  \right) 
\eta _{ \mu \nu }, \Lambda \left( x  \right)  > 0,  
\] Infinitesimally, 
we can view this as \[
x ^ \mu \to x ^ \mu \left( x  \right)   = x ^ \mu 
+ \epsilon v ^ \mu \left( x  \right)  + \dots 
\] where $ \epsilon $ is a small number, and $ v ^ \mu $ is 
the what we call the generator of our transformation.
\end{defn}
Now, in a similar fashion to 
coming up with a condition that a vector lies in 
the conformal Killing group, we will now 
cook up a condition to determine 
whether the transformation associated with $ v^ \mu $  
is a conformal transformation. 
To first order in $ \epsilon $, we have that 
\[
\eta _{ \mu \nu } \to \eta _{ \mu \nu } + \epsilon \left( 
\partial  _ \mu v _ \nu + \partial  _ \nu v _ \mu \right)  
 = \Lambda \left(  x  \right) \eta _{ \mu \nu }
\] Now we parametrise 
our scaling parameter and let $ \Lambda \left(  x  \right)   = e ^{ \epsilon w \left( x  \right)  } 
 \simeq 1 + \epsilon w \left( x  \right)  $. 
 Substituting this on both sides, 
 we arrive at the equation 
 \[
	 \partial  _ \mu v _ \nu + \partial  _ \nu v _ \mu  = \omega \left( x  \right)  
	 \eta _{ \mu \nu }
 \] We now solve for $ \omega $, and the 
 obvious way to do this is to take the trace 
 of the equation. Taking the trace of this equation, we 
 require 
 \[
	 \omega \left( x  \right)  = \frac{2}{d } \left( \partial  _ \mu v ^ \mu  \right) 
 \] Vector fields that generate 
 such conformal transformations 
 satisfy 
 \[
  \partial  _ \mu v _ \nu + \partial  _ \nu v _ \mu 
  = \frac{2}{d } \left( \partial  _ \lambda v ^ \lambda   \right)  \eta _{ \mu \nu } 
 \] In two dimensions, 
 we have that $ d = 2 $, 
and something special happens. 
Let us take the coordinates to be $ \sigma, \tau $, and 
we have enough gauge freedom 
choose $ \eta $ to be the Euclidean 
metric $ \begin{pmatrix}  1 & 0 \\ 0 & 1  \end{pmatrix} $. 
With this metric, our condition for a vector field 
which generates conformal transformations becomes, 
by choosing either $ \left( \mu , \nu  \right)  = 
\left( \sigma, \tau  \right)   $  or $ \left( \mu , \nu  \right)  
 = \left( \tau , \tau  \right)  $, 
\[
 \frac{\partial  v_ \tau  }{\partial  \tau }   = \frac{\partial  v _ \sigma }{\partial  \sigma } , 
 \quad \frac{\partial  v _\sigma }{\partial  \tau }  =  - \frac{\partial  
 v _ \tau }{\partial  \sigma }  
\] This is amazing, since you 
may recognise these equations as the Cauchy Riemann equations.

Let's make these conditions more compact. 
Let's first write $ v  = v _ \tau + i v _ \sigma $. 
If we introduce 
complex coordinates 
\[
 w = \tau + i \sigma ,\quad  \overline{ w }   = \tau  - i \sigma , \quad 
 w \in \mathbb{ C } 
\] The condition for $ v $ to generate a conformal transformation 
is that $ v  $ is holomorphic, 
where 
\[
 \overline{ \partial   }v = 0  
\] So, we have a 
conserved charge here, and we will show a 
way to generate an infinite amount of conserved 
charges which govern a lot of the dynamics here. 
A particularly useful 
set of coordinates for our 
worldsheet is 
\[
  z  = e ^{ \tau + i \sigma }, \quad 
 \overline{ z }  = e ^{ \tau  - i \sigma }
\] Under this map, the cylindrical worldsheet is 
mapped to the complex plane, 
if we choose a Euclidean metric. 
Time evolution $ \left( \tau   \right)   $ 
on the cylinder becomes radial evolution 
on the complex plane.

\subsubsection{Connection to the Witt Algebra}
Let's return to the Witt algebra in the context 
of CFT. 
We saw last time that the conformal transformations 
are generated by holomorphic and anti-holomorphic 
vector fields, $ v \left( z  \right)  $, and $ \overline{ v } \left( 
\overline{ z } \right)  $. 
In other words, we 
have the map 
\[
  z \to z + v \left( z  \right)  + \dots, \quad \overline{ z } \to 
  \overline{ z } + \overline{ v } \left( \overline{ z }  \right)  + \dots
\]  We can write $ v \left( z  \right)  $ as a series expansion 
\[
 v \left( z   \right)   = \sum _ n v _ n z ^{ n + 1  } 
\] we might have some issues here if 
$ v \left( z  \right)  $ has a pole. 
We may find the $ n + 1 $ factor a weird convention to put, 
but we'll see why this is later on. 
Then, we have infinitesimally, our generator gives 
$ z \to z + \sum _ n v _ n z ^{ n + 1 } $. 
We can think of this infinitesimal transformation 
as being generated by the operators $ l _ n  =  -z ^{ n + 1 } 
\frac{\partial }{\partial z} $. 
In other words, we have that 
\[
 z \to z  - \sum _ n v _ n l_ n  z 
\] The negative sign 
here right now is a stylistic convention. We have the similar analog for the 
conjugate $ \overline{  z  } $. 
The $ l_ n  $  satisfy
\[
 \left[  l _ n , l _ m  \right]   = 
 \left( n - m  \right)  l _{ n + m } 
\] which is the Witt algebra. This can 
be checked directly by substituting in 
our expressions. The proof is straightforward.  
We're going to talk about these sorts of transformations 
and how things in our field theory 
transform with these types of transformations. 
The defining feature of the string theory is that 
stress tensor. 

\subsection{Conformal Fields}
We'll need a bit of Jargon first. 
Let's have some definitions. To begin with. 

\begin{defn}{Chiral Field.}
	A chiral field is a field $ \Phi \left(  z  \right)  $ that depends on 
	$ z $ alone. This is also called 
	a homomorphic field. 
\end{defn}
\begin{defn}{Conformal Dimension.}
	The conformal dimension is a number 
	$ \Delta  = h + \overline{ h } $  which 
	tells us how a field transforms under scaling. 
	Suppose we rescale $ \left( z, \overline{ z }  \right)  $ by some 
	number as $ \left( z, \overline{ z }  \right)  \to \left( 
	\lambda z ,  \overline{\lambda} \overline{ z } \right)  $. 
	If our field transforms as 
	\[
	 \Phi \left( z, \overline{ z }  \right)  \to 
	 \Phi  ' \left( z', \overline{ z }   '  \right)  = 
	 \lambda ^ h \overline{ \lambda } ^{ \overline{ h } } \Phi 
	 \left( \lambda z , \overline{ \lambda } \overline{  z  }  \right) 
 \] A chiral field has $ \overline{ h }  = 0  $
\end{defn}

\begin{defn}{Primary Field.}
	Under the transformation $ z \to f \left( z  \right)  $, 
	a primary field transforms as 
	\[
	 \Phi \left( z, \overline{ z }  \right)  \to 
	 \left( \frac{\partial  f }{\partial  h }   \right) ^{ h } 
	 \left( \frac{\partial  \overline{ f } }{\partial  \overline{ z } }   \right)  
	 ^{ \overline{ z }  } \Phi \left( f \left( z \right)  , 
	 \overline{ f } \left( \overline{ z }  \right)  \right) 
 \] Here we will find that 
 the theory factors into two sectors. But we say almost 
 factors because we still have this level matching condition 
 which link the anti-holomorphic and holomorphic parts. 
 When we study the chiral theory, we might have 
the exact same thing in the anti-chiral sector. 
However, there are theories in which the left and 
right moving parts behave differently, 
and this is called heterotic string theory. 	
\end{defn}

We can write the transformation functions 
$ f \left(  z  \right)  $ and $ \overline{ f } \left( \overline{ z }  \right)  $ 
infinitesimally. Note however that 
$ \overline{ f } $  isn't necessarily the complex 
conjugate, just a different function! 
For an infinitesimal transformation 
$ f \left( z  \right)    = z + v \left( z  \right)   + \dots $, 
and the corresponding anti-holomorphic function, 
we have that 
\[
 \left( \frac{\partial  f }{\partial  z }   \right)  ^{ h } 
 \simeq \left(  1 + \partial  v  \right)  ^{ h }  = 1 + h \partial  v , 
 \quad \left( \frac{\partial \overline{ f } }{\partial \overline{ z }  } \right)
 ^{ \overline{ h } }  \simeq 1
\] Our corresponding field changes as 
\[
	\phi \left( f \left(  z  \right)   \right)   = 
	\phi \left( z + v + \dots  \right)   = 
	\phi \left( z  \right)  + v \partial  \phi \left( z  \right)  + \dots 
\] and similarly, we find that for $\overline{z} \to \overline{ z  } 
+ \overline{ v } \left( \overline{ z   }  \right)  $, 
and we find that 
\[
	\delta _{ v, \overline{ v }  } \phi \left( z, \overline{ z }  \right)  
	 = \left( h \partial  v + \overline{ h } 
	 \overline{ \partial   } \overline{ v } + 
 v \partial  + \overline{ v }  \overline{\partial } \right) \phi \left( z, \overline{  z } \right) 
\] 

\subsection{Conformal transformations and the Stress Tensor}

We start with the action 
\[
 S \left[  X  \right]   =  - \frac{1}{4 \pi \alpha ' } \int_{ \Sigma } 
 d ^ 2 \sigma \partial  _ a X ^ \mu \partial  ^ a X ^ \nu \eta _{ \mu \nu } 
\] Under a conformal transformation, we have that 
$ \delta _ v X ^ \mu  = v ^ a \partial  _ a X ^ \mu $. 
Under such a transformation, we have that the action 
changes by 
\[
 \delta _ v S \left[  X  \right]   = 
 \frac{1}{2 \pi } \int _{ \Sigma  } d ^ 2 \sigma \left( 
 \partial  ^ a v ^ b \right)  T _{ ab }  =  - \frac{1}{2 \pi } 
 \int _{ \Sigma } d ^ 2 \sigma v ^ a \left( \partial  ^ b 
 T _{ ab } \right)   = 0 
\] The first step is shown 
quite straightforwardly by 
substituting in our form of $ T _{ ab } $ 
where $ h _{ ab }  = \eta _{ ab } $, and 
then doing some integration by parts, 
and you should find that the factors of 
$\frac{1}{2 } $ in our definition of $ T _{ ab } $ add up 
to the right amount.
Then, we use the symmetry 
of the stress tensor and then integrate by parts 
again to get the second identity.
This implies a conserved current, which satisfies 
\[
 \partial  ^ a T _{ ab } = 0  = \partial  ^ b T _{ ab } 
\] We could define conserved 
charges $ Q _{  \pm  } $  which in 
light cone coordinates $ \sigma ^{ \pm }  = \tau \pm \sigma $ look like 
\[
 Q _{ \pm }  = \frac{1}{2 \pi } \oint d \sigma v ^{ \pm } 
 \left( \sigma  \right)  T _{ \pm \pm } \left( \sigma  \right) 
\] Why are these charges conserved? I 
am not sure. Infinitesimal conformal transformations 
are given by the Poisson bracket of a field with $ \phi_{\pm }  $. 
We have that 
\[
 \delta _ v X ^ \mu  = \left\{  Q _ + + Q _  - , X ^ \mu  \right\}  
 _{ PB}
\] where our bracket here is the classical 
Poisson bracket. 
We want to understand how symmetries 
give rise to space time symmetries. 
We want to recast this into having conformal symmetries 
give rise to space time structure. 

\subsection{Complex Coordinates}

To simplify things, we'll change our coordinate system 
We shall use $ z  = e ^{ \tau  +i \sigma }  $, $ \overline{ z }  = 
e ^{ \tau  - i \sigma } $ and take $ h _{ ab } $
to be Euclidean on $ \Sigma $ in the 
$ \sigma, \tau $ coordinates. First, if we
 write down our Polyakov action 
 with $ h _{ab  } $ chosen to be Euclidean, 
 and then change coordinates such that 
 \[
  \partial  _ \tau  = z \partial  + \overline{ z } \overline{ \partial   }, 
  \quad \partial  _ \sigma  = i \left( z \partial  
   - \overline{ z} \overline{ \partial  } \right) 
 \] In these coordinates, 
the stress tensor becomes: 
\[
 T_{ z z} :  = T  = - \frac{1}{\alpha ' } \partial X ^ \mu X ^ \nu \eta _{ \mu \nu } , 
 \quad T _{ \overline{ z } \overline{ z } }  = \overline{ T }  = 
  - \frac{1}{\alpha ' } \overline{ \partial   } X ^ \mu \overline{ \partial  } 
  X ^ \nu \eta _{ \mu \nu } 
\] 
Recall that, 
regardless of coordinate choice, 
we have that the stress tensor 
is traceless, since $ h _{ ab } T ^{ 
ab }  = 0 $. In these new set 
of coordinates, we have that $ T _{ z \overline{ z } }  = 0 $ as it is the trace 
of the stress tensor. 
We now explore how 
our action an equations of motion 
change under these change of coordinates. 
First, we look at how our measure 
in our integral for the action changes. 
Here we have that 
\[
 d \tau d \sigma  = - \frac{d z d \overline{ z } }{ 2 i | z | ^ 2 }
\] The path integral becomes 
\[
 \int \mathcal{ D } X e ^{ i S \left[  X  \right]  } \to 
 \int \mathcal{ D } X e ^{  - S\left[  X  \right]  } 
\]  our action is 
\[
 S \left[  X  \right]   = - \frac{1}{2 \pi \alpha ' } 
 \int _{ \Sigma } d ^ 2 z \partial  X ^ \mu 
 \overline{ \partial  } X ^ \nu \eta _{ \mu \nu }
\] The equation of motion for the $ X ^ \mu $ is 
now $ \partial  \overline{ \partial } X ^ \mu  = 0  $. 
We have that $ X ^ \mu \left( z, \overline{ z }  \right)   = 
X _ L ^ \mu \left(  z   \right)  + X _ R ^{ \mu } \left( \overline{ z }  \right)  $
The conservation equation $ \partial  ^ a T _{ ab }  =0 $, 
becomes 
\[
	\overline{ \partial   } T \left( z  \right)   =0 , \quad 
	\partial  \overline{ T } \left( \overline{ z }  \right)   = 0 
\] 
\subsection{Ward Identities and Conformal Transformations}
The point of this section is 
to explore the quantum version of Noether's 
theorem. 
Previously, we saw that 
if we transform a field to give $ \delta \phi   = 
\epsilon   f \left(  \phi , \partial  \phi , \dots  \right)  $, 
and the action stays invariant 
then if $ \epsilon $ is a constant, we have a 
\textbf{classical} symmetry of the action.

However, our generator $ \epsilon ^ a $ might also 
depend on our location on the worldsheet. 
As we saw before, this means that our 
change in the action $ \delta S $ (which is zero by 
construction), is now 
\[
	\delta S  = \int_{ \Sigma } d ^ 2 \sigma \, \left( 
	\partial  ^ a v ^  b  \right)  T_{ ab } 
\] Using integration by parts 
and shifting the derivative to hit 
the stress tensor, we get that 
out of this, we have a conservation law 
\[
 \partial  ^a  T _{ ab }  =0 
\] 
Now we will focus on the quantum analogue of 
this principle. 
We will describe a general primary field 
as some field which is a 
function of $ z , \overline{ z }  $, sufficiently 
well-behaved. $ \phi \left( z, \overline{ z }  \right)  $. 
Later we shall choose $ \phi $ to be either $ X ^ \mu $ 
or one of the $ b , c $ ghost fields. 
The first thing we do will be 
to consider the change in a correlation function 
\[
 \left< \phi _ 1 \dots \phi _ n  \right>  = 
 \int \mathcal{ D } \phi e ^{  - S \left[  \phi  \right]  } \phi _ 1 \dots 
 \phi _ n 
\] resulting from an infinitesimal transformation $ \phi \to \phi + \delta \phi = \phi  '$. 
In particular we would like to see what happens when we change 
the field under a conformal transformation. 
We have 
\[
	\left< \phi  ' \left( z _ 1  \right)  \dots 
	\phi  ' \left( z _ n  \right)   \right>
	 = \int \mathcal{ D } \phi  ' e ^{  - S \left[  \phi  '  \right]  } 
	 \phi _ 1  '\left( z _ 1  \right)  \dots \phi  ' \left( z _ n  \right) 
\] We shall assume that 
$ \mathcal{ D } \phi  = \mathcal{ D } \phi  ' $. 
When we change the field, we 
have that 
\[
	\left< \phi  ' (z_1 ) \dots \phi  ' \left( z _ n  \right)  \right> 
	 =\int \mathcal{ D } \phi e ^{  - S \left[  \phi  \right]  
	 - \delta S \left[  \phi  \right]  } \left( 
 \phi \left( z_1  \right)  + \delta \phi \left(  z_1  \right)  \right)  \dots 
 \left( \phi \left( z_ n   \right)  + \delta \phi \left( z _ n  \right)  \right) 
\]  This is, to the first order in 
$ \delta \phi $, 
\[
	\dots = \left< \phi \left( z_1  \right)  \dots \phi \left( z_ n  \right)   \right> 
	 - \int \mathcal{ D } \phi e ^{  - S \left[  \phi  \right]   } 
	 \delta S \left[  \phi  \right]  \phi \left( z_1  \right)  
	 \dots \phi \left( z _ n  \right)  
	 + \int \mathcal{ D } \phi e ^{  - S \left[  \phi  \right]   } 
	 \sum _{ k = 1 } ^{ n } \phi \left( z _ 1  \right) \phi \left( z _{ k - 1 }  \right)  
	 \delta \phi  \left( z _ k  \right)  \dots \phi \left( z _ n  \right) 
\] to leading order. 
Now matter what kinds of 
fields we use, we expect 
that the physics of our theory to be invariant under the change 
$ \phi \to \phi  '  $. 
This means that we impose 
the condition that the correlation 
functions should be the same. 
We thus require
\[
  \left< \phi _ 1 ' \dots \phi _ n  '  \right> 
= \left< \phi _ 1 \dots \phi _ n  \right> 
\]
This means that we're 
left with the identity that 
the correlation function, including a change in 
the metric, is a sum of correlation 
functions with some insertions. 
\[
 \left< \delta S \left[  \phi  \right]  \phi 
 \left( x_1  \right)  \dots \phi \left( z _ n  \right)  \right>  = 
 \sum _{  k = 1 } ^ n \left< \phi \left( z_1  \right)  \dots \delta 
 \phi \left( z _ k  \right)  \dots \phi \left( z _ n  \right) \right> 
\] 
For a change $ \phi \to \phi + \delta \phi $, 
the action undergoes some $ S \left[  \phi  \right]  
\to S \left[  \phi  \right]  + \delta S \left[  \phi  \right]  $. 
We then found that 
our associated correlation function 
\[
 \left< \delta S \left[  \phi  \right]  
 \phi _ 1 \dots \phi _ n \right>  = \sum_{ k = 1 } ^ n 
 \left< \phi _ 1 \dots \delta \phi _ k \dots \phi _ n   \right> 
\] where we have that $ \phi _ i  = \phi \left( z_ i , 
\overline{ z } _ i \right)  $. 
Let's now take $ S $ to be the Polyakov 
action with fixed worldsheet metric, and 
$ \delta \phi  = v ^ a \partial  _ a \phi $. 
We can use the fact that the 
variation of the action 
under such a transformation 
gives 
\[
 \delta S  \left[ \phi  \right]  = \frac{1}{2 \pi } 
 \int_{ \Sigma } d ^ 2 \sigma \left( \partial  _ a v _ b  \right) T ^{ ab }  
\] where $ T ^{ ab } $ is the stress tensor. 
We have that $ v ^ a $ are just 
vector parameters which we can 
pull out of the integrand and the 
correlation function. 
From the equation above, we have 
\[
 \frac{1}{2 \pi }  
 \int_{ \Sigma } d ^ 2 \sigma \left( 
 \partial  ^ a v ^ b \right)  \left< T _{ ab }
 \phi _ 1 \dots \phi _ n \right>  = \sum _{ k =  1 } ^ n 
 \left< \phi _ 1 \dots \delta _  k \dots \phi _ n  \right> 
\] What does our worldsheet look like? 
In this case, in our $ \left( \sigma, \tau  \right)  $ 
coordinates, it looks like multiple cylinders 
coming in and then emerging out into different cylinders. 
However, we observed before that we can 
map cylinders in $ \left( \sigma, \tau  \right)  $ 
coordinates to circles in $ \left( z, \overline{ z}  \right) $ coordinates. 
Thus, our worldsheet looks like, in the $ \left( z, \overline{ z }   \right)  $ plane, 
as the complex plane except with discs centred on 
each insertion $ z _ i   $. We call these discs 
$ \mathcal{ D }_ i $, and we label the boundary 
$ c _ i $. 
(diagram here). 

we map each of these cylinders 
to the complex place using $ z  = e ^{ \tau + i \sigma } $,
and glue the regions together 
using holomorphic transition functions. 
If we take the boundaries 
$ c _ i $ to points 
(the local $ \tau $ coordinate goes to $ \tau \to  - \infty $ ), 
then these circles go to points.
We can associate the state on the boundary 
$ c _ i $ to a local 
operator $ \phi \left( z, \overline{  z} _ i  \right)  $. 
We'll choose some $ v $ which is 
completely general on the bulk of the worldsheet, 
and vanishes on all the boundaries except 1. 
We choose our parameter 
$ v ^ a $ to be 
\begin{itemize}
	\item zero on all $ c _ i $, as well  
		as in the disk $ \mathcal{ D } _ i$, 
		except $ c _{ \omega } $ which is 
		the boundary associated to 
		the operator at $ z _ i  = \omega $. 
		In other words, $ \delta _ v \phi _ i   = 0 $
		except for $ \delta_ v \phi \left( \omega, 
		\overline{ \omega } \right)  $. 
	\item We will choose $ v ^ a $ to be 
		of the form $ v ^ a  = 
		\left( v \left( z  \right)  , \overline{ v } 
		\left( \overline{ z }  \right)  \right)  $. 
	\item We could choose multiple fields to vary, 
		but for simplicity we'll choose 
		just one field to vary right now. 
	\item  $ v ^ a $ is arbitrary otherwise in $ \Sigma $ 
		otherwise. 
\end{itemize}
To summarise, we have that 
for $ i \neq k$, our value of the field $ v ^ a $ satisfies 
 \[
    v ^  a \left( z, \overline{ z }  \right)  \mid _{ C _ i} =0 
\] On the other hand, we 
on the boundary of the circle 
which contains $ z _ k  = \omega $, 
we have that our generator field $ v ^ a $ is arbitrary, 
and is defined to be split up 
into the holomorphic and anti-holomorphic parts 
\[
	v ^{ z } \left( z, \overline{ z }  \right)   = v \left(  z  \right) , 
	\quad v ^{ \overline{ z } } \left( z, \overline{ z }  \right)   = 
	\overline{ v } \left( \overline{ z }  \right)  
\] By construction, we take the boundary of 
our worldsheet to be these discs, specifically so that 
\[
 \partial  \Sigma  = C _{ \omega } \cup _{ i \neq k  } C _ i 
\] and we have that the value of $ v ^ a $ 
is arbitrary everywhere else on the worldsheet. 

Thus, substituting this in 
the only value of $ \delta \phi _ k $ which 
is non-zero on this world sheet is when 
$ z _ k  = \omega $, for 
a particular choice of disc indexed 
with $ k$ . We denote 
$ \delta \phi _ k  = \delta \phi \left( \omega, \overline{ \omega }  \right)  $. 
This means that, for this choice of $ v ^ a $, 
we have 
\[
 \frac{1}{2 \pi } \int _{ \Sigma } d ^ 2 \sigma 
 \left( \partial  _ a v _ b  \right) \langle T ^{ ab }
 \phi _ 1 \dots \phi _ n \rangle = 
 \left< \phi _ 1 \dots \delta \phi \left( \omega , 
 \overline{ \omega } \right)  \dots \phi _ n  \right>
\] 
The left hand side expression is, 
using integration by parts 
to take out a boundary term, is 
\[
 \frac{1}{2 \pi } \int d ^ 2 \sigma 
 \partial  ^ a \left(  v ^ b \left< 
 T _{ ab } \left( \sigma  \right)  \phi _  1 \dots \phi _ n \right>   \right)  
   - \frac{1}{2 \pi } \int_{ \Sigma } d ^ 2 \sigma 
   v ^ b \partial  ^ a \left< 
   T _{ ab } \left( \sigma  \right)  \phi _ 1 \dots \phi _ n \right> 
\] To evaluate the first 
term in the expression above, we 
appeal to Stokes' theorem 
on the complex plane. In differential 
geometry, Stokes theorem 
is the statement that, in two dimensions
\[
 \int_{ M}  dw = \int _{ \partial  M } w , \ d w  = 
 \left( \frac{\partial w _ 2 }{\partial  x ^ 1  }  - \frac{\partial w_ 1 }{\partial  
 x ^  2 }   \right)  dx ^ 1 \wedge  dx ^ 2, 
 \quad w = w_1 dx ^ 1 + w_2 dx ^ 2 
\] We set, in the above, 
that $ x^ 2  = z , x^ 2 = \overline{ z } , w _ 1  = - j ^  z, w_2  = j ^{ \overline{ z } }$. 
The first term 
is a boundary term $ \partial  \Sigma  = 
\cup _ i C _ i $. 
The only boundary 
contribution comes from 
$ c _{ \omega } $ where $ v ^ a $ has homomorphic 
$ v \left(  z  \right)  $ and anti-holomorphic 
$ \overline{ v } \left( \overline{ z }  \right)  $ components
and is 
\[
	\frac{1}{2 \pi i } \oint _{ c _ \omega } dz v \left(  z  \right) 
	\left< T \left(  z  \right)  \phi _ 1 \dots \phi _ n  \right> 
	 - \frac{1}{2 \pi i } \oint _{ c _ \omega } 
	 d \overline{ z } \overline{ v } \left( \overline{ z } \right)  \langle \overline{ T } \left( \overline{ z }  \right)  
	 \phi _ 1 \dots \phi _ n \rangle 
 \] where $ T _{ zz } \left( z  \right)   =  T\left(  z  \right)   $, 
 Thus, we have 
 that 
 \begin{align*}
 	\left< \phi _ 1 \dots 
	\dots \delta _ v \phi \left( \overline{ \omega } , \omega  \right)
	\dots \phi _ n \right> &=   -\frac{1}{2 \pi } 
	\int _{ \Sigma } d ^ 2 \sigma v ^ b 
	\left( \partial  ^ a \left< T _{ ab } \phi _ 1 \dots 
	\phi _ n \right>  \right)  \\
			       & + \frac{1}{2 \pi i } \oint_{ c _{ \omega } } 
			       d zv \left(  z  \right)  \left< T ( z) 
			       \phi _ 1 \dots \phi _ n \right>
			        - \frac{1}{ 2 \pi i } \oint _{ c _ \omega  } 
				d \overline{ z  } \overline{ v } 
				\left( \overline{ z }  \right)  \left< 
				\overline{ T } \left(  \overline{ z }  \right)  
				\phi _ 1 \dots \phi _ n \right> 
 \end{align*}
 Since this is valid for arbitrary $ v $ 
 we have that this is 
 \[
	 \partial  ^ a \left< T _{ ab } \left( \sigma  \right), 
	 \phi _ 1 \dots \phi _ n \right>  =0 
 \] This is the analogue of the classical 
 Nether statement that $ \partial  ^ a T _{ ab }  =0  $. 
 This then means that 
 \[
  	\left< \phi _ 1 \dots 
	\dots \delta _ v \phi \left( \overline{ \omega } , \omega  \right)
	\dots \phi _ n \right>  = \frac{1}{2 \pi i } \oint_{ c _{ \omega } } 
			        - \frac{1}{ 2 \pi i } \oint _{ c _ \omega  } 
				d \overline{ z  } \overline{ v } 
				\left( \overline{ z }  \right)  \left< 
				\overline{ T } \left(  \overline{ z }  \right)  
				\phi _ 1 \dots \phi _ n \right> 
 \] 
 Note that $ \phi _ i \neq \phi \left(  \omega, \overline{ \omega }  \right)  $ 
 didn't contribute to the 
 calculation and we conclude that 
 \[
	 \delta _{ v } \phi \left( \omega, 
	 \overline{ \omega } \right)   = 
	 \oint_{ c _ \omega } \frac{d z }{ 2 \pi i } v \left( z  \right)  
	 T \left(  z  \right)  \phi \left( \omega, \overline{ \omega }  \right)   - 
	 \oint _{ c _ \omega } \frac{ d \overline{  z } }{ 2 \pi i } 
	 \overline{ v } \left(  \overline{ z }  \right)  
	 \overline{ T } \left(  \overline{ z }  \right)  \phi 
	 \left( \omega, \overline{ \omega }  \right) 
 \] this is understood to hold up on insertion 
 into a correlation function. 

 \subsection{Radial Ordering}
 Here, we're 
 working with the Euclidean metric, 
 so we don't have an a-priori notion of 
 time. There is  a
 notion that radial separation plays 
 an analogous role of time. 
We see that time ordering 
$ \tau _ 1 > \tau _ 2 $ translates into radial 
ordering $ | z _ 1 | > | z _ 2 |  $. We introduce the
notion of radial ordering. 
The radial ordering of 
two operators 
is 
\[
	\mathcal{ R } \left( A \left(  z \right)  B \left(  
	w\right)  \right)
	 = \begin{cases}
		 A \left(  z  \right)  B ( w) &  \text{ if  } | z | > | w | \\
		 B \left(  w  \right)  A \left(   z \right)  & 
		 \text{ if  } | z | < | w| 
	 \end{cases}
\] Consider 
the expression $ \oint_{ C \left( w \right) } 
dz \mathcal{ R } \left( A \left( z  \right)  B \left(  w  \right)   \right)  $, 
where $ C \left( w  \right)  $ is the contour 
around $ z  =w $ . 
How do we make sense of this? As 
shown in the diagram, on the circle around 
$ w  $, some parts of the contour 
have $ | z |  $ larger than $ | w | $, 
and smaller on other parts. We can 
make sense of this contour 
by looking at it 
as the difference of two contours, 
which both have a well defined 
ordering.
\begin{figure}[htpb]
	\centering
	\input{ordering.pdf_tex}
	\caption{}%
	\label{fig:}
\end{figure}
On $ C _ 1 $, we have $  | z | > | w |  $, 
and otherwise on $ C_ 2 $. 
This means that on $ C _ 1 $, we have 
that 
\[
	\oint_{ C _ 1 } dz \mathcal{ R } \left( 
	A \left(  z \right)  B \left(  w  \right)  \right)  
	= \oint dz A \left(  z \right)  B \left(  w  \right) 
\] On $ C _ 2 $, we have that 
\[
	\oint_{ C _  2} dz \mathcal{ R } \left( 
	A \left( z  \right)  B \left(  w  \right)  \right)  
	= \oint _{ C _ 2 } dz B \left(  w  \right)  A \left(  z  \right) 
\] Thus, we have that, on the 
specific contour $ C _ w $ around $ w $, 
we have decomposed this into a commutator of sorts. 
\[
	\oint_{ 
		C_ w } dz \mathcal{ R  } \left( A \left(  z  \right) B 
	\left(  w  \right)  \right)   = 
	\oint_{ C _ 1 } dz A \left( z  \right)  B \left(  w  \right)  
	- \oint_{ C _  2 } B \left(  w  \right)  A 
	\left(  z  \right) 
\] In particular, 
consider 
\[
	O  = \oint dz A \left(  z  \right) \dots , \dots B \left(  w  \right) 
\] Consider 
the integral which we define as 
the radial ordering part. 
\[
	\oint_{ C \left(  w  \right)  } 
	dz \mathcal{ R } \left( A \left(   z \right)  , 
	B \left(  w  \right)   \right)  : = 
	\left[  O , B \left(  w  \right)   \right] 
\] This plays the role of a 
commutator in our radially ordered theory. 
If we consider the variation of a 
chiral primary field 
$ \phi \left(  w  \right)  $, 
we get a contribution 
from our stress energy tensor $ T_{ zz } \left(   z  \right)   $ 
(since $ \phi $ is chiral we have zero contribution from 
the anti-holomorphic component of our stress tensor $ \overline{T }  $ ). 
This can be encoded into a 
charge $ Q$, and so the $ \delta $ function 
can be written as such. 
\begin{align*}
	 \delta _ v \phi \left(  w  \right)  
	 &=  \oint _{ C \left(  w  \right)  } 
	 \frac{ d z }{ 2 \pi i } \mathcal{ R } \left( v 
	 \left(  z  \right)  T \left(  z  \right)  \phi \left(  w  \right)  \right)  \\
	 &=  \oint_{ | z |  >  | w |      }
	 \frac{d z }{ 2 \pi i  } v \left(  z  \right)  T \left( z   \right)  
	 \phi \left(  w  \right)   - \oint_{ 
	  | z | \leq | w |  } \frac{ d z }{ 2 \pi i } 
	  \phi \left( w  \right)  v \left(  z  \right)  T \left(  z  \right)  \\
	  &=  \left[  Q , \phi \left(  w  \right)  \right] 
\end{align*}
where the charge $ Q $ is given 
by 
\[
	Q = \int_{ C \left(  w  \right)  } 
	\frac{ d z }{ 2 \pi i } v \left(  z  \right)  T \left(  z  \right) 
\] This is similar to 
the classical poisson bracket expression 
for the transformation of a field $ \phi \left(  w  \right)  $. 
We're extracting 
some information about the pole structure here, 
and we want to explore this. 

\subsection{Mode expansions and Conformal Weights}
We previously considered the 
transformation from a cylinder to 
a flat plane. We want to see how the 
mode expansions change from the 
cylindrical worldsheet to the plane. 
On the cylinder the mode 
expansion for a field $ \phi \left( \sigma, \tau  \right)  $
which only depends on $ w  = \tau + i \sigma $, 
and therefore only depends on $ z  = e ^{ w } $ (and
is hence a chiral field)
might be naturally written as 
\[
	\phi _{\text{cyl}} \left( \omega  \right)  
	  = \sum_ n \phi _ n e ^{  - n \omega } 
  \] and similarly for a field with $ \overline{ \omega } $ 
  dependence. This would be the left-moving part. 
  Suppose $ \phi _{ \text{cyl}}$ has 
  conformal weight $ h $, in other words 
  $ \phi _{\text{ cyl} } $ is a chiral primary.
  The definition of a chiral primary field 
  is that it transforms like 
\[
	\phi \left( w  \right)  \to \phi ' \left( w  \right)  
	 = \left( \frac{\partial  z }{\partial  w }   \right)^{ h } 
	 \phi \left( z \left(  w  \right)   \right)  
\] Now, if we switch around 
our $ z $ and $ w $ variables, we get that 
\[
	\phi' \left(  z  \right)  
	= \left( \frac{\partial w }{\partial  z  }   \right)^{ 
	h  } \phi \left(  w  \right)  
\] 
  If we now transform to the complex plane $ 
   z = e ^{ w }  = e ^{ \tau + i \sigma  } $, 
   we have that 
   \[
	   \phi _{ \text{cyl } } \left( w  \right)  \to \phi
	   _{\text{plane}}\left( z   \right) 
	    = \left( \frac{\partial  z }{\partial  w }   \right)  ^{ - h } 
	    \phi _{\text{cyl} } \left( w  \right)   = 
	    z ^{ - h } \phi _{\text{cyl}} \left( w  \right) 
   \] This is what 
   our field on our cylinder looks like 
   mapped to our complex plane. 
   To save ink, we'll instead 
   write $ \phi _{\text{plane}}\left(  z  \right)  =\phi\left( z \right)  $. 
   We have that 
   \[
    \phi \left(  z  \right)   = \sum _ n \phi _ n z ^{ -  n - h } 
   \] Previously, the natural 
   thing to do was to do a Fourier 
   expansion. Now, we have 
   that on the complex plane with Euclidean worldsheet 
   that this is the natural form 
   of mode expansion, for chiral primary fields. 
   More generally, for 
   a primary of weight $ \left( h , \overline{ h }  \right)   $, 
   we have that 
   \[
	   \phi \left( z, \overline{ z }  \right)  
	    =\sum _{ n , m } \phi _{ n , m } z^{ - n - h } \overline{ z } 
	    ^{ - m - \overline{ h } }
   \] For example, as 
   we shall see, the stress tensor has weight 
   $ \left( h , \overline{ h }  \right)   = \left( 2, 0  \right)  $ 
   for $ T \left(  z  \right)  $ and $ \left(  0 , 2  \right)  $ 
   for $ \overline{ T } \left( \overline{ z }  \right)  $. 
   We're shifting how 
   we count the modes here. 
   \[
	   T \left( z  \right)   = \sum _ n 
	   L _ n z ^{ -n - 2 } , \quad 
	   \overline{ T } \left(  \overline{ z }  \right)  
	    = \sum _ n \overline{ L } _ n \overline{ z } 
	    ^{ -n  - 2  }
   \]  For reference 
   \[
    X ^ \mu \left( z, \overline{ z }  \right)  
     = x ^ \mu  - i \frac{ \alpha ' }{ 2 } p ^ \mu \ln | z | ^ 2 
     + i \sqrt{ \frac{\alpha '  }{ 2 } }  \sum _{ n \neq 0 }
     \frac{1}{n } \left( \alpha _ n ^ \mu z ^{ - n }  - \overline{ \alpha } _ n 
     ^{ \mu } \overline{ z } ^{ - n } \right) 
   \] and we have that 
   our holomorphic 
   derivative gives 
   \[
	   \partial X ^ \mu \left(  z  \right)  
	    = - i \sqrt{ \frac{ \alpha ' }{ 2  } }  \sum _ n 
	    \alpha _ n ^ \mu z ^{ - n - 1 } , \quad \alpha _ 0 ^ \mu 
	     = \sqrt{ \frac{ \alpha  ' }{  2 } }  p ^ \mu 
     \]  in other words, it looks like 
     $ \partial   X^ \mu  $ looks to be a conformal 
     field with weight $ \left( 1, 0  \right)  $. 
     In some sense, we're working backwards here.

\subsection{Radial Ordering and Normal Ordering}
To a large extent 
we can go through the same route in 
QFT to relate radial ordering to normal ordering. 
Consider the weight $ \left( 1, 0  \right)  $ chiral 
field $ j ^ \mu \left(  z  \right)  = \sum _ n 
\alpha _ n ^ \mu z ^{ - n - 1 } $. 
We shall think of $ j ^ \mu \left(  z  \right)  $ 
as 
\[
 j ^ \mu \left( z  \right)   = i \sqrt{\frac{\alpha  }{  2  } } \partial  
 X ^ \mu \left(  z  \right) 
\] 
Since $ \alpha  n ^ \mu \ket{ 0 } $ for $ n \geq 0 $ , 
we can split $ j ^ \mu \left( z   \right)  $ into 
creation and annihilation parts. 
\[
	j ^ \mu \left(  z \right)   = j _ + ^ \mu (z) + 
	j _ - ^ \mu \left(  z  \right) 
\] where we have $ j _ + ^ \mu \left(  z  \right)  
 = \sum _{ n \geq 0 } \alpha _ n ^ \mu z ^{ - n - 1 } $. 
 We define normal ordering 
 $ :  : $ in the usual way, as moving 
 all creation operators to the left, 
 in any string of operators. In this case, 
 in particular, that means 
 \[
	 : j ^ \mu \left(  z  \right)  j ^ \nu \left(  w  \right)  : 
	  = j _ + ^ \mu \left( z  \right)  j _ + ^ \nu 
	  \left(  w  \right)  + j _ - ^{ \mu  } \left(  z  \right)  
	  j _ + ^{ \nu } \left(  w  \right)  
	  + j _ - ^ \nu \left(  w  \right)  j _ + ^ \mu \left(  z  \right)  
	  + j _ - ^ \mu \left(  z  \right)  j _ - ^ \nu \left(  w  \right) 
 \] Note that 
 \[
	 : j ^ \mu \left(  z  \right)  j ^ \nu \left(  w  \right)  : 
	 = j ^ \mu \left( z  \right)  j ^ \nu \left(  w  \right)  
	 + \left[  j _ - ^ \nu \left(  w  \right)  , 
	 j _ + ^{ \mu } \left(  z  \right)  \right]  
 \] Recall from the point pf Wick's theorem 
 that we could think of this commutator as a 
 contraction. 
 There are many ways we could evaluate this 
 commutator. We can 
 use the mode expansion to evaluate 
 the commutator. So, 
 \[
	 \left[  j _ - ^ \mu \left(  w  \right)  , 
	 J _ +^\nu  \left(  z   \right)  \right] 
	  = \sum_{ m \ge  0 , n \ge 0 } \left[  
	  \alpha _{ - m  } ^ \mu , \alpha _{ n  }^{ \nu } \right]  w 
	  ^{ m - 1  } z ^{ - n -  1} = - n \sum 
	  _{ m , n  } n \delta _{ m , n } \eta ^{ \mu \nu } w ^{ m - 1 } 
	  z ^{ - n -  1 }  
 \] This gives us 
 \[
  \dots =  - \frac{ \eta ^{ \mu \nu  } }{ z ^  2 } \sum _{ n > 0  }
  n \left( \frac{ w }{ z }  \right)  ^{ n - 1 } = \frac{\eta ^{ \mu \nu  } }{ 
  \left(  z -  w  \right)^ 2  }
 \] This will converge if $  | z | > | w |  $. 
 This tells us that 
 the relationship 
 \[
	 : j ^ \mu \left(  z  \right)  j ^ \nu \left(  w  \right)  : 
	 = j ^ \mu \left(  z  \right)  j ^ \nu \left(  w  \right)  
	 + \frac{ \eta ^{ \mu \nu  } }{ \left(  z - w  \right)  ^  2 } 
 \] where $ | z |  > | w |  $. 
 We see that this relationship, 
 the different between the normal product and 
 the generic product is something that is 
 singular. This is useful 
 when evaluating contour integrals. 

 Recall that $ j \left(  z  \right)  $ is a chiral 
 primary, with conformal weight $\left( h , \overline{ h }  \right)   = 
 \left( 1, 0  \right)  $. 
 We also said that 
 for $ | z | > |w | $, we have that 
  \[
	  j ^ \mu \left(  z  \right)  j ^ \nu \left(  w  \right)  
	  = : j ^ \mu \left(  z  \right) j ^ \nu \left(  w   \right)  : 
	  + \frac{\eta ^{ \mu \nu  }}{ \left(  z - w  \right)  ^{ 2 } } 
 \] Similarly, we have 
 that the converse statement 
 reads for $ | w | > | z| $, 
  \[
	  j ^ \nu  \left(  w  \right)  j ^ \mu \left( z  \right)  
	  = : j ^ \mu \left(  z  \right)  j ^ \nu \left(  w  \right)  : 
	  + \frac{\eta ^{ \mu \nu } }{ \left(  z - w  \right)  ^ 2 } 
 \] This means 
 we have an expression for the 
 radial ordered product. 
 In the spirit for Wick's theorem, we 
 write the radial ordering 
 of the $ j $ operators 
 as the sum of a normal ordered part plus a contraction part. 
 We have 
 \[
	 \mathcal{ R } \left( j ^ \mu \left(  
	 z \right)  j ^ \nu\left( w  \right)   \right) 
	 = : j ^ \mu \left( z  \right)  j ^ \nu \left( w  \right)  : 
	 + \wick{\c j ^\mu \left(  z  \right)  \c 
	 j ^ \nu \left( w  \right)  }
 \] Where, we have 
 that in this case, $ \wick{\c j ^ \mu \left(  z  \right)  
 \c j ^ \nu \left(  w  \right)  }$.  
 This is referred to as a contraction. Recall from 
 earlier that we defined the derivative of the 
 mode expansion as $ \partial  X ^ \mu \left( z  \right)  
 = - i \sqrt{\frac{\alpha'  }{ 2 }} j ^ \mu \left(  z  \right)    $. 
 This means that the 
 corresponding contraction 
 between these derivative fields is 
 \[
	 \wick{ \c \partial  _ z X ^ \mu 
	 \left(  z  \right)  \c \partial  _ w X ^ \nu 
 \left(  w  \right)  } = \frac{- \alpha'  }{ 2 } \frac{\eta ^{ \mu \nu } }{ 
 \left( z - w  \right)  ^ 2  } 
 \] We get a contraction for 
 free between the fields $ X ^ \mu \left( z  \right)  $ 
 and $ X ^ \nu \left(  w  \right)  $ when 
 we integrate this object through. 
 When we integrate with respect to $  w $ then $ z $, 
 we then get 
 \[
	 X ^ \mu \left(  z  \right)  X ^ \nu \left( w  \right)   = 
	 - \frac{\alpha '   }{ 2 } \eta ^{ \mu \nu } 
	 \log \left(  z-  w  \right) 
 \] This is 
 generalised in Wick's theorem
 where we take all possible contractions 
 of the fields in normal 
 ordering. 
 \begin{align*}
	 \mathcal{ R } \left( \phi _ 1\left( z _ 1  \right)  \dots 
	 \phi _ n \left( z _ n  \right)  \right)  &=  : \phi _1  
	 \left(  z_ 1  \right)  \dots \phi _ n \left( z _ n  \right)  : 
	 \\
						  &+ \sum_{ i , j } 
						  : \phi _ 1 \left( z _ 1  \right)  \dots 
						  \wick{\c \phi _ i \left( z _ i  \right)  
						  \dots \c \phi _ j \left( z _ j  \right)  }
						  \dots \phi _ n \left(  z _ n   \right) \\
						  + \dots 
 \end{align*}

 \subsection{Operator Product Expansions}
 We construct something called an 
 operator expansion.
 This is a way to characterise the 
 behaviour of a set of local 
 operators at short distances. 
 In particular, 
 we are interested in the limit 
 \[
	 \lim_{ z \to w } O _ i \left( w  \right)  
	 O _ j  \left( z \right)  \sum_ k f_{ij } ^ k \left( z - w  \right)  O _ k \left( w  \right)  
 \] For example, we showed 
 just now that 
 \[
	 \mathcal{ R } \left( \partial  X ^ \mu \left(  z  \right)  X ^ \nu \left(  w  \right)   \right)  
	 = : \partial X ^ \mu \left(  z  \right)  \partial X ^ \nu \left(  w  \right)  : 
	 - \frac{\alpha' }{  2  } \frac{\eta ^{ \mu \nu } }{ \left(  z -w  \right)  ^ 2 } 
 \] This term 
 is regular as $ z \to w $. 
 We write that the average 
 expected value, since our normal ordering term vanishes
 \[
	 \left< \mathcal{ R } \left(\partial  X^ \mu 
		 \left(  z  \right)  \partial  X ^ \nu \left( w  \right)  \right>
		 =  - \frac{\alpha'  }{ 2 } \frac{\eta ^{ \mu \nu } }{ \left( 
		 z - w \right)  ^ 2 } 
 \]  From now on, we weill 
 implicitly assume radial ordering everywhere unless otherwise stated. 
We could compute the correlation 
function, which as in quantum field theory is just the
correlation function between 
\begin{align*}
	\left< X ^ \mu \left(  z  \right)  X ^ \nu \left(  w  \right)   \right> 
	&= \left< : X ^ \mu \left( z  \right)  X^ \nu \left(  w  \right)  :\right> 
	- \frac{\alpha'}{2}  \eta ^{\mu \nu } \log \left(  z - w  \right)  \\
	&=   - \frac{\alpha'  2 }{ 2 } \eta ^{ \mu \nu } \log \left( z - w   \right) 
 \end{align*}
 Recall from QFT last term that the two point 
 propagator is 
 precisely the Green's function for 
 our Klein Gordon equation. Thus, we have 
 that $ - \frac{\alpha'  }{ 2 } \eta ^{ \mu \nu } \log \left( z - w  \right)   $ 
 is the solution for the Green's function for $ \frac{1}{ 2 \pi \alpha'  } \partial  
 \overline{ \partial    } $. Similarly, we 
 have our OPE expansions for $ \overline{  X }^ \mu  $ 
 and their respective anti-holomorphic variables. 
 This means that 
 \[
	 \overline{  X} ^ \mu \left( \overline{ z }   \right)  
	 \overline{  X} ^ \nu \left( \overline{ w }  \right)  = - \frac{\alpha' }{  2 } 
	 \eta ^{ \mu \nu } \log \left( \overline{ z } - \overline{ w }  \right) 
 \]  So, we still have a singular 
 part when we have $ z $ approach $ w $. 
 On the other hand, when we take 
 the radial ordering $ \overline{ X } ^ \mu X ^ \nu $, we 
 have that this object is regular. We can use the knowledge 
 of the  $ X ^ \mu $ OPE to now define operators like the stress tensor. 
 In the classical treatment of our theory, we 
 had that our stress tensor 
 in terms of holomorphic variables is 
 given by 
 \[
	 T \left(  z  \right)   = -\frac{1}{\alpha'}\partial  X ^ \mu \left(  z  \right)  
	 \partial  X _ \mu \left(  z  \right) 
 \] But, this doesn't apply when 
 we are working with conformal field theory. 
 In this case, we need to 
 define the normal ordered part 
 as the physical part of 
 what we require. 
 This means we take the sum of our radial ordered part 
 as well as our contraction. 
 We can take a look at why we 
 need to do the $ T \left( z  \right)   X ^ \mu \left( z  \right)  $ 
 OPE, for the purpose of finding out how things transform 
 conformally. 
 and take the sum. 
 Since by definition, we have that 
 the stress tensor is $ T (z ) $ physical, and therefore normal 
 ordered, 
 we have the product 
 \[
	 T \left(  z  \right) X ^ \mu \left( w  \right)   = 
	 - \frac{1}{\alpha' } : \partial  X ^ \nu \left(  z  \right)  \partial  X _ \nu \left(  z  \right) : 
	 X ^\mu \left(  w  \right) 
 \] 
 We make use of the contraction expression 
 for $ \partial  ^ \mu \left( z   \right)  X ^ \nu \left(  w   \right)  = - \frac{\alpha' }{ 2 } \frac{\eta 
 ^{ \mu \nu } }{ z -  w }  $. 
Observing that the contraction 
of a normal ordered product has a zero 
contraction, we get that 
when we apply Wick's theorem, 
\begin{align*}
	T \left(  z  \right)  X ^ \mu \left(  w  \right)  
	&=  - \frac{2}{\alpha' } : \partial  X^ \nu \wick{
	\c \partial  X _ \nu \left( z \right) : \c X^ \mu \left(  w  \right)  } + \dots \\
	&=  - \frac{2}{ \alpha'  } \partial  X _ \nu \left(  z  \right)  
	\left( - \frac{\alpha '}{ 2} \frac{\eta ^{ \mu \nu } }{ z - w }   \right) + \dots \\
	&=  \frac{\partial  X^ \mu \left( z  \right)  }{z- w } + \dots
\end{align*} 
We could now just Taylor expand $ \partial  X^ \mu \left(  z  \right)  $ 
about $ z = w $, which gives us 
\[
	\partial X^ \mu \left( z  \right)   = \partial  X ^ \mu \left( w  \right)  
	+ \left( z - w  \right)  \partial  ^ 2 X ^ \mu \left( w  \right)  + \dots 
\] The second part of this equation is regular so we 
don't really care about this. Substituting this 
into the above, we find that our expression for 
the stress tensor combined with our 
position field is 
\[
	T \left( z  \right)  X^ \mu \left(  w  \right)  
	 = \frac{\partial  X ^ \mu  }{ z - w } 
\] We can now compute the conformal transformation 
associated with the generating vector $ v  $, of 
$ X ^ \mu \left( w  \right)  $. We expect that 
since this is a field, that we have $ X^ \mu \left(  w  \right)  $
not to transform, so that $ h = 0$. Let's confirm our 
suspicions here. Our conformal 
transformation of the field is 
\begin{align*}
	\delta _ v X ^ \mu \left( w  \right)  &=  
	\oint_{ z = w } \frac{dz }{ 2 \pi i } v \left( z  \right)  T \left(  z  \right)  X ^ \mu 
	\left(  w  \right)  \\ 
					      &=  \oint_{ z = w } \frac{dz }{ 2 \pi i } v \left(  z \right)  \frac{
					      \partial  X ^ \mu \left( w   \right)  }{ \left(  z- w  \right)  }  \\
					      &=  v \left(  w  \right)  \partial  X ^ \mu \left(  w  \right)  
\end{align*} One can check 
that this gives the correct conformal transformation 
for $ X ^ \mu \left( z  \right)  $. Recall that 
a general operator transforms under conformal transformations $ f \left(  z  \right)  
 = z + v \left( z  \right)  $ exactly and infinitesimally as follows: 
 \[
	 \phi \left(  z  \right) = \left( \frac{df }{ dz }  \right)^{ h } \phi \left( 
	 f \left(  z  \right)  \right) , \quad 
	 \delta _ v \phi \left(  z  \right)   = 
	 v \partial  \phi + h ( \partial  v  )  \phi 
 \] Comparing this to the expression of 
 $ \delta _ v \phi \left( v  \right)  $, this suggest 
 that we have $ h = 0 $ for the conformal transformation 
 of $ X ^ \mu (z)$. Recall that in 
 general, we have that a conformal transformation 
 can be written as 
 and this gives us 
\[
   \delta _ v \phi \left(  z  \right)   = 
   \int \frac{d w}{2 \pi i  } \mathcal{ R } \left( v \left( w  \right)  T \left(  w  \right) \phi 
   \left(  z  \right)  \right)  
\] Making use of the residue theorem, 
which is 
\[
	\frac{1}{\left( n- 1  \right)  ! } 
	\partial  _ z ^{ n -1  } f \left(  z \right)   = \oint_{ w = z } 
	\frac{d w}{ 2 \pi i } \frac{ f \left(  w  \right)  }{\left(  w - z  \right) ^ n}
\] This requirement fixes the 
$ T \left( w  \right)  \phi \left(  z  \right)  $ OPE 
to have the form
\[
	T \left(  w  \right)  \phi \left(  z  \right)  = \frac{h }{ \left( z - w  \right)  ^ 2 } 
	\phi \left( z    \right)  + \frac{1}{ z - w  } \partial  \phi \left(  z  \right)  + \dots 
\] For formal purposes, we will now take this AS 
the definition for a field $ \phi \left(  z  \right)   $ 
to be a chiral primary of weight $ \left( h , 0  \right)  $. 

\subsection{The OPE of $ T \left(  z  \right)  : e ^{ i k \cdot  X \left(  w  \right)  } $}
Let's calculate the OPE 
expansion of the object $ T \left(  z  \right)  : e ^{ i k \cdot  X \left(  w  \right)  } : $. 
Throughout this calculation we'll use a 
few tricks, including contracting 
things multiple times. 
The first thing to do is to 
recall the definition of the stress tensor 
in holomorphic coordinates, the $ T _zz \left(  z  \right)  = T \left(  z  \right)  $ 
component of the stress tensor is given by 
\[
 T \left(  z  \right)   = - \frac{1}{\alpha ' } : \partial  _ \mu X_1
\] After single 
contractions, and Taylor expanding out 
$ \partial  ^ \nu \left( z \right) = \partial X ^ \nu \left( w   \right) 
+ \left( z - w  \right)  + \partial  ^  2 X ^ \nu \left(   w  \right)  $, 
we find that the expansion from 
single contractions is $ \frac{1}{z - w } \partial  \left(  e 
^{ i k \cdot  X\left(  w  \right)  }\right) $. 
The only other contraction we can 
have are with two sets of contractions 
from both of the terms in $ T \left( z   \right)  $. 
Using the fact that we can have $ n \left( n - 1  \right)  $ contractions. 

This term is 
given by 
\begin{align*}
	T _ 2 &= - \frac{1}{\alpha '  } \sum _{ n \geq  2 } k _{ \mu _ 3 } \dots k _{ \mu _ n } 
	\frac{i^ n  }{ n ! } n \left( n - 1  \right)  X^{ \mu _ 3 } \left(  w  \right)  
	\dots X ^{ \mu _ n } \left( - \frac{\alpha '  }{ 2  } \right)  ^2 \frac{k ^  2 }{ \left( z - w  \right)  ^ 2 } \\
	      &=  -\frac{\alpha'  }{4 } \frac{k ^ 2  }{ \left( z - w  \right)  ^ 2 } 
	      \sum_{ n \geq 2 } \left( k \cdot  X  \right) ^{n - 2 } i ^ 2 i ^{ n - 2  }
	      \frac{n  ! }{n ! \left( n - 2  \right)! }\\
	      &=  \frac{\alpha ' }{ 4 } \frac{k ^ 2  }{ \left( z - w  \right)  ^ 2 } : e ^{ i k \cdot  X } : 
\end{align*}
Since this is all the possible 
contractions, we 
have that our OPE is 
\[
	T \left( z  \right)  : e ^{ i K \cdot  X \left(  w  \right)  } : = 
	\left( \frac{\alpha ' k ^ 2 /  4 }{ \left(  z - w  \right)  ^ 2 } + \frac{\partial   }{ 
	\left( z - w  \right)   }  \right)  e ^{ i k \cdot  X }
\] This gives us an 
indication of what the value of $ h = \frac{\alpha ' k ^  2 }{ 4} $. 
It is easy to see that a similar 
result holds for $ \overline{ T } \left( \overline{ z }  \right)  e ^{ i k \cdot  \overline{ X  } \left( 
\overline{ w } \right)  } $. 
More generally, 
the operator which is a function of 
both variables $: \exp \left( i k  _\mu X ^ \mu \left( w , \overline{ w }  \right)   \right)  :$
has the conformal weight $ \left( h , \overline{ h }  \right)   = \left( 
\frac{\alpha  'k ^ 2 }{ 4} , \frac{\alpha ' k ^ 2 }{ 4 } \right) $. 
I'm not sure why this is true, but there's 
a sense this is a purely quantum mechanical result. 

Now, since $ T $ is such an important 
object, it is of interest of us 
to explore how $ T $ itself transforms 
conformally. What is it's conformal weight, for example? 

We now try to find out what the 
$ T \left( z  \right)  T \left(  w  \right)  $ operator product expansion is. 
This is related to something called the Virasoro algebra. 
We use the fact that 
\[
 T \left( z  \right)   = - \frac{1}{\alpha' } : \partial  X \partial  X :
\] From our previous analysis we found 
that $ \partial X ^ \mu \left(  z  \right)  \partial  X ^ \nu \left(  w  \right)   = 
- \frac{\alpha ' }{ 2 } \frac{\eta ^{ \mu \nu } }{ \left( z -  w  \right)  ^{ 2 } } $. 
Again, we use Wick's theorem 
and do some contractions. In terms 
of single contractions, we have 4 of the same type of contraction, 
where we contract $ \partial  X _ \mu \left( z \right) $ with $ \partial  X _ \nu 
\left( w  \right)  $. In terms of double contractions, 
we have two contributions. This 
means that we ultimately have 
\[
	T \left( z  \right)  T \left( w  \right)  
	= -\frac{2}{\alpha ' } \frac{\eta ^{ \mu \nu } }{ \left( z - w  \right)  ^ 2 } 
	: \partial  X ^ \mu \left(  z  \right)  \partial  X ^ \nu \left(  w  \right)  : 
	+ \frac{1}{2 } \frac{\eta ^{ \mu \nu } \eta _{ \mu \nu } }{\left( z  - w  \right)  ^ 4 }
\] Ideally we want to rewrite this in terms 
of $ T \left( w  \right)   $ so we 
can compare it to the form of the conformal transformation. 
To do this, we Taylor expand $ \partial  X ^ \mu \left(  z \right)  
= \partial  X ^ \mu \left( w   \right)  + \left( z - w  \right)  \partial  ^ 2 X ^ \mu 
\left( w  \right)  + \dots $. 
This means we get a total expansion of 
the form 
\[
	T \left( z  \right)  T \left( w  \right)   = 
	\frac{D  / 2 }{ \left( z - w  \right)  ^{ 4 }  }  - \frac{2}{\alpha ' } \frac{1}{\left( z- w  \right)  ^ 2 } 
	: \partial  X ^ \mu \partial  X _ \mu \left( w  \right)  :  - \frac{2}{\alpha ' } 
	\frac{1}{z - w } : \partial  X _ \mu \partial  ^  2 X ^ \mu : 
\] Therefore, we have 
that, in terms of the 
expression $ T \left( w  \right)  $, 
that our operator product expansion 
\[
	T \left( z  \right)  T \left( w  \right)  
	= \frac{D}{2 } \frac{1}{\left( z - w  \right)  ^{ 4 } } + 
	\frac{2}{\left( z - w  \right) ^ 2  } T \left(  w \right)  
	+ \frac{1}{z - w  } \partial  T + \dots 
\]  The pole 
of order 4 only vanishes if $ D = 0 $, which 
suggests that $ T $ only transforms conformally if $ D  =0 $. 
This suggests that $ T $ in fact isn't 
the full form of the stress tensor - we need to consider 
ghost fields. 

\subsection{The Virasoro Algebra}
We think a bit harder about this and introduce 
a mode expansion. 
From our reasoning above, 
we have that the stress tensor is a field with weight $ 2 $, 
so we can expand it in terms of these modes $ L _ n $. 
\[
 T \left(  z \right)   = \sum_ n L _ n z ^{ - n - 2 } 
\] This may be 
 inverted to give 
 \[
	 L _ n  = \oint_{ z  =0 } \frac{dz }{ 2 \pi i } z ^{ n + 1  } T \left(  z \right) 
 \] and similarly for $ \overline{ T } \left( \overline{ z  }  \right)  $. 
Consider the commutator 
of these fields. 
\[
 \left[  L _ m , L _ n  \right]   = \oint_{ z = 0 } 
 \frac{dz }{ 2 \pi i } \oint_{ w = 0 } \frac{dw }{ 2 \pi i } z ^{ m + 1 } w ^{ n + 1 } \left[ 
 T \left( z  \right)  , T \left(  w  \right)  \right]  
\] We have already argued 
that we should think of 
\[
	\oint_{ z = 0 } \frac{dz }{ 2 \pi i } z ^{ m + 1 } \left[  T \left(  z  \right)  , T \left( w  \right)   \right]  
	= \oint _{  z = w }  R\left( T \left( z  \right)  T \left( w  \right)   \right)  z ^{ m + 1 } 
\] and so, what we're interested in then 
is 
\[
 \left[  L _ m , L _ n  \right]   = \oint _{ w = 0 } 
 \frac{dw }{ 2 \pi i } w ^{ n + 1 } \oint _{ z = w  } \frac{dz }{ 2 \pi i } z ^{ m + 1 } \mathcal{ R } 
 \left( T \left(  z   \right)  T \left(   w  \right)  \right) 
\] we use the TT OPE to evaluate this. 
So, the only thing we care about is the 
singular structure, since this is a 
countour integral and regular terms 
don't contribute. 
Hence, this is 
\[
  = \oint_{ w = 0 } \frac{dw  }{ 2 \pi i } w ^{ n + 1 } 
  \oint \frac{dz }{ 2 \pi i } z ^{ m + 1 } \left( 
  \frac{D  /  2 }{ \left( z - w  \right)  ^{ 4 } } + \frac{ 2 T \left(  w  \right)  }{\left( 
  z -  w \right)  ^{  2}   } + \frac{\partial  T \left(  w  \right)  }{z - w   }\right) 
\] 
Noting that 
\begin{align*}
	\oint \frac{dz  }{ 2 \pi i } \frac{z ^{ m + 1 } }{ \left( z - w  \right)  ^ 2 }  & = \left( m + 1  \right)  w ^{ m } \\
	\oint_{ z = w } \frac{d z }{ 2 \pi i } \frac{z ^{ m + 1 } }{ \left( z - w  \right)  ^ 4 }
											 & = \frac{1}{3 ! } \left( m + 1  \right)  m \left( m - 1  \right)  w ^{ m - 2 } 
\end{align*}
And so, we have that 
the final result of the 
commutator is that 
\begin{align*}
	\left[  L _ m ,  L _ n  \right]  &=  \oint_{ w = 0 } 
	\frac{dw }{ 2 \pi i } w ^{ n + 1 } \left( \frac{D}{12 } m \left( m ^ 2 - 1  \right)  w ^{ m - 2 } 
	+ 2 T \left(  w  \right)  \left( m + 1  \right)  w ^{ m } + \partial  T \left(  w  \right) w ^{ m + 1 } \right) \\
					 &=  \frac{D }{ 12 } m \left(  m^ 2 - 1  \right)  \delta _{ m + n , 0} 
					 + \left( m - n  \right)  L _{ m + n } \\
\end{align*}
The additional term is sometimes referred to as a 
central extension or an anomaly term. 
This is the Virasoro algebra. If you look 
at just the terms generated by $ L _ 0 , L _{ \pm 1 } $, there is no 
anomaly. 
Similarly, for $ \left[  \overline{ L } _ m , \overline{ L } _ n  \right]  $ 
this result holds. Note that the 
$ T \left( z    \right)  \overline{ T } \left( w  \right)  $ OPE 
is regular so $ \left[  L _ m , \overline{ L } _ n  \right]   = 0 $. 
 
\subsection{The $ b , c $ ghost system}
We will now work on a way to 
remove the anomalous terms that we 
get in the Virasoro algebra. To do this, we 
will need to introduce ghost fields as we did 
looking at our FP determinant. 
The Faddeev Popov procedure 
required us to to introduce 
anti-commuting ghost fields: $ b _{ ab } \left(  z  \right)   $ 
and $ c ^ a \left( z  \right)  $ such that 
$ c ^ a \left( z   \right)  c ^ b \left(  w  \right)   = - c ^ b \left(  w  \right)  c ^ a \left(  z \right)  $. 

Our ghost action is 
\[
	S \left[  b, c  \right]   = \frac{1}{2 \pi } \int_{ \Sigma } d ^ 2 \sigma \sqrt{ - h }  h ^{ ac } b _{ ab } \left( 
	\nabla _ c c ^ b\right) 
\] So, the total action 
$ S = S _ p \left[  X  \right]  + S _{ gh } \left[  b, c \right]  $. 
This allows us to cook up a stress tensor 
which exhibits the conformal symmetry 
that we want. 
The ghost stress tensor is then, upon varying 
the metric $ h _{ ab } $ appropriately in 
our action, 
\[
	T_{ ab }  = - i \left( \frac{1}{2 } c ^ c \nabla _{ ( a  } b _{ b ) c  } 
	+ \left( \nabla _{ ( a  } c ^ c  \right) b _{ b ) c }    - h _{ ab } \text{trace}\right)  
	\] Notice that by
	construction, this object is traceless, 
	since we've removed the 
	trace term at the end.
	This is a bit of a difficult expression to work with, 
	so let's fix $ h _{ ab }  = e ^{ \phi } \delta _{ ab } $ 
	by working in the conformal gauge. 
The total stress tensor is the contribution that we 
get from the field $ X $ and the stress tensor which we got from the 
ghost field. 
\[
 \mathcal{ T }  = T _ X + T _{ \text{gh}}
\] Instead of working in 
general indices $ b , c $  lets 
now work with a Euclidean metric on 
$ \Sigma $ and complex coordinates as before. 
We cam write the two degrees of freedom 
in $ b _{ab } $ as $ b \left( z  \right)  $ and $ \overline{ b } \left( \overline{ z }  \right)  $, 
and the 2 degrees of freedom in $ c ^ a $ 
as $ c \left( z  \right)   $ and $ \overline{ c } \left(  \overline{ z }  \right)  $. 
The action then becomes 
\[
 S _{\text{gh}} \left[  b,  c  \right]   = \frac{1}{ 2 \pi } \int_{ \Sigma } d ^ 2 z 
 b \overline{ \partial  } c + \frac{1}{2 \pi } \int_{ \Sigma }  d ^ 2 z \overline{ b } \partial  \overline{ c } 
\] The total stress 
tensor 
\[
	\mathcal{ T } \left(  z  \right)  = T _ X \left(  z  \right)  + 
	T_{\text{gh } } \left( z  \right)  
\] where we have 
\begin{align*}
	T \left( z   \right)  & =  - \frac{1}{\alpha ' } : \partial   X^ \mu \partial  X _ \mu \left( z  \right)  : \\
	T _{\text{gh}}\left( z  \right)  &=  : \left( \partial  b  \right)  c \left( z  \right)  : 
	-  2 \partial  \left( : bc \left( z   \right)  :   \right) 
\end{align*}
with similar expressions for the anti-homomorphic variables. 
Since we have now $ X $ dependence from the ghost fields 
and vice versa, we have only regular terms 
when we do $ T _{\text{gh}} T_{ X } $ OPE. 
So, we want to find the OPE of the ghost field on it's own, 
which is going to be spacetime independent. 
 
\subsection{The $b, c  $ OPE}
The two point function (a Green's function) gives 
the singular part of the $ b \left( z  \right)  c \left( w  \right)  $ OPE. 
The two point function is 
\[
	\left< b \left( z  \right)  c \left( w  \right)    \right>  = 
	\left< : b \left( z  \right)  c \left( w  \right)  :  \right> + \wick{ \c b \left(  z  \right)  \c 
	c \left(  w  \right)  } = \wick{\c b \left( z  \right)  \c c \left(  w \right) }
\]  This is also the green's function 
for the holomorphic part of the action. 
Recall that the 
propagator of some field configuration 
is the green's function of the 
associated equation of motion. 
In terms of the holomorphic part, our action is 
$ \int d ^ 2 z b \overline{ p   } c $. Our 
associated operator then, for the equation of motion of $ c $, 
is the $ \overline{ \partial  } $ operator. 
The classical green's function for $ \overline{ \partial  }  $ on the 
sphere is just $ \frac{1}{ z - w }  $ is just $ \frac{1}{ z - w } $. 
This can be shown using Stokes' theorem.
We want to show that 
\[
	\overline{ p } G \left( b, c  \right)   = 2 \pi \delta \left( z - w  \right)  
\] 
so we have 
\[
	b \left( z  \right)  c \left( w   \right)  = \frac{1}{ z - w } + \dots c \left( z  \right)  b \left(  w \right) 
\] if we swap $ w $ and $ z $ we get a change in sign, 
but if we commute $ b  $ and see past each other we also 
pick up a minus sign. 

\subsubsection{The conformal weight of $ b \left( z  \right)  $ }
The stress tensor is $ T  = \left( \partial  b  \right)  c  - 2 \partial  \left( bc \right)   $, 
and so the OPE we want is 
\begin{align*}
	T \left( z  \right)  b \left( w  \right)  &=  
	\left( \partial  b \left( z  \right)    \right) \wick{\c c \left( z  \right)  \c b \left(  w  \right) }
	- 2 \partial  _ z ( b \left(  z  \right)  \wick{  \c c \left(  z \right)   ) \c b \left(  w  \right)   }\dots  
	\\ 
						  &= \frac{\partial  b \left(  z  \right)   }{ z -w } 
						  - 2 \partial  _ z \left( \frac{b \left(  z  \right)  }{ z - w }  \right)  + \dots\\
						  &=  
						  \frac{\partial  b \left(  z  \right)  }{ \left( z - w   \right)   }
						  - 2 \frac{ \partial  b \left(  z  \right)   }{z - w } 
						  + 2 \frac{ b \left(  z  \right)   }{ \left( z - w  \right)  ^ 2 } + \dots \\
						  &=  - \frac{\partial  b \left(  z  \right)   }{ z - w } + 
						  \frac{ 2 b \left(  z  \right)   }{\left( z - w  \right)  ^  2 }  + \dots \\
						  &=  - \frac{\partial  b \left(  w \right)   }{z - w} 
						  + 2 \frac{ b \left(  w  \right)  }{\left(  z- w  \right)  ^ 2 } 
						  + 2 \frac{\partial  b \left(  w  \right)  }{ \left(  z - w   \right)   } + \dots\\
\end{align*} 
Now, putting this in the form where 
we can read off the conformal weight, 
we get the result as 
\[
	T _{ gh } \left(  z  \right) b \left( w  \right)   = 2 \frac{ b \left(  w  \right)  }{ \left(  z - w  \right)  ^ 2 } 
	+ \frac{\partial  b \left(  w  \right)   }{ z - w } 
\] We conclude $ b \left( w  \right)   $ has weight $ h = 2 $. 
We cought do $ c \left( w  \right)  $ to find $ h  = - 1 $,
with similar expressions for the anti-holomorphic part.
This gives the expression 
\[
	T _{gh }\left(  z  \right)  c \left(  w  \right)   = -\frac{1}{\left( z - w  \right)  ^ 2 } 
	c \left(  w \right)  + \frac{1}{z - w } \partial  c \left(  w  \right)  + \dots 
\] Hence we have that 
$ c \left( w  \right)   $ has weight $ \left( h , \overline{ h }   \right)   = \left(  - 1 , 0  \right)  $. 
This feels like the sort of thing 
we should be getting, because we have the object $ b _{ ab } $ 
which is some modified two form, and $ c ^ a $ has index 
upstairs. So, this is roughly what we 
expect. 
We can now find the OPE between the ghost stress tensor and 
itself, which is $ T_{gh}\left( z  \right)  T _{ gh } \left(  w  \right)   $. 
Life is two short to do 
this explicitly, and we get that 
\[
	T_{gh} \left( z  \right)  T _{gh } \left(  w  \right)  
	=  - \frac{26 /2 }{\left( z - w  \right)  ^ 4  } + \frac{2}{\left( z - w  \right)  ^ 2 } T _{gh } \left(  w  \right)  
	+ \frac{1}{z - w } \partial  T _{gh } \left( w  \right)  + \dots 
\] Recall that from our 
fields we have that 
\[
	T _ X \left(  z  \right)  T  _ X \left(  w  \right)  
	= \frac{D / 2  }{ \left( z - w  \right)  ^ 4  } + \frac{2}{\left( z - w  \right)  ^ 2  }
	T _ X \left(  w  \right)  + \frac{1}{z - w } \partial  T _  X \left(  w  \right) 
\] and we have that $ T _ X \left( z   \right)  T _{ gh } \left( w   \right) $ 
is regular. 
This means that the OPE of the total stress tensor, 
which is the stress tensor of the whole 
theory $ T \left( z  \right)   = T  _  X \left( z  \right)  + T _{gh } \left(  z  \right)   $ 
has the OPE expansion
\[
	\mathcal{ T } \left(  z  \right)  \mathcal{ T } \left( w  \right)  
	= \frac{\left(  D - 26  \right)   / 2 }{ \left(  z - w  \right)  ^  4 }
	+ \frac{2}{\left( z - w  \right)  ^ 2} \mathcal{ T } \left(  w  \right)  + \frac{1}{z -  w } \partial \mathcal{ T } \left(  w \right) 
\] This is assuming 
that we can treat the ghost fields independently. 
If $ D = 26 $, the anomaly term, which is a pole of order 4, 
vanishes and we have a consistent quantum conformal theory (the bosonic theory). 
We got this result by looking at the contributions of two sectors. 
The solution to resolving the tachyonic problem 
is using supersymmetry. 
To make this supersymmetric we need to add some world sheet fermions, 
and add this on to the stress tensor. 
If we do this, we get $ D = 10 $. 
We'll say something briefly about reactions to this fact. 
One reaction is that this is nonsense. Another reaction 
is that this is interesting because we have a theory which 
spits out a dimension, unlike other theories. 
It could be that macroscopically, we have 4 dimensions with 
the other dimensions compactified. 
It may be that in quantum theories, that dimensions of 
space-time are emergent from quantum theories in 
some classical limit. 

\subsection{Mode expansions for ghosts}
In this section, we'll explore how to 
build anti-commutation relation 
We have that the conventional mode expansion 
for the ghosts, given by the conformal weights, 
are 
\[
	b \left(  z  \right)   = \sum _ n b _ n z ^{  -n -  2 } , \quad c \left(  z  \right)  
	 = \sum _ n c _ n z ^{ - n + 1 } 
\] thus, we have that 
\[
	b _ n  = \oint \frac{dz }{ 2 \pi i } z ^{ n + 1 } b \left( z  \right)  , \quad 
	c _ n  = \oint \frac{dz  }{ 2 \pi i  } z ^{ n - 2 } c \left(  z  \right) 
\]  Then, we have that 
\[
 b _ m c _ n + c _ n b _ m  = \left\{  b _ m , c _ n  \right\} 
\] They obey fermionic statistics, 
integer spin. 
The above expression is 
\[
 \dots = \oint_{ z = 0 } \frac{dz   }{ 2 \pi i } z ^{ m + 1 } 
 \oint _{ z = 0  } \frac{dw   }{ 2 \pi i } w ^{ n -2  } \left\{  b \left(  z  \right)  , c \left(  w  \right)   \right\} 
   = \oint_{ z  =0 } \frac{dz  }{ 2 \pi i } \oint _{ z = w } \frac{dw }{ 2 \pi i } 
   z ^{ m + 1  } w ^{ n -  2 } \mathcal{ R } \left( b \left(  z  \right)  c \left(  w  \right)   \right) 
\] Now, using the 
$ b \left( z   \right)  c \left( w   \right)  $ OPE 
we can evaluate this anti-commutation relation as 
\[
 \left\{  b _ m , c _ n  \right\}   = \delta _{ m + n , 0 }
\] 

\subsection{The State-Operator Correspondence}
For example, 
we have the states $ \ket{ k }  = e ^{ i k \cdot  X } \ket{ 0 } $, 
and $ \epsilon _{ \mu \nu } \alpha _{ -  1 } \overline{ \alpha } _{ - 1 } \ket{ 0 } $. 
In 2 dimensional conformal field theory, to each physical 
state there is an operator in the 
operator algebra of the theory. 
For example, 
\[
	\partial  X ^ \mu \left( z  \right)   =  - i \sqrt{ \frac{\alpha }{ 2 } }  \sum _ n \alpha _ n ^{ \mu } 
	z ^{ - n - 1 } 
\] We can construct a non-physical state as follows. 
\[
	\lim_{ z \to 0 } \partial  X ^ \mu \left( z  \right)  \ket{ 0 } 
\] recalling that $ \alpha _ n ^ \mu \ket{ 0 }  =0 $ for $ n \geq 0$. 
This gives 
\[
  - i \sqrt{\frac{\alpha ' }{  2 } }  \lim _{ z \to 0 } \sum _ n \frac{\alpha _ n ^ \mu }{ z ^{ n + 1 }  } \ket{ 0 } 
   = - i \sqrt{\frac{\alpha '  }{  2 } }  \lim _{ z \to 0 } \sum _{ n \geq - 1 } \frac{\alpha _ n ^ \mu  }{ z 
   ^{ n + 1  } } \ket{ 0 } 
\] but since $ \alpha _ n ^ \mu \ket{ 0 }  = 0 $ for $ n \geq 0 $, this is 
the state 
\[
- i \sqrt{ \frac{\alpha }{  2 } }  \alpha _{ - 1 } ^ \mu \ket{ 0 } 
\] There are operators that  
do not ...
Similarly, we have that $ \lim _{ z , \overline{ z }  \to 0  } \partial  X ^ \mu \overline{ \partial   } X ^ \nu 
e ^{ i k \cdot  X \left( z, \overline{ z  }    \right)   } \ket{ 0 }  = \alpha _{ -  1 } ^ \mu \overline{ \alpha  } _{ - 1 }^ \nu 
\ket{ k }  $. 
More generally, if we have a weight $ h $ chiral field 
$ \phi \left(  z  \right)   $, then $ \phi \left( z  \right)   = \sum _ m \phi _ m z ^{ - n - h }  $. 
Then, for 
\[
	\lim _{ z \to 0 } \phi \left(  z  \right)  \ket{ 0 } \text{ to exist }
\] we require that $ \phi _ n \ket{ 0 }  = 0  $ for $ n > - h $. 
Then, we have that 
\[
	\lim _{ z \to 0 } \phi \left( z  \right)  \ket{ 0 }  = \phi _{- h } \ket{ 0 } 
\]  We will now look at the conditions for 
this to be a physical state. 

\subsection{BRST Symmetry}
After Faddeev-Popov we had the action 
\[
 S \left[  X , b, c  \right]   = S \left[  X  \right]  + S _{ gh } \left[  b, c  \right]  
\] and $ h _{ ab } $ was fixed to $ \hat{ h }_{ ab }  $. 
We would like to see how the choice 
$ \hat{ h}_{ab }  $ (doesn't influence the physics). 
Let usa introduce a Lagrange multiplier field 
$ B _{ ab } $ and the gauge fixing term in the action 
\[
 S _{gf} \left[  h , B  \right]   = \frac{1}{4 \pi \alpha ' } \int _{ \Sigma } 
 d ^ 2 \sigma \sqrt{ - h }  B ^{ ab } \left( \hat{ h } _{ ab }  - h _{ ab }  \right) 
\] This thoery has a residual rigid symmetry. 
Invariance gives us clear physcial consistency conditions 
and operators of the theory. 
\pagebreak 
\subsubsection*{Strings}
\begin{itemize}
	\item A string is two dimensional, 
		embedded with parameters $ \sigma, \tau $
		\[
			X^ \mu  \left( \sigma, \tau  \right)   = X^ \mu 
			\left(  \sigma + 2 \pi n , \tau  \right) , \quad 
			n \in \mathbb{ Z } 
		\]
	\item Our associated action is the Nambu-goto action 
	\[
		S[X] =  - \frac{1}{ 2 \pi \alpha ' }\int d \sigma d \tau 
		\sqrt{  - \det\left( \eta_{ \mu \nu } 
		\partial  _ a X ^ \mu \partial  _ b X ^ \beta \right) } 
	\] 
\item We add an extra degree of freedom $ h _{ ab } $ to introduce 
	the Polyakov action 
	\[
	 S[X, h ] = - \frac{1}{4 \pi \alpha ' } 
	 \int d ^ 2 \sigma \sqrt{  - h }  h ^{ ab } \eta _{ \mu \nu } 
	 \partial  _ a X ^ \mu \partial  _ b X ^ \nu 
	\]  
\end{itemize}


\subsubsection*{Quantising the Bosonic String}

\begin{itemize}
	\item Expand out position and conjugate 
		momenta in terms of Fourier modes
	\item We use our gauge invariance to 
		set 
		\[
			h_{ ab }  = e ^{ \phi } \begin{pmatrix}  - 1 &  0 
			\\ 0 & 1 \end{pmatrix} 
		\] The $ - 1 $ in the diagonal gives us a rough notion 
		of time. 
	\item Put this in our Polyakov action to get the 
		conjugate momenta, which is 
		\[
			P_\mu  = \fdv{S}{\dot{ X } ^ \mu  }
			 = \frac{1}{2 \pi \alpha '  } \dot{ X } ^ \mu  
		\]  
	\item Impose equal time commutation relations 
		which are equivalent to commutation 
		relations of the Fourier components. We use Poisson 
		brackets for this
		\[
		 \left\{ X ^ \mu, P_ \nu  \right\}  = 
		 \delta \indices{ ^ \mu _ \nu }  = \delta \left( \sigma  - \sigma '  \right) \iff \left\{ \alpha_ n , \alpha_ m  \right\}   = 
	 - i m \delta_{ n + m , 0 } , \quad \left\{  \overline{ \alpha }_ n, 
 \overline{ \alpha } _ m  \right\}   = - i m \delta _{ n + m , 0  } \]
	
	\item We can change coordinates to 
		\[
		 \sigma_{ \pm }  = \tau \pm \sigma 
		\] 
\end{itemize}
\section*{Example Sheet 1}

\subsection{Question 1}
Here we are showing equivalence 
of the Nambu-Goto action and the Polyakov action.
 

\pagebreak

\section{Example Sheet 2}

\subsection{Question 1}
We are asserting that 
our variations in moduli space should 
be orthogonal to arbitrary diffeomorphism 
variations generated by the vector $ v^ a  $. 
This means for arbitrary $ v ^ a  $, we 
have that 
\[
 \left( \delta _ v h _{ ab } \mid \delta _ t h _{ ab }  \right)   = 0
\] This implies that, by the definition of 
orthogonality of tensors provided in the definition of 
the question, 
\begin{align*}
	0 &=  \int_{ \Sigma } d ^ 2 \sigma \sqrt{ - h }  
	h ^{ ab } h ^{ cd } \delta _ v h _{ ab } \delta _ t h _{ cd} \\
	&=  \int_{ \Sigma } d ^ 2 \sigma \sqrt{ - h}  
	h ^{ ab } h ^{ cd } \left( \nabla _ a v _ b 
	+ \nabla _ b v _  a  \right)  \delta _ t h _{ cd } \\
	&=  2 \int _{ \Sigma  } d ^ 2 \sigma \left( 
	\sqrt{ -h }  h ^{ ab } \nabla _ a v _ b \right)  h ^{ cd } 
	\delta _ t h _{ cd }  \\
	&=  0 
\end{align*}
Now the thing to notice here is that 
this holds for arbitrary $ v $. This means that 
supposing we can choose $ v $ appropriately, we find 
$ h ^{ cd } \delta _ t h _{ cd } = 0 $.
This is because we have an integral 
of the form $ \int d ^ 2 \sigma f \left(  x  \right)  g \left( x  \right)    =0   $ 
for arbitrary $ f = \sqrt{ -h }  h ^{ ab } \nabla _ a v _ b  $.
So, $ g \left(  x  \right)   =0 $. This means 
we're left with the result that 
\[
 h ^{ ab } \delta _ t h _{ ab }  = 0 
\] 

For the second part of the 
question, we know we can write
a diffeomorphsim and Weyl transformation as
the sum of it's traceless part and a modified Weyl part. 
\[
 \delta _ v h _{ ab } + \delta _{\omega } h _{  ab }  
 = \left( P v  \right)  _{ ab } + \left(  2 \omega - \partial  _ c 
 v ^ c  \right)  h _{ ab }  = 
 \left( P v  \right)  _{ ab } +  \left( 2 \overline{ \omega }  \right) h _{ ab }  =0  
\] 
This means that we 
have 
\begin{align*}
	0 &=  \left( \delta _ t h _{ ab } \mid \left( P v  \right)  _{ ab } 
	+ 2 \overline{ \omega } h _{ ab } \right)    \\ 
	  &=  \left( \delta _ t h _{ ab } \mid (P v ) _{ ab }  \right)  
	  + 2 \int d ^ 2 \sigma \sqrt{ -h }  h ^{ cd } h ^{ ab } \overline{ \omega } 
	  h _{ cd } \delta _ t h _{ ab } \\ 
	  &=  \left( \delta _ t h_{ ab } \mid \left( P v  \right)  _{ ab }  \right)  
	  + 4 \int d ^ 2 \sigma \sqrt{ - h }  \overline{ \omega } h ^{ ab } \delta _ t h _{ ab }  \\ 
	  & = \left( \delta _t h _{ ab } \mid \left( P v  \right) _{ ab }  \right)  
\end{align*}

\pagebreak 

\subsection{Question 2}
To calculate the variation with respect to 
$ h $ in this question, we apply the usual rules 
of differentiation but with variations instead.
\[
 \delta _ h \left< O _ 1 \dots O _ n \right> 
 = \int \mathcal{ D } X i \delta _ h S e ^{ i S \left[  X, h  \right]  } O _ 1 \dots O _ n 
\]
However, from lectures, we know 
what the variation of $ S $ with respect to $ h $ is. 
\[
 \delta _ h S  = -\frac{1}{4 \pi } \int d ^ 2 \sigma 
 \delta h^{ ab  } T _{ ab }
\] This means that our 
corresponding variation in 
our expectation value of 
the observables is 
\[
 \delta _ h \left( O _ 1 
 \dots O _ n \right)   = \int \mathcal{ D } X \left( 
 \int d ^ 2 \sigma \sqrt{ - h  } \delta h ^{ ab } T _{ ab } \right)  e ^{ i S \left[  
 X,  h \right]  } O _ 1 \dots O _ n 
\] However, we switch 
the order of integration here 
and absorb $ T _{ ab} $ into 
the definition of our expectation value. Thus, we find 
that 
\[
	\delta _ h \left< O _1 \dots O _ n  \right> 
	= -\frac{1}{4 \pi } 
	\int d ^ 2 \sigma \sqrt{ -h } \delta h ^{ ab } 
	\left< T _{ ab } O _ 1 \dots O _ n  \right>
\] In the 
case of a Weyl transformation, we have 
that $ \delta h _{ ab }  = \omega h _{ ab } $. 
Since our expectation value 
should be invariant of transformations
to the metric, we have that 
\begin{align*}
	0 &=   -\frac{1}{4 \pi} \int d ^ 2 \sigma 
	\sqrt{ - h }  \delta h^{ ab } \left< T 
	_{ ab } O _ 1 \dots O _ n \right> \\
	&=  - \frac{1}{4 \pi } \int d ^ 2\sigma 
	\sqrt{ - h }  \omega h ^{ ab } \left< T _{ ab } O _ 1 
	\dots O _ n \right> \\ 
	&=   - \frac{1}{4 \pi } \int d ^ 2 \sigma 
	\sqrt{ - h }  \omega \left< T \indices{ ^ a _ a } 
	O _ 1 \dots O _ n \right>
\end{align*}
Since we can pick $ \omega $ to be arbitrary 
here, we must necessarily have 
\[
 0  = \left< T \indices{ ^ a _ a } O _ 1 \dots O _ n   \right>
\] 

\subsection{Question 3}
In this question, it's a matter of 
going through the motions. 
\begin{align*}
	\delta S &=  - \frac{1}{4 \pi \alpha ' } 
	\int d ^ 2 \sigma 2 \delta \left( \partial  _ a X  \right)  
	\cdot  \partial  ^ a X \\ 
	&=  - \frac{1}{2 \pi \alpha ' } \int d ^ 2 \sigma 
	\partial _ a \left( v ^ b \partial _ b X  \right)  \cdot  \partial  ^ a 
	X \\
	&=  - \frac{1}{2 \pi \alpha ' } 
	\int d ^ 2 \sigma \, \left(  \partial  _ a v ^ b  \right)  
	\partial  _ b X \cdot  \partial  ^ a ^  X + v ^ b 
	\partial  _ a \partial  _ b X \cdot  \partial  ^ a X \\ 
	&=  -\frac{1}{2 \pi \alpha '  } \int d^ 2 \sigma \, 
	\left( \partial  ^ a v ^ b   \right)  \partial  _ b X \partial  _ a 
	X + \frac{1}{2 } v ^ b \partial  _ b \left( \partial  _ c X
	\cdot  \partial  ^ c X  \right) \\
	&=  - \frac{1}{2 \pi \alpha ' } 
	\int d ^ 2 \sigma \left( \partial  ^ a v ^ b  \right)  
	\partial  _ a X  \cdot  \partial  _ b X - \frac{1}{2 } 
	\left( \partial  ^ a v ^ b  \right)  h _{ ab } 
	\left( \partial  _ c X \cdot  \partial  ^ c X  \right)  \\
	&=  \frac{1}{2 \pi  } \int d ^ 2 \sigma \partial  ^ a 
	v ^ b \left[  \left( - \frac{1}{\alpha' }  \right)  
	\left( \partial  _ a X \cdot  \partial  _ b X - 
\frac{1}{2} h _{ ab } \partial  _ c \cdot  \partial  ^ c X   \right) \right] \\
	&=  \frac{1}{2 \pi } \int d ^ 2 \sigma \left( \partial  ^ a 
	v ^ b \right)  T _{ ab }  
\end{align*} 
\end{document} 
