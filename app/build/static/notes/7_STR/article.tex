\documentclass[11pt, oneside]{article}   	% use "amsart" instead of "article" for AMSLaTeX format
\usepackage[margin = 1.1in]{geometry}            		% See geometry.pdf to learn the layout options. There are lots.
\geometry{letterpaper}                   		% ... or a4paper or a5paper or ... 
\usepackage[parfill]{parskip}    		% Activate to begin paragraphs with an empty line rather than an indent
\usepackage{graphicx}				% Use pdf, png, jpg, or eps§ with pdflatex; use eps in DVI mode
								% TeX will automatically convert eps --> pdf in pdflatex	
\usepackage{adjustbox}	
\usepackage[section]{placeins}


%% LaTeX Preamble - Common packages
\usepackage[utf8]{inputenc}
\usepackage[english]{babel}
\usepackage{textcomp} % provide lots of new symbols
\usepackage{graphicx}  % Add graphics capabilities
\usepackage{flafter}  % Don't place floats before their definition
\usepackage{amsmath,amssymb}  % Better maths support & more symbols
\usepackage[backend=biber]{biblatex}
\usepackage{amsthm}
\usepackage{bm}  % Define \bm{} to use bold math fontsx
\usepackage[pdftex,bookmarks,colorlinks,breaklinks]{hyperref}  % PDF hyperlinks, with coloured links
\usepackage{memhfixc}  % remove conflict between the memoir class & hyperref
\usepackage{mathtools}
\usepackage[T1]{fontenc}
\usepackage[scaled]{beramono}
\usepackage{listings}
\usepackage{physics}
\usepackage{tensor}
\usepackage{simplewick} 
\usepackage{tikz} 
\usepackage{import}
\usepackage{xifthen}
\usepackage{pdfpages}
\usepackage{transparent}
\usepackage{pgfplots}
\usepackage[compat=1.1.0]{tikz-feynman}
\usepackage{subfiles}
\usepackage{simpler-wick}
\usepackage{slashed}
\usepackage{fancyhdr}
\usepackage{enumitem}

\pagestyle{fancy}
\fancyhf{}
\rhead{Notes by Afiq Hatta}
\lhead{String Theory}
\rfoot{Page \thepage}

%% Commands for typesetting theorems, claims and other things.
\newtheoremstyle{slanted}
{1em}%   Space above
{.8em}%   Space below
{}%  Body font
{}%          Indent amount (empty = no indent, \parindent = para indent)
{\bfseries}% Thm head font
{.}%         Punctuation after thm head
{0.5em}%     Space after thm head: " " = normal interword space;
{}%         \newline = linebreak
{}%          Thm head spec (can be left empty, meaning `normal')

%% Commands for typesetting theorems, claims and other things. 

\theoremstyle{slanted}
\newtheorem{theorem}{Theorem}
\newtheorem*{thm}{Theorem}
\newtheorem*{claim}{Claim}
\newtheorem{example}{Example}

\newtheorem*{defn}{Definition}

\newcommand{\Lagr}{\mathcal{L}} 
\newcommand{\vc}[1]{\mathbf{#1}}
\newcommand{\pdrv}[2]{\frac{\partial{#1}}{\partial{#2}}}
\newcommand{\thrint}[1]{\int d^3 \vc{x} \left( {#1} \right)}

%% QFT specific macros 
\newcommand{\intp}{ \int \frac{ d^3 p }{ (2 \pi)^3 } \, }
\newcommand{\ann}[1]{a_{ \mathbf{ #1 }}}
\newcommand{\crea}[1]{a^\dagger_{ \mathbf{ #1 }}}
\newcommand{\ve}[1]{ \mathbf{ #1 } } 
\newcommand{\mode}[ 1]{ e^{ i \mathbf{ #1 } \cdot \mathbf{x} }}
\newcommand{\nmode}[1]{ e^{  - i \mathbf{ #1 } \cdot \mathbf{x} }}
\newcommand{\freq}[1]{\omega_\mathbf{ #1} } 
\newcommand{\scal}[1]{\phi ( \mathbf{ #1 })} 
\newcommand{\mom}[1]{ \pi (\mathbf{ #1 })} 
\newcommand{\arr}{\rightarrow} 

\newcommand{\incfig}[1]{%
\def\svgwidth{\columnwidth}
\resizebox{0.75\textwidth}{!}{\input{./figures/#1.pdf_tex}}
}

\newcommand{\anop}[2]{ #1_\mathbf{#2}}
\newcommand{\crop}[2]{#1_\mathbf{#2}^\dagger}

\usepackage{helvet} 

%tikz decoration commands 
\usetikzlibrary{decorations.pathmorphing}


\title{String Theory Lecture Notes}
\author{Lectured by Dr. Reid-Edwards} 
\begin{document} 
\maketitle
\tableofcontents

\pagebreak 

\section{Introduction}%
\label{sec:introduction}

If you'd like to read about string theory, 
check out 
\begin{itemize}
\item String Theory, Vol 1, Polchinksi, CUP 
\item Superstrings, Vol 1, Green et al, CUP
\item A String Theory Primer, Schomerus, CUP
\item David Tong's Notes, at arXiv: 09080333
\item 'Why String Theory' Conlon, CRC  
\end{itemize}

What is string theory? 
We don't really know what string theory is. 
The question itself requires a little more 
fleshing out to make sense of it. String theory 
is a work in progress. A final version of string 
theory would be a theory which we understand physically 
and mathematically. 
It's a work in progress - the final form will 
be quite different from what we have now. 

What do we know? 
Well, in some sense, string theory 
is an attempt to answer the question of how we quantise the 
gravitational field. A theory of \textbf{quantum gravity}. 
There are however, a number of obstacles.
In particular, naive quantisation of the 
Einstein Hilbert action presents a number of problems. 

There are deep conceptual problems associated with this. 
\begin{itemize}
\item  There's a question about the 
	nature of time in quantum gravity. 
If we think about time in quantum mechanics, time 
is treated as a fixed clock in which the Hamiltonian 
governs the evolution. In GR, space and time are on the same footing. 
Thus, the descriptions are on a different footing 
in QM versus GR. This is not neseccarily a technical problem, 
just something to think about. 
\item How do we quantise things without a pre-existing causal structure? 
What do we mean by this? We we quantise things in QFT, 
it's important to know whether two operators 
are timelike or spacelike separated. We have a notion 
that for all operators which are spacelike separated, 
commute. (One should not be able to influence the other). 
If we try to quantise GR, the metric is the object which 
we would like to quantise. But, this determines the casual 
structure. So, it's not immediately obvious 
what the algebra of operators should look like. 

\item GR has a very big symmetry - diffeomorphism symmetry (symmetry under
coordinate reparametrisations). This is diffeomorphism 
invariance. This is something we'll discuss a bit later on. 
This is considered to be a gauge symmetry. 
\item One thing we'll discuss is that there are no local 
diffeomorphism invariant observables in GR. It's 
not even clear what the observables should be. 
\end{itemize}

More importantly, 
there are technical obstacles. The previous issues are hard, 
but can we make some assumptions which help us make progress? 
We can look at perturbation theory; we can take our 
metric an expand it around some classical solution
\[
g _{ \mu \nu } \left( X  \right)   = \eta _{ \mu \nu } + h _{ \mu \nu } \left( x \right) 
\] for our purposes, this classical solution will be 
Minkowski space-time. 
This means we can use the causal structure of 
the background metric (Minkowski) to 
learn about the causal structure of the perturbation, 
and quantise. 
This immediately neutralises the first two 
conceptual problems. We can call the fluctuations $ h _{ \mu \nu } $ 
as gravitons. 
So, if we take a background static spacetime, add a field, 
then quantise. 

There are some unsatisfactory things about this. 
When we split the metric into two, 
we hide a lot of the deep structure that we want. 
Nonetheless, we can take our Einstein-Hilbert action and expand it out
\[
S \left[ g  \right]   = \frac{1}{K_ 0 } \int d ^{ D } X \sqrt{  - g }  R ( g ) 
\] Choose a gauge and expand out 
\[
S \left[  h  \right]   = \frac{1}{K_0 } \int  d ^ D X \left( 
h _{ \mu \nu } \Box h ^{ \mu \nu } \right) 
\] Since the Ricci scalar contains inverses, 
this expansion goes on forever. This is called 
a non-polynomial action. 
The first quadratic term gives the propagator, 
and the higher order term gives us our interactions. 

The propagator is represented by a wiggly line. 
The interaction term gives us vertices. (Three or four 
wiggly lines coming together) 
These lead to Feynman rules.

However, when we compute loops, 
we get divergences. But, in physical QFT, 
we can absorb these divergences into coupling constants. 
These can be dealt with using standard techniques. From
advanced quantum field theory, this technique 
is called renormalisation. 

Thus, the difficulties run deeper than conceptual ones. 
We simply don't know how to calculate. 
So, string theory 
provides a way to do quantum perturbation theory 
of the gravitational 'field' 
We put field in quotation marks because it's 
not really a field which we'll be dealing with. 

String theory answers some of the questions, 
in the sense that it gives 
us a framework to ask 
meaningful questions in quantum gravity. 
But not all of these questions are questions that 
we can answer. 

The viewpoint that  we're going to 
take for this course will be from 
perturbation theory. But we'll always 
try to understand what this tells us about the 
non-perturbative physics. 

\subsection{Worldsheets and Embeddings}
Let's try 
to put together some sort of language 
to get started. From any popular science book, 
you may find that particles are described 
as vibrating strings. 
The starting point is to consider a 
worldsheet $ \Sigma $ which is a 
2-dimensional surface swept out by a string. 
This is analogous to a 
worldline swept out by a particle.

(Insert diagram of line parametrised by $ \tau $  and pointlike 
object, and diagram of cylinder object with surface called 
$ \Sigma $, with two axes called $ \tau $ and $ \sigma $.   This 
diagram is in 3d Minkwoski space). 
We put coordinates $ \left( \sigma , \tau  \right)  $
on $ \Sigma $, at least locally. 
And, we can define an embedding of the worldsheet $ \Sigma $ 
in the background spacetime, $ \mathcal{ M } $ (Minkowski space),
by the functions $ X ^{ \mu } \left( \sigma, \tau  \right)  $, 
where the $ X ^ \mu $ are coordinates on 
$ \mathcal{ M }  $. So if you like, we have that 
\[
X : \Sigma \to \mathcal{ M }
\] Why is this an embedding? 
A choice $ \sigma $ and $ \tau $ on $ \Sigma $ gives 
us a location in Minkwoski space. 
There are rules (which we shall investigate), 
for gluing such worldsheets together 
in a way which is consistent with 
the symmetries of the theory. 

So, we can not only describe the embeddings 
of a single string propagating through space-time, 
but multiple strings coming together. 

(Draw a diagram of two tubes merging into one tube, 
then splitting back again into two tubes - this looks like 
a three point vertex)

We shall see that such diagrams like the above 
are in one-to-one 
correspondence with correlation functions in some 
quantum theory. 
It is natural to interpret such diagrams 
as Feynman diagrams in a perturbative expansion 
of some theory about a given vacuum. 

So what we have is a way of understanding 
writing down Feynman rules, and performing 
successively better approximations to 
an exact result in field theory which we don't have. 

So what we have here will turn out to be Feynman rules for a theory 
which we don't yet have. Where do these Feynman rules come from? 
Quantising is tremendously restrictive. 


\section{The Classical Particle and String}
In non-relativistic Q, we treat 
time $ \left( t  \right)  $ as a parameter 
and position $ \hat{ X } ^ i $ as an operator. 
Obviously, this kind of restiriction shouldnt 
survive very long in a relativistic string theory. 
So, there are choices to be made here. 
One of those choices is second quantisation.
Second quantisation is when both $ X ^ i $ and $ t $ 
are parameters. Then, 
we quantise fields, for example $ \phi \left( x, t  \right)  $ 
which are the fundamental objects of interest in our 
theory. 
We of course require that the fields 
transform in an appropriate way under field transformations. 
This is what we do in QFT. Most of what we know for example, 
in the standard model, omes from this approach. 

However, there is another way. 
This is \textbf{first quantisation}. 
We elevate $ t $ to be an operator, 
and have something else in the background. 
This is a natural framework for 
describing the relativistic 
embedding for worldline, worldsheet
or worldvolume 
in a spacetime.

And here, $ X ^ \mu  = \left( X ^ i , t  \right)   $ 
is an operator, which is the fundamental 
object which we quantise (our basic variable), 
and we have some other natural parameter 
entering the theory. 
We'll look at concrete examples of what that parameter 
is through this section. 

This other possibility, 
where we think of our fundamental degrees of 
freedom as some object embedded into spacetime, 
is the path we'll take in string theory. 
This is because this approach has been very successful. 

There is string field theory however, which takes the first 
approach, but this leads to most of the results 
of first quantised string theory. 

\subsection{Worldlines and Particles}
Suppose we want to take this approach 
with a particle. 
We consider an embedding 
of a worldline $ \mathcal{ L } $ 
into spacetime $ \mathcal{ M } $. 
We assume zero curvature. 
The basic field is the embedding $ X ^ \mu : 
\mathcal{ L } \to \mathcal{ M } $ 
and an action might be 
\[
S \left[  X  \right]   =  -m \int _{ x_0 } ^{ x_1 } ds  =  -m 
\int_{ \tau_1 }^{ \tau_2 } d \tau \sqrt{  - \eta _{ \mu \nu } \dot{ X  }^ \mu \dot{  X} ^ \nu   } 
\] (Insert diagram of line connecting nodes $ x_ 1 ^ \mu $ and $ x _ 2 ^ \mu $  )
where $ \tau $ (a parameter) is the proper time 
and 
\[
X ^ \mu \left( \tau _ 2  \right)   =  x _ 2 ^ \mu , \quad X ^ \mu \left( \tau _ 1  \right)   = x_1 ^ \mu  
\] are endpoints of the worldline. 
It makes sense 
that our action 
should be proportional to 
the length of the worldline. 
So, a reasonable guess for our action. 
We're taking our space-time metric as $ \left(  -, + , + , +  \right)  $. 
This seems like a reasonable starting point. 

The constant $ m $ has dimensions of mass, 
so a good guess is that this parameter 
is interpreted as the mass. 
We can do some things with this action. 
We can first compute the 
conjugate momentum
to $ X ^ \mu \left( \tau  \right)  $, which is 
\[
P _ \mu \left(\tau   \right)   =  - m \frac{ \dot{ X  }^ \mu   }{ 
\sqrt{  - \dot{ X} ^ 2   }  } , \quad \left( \dot{ X ^ 2 }  = \eta_{ \mu \nu } 
\dot{X } ^ \mu \dot{ X } ^ \nu   \right) 
\] This satisfies $ P ^ 2 + m ^ 2  =  0 $. 
This is what we call an 'on shell' condition. 
There are two symmetries 
associated with this action. 
\begin{itemize}
\item  We have a rigid symmetry, where 
	\[
	 X ^ \mu \left( \tau  \right)  \to \Lambda \indices{ ^ \mu _ \nu } 
	 X^ \nu \left( \tau  \right)  +  a ^ \mu 
 \] where $ \Lambda \indices{ ^ \mu _ \nu }  $ is a Lorentz transformation 
 matrix and $ a ^ \mu $ is a constant displacement. 
\item Also, this action has 
 re parametrisation invariance. In other words, 
 in the physical variable of the $ x $ 's, $ \tau $ is 
 just a parameter which measures the distance along the line. 
 So, we can replace it. 
 If we take 
 \[
	 \tau \to \tau + \xi \left( \tau  \right)  
 \] The embedding $ X ^ \mu $ changes as 
 \[
  X ^ \mu \left( \tau  \right)  \to 
  X ^ \mu \left( \tau + \xi  \right)   = X ^ \mu \left( \tau  \right)  
  + \xi \dot{ X } ^ \mu \left( \tau  \right)  + \dots  
\] To first order, we have that $ \delta X ^ \mu \left( \tau  \right)  
\xi \dot{ X } ^ \mu \left( \tau  \right)   $. 
\end{itemize} 

There's a rewriting 
of this action which makes life a little bit easier. 
Specifically, the action above is 
hard to interpret in the massless case. 
We can rewrite the action 
as \[
X\left[  X , e  \right]    = \frac{1}{2 } \int 
d \tau \, (  e ^{ -1 } \eta _{ \mu \nu } \dot{ X } ^ \mu \dot{ X} ^ \nu   
- e m ^ 2 ) 
\] There are no square roots, and we can take the massless limit. 
We will show that this new action is equivalent to 
the one we wrote down earlier. 
$ e \left( \tau  \right)  $ is some new field on the worldline. 
If you like, you might want to 
think of $ e $ as some one-dimensional metric which 
sets the scale of distances on the line. 

The equations of motion for $ x ^ \mu \left( \tau  \right)  $ 
and $ e \left( \tau  \right)  $ are as follows 
\[
\frac{d }{ d \tau } \left( e ^{ - 1 } \dot{ X }   \right)  = 0 
\] We notice that interestingly, $ e $ does not appear with a time 
derivative. So its equation of 
motion is purely algebraic. If you like, you 
can think of $ e $ as being a lagrange multiplier 
for every single point $ \tau $ on the worldline. 
The $ e $ equation of motion is 
\[
\dot{ X } ^ 2 + e ^ 2 m ^ 2  = 0  
\] $ e \left( \tau  \right)  $ enters 
algebraically 
and it can be thought of 
as a constraint! 
The momentum conjugate to $ X ^ \mu $ 
is 
\[
P _ \mu  = e ^{ - 1 } \dot{ X } ^ \mu  
\] If we combine this with the algebraic constraint
for $ e $, we can combine this with the mass shell 
condition 
to get $ P ^ 2 + m ^ 2  = 0 $. 
So interestingly, this auxillary field $ e $ 
imposes a constraint but is equivalent to 
the space-time energy momentum condition. 

We can write $ e ^{ - 1 }  = \frac{ m }{ | \dot{ X }  |  } $, 
plug this into the action 
to find $ S \left[  X , e  \right]  $ 
subject to the equations of motion 
for $ e \left( \tau  \right)  $ gives 
precisely the action 
\[
S\left[  X  \right]   =  - m \int_{ C } \sqrt{  - \eta _{ \mu \nu } \dot{ X } ^ \mu 
\dot{ X } ^ \nu  }  
\] What guarantees that $ e ^{ - 1 }  $ exists? 
Well, a priori nothing. But, we 
can motivate $ e $ coming from 
the interpretation as being a metric, 
which is invertible. 

With a bit more work, 
we can argue that the $ m $ goes to 
$ 0 $ limit gives 
us the description for null light. 
This action is overall a lot nicer. 

The action $ S \left[  X , e  \right]  $ has the 
symmetries 
\begin{itemize}
\item  Poincare invariance, where $ e $ is 
	invariant. 
\item This also has re-parametrisation 
	invariance, but since $ e $ depends on $ \tau  $, 
	it also has to transform. 
	Infinitesimally, 
	\[
	 \delta X ^ \mu  = \xi \dot{ X } ^ \mu , \quad \delta e  = \frac{ 
	 d }{ d \tau } \left( \xi e  \right)  
	\] provided these variations vanish on the endpoints. 
	$ e $ is not a scalar function on the worldline, 
	but this is the natural way to choose how it 
	transforms so that the action is invariant. 
\end{itemize}
we have a couple of comments. 
The first thing we could to is add curvature 
to our spacetime. 
We could generalise $ \eta _{ \mu \nu } \to g _{ \mu \nu } \left( X \left( \tau  \right)   \right)  $. Then, this becomes 
a highly non-linear model. 

\subsection{Classical Strings}
\subsubsection{Nambu-Goto Action}
The Nambu-Goto action 
is the analog of the action above 
but for a string. 
(Diagram of cylinder with open ends with axes $ \sigma , \tau $ ). 
The fundamental degree of freedom is 
\[
X : \Sigma \to \mathcal{ M } 
\] In this context, we refer to the object which 
the sheet is embedded into as the target space. 
Often, the thing we're embedding into may not 
be space-time for historical reasons. 
The Nambu-Goto action 
is the proposed generalisation 
so that for $ X ^ \mu \left( \sigma , \tau  \right)  $, we have 
\[
S \left[  X  \right] =  - \frac{1}{2 \pi \alpha '  } 
\int d \sigma d \tau \sqrt{  - \det \left( \eta _{ \mu \nu } \partial  _ a X ^ \mu 
\partial  _ b X ^ \nu \right) } 

\] This is an action which 
is proportional to the area 
spread out by the wordsheet. 
$ \alpha  ' $ is a historically labelled 
constant with dimensions of area as measured in 
spacetime. 
One often speaks of the string length $ l _s = 2 \pi \sqrt{ \alpha ' } $. 
We introduce the string tension $ T  = \frac{1}{ 2 \pi \alpha ' } $, 
where we assume throughout that $ \hbar = 1  =c  $. 
These are some characteristic scales in the 
theory. 

The usual sort of issues from Nambu-Goto 
are similar to the issues we faced from 
the original worldline action. 
A much better starting point 
for us is the Polyakov action. 
We place this game of removing the 
square root at the price of introducing an 
extra non-dynamical field. 
\[
S\left[  X, h   \right] =  - \frac{1}{ 4 \pi \alpha ' } 
\int _{ \Sigma } d ^ 2 \sigma \sqrt{  -h }  h ^{ ab } \eta _{ \mu \nu } 
\partial  _ a X ^ \mu \partial  _ b X ^ \nu 
\]  Our entire lecture course 
will start from the 
quantisation of this. 
$ h _{ ab } $ is a metric on  $ \Sigma $ and is 
non-dynamical  - there are no terms 
involving derivatives $ h$ - it is 
merely a constraint field like $ e $ was. 
We will find however that $ h $ plays 
an important role. 

If we remember $ h $ as a metric, 
this is a two dimensional quantum Klein-Gordon 
field which is massless and in $ 2 $ dimensions. 
if we treat this as a two dimensional quantum field theory, 
there are other terms which we may want to add. 

Let's have a look at 
some equations of motion 

...

\subsection{Classical Hamiltonian Dynamics of the String}
Today, we're 
going to be interested in the quantisation 
of our closed bosonic string. 
To a first approximation, 
the canonical quantisation theory is 
quite straight forward. 

We're going to continue to
work in what we're going to 
call conformal gauge. 
This is when we take the metric 
on the worldsheet to take 
the form 
\[
h _{ ab }  = e ^{ \Phi } \begin{pmatrix}  
- 1 ^ 0 \\ 0 & 1 \end{pmatrix} 
\] So we have a sort of natural notion of time. 
We can then define the 
canonical momentum field 
conjugate to $  X^ \mu $ 
in the usual way. 
This is just 
\[
P _ \mu \left( \sigma, \tau  \right)   = \frac{\delta S \left[  X  \right]  }{ 
\delta X ^ \mu \left( \sigma, \tau  \right)  }  = \frac{1}{2 \pi \alpha   ' }\dot{ X } _ \mu   
\] We can also do the usual stuff 
and write down the Hamiltonian density. 
Given the Lagrangian density 
$ \mathcal{ L } $, the Hamiltonian 
density is 
\[
\mathcal{ H }  = P _ \mu \dot{ X } ^ \mu  - \mathcal{ L }  = \frac{1}{4 \pi \alpha  ' }
\left( \dot{ X } ^ 2 + X ^{ ' 2 }   \right)  
\]  Recall that the 
dot derivative is the derivative 
with respect to $ \tau $, and the prime is the 
derivative with respect to $ \sigma $. 
It is always useful when looking at 
Hamiltonian dynamics 
to define the Poisson brackets. 
We introduce the bracket as $ \left\{  ,  \right\}  _{ P B } $. 
In particle theory, 
where our coordinates 
$ x ^ \mu \left( \tau  \right)  $ and 
momenta $ p _ \mu \left( \tau  \right)  $ are 
on fundamental variables, it is useful to define the following. 
\[
\left\{  f, g,  \right\}  _{ P B }  = 
\frac{\partial  f }{\partial  X ^ \mu }  \frac{\partial  g }{\partial  p_ \mu }  
- \frac{\partial  f }{\partial  p _ \mu }  \frac{\partial  g }{\partial  x^ \mu } 
\]  So, for example, we have that 
$ \left\{ x ^ \mu , p _ \nu  \right\}   = \delta \indices{ ^ \mu _ \nu }  $. 
The Hamiltonian $ \mathcal{ H } $ plays the 
role as a generator 
of time translations. For example, if we 
have some function of $ x, p $ which is 
\[
\frac{df }{ d \tau }  = \frac{\partial  f }{\partial \tau } + \left\{  f, H  \right\}   
\] which can be shown from the chain rule. 
Here, we're doing something slightly more 
general because $ x , p $ depends 
two parameters, not just the $ \tau $. 
So, if you like, this 
is a field theory generalisation. 
Our field theoretic generalisation 
requires that 
\[
\left\{ X ^ \mu \left( \sigma, \tau  \right) , P _ \nu 
\left( \sigma ' , \tau  \right)  \right\}    = \delta \indices{ ^ \mu _ \nu } 
\delta \left( \sigma - \sigma  '  \right) 
\]   This is a precurcsor 
for equal time commutation relations. 
This is a nice construction! 
If we recall the form of $ X ^ \mu \left( \sigma, \tau  \right)  $ 
to be written in terms of fourier modes $ \alpha^ \mu _ n  $
and $ \overline{ \sigma } ^ \mu _ n $. 
We have the 
mode expansion 
\[
X^ \mu \left( \sigma, \tau  \right)   = x ^ \mu 
+ \sqrt{ \alpha  ' }  p ^ \mu \tau  +  i \sqrt{ \frac{\alpha ' }{ 2 } }  
\sum_{ n \neq 0 } \frac{1}{n } \left( 
\alpha ^ \mu _ n e ^{   -in \left( \tau  - \sigma  \right)  } + 
\overline{ \alpha } ^ \mu _ n e ^{  - in \left( \tau + \sigma  \right)  } \right) , 
\quad \sigma \sim \sigma + 2 \pi 
\]  So imposing 
the natural commutation relations on 
$ X ^ \mu $ and $ P ^ \mu $, gives 
us natural Poisson bracket relations 
for $ \alpha , \alpha ' $. 
This requires 
\[
\left\{  \sigma _ \mu ^ \nu , \sigma _ n ^ \nu  \right\}  _{ P B } 
= - i m \eta ^{ \mu \nu } \delta _{ m + n , 0 }, \quad 
\left\{  \alpha ^ \mu _ m , \overline{ \alpha } ^ \nu _ n  \right\}  _{ P B }  = 0 , \quad 
\left\{  \overline{ \alpha } ^ \mu _ m , \overline{ \alpha } ^ \nu _ n  \right\}  _{ P B } 
=   - i m \eta ^{ \mu \nu } \delta _{ m + n , 0 }
\] The left hand side 
vanishes unless $ n  =- m $. Why are we considering closed strings? 
Apparently, they make life easier later, 
and closed strings give rise to gravity. 
Let's see if this is plausible. 
The relation above is for equal times, 
and the $ X , P $ commutation relation 
is valid for equal times, 
so let's check this holds for $ \tau  = 0$. 
We'll see later on that this 
choice of $ \tau $ is in actual fact not a special case. 
Just for simplicity however, we'll do it this way. 

Our string looks like, at $ \tau  =0 $, 
\[
X ^ \mu \left( \sigma  \right)  = x ^ \mu 
+ i  \sqrt{ \frac{ \alpha ' }{ 2 }}  \sum _{ n \neq  0 } \frac{1}{n } \left( 
\alpha ^ \mu _ n e^{ i n \sigma } + \overline{ \sigma } ^ \mu _ n e ^{  - i n \sigma } \right)
\] Our conjugate momenta is then  
\[
P _ \mu \left( \sigma  \right)   = \frac{p ^ \mu }{ 2 \pi } 
+ \frac{1}{ 2 \pi } \sqrt{  \frac{1}{2 \alpha  ' }}  \sum _{ n \neq  0 } \left( 
\alpha _ n ^ \mu e ^{ i n \sigma } + \overline{ \alpha } _ n ^ \mu e ^{  - i n \sigma } \right)
\] 
Computing the commutation relation from this, 
we have that 
\begin{align*}
\left\{  X ^ \mu \left( \sigma  \right)  , P ^ \nu  \left( \sigma '  \right)   \right\}  _{ P B } &  = \frac{1}{2 \pi } \left\{  x ^ \mu , p^ \nu   \right\}  + 
\frac{i}{4 \pi } \sum _{ m , n \neq 0 } \frac{1}{m } 
\left( \left\{  \alpha _ m ^ \mu , \alpha _ n ^ \nu  \right\}  
e ^{ i \left( m \sigma + n \sigma  '  \right)   } + 
\left\{  \overline{ \alpha } _ m ^ \nu , \overline{ \alpha } _ n ^ \nu  \right\}  
_{ P B } e ^{  - i \left( m \sigma + n \sigma  '  \right)  }\right) \\
&= \frac{\eta ^{ \mu \nu } }{ 2 \pi } + \frac{\eta ^{ \mu \nu } }{ 2 \pi } 
\sum _{n \neq  0 } e ^{  i n \left( \sigma - \sigma '  \right)   }  \\
&=  \frac{\eta ^{ \mu \nu }}{ 2 \pi } \sum _{ n  } e ^{ i n \left( \sigma - \sigma '  \right)  
\end{align*}
But, we have that $ \frac{1}{2 \pi } \sum _{ n } e ^{ i n \left( \sigma - \sigma '  \right)  } $ is just a periodic version of the Dirac delta function. 
Thus, we get 
exactly what we're looking for. 
So, 
\[
\left\{  X ^ \mu \left( \sigma  \right) , P ^ \nu \left( \sigma  '  \right)   \right\} _{ P B }   = \eta ^{\mu\nu 	 } \sigma \left( \sigma  - \sigma '  \right)  
\] 
The other way to go about 
doing this is to find $ \alpha , \overline{\alpha } $ 
in terms 
of integral expressions, and 
then use this to prove the commutation relations 
the other way around. 

\subsection{The Stress Tensor and Wit Algebra}
Introduce the 
worldsheet lightcone coordinates 
\[
\sigma ^{ \pm }  = \tau \pm \sigma 
\] which are a  natural choice of 
coordinates in this gauge to 
help us understand the structure 
of this space a little but more. 

In these coordinates, and this gauge, the 
worldsheet metric 
now looks like 
\[
h  =e ^{ \phi } \begin{pmatrix}   0 & \frac{1}{2 } \\
\frac{1}{2 } & 0 \\ \end{pmatrix} 
\] and $ \partial  _{ \pm }  = \frac{\partial  }{\partial  \sigma ^{ \pm } }   $. 
The action and equations of motion 
becomes 
\[
S =  - \frac{1}{2 \pi \alpha ' } \int_{ \Sigma } d \sigma ^ + d \sigma ^  - 
\partial  _ + X \cdot  \partial  _ - X , \quad \partial  _+ \partial  _ - X ^ \mu  =0 
\] The stress 
tensor $ T _{ ab } $ becomes in these coordinates 
\[
T _{ ++ }   =  - \frac{1}{\alpha ' } \partial _ + X \cdot  \partial _ + X 
, \quad T _{ -- }  = - \frac{1}{ \alpha ' } \partial _ - X \cdot  \partial  _ - X , 
\quad T _{ + - }   = T _{ - +  } =0 
\] So we have two fields on the worldsheet, 
the $ X  $s, and the metric which encodes 
the vanishing stress tensor constraint. 

The constraint is  $ T _{ \pm \pm }  =0 $. 
Let's 
pause a little and think
about what these constraints 
might look like in 
terms of these modes. 
It;s going to be very useful to introduce 
the Fourier modes of $ T _{ \pm \pm } $. 
We define at $ \tau  = 0$, the charges 
\[
L _ n  = - \frac{1}{2 \pi } \int_{ 0 } ^{ 2 \pi } d \sigma T _{ -- } \left( \sigma  \right)  e ^{  - i n \sigma } , \quad 
\overline{ L } _ n  = -\frac{1}{2 \pi } \int _ 0 ^{ 2 \pi } d \sigma 
T _{ ++ } \left( \sigma  \right)  e ^{ i n \sigma }
\] We're just starting out to take $ \tau  = 0 $, 
but we'll see later that this doesn't matter. 
What do these modes look like in terms of $ \alpha $ ? 
We'll choose to explore one of them in detail, 
with the awareness that the other moving mode 
will have similar properties. 

If we differentiate $ X ^ \mu $ as 
\[
\partial  _ - X ^ \mu \left( \sigma , \tau  \right)    
\sqrt{ \frac{ \alpha  ' }{ 2 }  }  \sum _{ n } \alpha _ n e ^{  - i n \sigma ^{  -}  } , \quad \alpha _ 0 ^{ \mu }  = \sqrt{ \frac{ \alpha '  }{ 2 } }  p ^ \mu 
\] We find 
that 
\begin{align*}
L_ n  &=  \frac{1}{2 \pi \alpha  ' } \int _ 0 ^{ 2 \pi } 
d \sigma \partial  _ - X ^ \mu \left( \sigma  \right)  \cdot  \partial  _  - 
X _ \mu \left( \sigma  \right)  \\ 
&=  \frac{1}{4 \pi } \sum _{ m , p } \alpha _ m \cdot  \alpha _ p \int 
_ 0 ^{ 2 \pi } d \sigma e ^{  - i \left( m + p - n  \right)  \sigma } \\
&=  \frac{1}{ 4 \pi } \sum _{ m , p } \alpha_ m \cdot  \alpha _ p 
2 \pi \delta _{ p , n  -m } 
\end{align*}
Similarly, this 
holds for $ \overline{ L } _ n $. 
This gives us the 
relation that 
\[
L _ n  = \frac{1}{2 } \sum _ m \alpha _{ n  - m } \cdot   \alpha _ m , 
\quad \overline{ L } _ n  = \frac{1}{2 } \sum _ m \overline{ \alpha } _{ n - m } \cdot  
\overline{ \alpha } _m 
\]  This constraint can be written 
as $ L _ n  = 0  = \overline{ L } _ n $. 
Using the algebra for the $ \alpha _ n ^ \mu $ $ \left( \overline{ \alpha } _ n ^ \mu  \right)  $, we can compute the algebra for $ L _ n \left( \overline{ L } _ n  \right)  $. 
It's 
not to hard to show that the algebra 
of these objects obey 
the algebra 
\[
\left\{  L _ m , L _ n  \right\}   = - i 
\left( m - n  \right)  L _{ m + n }, \quad 
\left\{  L _ m , \overline{ L } _ n  \right\}   = 0 , 
\quad \left\{  \overline{ L } _ m , \overline{ L } _ n  \right\}   = - i \left( m - n  \right)  \overline{ L } _{ m + n }
\] 
This is often called the Witt algebra. 
We shall see, when we promote this 
to quantum operators, 
we see something very similar, but with an important difference. 
We will see that if we set $ L _ n  = 0  = \overline{ L } _n $, 
at a given  $ \tau $ , then 
the evolution of the system 
preserves $ L _ n   = 0  = \overline{ L } _ n $. 
So what we find, underlying the constraints of 
this theory, that there is this underlying 
infinite dimensional symmetry. 

\subsection{A First Look at the Quantum Theory}
Given the work we've done
on the classical theory, quantising 
this to build a theory on Hilbert space will be a straightforward extension. 

Recall from earlier, that we decided to work with the 
Plyakov action.
We transformed our metric locally due to Weyl rescaling, 
which gives us a two dimensional massless Klein-Gordon theory. 
But, as a result, we 
get constraints from the stress-energy tensor. 

As we did in quantum field theory, 
there are two ways we can proceed in imposing 
the constraints. 

\begin{itemize}
\item We can constrain states then quantise, which 
is an approach that has been reasonably successful
(this is called light-cone quantisation). However, we 
will not be following this path of quantisation. 
\item The second approach is 
to quantise the unconstrained 
theory, then impose constraints 
as a physical condition 
on our Hilbert space. For specifically, recall that the only algebraic 
constraint we got from earlier was our condition on our stress energy tensor, $ T_{ab }  = 0 $. 
Thus, this is the only thing we will impose on our Hilbert space. 


\end{itemize}

\subsubsection{Canonical Quantisation}
We quantise by 
replacing out Poisson brackets 
with commutators. So, functions on phase 
space for example, are replaced by operators. 
We do the transformation 
\[
 \left\{  , \,   \right\}  _{ P B  }  \to  - i \left[  , \,  \right] 
\] So, the structure 
of quantum mechanics is 
similar to Hamiltonian dynamics. 
These give rise to equal time comuttation 
relations: 
\[
	\left[  X ^ \mu \left( \sigma \right) , X ^ \nu \left( \sigma '  \right)    \right]   = 0 , \quad \left[  P _ \mu \left( \sigma  \right)  , 
	P _ \nu \left( \sigma '  \right)   \right]   = 0 , \quad 
	\left[  P _ \mu \left( \sigma  \right)  , 
	X ^ \nu \left(  \sigma '  \right)  \right]   =  - i \delta 
	\indices{ _ \mu ^ \nu  } \delta \left( \sigma - \sigma '  \right)   
\] Now, recall the mode 
expansions we get from expanding 
our position operator $ X ^ \mu $, 
\begin{align*}
	X ^ \mu \left( \sigma , \tau  \right)   &= 
 x ^ \mu +  \alpha' p ^ \mu \tau + i \sqrt{ \frac{ \alpha ' }{ 2 } }  
 \sum _{ n \neq 0   } \frac{1}{n } \left( \alpha _ \nu ^ \mu 
 e ^{  - i n \left( \tau  - \sigma  \right)  } + \overline{ \alpha } _ n ^ \mu 
  e^{  - i n \left( \tau + \sigma  \right) }\right) \\
  P ^ \mu \left( \sigma, \tau  \right)  &=  
  \frac{p ^ \mu }{ 2 \pi } + \frac{1}{2 \pi  } 
  \sqrt{ \frac{1}{2 \alpha  ' }  }  
  \sum _{ n \neq 0 } \left( \alpha _ n ^ \mu e ^{  - i n\left( \tau + \sigma  \right)  } 
  + \overline{ \alpha } ^ \mu _ n e ^{  - in \left( \sigma - \tau  \right)  } \right) 
\end{align*} One can show that the commutation 
relations for the mode operators $ \alpha $ and $ \overline{ \alpha }  $  are 
consistent which changing our Poisson brackets to 
commutators. This gives our relations as 
\[
 \left[  \alpha _ m ^ \mu , \alpha _ n ^ \nu  \right]   =
 m \delta _{ m + n , 0 } \eta ^{ \mu \nu } , \quad 
 \left[  \alpha _ m ^ \mu , \overline{ \alpha } _ n ^ \nu  \right]   = 0 , 
 \quad \left[  \overline{ \alpha }_ m ^ \mu , \overline{ \alpha } _ n ^ \nu  \right] 
  = m \delta_{ m +  n , 0 }\eta ^{ \mu \nu } 
\] For each 
dimension $ \mu $, these are an 
infinite number of ladder operators, where the annihilation  
operators are given by 
$ \alpha _ n $ for $ n > 0 $. However, by looking at the
fact that $  X ^ \mu $ and $ P ^ \mu $ should be real 
and by comparing coefficients, we find the relations 
\[
 \left( \alpha_ n ^ \mu  \right)  ^{ \dagger }  = \alpha _{  -n } ^ \mu 
\] These are 
creation and annihilation ladder 
operators. 
They are creating and annihilating 
different modes on the string. 
It use necessary to 
introduce a vacuum state on $ \Sigma $, 
$ \ket{ 0 } $ such that 
\[
 \alpha _ n ^ \mu \ket{ 0 }  = 0, \quad n \geq 0 
\] It is 
important to note that this is not a vacuum in spacetime, 
it is a vacuum of vibrational modes. 
We recall the 
Fourier modes of $ T _{ ab } $
are $ L _ n $ and $ \overline{ L } _ n $. 
Just as before, 
we can write those modes in terms of $ \alpha $ and $ \overline{ \alpha } $. 

\[
 L _ m  = \frac{1}{2 } \sum_ n \alpha _{ m - n } \cdot  \alpha _ n 
\] 
We call $ L _ m $ Virasoro 
operators. This expression is 
ambiguous for $ m   = 0 $, 
going from the classical to the quantum theory, 
operator ordering starts to matter.
 This is because $ \alpha _ n $ and $ \alpha _{  - n  } $
 do not commute for $ n \neq 0 $, and hence
 we don't know how to order these in our expression for $ L _ 0 $. 
 In this case, we shall take 
 \[
   L _ 0  = \frac{1}{2 } \alpha _ 0 ^ 2 + \sum _{ n > 0  } \alpha 
   _{ -n  } \cdot  \alpha _ n 
 \] 
 We can adopt the usual notion of 
 normal ordering. 
 There are other normal ordering prescriptions 
 we could use.

 \subsubsection{Physics State Conditions}
 Instead of thinking about 
 the stress tensor and $ X  , P $, it's
 more useful to talk about 
 the Virasoro operators $ L _ i $ to 
 impose $ T _{ ab }  =0 $ on the Hilbert space 
 of our theory. 

 Let's first define 
 some notation to make our lives easier. 
 Define 
 \[
  N  = \sum _{ n > 0 } \alpha _{  -n  } \cdot \alpha _ n ,\quad  
  \overline{ N }  = \sum _{ n > 0 } \overline{\alpha _{ - n }}  \cdot  
  \overline{\alpha} _ n 
 \] These operators have the interpretation of 
 being 'weighted number operators' on physical states. 
 Given this, 
 we can write for example, that
 \[
  L_0 = \frac{\alpha ' }{ 4 } p ^2 + N , \quad  
  \overline{ L } _ 0   = \frac{\alpha '  }{ 4 } p ^ 2 + \overline{ N } , 
  \quad \left( \alpha _ 0 ^ \mu  = \sqrt{ \frac{\alpha '  }{ 2 } }  p ^ \mu  \right) 
 \] We require
 that $ T_{ ab   } \ket{ \phi }  = 0 $ 
 for $ \ket{ \phi } $ to be a physical state. 
 This means that 
 \[
  L _ n \ket{ \phi }   = 0, \quad \text{ for } n > 0 
 \] under Hermitian conjugation, this also implies 
 that $ \bra{ \phi } L _{  - n }  = 0  $. 
 We shall also require 
 \[
  L _ 0 \ket{ \phi }  = a \ket{ \phi } , \quad \overline{ L } _ 0 
  \ket{ \phi }  = a \ket{ \phi } 
 \] allowing for the fact that $ L _ 0 $ 
 may be true up to some constant, which is related to 
 the normal ordering. The constant $ a $ reflects 
 the ambiguity in defining $ L_0 $ in the quantum theory. 
We'll come back as to what values $ a $ should take 
depending on our perspective. For 
now, we should take $ a = 1 $. 
This justification is a posteori. 

For convenience, we also define the new operators 
\[
 L _ 0 ^{ \pm }  = L _ 0 \pm \overline{ L } _ 0 
\] Hence, in terms of our newly defined 
operators, our physical 
conditions are 
\[
 \left( L_0 ^{ +  }  - 2 \right)  \ket{ \phi }  =0, \quad 
 L_ 0 ^ - \ket{ \phi }  = 0 , \quad L_ n \ket{ \phi }  = 0 
  = \overline{ L } _ n \ket{ \phi } \text{ for  } n > 0 
\] The $ L _ n $ are difficult to interpret physically, 
but are related to polarisation 
conditions. The first two conditions 
are related to rotational invariance. 
With these physical constraint conditions, we can start to look at the
spectrum of the system.  

\subsection{The Spectrum}

\subsubsection{The Tachyon}
Now it's time to start constructing 
some actual states in our Hilbert space. 
We can construct a space-time momentum 
eigenstate as 
\[
\ket{ k }  = e ^{  i k \cdot  x } \ket{0} 
\] For now, we just work in the 
context where $ x $ is just a position variable 
and not an operator, like in the expression $ e^{ i k \cdot  X } \ket{ 0 } $. 
However, we will see what this looks like 
at a later stage. 
From quantum mechanics, we know 
how $ p_ \mu $ acts  in the position basis. In terms of a position basis in 
the target space, the 
momentum operator is $  -i \frac{\partial   }{\partial  x ^ u }  
 = p _ \mu $, so 
 $ p _ \mu \ket{ k }  = k _ \mu \ket{ k } $. 
 In addition, $ L _ n \ket{ k }  = 0  = \overline{ L } _ n \ket{ k } $
 straightforwardly, although this 
 has yet  to be shown in the lectures and I'm not sure why 
 this is true. 
 We can write $ L_0 ^ -  $ as 
 \[
 L_0 ^  -  = N - \overline{ N } 
 \] So the vanishing condition 
 $ L_0 ^  - \ket{ \phi }  = 0 $
 suggests a symmetry of 
 right movers versus left movers. 
 This is sometimes called level matching. 
 It is the weighted count difference from 
 either side. 
 If we apply $ N $ or $\overline{ N } $ 
 to any of the states of the form $ \ket{ k } $, 
 we will always find that since $ N $ is the sum 
 of $ \alpha _ n \cdot  \alpha _{ - n } $ , we 
 can always commute the annihilation operator forward.
 $ N  = \overline{ N }  = 0  $. 
 We check that $ \left( L _ 0 ^ +   -2  \right)  \ket{ k } =0   $. 
 Thus, we find that 
 \begin{align*}
	 \left( L_0 ^ +  - 2  \right)  \ket{ k} &=  
	 \left( \frac{\alpha  ' }{ 2 } + N + \overline{ N }  - 2  \right)  \ket{ k } \\
						&=  \left( 
						\frac{\alpha ' }{  2 }
					k ^ 2 - 2  \right)  \ket{ k } = 0  \\
 \end{align*} 
 This gives 
		us the condition on $ k ^ 2  $, as $ k ^ 2   - \frac{4}{\alpha ' }  = 0 $. 
		If we compare this with the energy-momentum condition 
		$ k ^ 2 + M ^ 2   =0  $, 
		with gives 
		\[
		 M ^ 2  =  - \frac{4}{\alpha ' } 
		\]
The state $ \ket{ k } $ has a spacetime interpretation as a tachyon. 
This problem is 
not going to go away. 
We do have the tachyon, and its 
cured by promoting this to the supersymmetric 
string, and adding supersymmetry. 

We'll now look at the next excited state, 
which are the massless states.
Consider states of the form 
\[
 \ket{ \epsilon }  = \epsilon _{ \mu \nu } \alpha _{ -1 } ^ \mu \overline{ \alpha } _{ - 1 } ^ \nu 
 \ket{ k } 
\]  NOw we want to look at the 
condition which allows us to 
view this as a physical state. 
Clearly, $ N = \overline{ N }  = 1 $. 
We can look at the energy momentum condition. 
The condition 
\[
 \left( L_0 ^ +  - 2  \right) \ket{ \epsilon }  = 0 
\] implies $ \frac{\alpha ' }{ 2 } k ^ 2  = 0 $, 
so we require that $ k ^ 2  = 0 $ (null). 
Consider the next condition, which 
gives $ L _ 1 \ket{ \epsilon }   =0 $. 
This condition gives 
\[
	\frac{1}{2 } \left( \sum _ n \alpha _{ 1 - n} \cdot  \alpha _ n  \right)  \epsilon 
	_{ \mu \nu } \alpha _{ - 1 } ^ \mu \overline{ \alpha } _{  - 1  }^{ \nu } \ket{ k } 
	= \epsilon _{ \mu \nu } \overline{ \alpha  }_{ - 1 } 
	^ \nu \alpha _ 0 \cdot  \alpha_ 1 \alpha _{ -  1 } ^ \mu  \ket{ k } 
\] recall that $ \alpha _ 0 ^ \mu  = \sqrt{ \frac{\alpha ' }{ 2 } }  p ^ \mu $. 
We require then 
\[
 k _ \rho \alpha _ 1 ^ \rho \alpha _{ - 1 } ^ \mu \ket{ k }  = 0 
\] we can use our commutator 
to show that the above is 
equal to 
\[
 \epsilon _{ \mu \nu } k _ \rho \left( \left[  \alpha _ 1 ^ \rho , \alpha _{ - 1 } ^ \mu  \right]  + 
 \alpha _{ - 1 } ^ \mu \alpha _ 1 ^ \rho  \right)  \ket{ k }  = \epsilon _{ \mu\nu  } 
 \eta ^{ \mu \rho } k _ \rho \ket{ k }  = \epsilon _{ \rho \nu } k ^ \rho \ket{ k }  = 0
\] 
This condition $ L _ 1 \ket{ \epsilon }   = 0 $ requires 
us to impose $ \epsilon _{ \nu \mu } k ^ \mu  = 0 $. 
In other words, we 
can think of this condition as the fact that 
there are no longitudinal polarisations.
Similarly, $ \overline{ L }_ 1 \ket{ \epsilon }  =0 $  requires 
$ \epsilon _{ \mu \nu } k ^ \nu  =0 $. 
So, 
we have three physical state conditions given to us. 
There are no further conditions on this tensor. 

We then have the conditions on $ \ket{ \epsilon } $.
Our first condition is that it is null and massless, 
so $ k ^ 2  = 0$, and no longitudinal polarisations 
$ \epsilon _{ \mu \nu } k ^ \mu  = 0 $ and $ \epsilon _{ \mu \nu  } k ^ \nu  = 0 $, 
thinking of 
$ \epsilon $ as a polarisation tensor. 


We can decompose $ \epsilon _{ \mu \nu } $ into 
symmetric $ h _{ \mu \nu } $, anti-symmetric ($ b _{ \mu \nu }  $  ), 
and trace $ \phi $ parts. 
We first extract the trace bit, 
which we call the Dilaton. 
\[
 \ket{ \phi }  = \phi \alpha _{ -  1 } ^ \mu \overline{ \alpha } _{ - 1 \mu } \ket{ k }  
\]  

We also have two other particles from this 
\begin{align*}
	\ket{ h }  &=  h _{ \mu \nu  } \alpha _{ - 1 } ^ \mu \overline{ \alpha }_{ - 1 } ^ \nu 
	\ket{ k } \quad \text{Graviton}, \quad h _{ \mu \nu }  = h _{ \nu \mu  } \\
	\ket{ b } &=  b _{ \mu \nu } \alpha _{ - 1 } ^ \mu \overline{ \alpha } _{ - 1 }^ \nu 
	\ket{ k }, \quad \text{B-Field}, \quad b _{ \mu \nu }  = - b _{ \nu \mu } \\
\end{align*}

Now let's look at massive states. 
We can now look at states with $ N = \overline{ N }  = 2 $. 
We have 
\[
 A _{ \mu \nu } \alpha _{ - 2 } ^ \mu \overline{ \alpha } _{ - 2 } ^ \nu \ket{ k } 
 + A_{ \mu \nu \lambda } \alpha _{ - 2 } ^ \mu \overline{ \alpha } _{ - 1 } ^ \nu 
 \overline{ \alpha } _{ -  1 } ^ \lambda \ket{ k } + 
 \tilde{ A }_{ \mu \nu \lambda } \overline{\alpha } _{ - 2 } ^ \mu \alpha _{ - 1 }^ \mu  \alpha _{ - 1}  
 ^ \lambda \ket{ k } + A _{ \mu \nu \lambda \rho } \alpha _{ - 1 } ^ \mu \alpha _{ -1 } ^ \nu 
 \overline{ \alpha  } _{ - 1 } ^ \lambda \overline{ \alpha } _{ -1 } ^ \rho \ket{ k } 
\] and $ N $ and $ \overline{ N  } $ count 
number of quanta going around the string. 
There's no profit 
in us solving the mass shell conditions. 
The mass of such states is $ m ^ 2  = \frac{4}{\alpha ' }  $. 
So the string is describing not only the tahcyon 
and massless fields, but an infinite amount of massive 
fields.

For the most part, string theorists 
have concerned themselves with the massless spectrum. 

\subsection{The Big(ish) Picture}
We started with our Polyakov action 
which describes the embedding of our string in our manifold. 
\[
 S  = -\frac{1}{4 \pi \alpha ' } \int_{ \Sigma } d ^ \sigma \eta _{ \mu \nu } \partial  _ a X ^ \mu 
 \partial  ^ a X ^ \nu 
\] We could deform this 
theory and add a perturbation to the metric. 
This can be achieved by adding a 
small plane wave deformation to the spacetime 
metric.
For example, we might take 
$ \eta _{ \mu \nu } $ and replace it 
by adding some plane wave. 
\[
 \eta _{ \mu \nu } \to \eta _{ \mu \nu } + h _{ \mu \nu  } e ^{ i k \cdot  x } 
\] 
The action changes by 
\[
 \Delta S  =  - \frac{1}{4 \pi \alpha ' } \int_{ \Sigma } d ^ 2 \sigma 
 h _{ \mu \nu } \partial  _ a X ^ \mu \partial  ^ a X ^ \nu e ^{ ikx } 
\] 
For every deformation of the theory, 
there is an associated operator 
\[
 \mathcal{ O }  = h _{ \mu \nu } \partial  _ a X ^ \mu \partial  ^ a X ^ \nu e ^{ ikx  }
\] This is clearly associated with a 
deformation of the spacetime metric. 

In string theory, (2-dimensional CFTs), 
to each operator $ \mathcal{ O } $ that 
corresponds to a physical deformation 
of the theory, (something that seems reasonable 
is what we mean by a physical deformation), 
there is a state in the Hilbert space. 

In this case, 
\[
 \lim_{ \tau \to  - \infty } \mathcal{ O } \ket{ 0 }  = \ket{ h } 
\] We will 
do this more carefully later on. 
This is called the state-operator correspondence. 
One last comment on this. 

What if we choose to start with a general metric?
For example, we could've started off with 
\[
 S  =  - \frac{1}{4 \pi \alpha '  } \int _{ \Sigma } d ^ 2 \sigma 
 g _{ \mu \nu } \left( x  \right)  \partial  _ a X^ \mu \partial  ^a X ^ \nu 
\]  Thhis is highly non-linear. 
How would we deal with this? 
We could try expand this in terms of a power series 
and then try to deal with the theory perturbatively.
This is now an interacting theory, 
and this is hard to do.

Moreoever, the condition of Weyl invariance 
in the quantum theory constrains what $ g _{ \mu \nu }\left( X  \right)   $ 
we can have. 
Weyl invariance is $ h _{ ab } \to e^{ \omega \left( \omega, \tau  \right)  } h _{ ab } $.
One finds that $ G _{ \mu \nu } $ has to satisfy 
\[
	R _{ \mu \nu } \left( g  \right)  + \mathcal{ O } \left( \alpha '   \right)   =0 
\] Which is the Einstein 
tensor up to string corrections. 
More generally, if we have other 
background fields, 
we find Weyl invariance requires 
the full Einstein equations to be satisfied to 
leading order in $ \alpha ' $.

\subsection{Spurious states and gauge invariance}
Before, when we mentioned $ L _ 0 $ to 
have an ordering ambiguity, 
we put our normal ordering constant  as $ a =1 $. 
We took $ a = 1  $ in the 
conditions $ \left( L_0  - a\right) \ket{ \phi }  = 
\left( \overline{ L } _ 0 - a  \right)  \ket{ \phi }  = 0  $. 
Why did we do this? 
This is so that we can interpret states easily. 
Consider the state 
\[
 \ket{ \chi }  = \sqrt{ \frac{2}{\alpha  ' } }  
 \left( \lambda _ \mu \alpha _{ - 1 } ^ \mu \overline{ L } _{  1 } + \tilde{\lambda }_ \mu 
 \overline{ \alpha } _{ - 1 } ^ \mu L _ 1  \right)  \ket{ k } 
\] Clearly, $ \ket{ \chi } $ is orthogonal 
to all physical states. 
If $ \ket{ \phi } \in \mathcal{ H } $, then $ \bra{ \phi }\ket{ \chi }  = 0  $
because $ L_1 \ket{ \phi }  = \overline{ L } _ 1 \ket{ \phi }   =0 $. 
What conditions do $ \lambda _ \mu , \tilde{\lambda }_ \mu  ,k   $ 
have to satisfy for $ \ket{ \chi }  $ to be physical. 

Is is useful to write $ \ket{ \chi } $ as 
\[
	\ket{ \chi }  = \left(  \lambda _ \mu 
	k _ \nu + \tilde{ \lambda } _ \nu k _ \mu  \right)  \alpha _{ - 1 } ^ \mu 	\overline{ \alpha }_{ - 1 } ^ \nu \ket{ k } 
\] Keeping $ a  $ arbitrary, 
we find that 
\[
	\left( L_0^ +  - 2 a  \right) \ket{ \chi }  = 0 \implies $ k ^ 2  = \frac{ 4 \left( 
	a  - 1  \right)  }{ \alpha ' } $
\] We also have that, by symmetry, 
\begin{align*}
	L_ 1 \ket{ \chi } =0 & \text{ if } \left( \lambda \cdot  k  \right)  
	k _ \mu + \tilde{ \lambda } _ \mu k ^ 2  = 0 \\
	\overline{ L } _ 1 \ket{ \chi }  = 0 & \text{ if } \left( 
	\tilde{ \lambda } \cdot  k  \right)  k _ \mu + \lambda _ \mu k ^ 2  = 0 
\end{align*}
Finally, we have that 
\[
	\bra{ \chi }\ket{ \chi }  = \lambda ^ 2 k ^ 2 + 2 \left( 
	\lambda \cdot  k \right)  \left( \tilde{ \lambda } \cdot  k   \right)  + \tilde{ \lambda } ^ 2 k ^ 2  
\] If $ a = 1 $, $ k ^ 2  = 0  $ , $ \lambda \cdot  k , \tilde{ \lambda } \cdot  k  =0  $, 
so $\bra{ \chi }\ket{ \chi } =0 $. 
\subsubsection*{Worldline actions}
\begin{itemize}[leftmargin=*]
	\item Our action is 
		\[
			S =  - m \int_{ s_1 } ^{ s_2 } ds 
			 = - m \int_{ \tau_1 } ^{ \tau_2 } d \tau \sqrt{ -
			 \eta ^{ \mu \nu } \dot{ x } ^ \mu \dot{ x } ^ \nu   } 
		\] 
	\item We have conjugate momenta with on-shell mass condition  
		\[
		 P^ \mu = -  \frac{m \dot{ x } ^ \mu  }{\sqrt{  - \dot{ x } ^ 2  }  }, 
		 \quad  P ^ 2 + m ^ 2 = 0 
		\]  
	\item It makes more sense to work with Einbeins,
		since we can work in the $ m \to 0 $ limit 
	\[
		S  = \frac{1}{2 } \int d\tau\, \left( e^{ - 1} 
		\eta _{ \mu \nu } \dot{ x } ^ \mu \dot{ x } ^ \nu   - e m ^ 2   \right)  
	\] 
\item Two equations of motion come from the Einbein action 
	\[
		\frac{d }{ d \tau } \left( e ^{ - 1} \dot{ x } ^ \mu   \right)   = 0 , \quad 
		\dot{ x } ^ 2 + e ^ 2 m ^ 2 = 0  
	\]
\item This has symmetries 
	\[
	 \delta x ^ \mu  = \xi \dot{ x } ^ \mu , \quad \delta e  = \frac{ d }{ d \tau } 
	 \left( \xi \dot{ e }   \right) 
	\]
\item In the massless limit, if we replace our Minkowski metric 
	with a general metric, we recover the geodesic equations 
	\[
	 S  = \frac{1}{2 } \int d \tau  e^{ - 1} g_{ \mu \nu } \dot{ x } ^ \mu \dot{  x } ^ \mu , 
	 \quad \ddot{x } ^ \mu + \Gamma ^ \mu _{ \alpha \beta }  \dot{ x } ^ \alpha 
	 \dot{ x } ^ \beta  = 0 
	\]  
\end{itemize}

\subsubsection*{Strings}
\begin{itemize}
	\item A string is two dimensional, 
		embedded with parameters $ \sigma, \tau $
		\[
			X^ \mu  \left( \sigma, \tau  \right)   = X^ \mu 
			\left(  \sigma + 2 \pi n , \tau  \right) , \quad 
			n \in \mathbb{ Z } 
		\]
	\item Our associated action is the Nambu-goto action 
	\[
		S[X] =  - \frac{1}{ 2 \pi \alpha ' }\int d \sigma d \tau 
		\sqrt{  - \det\left( \eta_{ \mu \nu } 
		\partial  _ a X ^ \mu \partial  _ b X ^ \beta \right) } 
	\] 
\item We add an extra degree of freedom $ h _{ ab } $ to introduce 
	the Polyakov action 
	\[
	 S[X, h ] = - \frac{1}{4 \pi \alpha ' } 
	 \int d ^ 2 \sigma \sqrt{  - h }  h ^{ ab } \eta _{ \mu \nu } 
	 \partial  _ a X ^ \mu \partial  _ b X ^ \nu 
	\]  
\end{itemize}


\subsubsection*{Quantising the Bosonic String}

\begin{itemize}
	\item Expand out position and conjugate 
		momenta in terms of Fourier modes
	\item We use our gauge invariance to 
		set 
		\[
			h_{ ab }  = e ^{ \phi } \begin{pmatrix}  - 1 &  0 
			\\ 0 & 1 \end{pmatrix} 
		\] The $ - 1 $ in the diagonal gives us a rough notion 
		of time. 
	\item Put this in our Polyakov action to get the 
		conjugate momenta, which is 
		\[
			P_\mu  = \fdv{S}{\dot{ X } ^ \mu  }
			 = \frac{1}{2 \pi \alpha '  } \dot{ X } ^ \mu  
		\]  
	\item Impose equal time commutation relations 
		which are equivalent to commutation 
		relations of the Fourier components. We use Poisson 
		brackets for this
		\[
		 \left\{ X ^ \mu, P_ \nu  \right\}  = 
		 \delta \indices{ ^ \mu _ \nu }  = \delta \left( \sigma  - \sigma '  \right) \iff \left\{ \alpha_ n , \alpha_ m  \right\}   = 
	 - i m \delta_{ n + m , 0 } , \quad \left\{  \overline{ \alpha }_ n, 
 \overline{ \alpha } _ m  \right\}   = - i m \delta _{ n + m , 0  } \]
	
	\item We can change coordinates to 
		\[
		 \sigma_{ \pm }  = \tau \pm \sigma 
		\] 
\end{itemize}
\section*{Example Sheet 1}

\subsection{Question 1}
Here we are showing equivalence 
of the Nambu-Goto action and the Polyakov action. 
\end{document} 
