\documentclass[11pt, oneside]{article}   	% use "amsart" instead of "article" for AMSLaTeX format
\usepackage[margin = 1.1in]{geometry}            		% See geometry.pdf to learn the layout options. There are lots.
\geometry{letterpaper}                   		% ... or a4paper or a5paper or ... 
\usepackage[parfill]{parskip}    		% Activate to begin paragraphs with an empty line rather than an indent
\usepackage{graphicx}				% Use pdf, png, jpg, or eps§ with pdflatex; use eps in DVI mode
								% TeX will automatically convert eps --> pdf in pdflatex	
\usepackage{adjustbox}	
\usepackage[section]{placeins}


%% LaTeX Preamble - Common packages
\usepackage[utf8]{inputenc}
\usepackage[english]{babel}
\usepackage{textcomp} % provide lots of new symbols
\usepackage{graphicx}  % Add graphics capabilities
\usepackage{flafter}  % Don't place floats before their definition
\usepackage{amsmath,amssymb}  % Better maths support & more symbols
\usepackage[backend=biber]{biblatex}
\usepackage{amsthm}
\usepackage{bm}  % Define \bm{} to use bold math fontsx
\usepackage[pdftex,bookmarks,colorlinks,breaklinks]{hyperref}  % PDF hyperlinks, with coloured links
\usepackage{memhfixc}  % remove conflict between the memoir class & hyperref
\usepackage{mathtools}
\usepackage[T1]{fontenc}
\usepackage[scaled]{beramono}
\usepackage{listings}
\usepackage{physics}
\usepackage{tensor}
\usepackage{simplewick} 
\usepackage{tikz} 
\usepackage{import}
\usepackage{xifthen}
\usepackage{pdfpages}
\usepackage{transparent}
\usepackage{pgfplots}
\usepackage[compat=1.1.0]{tikz-feynman}
\usepackage{subfiles}
\usepackage{simpler-wick}
\usepackage{slashed}
\usepackage{fancyhdr}
\usepackage{enumitem}

\pagestyle{fancy}
\fancyhf{}
\rhead{Notes by Afiq Hatta}
\lhead{Standard Model}
\rfoot{Page \thepage}

%% Commands for typesetting theorems, claims and other things.
\newtheoremstyle{slanted}
{1em}%   Space above
{.8em}%   Space below
{}%  Body font
{}%          Indent amount (empty = no indent, \parindent = para indent)
{\bfseries}% Thm head font
{.}%         Punctuation after thm head
{0.5em}%     Space after thm head: " " = normal interword space;
{}%         \newline = linebreak
{}%          Thm head spec (can be left empty, meaning `normal')

%% Commands for typesetting theorems, claims and other things. 

\theoremstyle{slanted}
\newtheorem{theorem}{Theorem}
\newtheorem*{thm}{Theorem}
\newtheorem*{claim}{Claim}
\newtheorem{example}{Example}
\newtheorem*{defn}{Definition}

\newcommand{\Lagr}{\mathcal{L}} 
\newcommand{\vc}[1]{\mathbf{#1}}
\newcommand{\pdrv}[2]{\frac{\partial{#1}}{\partial{#2}}}
\newcommand{\thrint}[1]{\int d^3 \vc{x} \left( {#1} \right)}

%% QFT specific macros 
\newcommand{\intp}{ \int \frac{ d^3 p }{ (2 \pi)^3 } \, }
\newcommand{\ann}[1]{a_{ \mathbf{ #1 }}}
\newcommand{\crea}[1]{a^\dagger_{ \mathbf{ #1 }}}
\newcommand{\ve}[1]{ \mathbf{ #1 } } 
\newcommand{\mode}[ 1]{ e^{ i \mathbf{ #1 } \cdot \mathbf{x} }}
\newcommand{\nmode}[1]{ e^{  - i \mathbf{ #1 } \cdot \mathbf{x} }}
\newcommand{\freq}[1]{\omega_\mathbf{ #1} } 
\newcommand{\scal}[1]{\phi ( \mathbf{ #1 })} 
\newcommand{\mom}[1]{ \pi (\mathbf{ #1 })} 
\newcommand{\arr}{\rightarrow} 

\newcommand{\incfig}[1]{%
\def\svgwidth{\columnwidth}
\resizebox{0.75\textwidth}{!}{\input{./figures/#1.pdf_tex}}
}

\newcommand{\anop}[2]{ #1_\mathbf{#2}}
\newcommand{\crop}[2]{#1_\mathbf{#2}^\dagger}

\usepackage{helvet} 

%tikz decoration commands 
\usetikzlibrary{decorations.pathmorphing}


\title{Standard Model Lecture Notes}
\author{Notes taken by Afiq Hatta} 
\begin{document} 
\maketitle
\tableofcontents

\pagebreak 

\section{Introduction}%
\label{sec:introduction}

\subsection{Introduction} 
What even is the standard model? You may have 
heard of this before but we'll try to start from scratch 
in this course. 

\begin{defn}{Standard Model}
The Standard model is a theoretical physics construction 
(what we call a theory, or a model), which describes all 
known elementary particles and their interactions, based on 
relativistic quantum field theory (QFT). 
This is a major thing - one of the biggest achievements 
in the history of science so far. 
\end{defn}

What are the 'ingredients' of the Standard Model? 
The SM is mostly based on the principle of symmetries, which is the key word here. 
The most important symmetry to consider here are spacetime symmetries, 
and we'll list these as well as some other important ones 
below. 
\begin{itemize}
\item  Spacetime: 3 + 1 dimensional Minkwoski space. 
	Symmetry: The Poincare group.
\item We can list the particles of the SM. 
	\begin{itemize}
		\item spin $ s  = 0 $: Higgs. A big part of this 
			course will be to describe the statistics of 
			the Higgs boson. 
		\item spin $ s = \frac{1}{2 } $ : 3 families of quarks and leptons. 
	\end{itemize}
\item How do these particles interact? The interactions are
	\begin{itemize}
		\item $ s = 1 $: 3 gauge interactions
		\item $ s= 2 $  : 1 gravity 
	\end{itemize}
The above gauge interactions are based on Gauge symmetry. 
This is a local symmetry. In QFT for example, we 
saw an example of a $ U ( 1 ) $ gauge symmetry. 
For the standard model this symmetry is $ SU ( 3 )_ C \times 
SU ( 2 ) _{ L } \times U ( 1 ) _{ \gamma }$. 
The $ c $ stands for colour, which describes the 
strong interaction. The $ L $ stands for left, which 
describes the elctroweak force, and the $ \gamma $ stands 
for hypercharge. This symmetry is 
spontenously broken to get 
\[
 SU (  3 ) _  c \times SU ( 2 ) _ L \times U ( 1 ) _{ \gamma  } 
 \to SU ( 3 ) _ c \times U ( 1 ) _{ EM } 
\]  
The Higgs particle, can take 
a value different from zero at the minimum. 
The symmetry of the vacuum is then no longer the 
same as the symmetry of the theory. 
This is called symmetry breaking. 

We have the following particle 
representations 
\begin{itemize}
	\item Quarks and leptons have the representation 
		\[
		 3 \left[ 
		 \left( 3, 2;    \frac{1}{6 }  \right) + \left( 
	 \overline{ 3 } , 1 ; - \frac{2}{3 } \right)  + \left( 
\overline{ 3 }, 1; -\frac{1}{3 } \right)  + \left( 1, 2;  - \frac{1}{2 }  \right)  + 
\left(  1, 1; -1  \right)  + \left( 1, 1; 0  \right) \right] 
		\] In this order, 
		we have $ Q _ L , U _ R , d _ R , L _ L , e _ R , \nu _ R $
		The factor of $ 3 $ is non-trivial (this is called flavour). 
		So, we have three flavours of quarks, 
		three flavours of leptons, etc. This applies for $ s = \frac{1}{2 ;} $
	\item The Higgs representation is $ \left(  1, 2  ; - \frac{1}{2 }  \right)  $, 
		which applies for $ s = 1 $. 
	\item Our gauge is represented by
		\[
			\left( 8, 1 ; 0  \right)  + \left(  1, 3 ; 0  \right)  
			+ \left( 1, 1; 0  \right) 
		\] From left to right, these are gluons, $ W ^{ \pm } , \mathbb{ Z } $ bosons, 
		and $ \gamma $, which is for $ s = 1 $. 
	\item Gravity is $ s = 2 $. 
\end{itemize} 
\end{itemize}

We have some comments 
\begin{itemize}
\item Interactions given by $ QFT $. 
\item Our main tool us symmetry! 
\item Total symmetry: spacetime and internal symmetries (gauge). 
\item There are also accidental (global) symmetries like lepton and baryon number. 
\item We also have symmetries which are approximate symmetries 
	like flavour. 
\item Look at the sum of the hypercharges $ \sum \gamma  = \sum \gamma ^ 3  = 0 $. 
	We also have that $ # 3 = # \overline{ 3 } $, and $ # 2 $ even. 
	(Multiply everything in the vector to get hypercharge). 
\item Gluons are confined. The same way that quarks are confined. 
	This is an open question to prove that quarks are confined.
\item There is rich structure (3 phases, Coulomb charge, Higgs and confinement). 
\end{itemize}
There are other symmetries like baryon number and lepton number, 
which are accidental and not fundamental - they just happen to be there. 

\subsection{Motivation}
Why should we learn about the SM? 
\begin{itemize}
\item It is fundamental! 
\item It is based on elegant principles of symmetry. 
\item It is true! The SM has been tested in many different ways, 
	with some outstanding prediction. 
	For example $ (\left( \mathbb{ Z } ^ 0 , W ^{ \pm }  \right) $,
	the Higgs, and so on. 
	There are also precision tests; but 
	there is an anomalous magnetic dipole moment electron. 
	\[
	 a = \frac{ g  - 2 }{ 2 }
	\] 
	We also have measured the fine structure 
	constant
	\[
	 \alpha ^{ - 1 }  = $ \overline{h} \frac{c}{e ^ 2 }$
	\]
\item The standard model is incomplete!  
\end{itemize}

Just to recall; $ SU_L ( 2 ) $ means that 
it's the left handed fields which transform. 

\subsection{The History of the Standard Model}
We should know the history 
of theories to gain a better understanding of why they work. 
Before the 20th Century, there were only 
two interactions known, which were gravity and electromagnetism. 
These interactions were studied famously by Newton and Maxwell from Cambridge. 
At this point, the discreteness of matter wasn't discovered, 
and it was not clear that matter was made of atoms. 

In 1896, radioactivity was studied by the Curies, Becquerel and Rutherford. 
$ \alpha , \beta  $ and  $ \gamma $ rays were discovered. $ \alpha $ rays 
are nuclei of helium, $ \beta $ rays are just electrons, 
and $ \gamma $ is light. Each of these 
are indicators of other interactions. For example, 
$ \alpha $ and $ \beta $ decay comes from 
the strong and weak interactions respectively. 
In $ 1897$, J. Thompson was playing with some cathode rays, 
and from computing charge and mass, discovered the electron. 
He computed the ratio of charge to mass. 
Years later, his son proved that electrons behaved as waves using diffraction 
experiments. 

From 1900 - 1930, quantum mechanics developed. 
In 1905, special relativity was discovered by Einstein. These 
are two basic theories which hints at the existence of the photon. 
In the same decade, Rutherford and Geiger 
carried out the gold scattering experiment which gave rise 
to the model of the atom. 

Once we had this, in the 1910s, Aston in 1919 
found the 'whole numbers rule',
where the nuclei of atoms had whole multiples, which implied the existence of the proton. 
In that decade, it was also an important 
experimental development 
that cosmic rays were discovered through cloud chambers. 
Cloud chambers are chambers of vapour where 
you can see the tracks of particles. 
Also, Einstein discovered general relativity. 

In the 1920s, Bose and Fermi 
came up with quantum statistics (bosons behave differently from fermions). 
During this time, we also saw 
the start of QFT, from Jordan, Heisenberg and Dirac. 
The Dirac equation described the electron but also predicted
the existence of the positron (he initially thought it 
was the proton). 
In 1931, Dirac came out with this prediction ($ e ^ + $  ). 
On a side note, he also mentioned this in the introduction 
of his paper on monopoles. 

In 1932, Anderson discovered the positron 
experimentally. 
Anderson was in Caltech, and in London Blackett also 
discovered the positron. But, Anderson got there first. 
This was found by finding particles that reflect in a 
different direction in a magnetic field. 

Chadwick in 1932 discovered the neutron. 
Pauli predicted the neutrino in 1930.  
In 1934, Fermi came up with the theory 
of weak interactions. 
This was found in $ \beta $ decay, 
\[
\eta \to p  + e ^ - + \overline{ \nu } 
\] (Diagram of $ n $ splitting to $ e ^  - , p , \overline{ \nu }  $ ). 
There is a coupling here 
called the Fermi coupling. 
We still use this theory today at low energies. 

In 1934, Yukawa had a theory of strong interactions - which 
is the force which keeps nucleons together 
in the atom. 
He proposed some particles called pions -
which are scalar mediators 
of strong interactions. 
He did the calculation and computed 
the strength of the interaction, 
\[
V \left( r  \right)   \sim  \frac{ e ^{   - m r } }{ r } 
\] This force goes very sharply to zero, 
since the strong interactions are very short range. 
He estimated $ m $, the mass of the pion, was 
$ 100 Me V $. 
With this mass, the strong interaction 
would be very important at very short distances. 


People started searching for pions, and in 
1936, Anderson discovered what we now call the muon, 
with a mass similar to the pion ($ m \sim 100 M eV $  ). 
But, then people discovered the muon was interacting 
via the weak interaction. 
The pion was discovered later, 
which decay into muons. 

In 1932, Heisenberg 
and in 1936, Condor et al, introduced a nice idea called isospin. 
People started seeing too many particles and Heisenberg realised 
that the neutron and proton were very similar. 
So, they thought that in the same way that electrons have two spins, 
there should be another internal symmetry that relates the newton and 
the proton. 

In the 1940s, 
nothing happened in the early 1940s, but 1947  
was a wonderful year for physics. This is when people 
started thinking again about understanding the world 
and the Lamb shift was discovered. 
This is what happens when we look at the two energy levels of hydrongen, 
which have slight difference of energies which could not be explained 
using quantum mechanics. 
Lamb presented at a conference, QED was discovered by 
Scwhinger, Feynman, Dyson and Tomonaga. 
QED started, renormalisation was a success, and this 
was a huge triumph for QFT. After this, 
people started believing in the theory. 
Three of the above won a Nobel prize for this. 
At this time, pions $ \pi $ were discovered. 

In the 1950s, 
particle accelerators 
and bubble chambers were improved. 
This allowed scientists 
to produce energies $ E \ge  Me V $. 
People always said that the 1950s were a decade of wealth, 
and this also applies to the discovery of 
particles. Dozens or particles were discovered, 
mostly from the strong interaction, 
and these were called hadrons.
They came with different names (kaons, hyperons... )
but in the end they classified t into two classes, 
mesons and baryons. 

Mesons are bosons and baryons were fermions. 
to put some order on this, 
new ideas were introduced. 
ONe of these ideas was strangeness, which 
came from Gell-Mann, Nishijima and Pais. 
This was kind of like a new charge, 
which also acts in pairs in decays. 
Another funny thing happened, 
which was Parity violation discovered by Lee and Yang. 
This was in 1956. They went to talk to Wu 
and she did an experiment with cobalt to produce 
beta decay, and changed the orientation. 
She discovered this in 1957. 

Parity is not a fundamental symmetry of nature. 

After this, there was the discovery of 
the (anti) neutrino by Cowan. 
At the same time the V- A property 
of weak interactions (vector and axial vector)
proposed by Marshak and Sudarshan 
In the same year 
Pontecorvo proposed neutrino oscillations. 
Then, 
Yang and Mills in 1954 came out with an 
interesting idea which would be a very important component. 
They realised that something that Schwinger proposed about 
symmetry. They thought about 
symmetries which are non-Abelian. 
In QED, there is $ U ( 1 ) $ symmetry mediated by the 
photon. 
In Yang-Mills, non-Abelian symmetry is proposed with 
also mediating particles which have not been discovered, so Pauli said it was nonsense. 

In the 1950s 
so many particles were discovered so no one 
new which particles were fundamental and which 
were not. 
However, in 1961 the Eightfold way 
was discovered by Gell-Mann and Neeman.  
This was a great development. 
Neeman was less famous and did a PhD at 
Imperial after working at the Israeli embassy. 
They had the idea of adding some structure to 
the particles, by organising particles 
as multiplets of representations of 
groups. 

They looked at $ SU ( 3 ) _{ \text{ Flavour } } $. 
The neutron and proton were two states 
of one single multiplet, which 
was part of the 
8 dimensional representation 
of $ SU ( 3 ) $. 
$ SU ( 2 ) $ has rank $ 1  $ 
so it's a one dimensional lattice, 
and $ SU ( 3 ) $ has rank $ 2 $ so it's 
a two dimensional lattice. 
This is represented 
in the diagram below, 
of Isospin verus Hypercharge. 

There is also a $ 10 $ dimensional 
representation of $ SU ( 3 ) $, 
and the $ \Omega^  - $ particle 
was not yet discovered, but 
from the lattice picture, 
Gell-Mann predicted its existence 
and then found shortly after. 


Then, in 1964, again, Gell-Mann and 
Zweig both in Caltech and CERN 
came out with a very nice idea. 
We all know that from $ SU ( 3 ) $
has fundamental representations, 
and from those we get the other 
representations. The fundamental 
representation of $ SU ( 3 ) $ 
is three dimensional, and this motivated 
the idea of quarks. 
The quarks are up, down and strange: 
\[
u , d, s 
\] 
However, people didn't 
take this seriously at first. 
But, how did people figure this out? 
\begin{align*}
3 \otimes 3 \to \text{mesons} & , \quad s = 0, \quad 
3 \times \overline{ 3 }  = 8 + 1 \\
3 \otimes 3 \otimes 3 \to \text{baryons}, \quad 
s = \frac{1}{2 }, \quad 3 \times 3 \times 3  = 
10 + 8 + 8 + 1 
\end{align*}
There was a problem with this. In the 10 representation, 
we have fermions so we couldn't have 
3 quarks in one state. 
However, in 1964 Greenberg and in 1965 Nambu and Han 
introduced the idea of colour so that 
we could resolve this issue. 

In 1967, Deep inelastic 
scattering gave the first piece of 
evidence that there 
was something inside protons and neutrons. 
There was evidence of substructure in protons and neutrons. 
Then people started to take the 
idea of quarks seriously. 

Back in 1961, the idea of symmetry breaking 
by Nambu, Goldstone, Salam and Weinberg 
was put out. Yang originally proposed the 
idea of non-Abelian symmetries. 
Nambu got the idea of symmetry breaking 
from superconductivity. 
Goldstone realised that 
when you break a symmetry 
into a smaller symmetry group 
there should be one particle 
that should be massless. 

This was the idea of Goldstone bosons 
which are massless. 

In 1964, 
the Higgs mechanism was discovered. This 
was due to Higgs, Brout, Englert, Kibble. 
If there is a broken symmetry locally then 
the gauge field is massive and 
and Goldstone boson is eaten to give a massive particle. 
This solves two problems at once. 
We can have non-Abelian gauge symmetries 
and brokem gauge symmetries. 

Everyone wanted to do this 
to learn about the strong interactions. 
In 1967 and 1968,
Weinberg and Salam did this 
for the weak interaction and 
this gave rise to electroweak 
unification. 
\[
SU ( 2 )_ L \times U \left( 1  \right)  _{ \gamma } 
\to U (1 ) _{ \text{ EM } } 
\] Glashow identified in 1962 $ SU ( 2) \times SU ( 1 ) $. 

In 1964, CP violation 
by Cronin and Fitsch was 
shown experimentally. 
This is when a particle changes to 
and anti-particle. 

In the 1970s, 
the Glashow-Iliopoulos-Maiani (GIM) 
mechanism was discovered to 
explain no flavour changing 
NCs, which implies the existence of a 
new quark - the charm quark ($ C $  ). 

In hindsight, it was 
totally obvious that was had to 
have a fourth quark. 
In 1969, Jackiw, Bell and Adler 
discovered anomalies, 
which suggested that the strange quark 
needed a partner. 

In 1974, $ J / \psi $ 
particles were discoverd so this implied 
the existence of the charm quark.
In 1973, weak neutral 
currents were discovered. 
This was one of the thinks 
put forward by the Weinberg-Salam model. 

Then, in 1975 and 1979, 
jets (quarks, gluons). 
Take a pion give it more and more energy, 
to get the quarks out of it. 
But, this stretches out 
the pion into a string and 
creates quark and anti-quark pairs
which is evidence for their existence. 
For example, 
\[
e ^ + e ^  - \to \vec{q} \vec{q} \to 2 -\text{jets}
\] 
We also have 
\[
\vec{q} g \vec{q} \to 3 - \text{ jets }
\] This 
was evidence for 3 colours. 

In 1973, asymptotic freedom was discovered 
by Gross and Politzer. 
The strength 
of the coupling of the strong interactions 
(Insert graphs here). 

In the 1970s, the $ \tau $ 
lepton was discovered.
In the 1980s, 
the $ \mathcal{ Z } ^ 0 , \mathcal{ W } ^ + $ 
bosons were discovered. 
In the 1990s, the top quark was discovered. 
Now we have the three families
\[
\begin{pmatrix} u \\ d  \end{pmatrix}  , 
\begin{pmatrix}  c \\ s  \end{pmatrix} , 
\begin{pmatrix}  t \\ b  \end{pmatrix} 
\] which are 3 families. 
In 2000s, the tau neutrino 
was discovered. And, 
the Higgs boson was discovered. 
Why was it that the $ 8 $ and $ 10 $ 
dimensional representations 
correspond to particles and 
other representations don't? 

\section{Spacetime Symmetries} 

In the standard model, 
we can decompose symmetries into a tensor product 
of spacetime and internal symmetries. 
\[
\text{symmetries} = \text{spacetime} \otimes \text{internal}
\] In 1967, Coleman and Mandula 
showed that you cannot mix the two. 
Internal symmetries are how fields which 
describe particles transform. 
As for spacetime symmetries, 
we know that special relativity provides the right description 
locally, so we know that the spacetime symmetries 
are the Poincare group. 

In this chapter, we'll 
be looking at spacetime symmetries. 

\subsection{Poincare Symmetries and Spinors}
A general transformation 
in the Poincare group is of the form 
\[
x ^ \mu \to x^{ ' \mu }  = \Lambda \indices{ ^ \mu _ \nu } x ^ \nu + a ^ \mu  , 
\quad \mu = 0 , 1, 2, 3 
\] where $ \Lambda \in O \left( 3, 1,  \right)  $, 
the Lorentz group, and $ a ^ \mu   $
is a translation in $ \R ^ 4$. 
The whole transformation group is 
referred to as the semi-direct product 
$ O \left( 3, 1  \right)  \wedge  \R ^ 4 $, 
since the two component groups don't
commute. 

This set of transformations 
are motivated by the fact that 
they leave invariant 
\[
ds ^ 2  = dx^ \mu \eta _{ \mu \nu } dx ^ \nu , \quad \eta _{ \mu \nu }  = \text{diag}\left( 
+, - , - , - \right) 
\] The Lorentz boost is defined 
such that for $ \Lambda \in O \left( 3, 1  \right)  $, 
\[
\Lambda \indices{ ^ \mu _ \rho } \eta _{ \mu \nu } \Lambda \indices{ ^ \nu _ \sigma } 
= \eta_{ \rho \sigma } \quad \Lambda ^ T \eta \Lambda  =  \eta 
\] This implies that,
taking the determinant of both sides of the equation 
implies that $ \det \Lambda  = \pm 1 $. 
Taking the 00 component of the equation 
implies that 
\[
\left( \Lambda \indices{ ^ 0 _ 0 }   \right)  ^ 2  - \left( \Lambda 
\indices{ ^ 1 _ 0 }  \right)  ^ 2 - \left( \Lambda \indices{ ^ 2 _ 0 }   \right)  ^ 2 
 - \left( \Lambda \indices{ ^ 3 _ 0 }   \right)  ^ 2  = 1
\] This implies that $ | \Lambda \indices{ ^ 0 _ 0 } | \geq 1  $, 
so we have that $ \Lambda \indices{ ^ 0 _ 0 } \leq 1   $ 
or $ \Lambda \indices{ ^ 0 _ 0 } \geq 1   $. 
With the choice of determinant and 
this choice, 
we have that $ O \left( 3, 1,  \right)  $ has 
four disconnected components. 

We denote $ SO  (3, 1 ) ^{ \uparrow } $ 
as the proper orthochronous Lorentz group 
as the choice of $ \det \Lambda  = 1 , \Lambda \indices{ ^ 0 _ 0} \geq 1   $  
This keeps the orientation and keeps the object 
in its respective lightcone. 

Any other element of $ O \left( 3, 1  \right)  $ 
can be obtained by combining elements of 
$ SO ( 3 , 1 ) ^{ \uparrow } $ with 
\[
\left\{  I , \Lambda _ P , \Lambda _ T , \Lambda _{ PT }  \right\}  
\] which corresponds to identity, parity, time-reversal, 
and the combination of the two. 
\begin{align*}
\Lambda_ P &  = \text{diag}\left( + 1, -1 , -1, -1  \right) \quad \text{Parity} \\
\Lambda _ T &=  \text{diag} \left( -1, +1, + 1, + 1  \right)  \quad \text{Time reversal} \\
\end{align*} 
$ \Lambda_{ PT}$ is the product of the above. 
These four elements form the Klein group. 
From now on, to save ink, we'll just call $ SO ( 3, 1 ) ^{ \uparrow } \to SO ( 3, 1 ) $. 
From now on, we're also working with the component 
which is connected to the identity. 
From this, we can construct the corresponding Lie algebra, 
which is called the Poincare Algebra. 
We do the standard thing and do an infinitesimal 
transformation. 
We write 
\[
\Lambda \indices{ ^ \mu _ \nu }  = \delta \indices{ ^ \mu _ \nu  } + 
\omega \indices{ ^ \mu _ \nu } , \quad a ^ \mu  = \epsilon ^ \mu , \quad 
\omega \indices{ ^ \mu _ \nu } , \epsilon ^ \mu \ll 1  
\] When we do the identity transformation 
infinitesimally, we have that 
\[
\Lambda ^  T \eta \Lambda  = \eta \implies \left( 
\delta \indices{ ^ \mu _ \rho } + \omega \indices{ ^ \mu _ \rho }   \right)  
\eta _{ \mu \nu } \left( \delta \indices{ ^ \nu _ \sigma } + 
\omega \indices{ ^ \nu _ \sigma }  \right)   = \eta _{ \rho \sigma }
\] This implies that $ \omega _{ \rho \sigma }  =  - \omega _{ \sigma \rho  } $. 
This implies that there are only $ 6 $ parameters for the Lorentz 
transformations. Also, we have an extra 4 dimensions 
which come from translations from $ \epsilon ^ \mu $. 
Therefore, in total, we have 10 dimensions in the Poincare group. 

The Poincare group is a group with ten dimensions and 
ten non trivial generators. 
The group is a manifold. We can take the coset of the Poincare group, 
quotient it out by the Lorentz group, then we get 
Minkowski spacetime. 

We want to study the algebra of 
representations of the Poincare group on Hilbert space. 
We're working with a state $ \ket{ \psi } $, 
and then mapping it with a unitary transformation 
\[
\ket{ \psi } \to U \left( \Lambda , a  \right)  \ket{ \psi } , \quad 
U ^{ -1 }  = U ^{ \dagger }
\] Near the identity, we have 
\[
U \left( 1 + \epsilon , \epsilon  \right)  = I + \frac{ - i }{ 2 } \omega_{\mu \nu } \mathcal{ M } ^{ \mu \nu } + i i \epsilon _ \mu P ^ \mu + \dots  
\] $\mathcal{ M }^{ \mu \nu } $ 
are generators of $ SO ( 3,1 ) $, and $ P ^ \mu $ 
generates translations. Thus, $ \mathcal{ M } ^{ \mu \nu }  =  - \mathcal{ M } ^{ \nu \mu } $. 

In terms of the algebra
\begin{itemize}
\item  Translators commuting implies that 
	\[
	 \left[ P _ \mu , P _ \nu  \right]  = 0
	\]  
\item What is $ \left[  P ^ \sigma , M ^{ \mu \nu }  \right]  $ ? 
	For this we will use the following. 
	$ P  ^ \mu $ has an index, so it's a vector. 
	But, it's also an operator acting on 
	Hilbert space. So since it's a vector, 
	we know how it transforms around boosts, namely, 

	\[
	 P ^ \sigma = \Lambda \indices{ ^ \sigma _ \rho } P ^ \rho  
	 \simeq \left( \delta \indices{ ^ \sigma _ \rho } + 
	 \omega \indices{ ^ \sigma _ \rho }  \right)  P ^ \rho  = 
	 P ^ \sigma + \frac{1}{ 2 } \left( \omega _{ 
	 \alpha \rho } + \omega _{ \rho \alpha }  \right)  \eta ^{ \sigma \alpha } 
	 P ^{ \rho }  = 
	 P ^{ \sigma } + \frac{1}{2 } \omega _{ \alpha \rho } 
	 \left( \eta ^{ \sigma \alpha } P ^ \rho  - \eta ^{ \sigma \rho } P ^ \alpha  \right) 
	\]  We're just playing with 
	the indices here. That comes from the fact that 
	$ P $ transforms as a vector. 
	But, we also require that $ P $ transforms as an operator
	in Hilbert space. This means that 
	\[
	 P ^ \sigma \to U^{ \dagger q} P ^ \sigma U  = 
	 \left( 1  + \frac{i}{2 } \omega _{ \mu \nu } M ^{ \mu \nu  }  \right)  
	 P ^ \sigma \left(  1 - \frac{i}{2 } \omega _{ \mu \nu } M ^{ \mu \nu }  \right) 
	\] Comparing 
	this transformation from operator 
	with the transformation as a vector, 
	we get that 
	\[
	 \left[  P ^ \sigma , M ^{ \mu \nu }  \right]   = 
	  -i \left( P ^ \mu \eta ^{ \nu \sigma }  - P ^ \nu \eta ^{ \mu \sigma } \right) 
	\] It is left the reader to prove why the commutator
	of the translation generators vanishes. 
\item Similarly, we have that 
	\[
	 \left[  M ^{ \mu \nu } , M ^{ \rho \sigma }  \right]   = 
	 i \left( M ^{ \mu \sigma } \eta ^{ \nu \rho } + 
	 M ^{ \nu \rho } \eta ^{ \mu \sigma } - 
 M ^{ \mu \rho } \eta ^{ \nu \rho }  - M ^{ \nu \sigma } \eta ^{ \mu \rho } \right)  
	\] This is how we define 
	the generators of the transformation. 
\item A four dimensional matrix representation of $ M ^{ \mu \nu } $ 
	is 
	\[
	 \left( M^{ \rho \sigma }   \right)\indices{ ^ \mu _ \nu } 
	  = - i \left( \eta ^{ \mu \sigma } \delta \indices{ ^ \rho _ \nu } 
	   - \eta ^{ \rho \mu } \delta \indices{ ^ \sigma _ \mu }  \right) 
	\] 
\end{itemize}

We have some comments about 
what there transformations really are. 
\begin{itemize}
\item $ P ^ 0 = \mathcal{ H }  $  implies that $ \left[  
	P ^ 0 , P ^ \mu \right]   =0   $, which 
	is energy and momentum conservation. 
\item $ \left[  P ^ 0 , M ^{ ij }  \right]   =0 $ implies 
	angular momentum conservation. 
	$ M ^{ ij } $ are the generators 
	of rotations.
\item $ \left[  P ^ 0 , M ^{ 0i }  \right] \neq  0  $ 
	This means that energy is not conserved 
	under Lorentz boosts! 
	implies no conservation laws here. 

\end{itemize}

The algebra of $ SO ( 3, 1 ) $
is determined by the 
algebra of $ SU ( 2 ) \otimes SU ( 2 )  $,
not equal to. We will see why 
this is the case. 
Define $ J _ i  = \frac{1}{2 } \epsilon _{ ij k } M ^{ jk } $. 
We also define $ M _{ 0i  } = K _ i  $, Hermitian. 
We can see from this that 
these satisfy the algebra 
\[
\left[  J _ i , J _ j  \right]   = 
i \epsilon _{ i j k  } J _ k, \quad 
\left[  J _ i , K _ j  \right]  = 
i \epsilon _{ ij k  } K _ k , \quad 
\left[  K _ i , K _ j  \right]   =  - i \epsilon _{ ij k  } J _ K 
\] We define 
\[
A_ i   = \frac{1}{2 } \left( J _ i  + i K _ i   \right) , \quad 
B _ i  = \frac{1}{2 } \left( J _ i  - i K _ i  \right) 
\] Note that these operators are not 
Hermitian. These satisfy the algebra 
\[
\left[ A_ i , A _ j  \right]  = i \epsilon _{ ij k } A _ k , 
\quad \left[  B _ i , B _ j  \right]   = i \epsilon _{ ij k }  B_ k , 
\quad \left[  A _ i , B _ j  \right]   =0 
\]  The commutation relations from $A_ i $  and $ B _ i$ 
follow the $ SU ( 2 ) $ algebra. 
Since we know how to handle the 
algebra of $ SU ( 2 )  $, 
we can write down some representations. 

For representations of $ SU ( 2 ) \otimes SU ( 2  ) $, 
we have states labelled by $ j = 0 , \frac{1}{2 } , \dots $. 
Then, the $ A _ i $ algebra has states 
labelled by $  A  = 0 , \frac{1}{2 } , \dots $, 
and the $ B _ i $ algebra has states labelled by $ B  = 0 , \frac{1}{2 } \dots  $. 
We then label representations 
of $ SO ( 3, 1 ) $ as $ \left( A, B  \right)  $. 

Note that under parity, 
we map 
\[
P : J _ i \to J _ i ; \quad K _ i \to - K _i 
\] this can be seen because $ J $ has 
two spatial indices, where as $ K _ i $ only relies on one. 
Thus parity swaps $ A _ i $ and $ B _ i $. 
Thus, 
\begin{align*}
P : & A _ i \leftrightarrow B_ i \\ 
\left( A, B  \right)  & \leftrightarrow \left( B , A  \right)  \\
\text{left} & \leftrightarrow \text{right} \\
\text{right} & \leftrightarrow \text{left} 
\end{align*}

We have another comment on this as well. 

We have $ SO ( 3 , 1 ) \simeq SL \left( 2, \mathbb{ C }  \right)  $, 
homomorphic. This is important, so we'll prove this. 
We have the four vector 
\[
X  = X _ \mu e ^{ \mu }  = \left( X _ 0 , X_1 , X_2, X_3  \right)  
\] Under a Lorentz transformation, 
$ X \to \Lambda X $, with $ | X | ^ 2  = X_0 ^ 2  - X_1 ^ 2  - X_2 ^ 2 - X_3 ^ 2 $
invariant. 
Now we do something a bit 
different. Instead of using $  e_ i$ 
as a basis, we now change basis. 
Take the basis of $ 2 \times 2 $ matrices. 
\[
\mathcal{ B }  = \left\{  I, \sigma _ x , \sigma _ y , \sigma _ z  \right\} 
\] which we denote with an index as $ \sigma ^{ \mu } $. 
We write $ \tilde{ X }   $ as a linear 
combination of these matrices, and this is 
\[
\tilde{ X }  = X _ \mu \sigma ^ \mu  = \begin{pmatrix}  
X _ 0 + X _ 3 & X _1  - i X _ 2 \\ X_ 1 + i X+ 2 & X _ 0  - X_ 3 \end{pmatrix}   
\] Furthermore, an action 
of $ SL \left( 2, \mathbb{ C }  \right)  $ on this matrix 
will do the following.
Define the action of $ S L ( 2 , \mathbb{ C } ) $ acting on $ \tilde{ X }  $  as 
\[
\tilde{ X } \to N \tilde{ X } N ^{ \dagger }   \quad N \in SL ( 2, \mathbb{ C } ) 
\] Since the determinant $ \det N  = 1 $, 
this leaves $ X_0 ^ 2 - X_1 ^ 2 - X_2 ^ 2  - X_3 ^ 2  $ invariant. 

This is a map from $ SL ( 2, \mathbb { C } ) \to SO (  3, 1 ) $. 
Hence, this these groups are homomorphic. 
The map is 2 to 1 since $ N  = \pm 1 $ is mapped to 
$ \Lambda  = 1 $. 
$ S L ( 2, \mathbb{ C } ) $ is nice 
to work with since as a manifold,
it is simply connected unlike $ SO ( 3 ,1 ) $. 
This means that we can work 
with the geometry of the whole space. 
$ SL ( 2, \mathbb{ C } ) $, is the 
covering group of $ SO ( 3, 1 ) $. 

We will prove that $ SL( 2, \mathbb{ C } ) $ is 
simply connected, which then implies that $ SO ( 3, 1 ) $ 
is doubly connected. 

we use the fact that 
we can use polar decomposition to 
decompose $ N $. 
\[
N  = e ^{ h } U , \quad h \text{hermitian}, \quad U \text{unitary}
\] Since $ \det N  = 1 $, 
this implies that $\tr h  = 1 $ 
and $ \det U  = 1 $. 
This is because $ \det e ^{ h }  = e ^{\tr h } $. 
The most general $ h $ 
we can write down is 
\[
h  = \begin{pmatrix}  
a & b + i c \\ b  -i c &  - a  \end{pmatrix}  
\quad U   = \begin{pmatrix}  x + i y & z +i w \\ 
 - z + i w & x - i y \end{pmatrix} 
\] This imposes the condition 
that $ a, b c \in \mathbb{ R  } $. 
We also have that $ x ^2 + y ^2 + z ^ 2 + w ^ 2  = 1 $,
which is  $ S^ 3 $. 
This means that the $ SL ( 2, \mathbb{ C } ) $ 
manifold is $ \mathbb{ R } ^ 3 \times S ^ 3  $.
Then the $ SO ( 3, 1 ) $ manifold is $ \mathbb{ R } ^ 3 \times S ^ 3  / \mathbb{ Z } _ 2 $.

Let's look at 
The representations of $ SL ( 2, \mathbb{ C } ) $. 
The fundamental representation 
of $ \psi _ \alpha , \alpha =  1 ,2  $ 
is given by 
\[
\psi _ \alpha  =  N \indices{ _ \alpha ^ \beta } , \quad \alpha, \beta  = 1 , 2  
\] We call $ \psi _ \alpha $ left-handed Weyl spinors. 
We also have the conjugate 
representation 
with with the right handed Weyl spinors 
$ \overline{ \chi } _{ \dot{ \alpha }  } $. 
\[
\overline{ \xi } ' _{ \dot{\alpha }  }  = N ^ * \indices{ _{ \dot{ \alpha }  } ^{ 
\dot{ \beta }  } } \overline{ \chi } _{ \dot{ \beta }  }  
\] 
We have the contra variant representations. 
\[
\psi ^{ ' \alpha }  = \psi ^ \beta \left( N ^{ 
 - 1 }  \right)  \indices{ _ \beta ^{ \alpha }} ; 
 \overline{ \chi } ^{ ' \dot{ \alpha }  }  = 
 \overline{ \chi } ^{ \dot{ \beta }  } \left( 
 N ^{ * - 1 } \right)  \indices{ _ \dot{ \beta } ^{ \dot{ \alpha  } }  } 
\] We 
use invariant tensors 
\begin{itemize}
\item $ SO ( 3, 1 ) $, which has invariant tensor 
	$ \eta ^{ \mu \nu }  = \left( \eta _{ \mu \nu }  \right)  ^{  -1  } $. 
	Invariance means we rase and lower indices. 
\item For $ SL ( 2, \mathbb{ C } ) $, we have that 
	\[
	 \epsilon ^{ \alpha \beta }  = \epsilon ^{ \dot{ \alpha } \dot{ \beta }   } 
	 = \begin{pmatrix}   0 & 1 \\ -1 & 0  \end{pmatrix}  
	  =  - \epsilon _{ \alpha \beta }  = - \epsilon _{ \dot{ \alpha } \dot{ \beta }   } 
	\] which is the invariant tensor. 
\item We have $ \epsilon ^{  ' \alpha \beta }  = \epsilon^{ \rho \sigma } 
	N \indices{ _ \rho ^ \alpha }  N \indices{ _ \sigma ^{ \beta } } 
	 = \epsilon ^{ \alpha \beta } \det N  = \epsilon ^{ \alpha \beta }$. 

\end{itemize}


Recall that Dirac spinors 
are given by $ \psi _ D =  \begin{pmatrix} \phi _ \alpha \\ \overline{ \chi } ^{ \dot{ \alpha }  } \end{pmatrix} $ the top spinor is represented as 
$ \left( \frac{1}{2 } , 0  \right)  $, and the bottom spinor is represented as $ \left( 0 , \frac{1}{ 2}  \right)  $. 

\subsection{Unitary Representations of the Poincare Group}

\subsubsection{Warming up with representations of $ SO ( 3 )  $ }
So far, using 
special relativity and quantum mechanics, 
we've achieved alot. To proceed further, 
we need to build representations of the Poincare group. 
A lot of this work was done by Wigner in 1939. 

To warm up, we'll refresh our memory 
of how to build finite irreducible representations of the rotation group in three dimensions. 
The rotation group in 3D is 
$ SO ( 3 ) \simeq SU (  2 )  $. 
The Lie algebra has the generators $ J _ i $
with the algebra $ \left[  J _ i , J _ k  \right]  = i \epsilon _{ ijk } J _ k  $. 
In terms of matrices, 
these things cannot be simultaneously diagonalised 
since they do not commute with one another. 

However, there is an operator called the
Casimir operator. 
This operator is something we've seen before; 
it's the total angular momentum operator \[ 
J ^ 2  = J_ 1 ^ 2 + J _ 2 ^ 2 + J _ ^ 2 \] 
which has the property that  $ \left[  J ^ 2 , J _ i  \right]   = 0 $,
so it commutes with everything in our algebra. 
This means, we can label representations with 
the eigenvalues of this operator. 
$ J ^ 2 $ labels a particular representation or 'multiplet'. 
We set 
\[
J ^ 2 \ket{ j }  = j \left( j + 1   \right)  \ket{ j } , \quad j = 0 , \frac{1}{2 }, 1 , \dots 
\] So, we want to find representation which are labelled by this. 
For compact group, the number of Kashmir operators 
is equal to the rank of the group. The rank 
of $ SU(2 ) $ is one, so this is the 
only Casimir operator we have. 

Now, within one single representation, we can have 
several states which are part of the representation.
How do we differentiate between states in 
one representation? To do this, we pick one of the $ J  _i   $'s, 
then simultaneously diagonalise with the 
Casimir operator. 

For instance, take $ J _ 3 $, with eigenvalues 
$ j _ 3 $. 
So now we have the states 
\[
J _ 3  \ket{ j ;  j _ 3 }  = j _ 3 \ket{ j ;  j _ 3 } , \quad j _ 3 = - j ,  - j + 1 , \dots , j 
\] 

\subsubsection{Building representations of the Poincare group}
We will follow exactly the same steps, 
but now for the Poincare group. 
For the Poincare group, 
we have the generators $ P ^ \mu $ and $ M ^{ \mu \nu } $. 
The next step, is to identify the 
Casimirs. 

Our two operators are 
\[
C _ 1  = P ^ \mu P _ \mu , \quad C _ 2  = W ^ \mu W _ \mu 
\] where 
\[
W _ \mu : = \frac{1}{2 } \epsilon _{ \mu \nu \rho \sigma  } P ^ \nu M ^{ \rho \sigma } , \quad 
\text{ the Pauli-Ljubanski vector}
\] These Casimir operators 
have the property that the 
commute with the generators. One can easily check that  
\[
\left[  C _{ 1, 2 }, P ^ \mu   \right]   = \left[  C _{ 1, 2 } , M ^{ \mu \nu }  \right]   = 0
\]  Since 
we have two Casimir operators, 
we require two labels for 
a given representation. Note that we have not fully justified that 
these indeed are the only two Casimir operators, 
but this is the case. 
Just for completeness, we'll write 
down the algebras for $ P  $ and $  W $. 
Using the fact that $ \left[  W _ \mu , P _ \nu  \right]   =0 $, 
we have that 
\[
\left[  W _ \mu , M _{ \rho \sigma  }  \right]   = i \left( 
\eta _{ \mu \rho  } W _ \sigma  - \eta _{ \mu \sigma } W _ \rho \right)  , \quad 
\left[  W _\mu , W _ \nu   \right]   = - i \epsilon _{ \mu \nu \rho \sigma } 
W ^ \rho P ^ \sigma 
\] Be careful, however. Note that this is not strictly an algebra since 
the last commutation 
involves a product of operators, 
not a linear combination. 

So, we can label 
representations by eigenvalues of $ C _ 1 $ and $ C _ 2 $. 
Now we have to pick some set of 
operators in the algebra which mutually commute. 
We can pick the momenta for this, since 
we found earlier that the set $ \left\{  P ^ \mu  \right\}  $ 
commute. 

Hence, within this representation, we've picked a subset of the generators that can be simultaneously diagonalised, which is $ \left\{  C _ 1, C _ 2   \right\}  \cup 
\left\{  P ^ \mu  \right\}  $ 
For example, $ P ^ \mu  $ with eigenvalue $ p ^ \mu $. 
Now, $ C _ 1  = P ^ \mu P _ \mu  $ has eigenvalues $ p ^ \mu p _ \mu > 0 $, or equal to zero, 
or less than zero. 
In the case that  $ p ^ \mu p _ \mu > 0 $, we use 
a standard technique
to Lorentz transform to a frame where $ p ^ \mu  = \left( m , 0 , 0  , 0  \right)  $. 
Rotations $ M ^{ ij } $ 
with leave $ p ^ \mu $ invariant, and hence $ J _ i $ keep $ p ^ \mu$ invariant.
This is called the Little group $ SO ( 3 ) $. 

We can also thus simultaneously diagonalise 
by $ J ^ i $, which gives us a total of four 
parameters to label our states with. 
We call this the multiplet labelled by 
\[
 \ket{ C _1 , C _ 2 ; p ^ \mu , j _ 3 } 
\] 
Now, we wish to find the associated eigenvalues 
for the $ C _ 1 $ and $ C _ 2 $ operators. We have that 
\begin{align*}
	C_1  &=  P ^ \mu P _ \mu  = m ^ 2  \\ 
	C_2 &=  W ^ \mu W _ \mu  = ? , \quad W _ \mu  = \left( 0 , 
	 - m J _  i \right) \implies C _ 2  = m ^ 2 J ^ 2   
\end{align*}
$ C _ 1 $ is fairly obvious, but $ W _ \mu $ less so. 
If we work in the rest frame, we 
necessarily require that the $ \nu $ index in
the definition is zero, and this leaves 
the spatial generator part of $ M ^{ \mu \nu } $. 
This implies that the multiplet, in terms of eigenvalues, 
is labelled by 
\[
 \ket{ m , j ; p ^ \mu , j _ 3 } 
\] This is a particle for a 
massive one particle state! 


The next case we could have 
is taking, for $ P ^ \mu P _ \mu  =0   $, 
that 
\[
 P ^ \mu  = \left( E , 0 , 0 , E  \right)
\] In this case, $ C _  1 =0 $. 
Computing the vector of $ \left( W_ 0 , W _1 , W _ 2, W_  3  \right)  $,
is done similarly as before, by 
looking over the components. 
We have that 
\[
	\left( W_0, W_1 , W_2, W_3  \right)   = E \left( 
	J _ 3,  - J _1 + K _ 2 ,  - J _ 2  -  K_ 1 ,  -J _ 3 \right) 
\] this gives us the commutation 
relations 
\[
 \left[  W_1 , W _ 2   \right]  =0 , \quad \left[  W _ 3,  W_ 1  \right]  
  =  - i E W _2 , \quad \left[  W _ 3, W _ 2  \right]   = i E W_ 1 
\] This algebra gives us a combination 
of rotations and translations in 2 dimensions, 
which is the 2D Euclidean group. 
This group has
infinite dimensional representations, and thus 
we haven't seen these particles 
before. These are infinite dimensional representations 
which cause a problem! 

For now, we remedy this 
problem by setting some things to zero. 
To remedy this, set $ W_ 1  = W_ 2 = 0 $, which 
is still consistent with the algebra above. 
It's then easy to see that 
$ W _ 3   $ generates $ SO ( 2 ) $ on its 
own, which are rotations around $ x _ 3 $. 
Setting $ W_1 $ and $ W_2 $ to zero 
means that our new $ W _ \mu $ vector is 
as follows 
\[
 W_ \mu  = E J _ 3 \left( 1, 0 , 0 , - 1   \right)  \propto p _ \mu 
\] Crucially, 
we have that the vector $ W _ \mu = \lambda p _ \mu $, 
where $ \lambda $ is called the Lorentz scalar. Hence, $ C _ 1 = p ^ \mu p _ \mu  =  0$ , 
and $ C _ 2  =  W^ \mu W _ \mu  = 0 $. 
Thus, we have an irrep represented by 
\[
 \ket{ 0 , 0 , p ^ \mu , \lambda } : = \ket{ p ^ \mu , \lambda } 
\] the eigenvalues of $ J _  3 $ is called helicity. 
$ \lambda  = 0 , \pm \frac{1}{2 } , \pm 1 , \dots $, 
so in this case
\[
 e ^{  2 \pi i \lambda  } \ket{ p ^ \mu , \lambda }  = \pm \ket{ p^ \mu , \lambda } 
\] $ \lambda  = 0 $ is the Higgs, 
$ \frac{1}{2 }  = \lambda $ are quarks and leptons, 
$ \pm 1  = \lambda $ are gauge fields, 
$ \pm  2  $ are gravitons. 
We haven't seen $ \lambda  = \pm \frac{3}{2 } $, 
which are called gravitini. 
Beyond $ \lambda  = \pm 2 $. 

When $ p ^ \mu  = \left( 0 , 0 , 0 , 0  \right)  $, 
this is called the vacuum. When $ p ^ \mu p _ \mu < 0  $, 
these are called Tachyons. 

So, to summarise what we have done earlier, 

\section{Gauge Symmetries}%
\label{sec:gauge_symmetries}

\section{Symmetry Breaking}%
\label{sec:symmetry_breaking}

\section{Electroweak Unification}%
\label{sec:electroweak_unification}

\section{QCD}%
\label{sec:qcd}

\section{Phenomenology of the SM}%
\label{sec:phenomenology_of_the_sm}

\section{EFT's and open questions}%
\label{sec:eft_s_and_open_questions}



\pagebreak 
\section*{Example Sheet 1}


\subsection*{Question 2}
If our spinor transforms under parity as $ \psi \to P \psi P ^{- 1} = \gamma ^  0 \psi \left( x_ P  \right)  $, 
then it is easy to see that 
\[
	\overline{ \psi } \to \overline{\psi } \left( x _ P   \right) \gamma ^  0 
\] Thus, 
if we transform our Lagrangian with $ g' = 0 $  by inserting 
parity operators,
we get that 
\begin{align*}
	P \mathcal{ L } _ I P ^{ - 1 } &=  
	g \left( P \overline{ \psi } P ^{ - 1 }  \right) \left( P 
	\psi P ^{ - 1 } \right)  \left( P \phi P ^{ -1 }  \right)  
	\\
				       &=  g \overline{ \psi } \left( x _P  \right) 
				       \gamma ^ 0 \gamma ^  0 \psi \left( x _ P  \right)  
				       P \phi P ^{ - 1 } 
\end{align*}
If we wanted to enforce 
invariance, we require that $ P \phi \left( x  \right)  P ^{-1 }  = \phi \left( x _ P   \right)  $. This is because now, 
\[
	\int d ^ x \mathcal{ L }\left( x  \right)  
	\to \int d ^ 4 x_ P \, P \mathcal{ L }\left( x  \right)   P 
	= \int d ^ 4 x _ P \, \mathcal{ L } \left( x _ P  \right)  
\] 
Now we look at the case with $ g  =0 $. 
Similarly, we use the fact that $ \gamma ^ 5 $ anti-commutes 
with $ \gamma ^  0$  to show that for  
\[
	\mathcal{ L } _ I  = i g' \overline{ \psi } \left( x  \right)
	\gamma ^ 5 \psi \left( x  \right)  \phi \left(  x  \right) 
\] we require $ O \phi \left( x  \right)  P ^{ - 1 }  =  - \phi \left( x _ P  \right)  $, 
for our Lagrangian to be invariant. 

For parity to be conserved, 
we need that $ \left[  P , H  \right]   = 0 $, 
or equivalently that $ \left[  P , \mathcal{ L }  \right]  =0  $. 
If $ g $ and $ g ' $ are both non-zero, 
then parity is not even well defined. 

Note: commuting with the parity operator is equivalent to $ P \mathcal{ L } P ^{ - 1}  = 
\mathcal{ L } $. 
The axial vector transforms as follows. 
\begin{align*}
	j ^ \mu & \to P \overline{\psi  } P^{ - 1 } \gamma ^ \mu \gamma ^ 5 P \psi P ^{ -  1} \\
		&=  \overline{ \psi } \left( x _ P  \right)  \gamma ^ 0 
		\gamma ^ \mu \gamma ^ 5 \gamma ^  0 \phi \left( x _ P  \right)  \\
		&=  \overline{ \psi } \left( x _ P  \right)  \gamma ^  0 
		\gamma ^ \mu \gamma ^  0 \gamma ^ 0 \gamma ^ 5 \gamma ^   0 \phi 
		\left(  x _ P  \right)  \\ 
		&=   - \overline{ \psi } \left( x _ P  \right)  
		\gamma ^  0 \gamma ^ \mu \gamma ^  0 \gamma ^ 5 \phi \left( x_ P  \right) \\ 
		&=  \begin{cases}
			- \overline{ \psi } \left( x _ P  \right)  \gamma ^ \mu \gamma ^ 5 
			\phi \left( x _ P  \right)  &  \mu  = 0 \\
			\overline{ \psi } \left( x _ P  \right)  
			\gamma ^ i \gamma ^ 5 \psi \left( x _ P  \right) 
						    & \mu  = 1 ,2 , 3\\
		\end{cases} 
\end{align*}

\subsection*{Question 3}
Our strategy is to show that $ B ^{  -1 } \psi ^ * \left(  x  \right)  $ 
is a solution to the time-reversed Dirac equation 
satisfied by $ \psi \left(  x _ T  \right)  $: 
\[
	\left( i \gamma ^ \mu \partial  _{ \mu T }  - m  \right)  \psi \left( x _T  \right)  =
	 0 
\] This can be shown to 
be true by just using the substitution $ y  = x _ T $, 
which reduces it to the Dirac equation. 

We start from the Dirac equation then 
apply operations. 
\begin{align*}
	\left( i \gamma ^ \mu \partial  _ \mu  - m  \right)  \psi _ r  &=  0  \\ 
	\left(   - i \gamma ^{ \mu * } \partial  _ \mu  - m  \right) \psi _ r ^ * &=  0  \\ 
	B ^{ - 1} \left(  - i \gamma ^{ \mu * } \partial  _ \mu  - m  \right)  \psi _ r ^ *  &  = 0  \\ 
	\left(  -i B ^{  -1 } \gamma ^{ \mu * } \partial  _ \mu  - B  ^{ - 1 } m    \right)  
	\psi _ r ^ *  &=  0  \\ 
	\left(  - i \left( \gamma^ 0 ,  - \mathbf{\gamma}  \right) \partial  _ \mu 
- m \right) B ^{ - 1 } \psi ^ * _ r  & = 0 \\ 
  \left( i \gamma ^ \mu \partial _{ \mu T }  - m  \right)  B ^{ - 1 } \psi ^ * _ r  &=  0  \\
\end{align*}
Therefore, by uniqueness of solutions, 
we have shown that up to scaling, $ B ^{  -1 } \psi ^{ * } _ r  = \psi_{ r  ' } 
\left( x  _ T  \right)  $. 

Dirac spinors 
transform under time reversal as $ T \psi \left( x  \right)  T ^{  -1 } 
 = \eta _ T  B \psi \left( x _ T  \right)  $. 
 Therefore, under the whole transformation, 
 we have that 
 \begin{align*}
	 T \hat{ \psi } \left( x  \right)  T^{ -1 } 
	 &=  \sum _ r T a _ r \psi _ r \left(  x  \right)  T ^{ - 1 }  \\ 
	 &=  \sum _ r T a _ r T ^{ - 1 } T \psi _ r \left( x  \right)  T ^{ - 1 }  \\ 
	 &=  \sum _ r a _{ r ' } \eta _ T B \psi _r \left( x _ T   \right)  \\
	 &=  \sum _ r a _{ r ' } \eta _ T B B^{ - 1 } \psi _{ r  ' } ^ * \left(  x  \right)  \\
	 &=  \eta _ T \hat{ \psi }^ *  \left( x  \right)   , \quad 
	 \text{ assuming } a_ r \in \mathbb{ R }
 \end{align*}
 What is the phase factor 
 for this question? Does it factor out of the sum?

\subsection*{Question 4}
Note that $ \hat{ C } $ is unitary, 
and therefore linear. 
Thus, to find out how 
$ \hat{ C } \overline{ \psi } X \psi \hat{ C } ^{  -1 } $ 
transforms, 
we insert $ \hat{ C }^{  -1  }  \hat{ C }$ between 
$ \overline{ \psi } $ and $ X $, and then 
commute it past $ X $. 
So, 
\begin{align*}
	\hat{ C } \overline{ \psi} X \psi \hat{ C } ^{  -1  } &=  
	\hat{ C  } \overline{ \psi } \hat{ C } ^{  -1  } X \hat{ C } 
	\psi \hat{ C } ^{  -1  } \\ 
	&=  \hat{ C } \overline{ \psi } \hat{ C } ^{  -1  } X C \overline{ \psi } 
	^{ T } 
\end{align*}
Using the unitary property of $ \hat{ C }$,
and the fact that $ \gamma ^  0  C \gamma ^ 0  =  - C $, 
we can show that  \[
 \hat{ C } \overline{ \psi } \hat{ C } ^{  -1 }  = - \psi ^  T C^{  -1 } 
\]
So, we have that 
\[
 \hat{ C } \overline{ \psi  } X \psi \hat{ C } ^{  -1 }  = 
  - \psi ^ T C ^{  -1  } X C \overline{ \psi } ^{  -1  } 
\] But since this is just a number, 
we can take the transpose of this object. 
Because $ \psi $ and $ \overline{ \psi } $ 
anti-commute, this picks up a minus sign. 
So, the above is equal to $ \overline{ \psi } C ^ T X ^ T \left( C ^{-1} \right) ^ T \psi  $
Now, use the fact that $ C ^ T  = - C $ and this recovers 
the result.

The case for $ \hat{ T } \overline{ \psi  } X \psi \hat{ T } ^{ - 1 }  $ 
is entirely similar, except we need to 
use anti-linearity to commute the $ \hat{ T } $ past $X  $.

Substituting the appropriate 
matrices into this expression, 
we find that
\begin{align*}
	\overline{ \psi } \left(  x  \right)  \psi \left( x  \right)  
	\to_{ \hat{ C } } & \overline{ \psi } \left( x  \right)  \psi ( x ) \\
	\overline{ \psi } \left( x  \right)  \psi \left( x  \right)  
	\to _{ \hat{ T }} \overline{ \psi } \left( x _ T  \right)  \psi 
	\left( x  _ T  \right) \\
	\overline{ \psi } i \gamma ^ 5 \psi \to_{ \hat{ C } } & 
	\overline{\psi } i \gamma ^ 5 \psi \\
	\overline{ \psi } i \gamma ^ 5 \psi \to_{ \hat{ T } } & 
	- \overline{ \psi } \left( x _ T  \right)  i \gamma ^ 5 
	\overline{ \psi } \left(  x _ T  \right)  \\
	\overline{ \psi }\gamma ^ \mu \gamma ^ 5 \psi 
	\to_{ \hat{ C }  } & \overline{ \psi } \gamma ^ \mu \gamma ^ 5 \psi 	 
\end{align*}  
Finally we have that 
for our time transformed quantity
\[
	\overline{ \psi } \left(   x  \right)  \gamma ^ \mu \gamma ^ 5 
	\psi \left(  x  \right)  \to _{ \hat{ T }  } \begin{cases}
		\overline{ \psi } \left( x _ T  \right)  \gamma ^  0 
		\gamma ^ 5 \psi \left( x _ T  \right)  \\ 
		- \overline{ \psi } \left( x _ T  \right)  \gamma ^ i 
		\gamma ^ 5 \psi \left( x _ T  \right) 
	\end{cases}
\] 
\end{document} 
