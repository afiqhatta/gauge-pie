\documentclass[11pt, oneside]{article}   	% use "amsart" instead of "article" for AMSLaTeX format
\usepackage[margin = 1.1in]{geometry}            		% See geometry.pdf to learn the layout options. There are lots.
\geometry{letterpaper}                   		% ... or a4paper or a5paper or ... 
\usepackage[parfill]{parskip}    		% Activate to begin paragraphs with an empty line rather than an indent
\usepackage{graphicx}				% Use pdf, png, jpg, or eps§ with pdflatex; use eps in DVI mode
								% TeX will automatically convert eps --> pdf in pdflatex	
\usepackage{adjustbox}	
\usepackage[section]{placeins}


%% LaTeX Preamble - Common packages
\usepackage[utf8]{inputenc}
\usepackage[english]{babel}
\usepackage{textcomp} % provide lots of new symbols
\usepackage{graphicx}  % Add graphics capabilities
\usepackage{flafter}  % Don't place floats before their definition
\usepackage{amsmath,amssymb}  % Better maths support & more symbols
\usepackage[backend=biber]{biblatex}
\usepackage{amsthm}
\usepackage{bm}  % Define \bm{} to use bold math fontsx
\usepackage[pdftex,bookmarks,colorlinks,breaklinks]{hyperref}  % PDF hyperlinks, with coloured links
\usepackage{memhfixc}  % remove conflict between the memoir class & hyperref
\usepackage{mathtools}
\usepackage[T1]{fontenc}
\usepackage[scaled]{beramono}
\usepackage{listings}
\usepackage{physics}
\usepackage{tensor}
\usepackage{simplewick} 
\usepackage{tikz} 
\usepackage{import}
\usepackage{xifthen}
\usepackage{pdfpages}
\usepackage{transparent}
\usepackage{pgfplots}
\usepackage[compat=1.1.0]{tikz-feynman}
\usepackage{subfiles}
\usepackage{simpler-wick}
\usepackage{slashed}
\usepackage{fancyhdr}
\usepackage{enumitem}

\pagestyle{fancy}
\fancyhf{}
\rhead{Notes by Afiq Hatta}
\lhead{Standard Model}
\rfoot{Page \thepage}

%% Commands for typesetting theorems, claims and other things.
\newtheoremstyle{slanted}
{1em}%   Space above
{.8em}%   Space below
{}%  Body font
{}%          Indent amount (empty = no indent, \parindent = para indent)
{\bfseries}% Thm head font
{.}%         Punctuation after thm head
{0.5em}%     Space after thm head: " " = normal interword space;
{}%         \newline = linebreak
{}%          Thm head spec (can be left empty, meaning `normal')

%% Commands for typesetting theorems, claims and other things. 

\theoremstyle{slanted}
\newtheorem{theorem}{Theorem}
\newtheorem*{thm}{Theorem}
\newtheorem*{claim}{Claim}
\newtheorem{example}{Example}
\newtheorem*{defn}{Definition}

\newcommand{\Lagr}{\mathcal{L}} 
\newcommand{\vc}[1]{\mathbf{#1}}
\newcommand{\pdrv}[2]{\frac{\partial{#1}}{\partial{#2}}}
\newcommand{\thrint}[1]{\int d^3 \vc{x} \left( {#1} \right)}

%% QFT specific macros 
\newcommand{\intp}{ \int \frac{ d^3 p }{ (2 \pi)^3 } \, }
\newcommand{\ann}[1]{a_{ \mathbf{ #1 }}}
\newcommand{\crea}[1]{a^\dagger_{ \mathbf{ #1 }}}
\newcommand{\ve}[1]{ \mathbf{ #1 } } 
\newcommand{\mode}[ 1]{ e^{ i \mathbf{ #1 } \cdot \mathbf{x} }}
\newcommand{\nmode}[1]{ e^{  - i \mathbf{ #1 } \cdot \mathbf{x} }}
\newcommand{\freq}[1]{\omega_\mathbf{ #1} } 
\newcommand{\scal}[1]{\phi ( \mathbf{ #1 })} 
\newcommand{\mom}[1]{ \pi (\mathbf{ #1 })} 
\newcommand{\arr}{\rightarrow} 

\newcommand{\incfig}[1]{%
\def\svgwidth{\columnwidth}
\resizebox{0.75\textwidth}{!}{\input{./figures/#1.pdf_tex}}
}

\newcommand{\anop}[2]{ #1_\mathbf{#2}}
\newcommand{\crop}[2]{#1_\mathbf{#2}^\dagger}

\usepackage{helvet} 

%tikz decoration commands 
\usetikzlibrary{decorations.pathmorphing}


\title{Standard Model Lecture Notes}
\author{Notes taken by Afiq Hatta} 
\begin{document} 
\maketitle
\tableofcontents

\pagebreak 

\section{Introduction}%
\label{sec:introduction}

\subsection{Introduction} 
What even is the standard model? You may have 
heard of this before but we'll try to start from scratch 
in this course. 

\begin{defn}{Standard Model}
	The Standard model is a theoretical physics construction 
	(what we call a theory, or a model), which describes all 
	known elementary particles and their interactions, based on 
	relativistic quantum field theory (QFT). 
	This is a major thing - one of the biggest achievements 
	in the history of science so far. 
\end{defn}

What are the 'ingredients' of the Standard Model? 
The SM is mostly based on the principle of symmetries, which is the key word here. 
The most important symmetry to consider here are spacetime symmetries, 
and we'll list these as well as some other important ones 
below. 
\begin{itemize}
	\item  Spacetime: 3 + 1 dimensional Minkwoski space. 
		Symmetry: The Poincare group.
	\item We can list the particles of the SM. 
		\begin{itemize}
			\item spin $ s  = 0 $: Higgs. A big part of this 
				course will be to describe the statistics of 
				the Higgs boson. 
			\item spin $ s = \frac{1}{2 } $ : 3 families of quarks and leptons. 
		\end{itemize}
	\item How do these particles interact? The interactions are
		\begin{itemize}
			\item $ s = 1 $: 3 gauge interactions
			\item $ s= 2 $  : 1 gravity 
		\end{itemize}
	The above gauge interactions are based on Gauge symmetry. 
	This is a local symmetry. In QFT for example, we 
	saw an example of a $ U ( 1 ) $ gauge symmetry. 
	For the standard model this symmetry is $ SU ( 3 )_ C \times 
	SU ( 2 ) _{ L } \times U ( 1 ) _{ \gamma }$. 
	The $ c $ stands for colour, which describes the 
	strong interaction. The $ L $ stands for left, which 
	describes the elctroweak force, and the $ \gamma $ stands 
	for hypercharge. This symmetry is 
	spontenously broken to get 
	\[
	 SU (  3 ) _  c \times SU ( 2 ) _ L \times U ( 1 ) _{ \gamma  } 
	 \to SU ( 3 ) _ c \times U ( 1 ) _{ EM } 
	\]  
	The Higgs particle, can take 
	a value different from zero at the minimum. 
	The symmetry of the vacuum is then no longer the 
	same as the symmetry of the theory. 
	This is called symmetry breaking. 

	We have the following particle 
	representations 
	\begin{itemize}
		\item Quarks and leptons have the representation 
			\[
			 3 \left[ 
			 \left( 3, 2;    \frac{1}{6 }  \right) + \left( 
		 \overline{ 3 } , 1 ; - \frac{2}{3 } \right)  + \left( 
 \overline{ 3 }, 1; -\frac{1}{3 } \right)  + \left( 1, 2;  - \frac{1}{2 }  \right)  + 
 \left(  1, 1; -1  \right)  + \left( 1, 1; 0  \right) \right] 
			\] In this order, 
			we have $ Q _ L , U _ R , d _ R , L _ L , e _ R , \nu _ R $
			The factor of $ 3 $ is non-trivial (this is called flavour). 
			So, we have three flavours of quarks, 
			three flavours of leptons, etc. 
		\item The Higgs representation is $ \left(  1, 2  ; - \frac{1}{2 }  \right)  $. 
		\item Our gauge is represented by
			\[
				\left( 8, 1 ; 0  \right)  + \left(  1, 3 ; 0  \right)  
				+ \left( 1, 1; 0  \right) 
			\] From left to right, these are gluons, $ W ^{ \pm } , \mathbb{ Z } $ bosons, 
			and $ \gamma $.
	\end{itemize} 
\end{itemize}

We have some comments 
\begin{itemize}
	\item Interactions given by $ QFT $. 
	\item Our main tool us symmetry! 
	\item Total symmetry: spacetime and internal symmetries (gauge). 
	\item There are also accidental (global) symmetries like lepton and baryon number. 
	\item We also have symmetries which are approximate symmetries 
		like flavour. 
	\item Look at the sum of the hypercharges $ \sum \gamma  = \sum \gamma ^ 3  = 0 $. 
		We also have that $ # 3 = # \overline{ 3 } $, and $ # 2 $ even. 
		(Multiply everything in the vector to get hypercharge). 
	\item Gluons are confined. The same way that quarks are confined. 
		This is an open question to prove that quarks are confined.
	\item There is rich structure (3 phases, Coulomb charge, Higgs and confinement). 
\end{itemize}
There are other symmetries like baryon number and lepton number, 
which are accidental and not fundamental - they just happen to be there. 

\subsection{Motivation}
Why should we learn about the SM? 
\begin{itemize}
	\item It is fundamental! 
	\item It is based on elegant principles of symmetry. 
	\item It is true! The SM has been tested in many different ways, 
		with some outstanding prediction. 
		For example $ (\left( \mathbb{ Z } ^ 0 , W ^{ \pm }  \right) $,
		the Higgs, and so on. 
		There are also precision tests; but 
		there is an anomalous magnetic dipole moment electron. 
		\[
		 a = \frac{ g  - 2 }{ 2 }
		\] 
		We also have measured the fine structure 
		constant
		\[
		 \alpha ^{ - 1 }  = $ \overline{h} \frac{c}{e ^ 2 }$
		\]
	\item The standard model is incomplete!  
\end{itemize}
\section{Spacetime Symmetries} 

\section{Gauge Symmetries}%
\label{sec:gauge_symmetries}

\section{Symmetry Breaking}%
\label{sec:symmetry_breaking}

\section{Electroweak Unification}%
\label{sec:electroweak_unification}

\section{QCD}%
\label{sec:qcd}

\section{Phenomenology of the SM}%
\label{sec:phenomenology_of_the_sm}

\section{EFT's and open questions}%
\label{sec:eft_s_and_open_questions}



\section*{Example Sheet 1}
 
\end{document} 
