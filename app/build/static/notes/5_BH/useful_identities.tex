\section{Useful Identities in General Relativity for Black Holes} 

In this section, we'll cover 
some useful identities which may prove useful for 
doing general relativity.
The first one we'll prove is an equation that's useful for 
proving the divergence theorem for curved space. 

\begin{thm}{Divergence of a vector field in terms of $\sqrt{ - g }  $}
	\begin{equation}
		\nabla _ \mu V ^ \mu = \frac{1}{\sqrt{ - g }  } \partial  _ \mu \left( 
		\sqrt{ - g}  V ^ \mu \right)  
	\end{equation}
\begin{proof}
	We first apply the product rule, then 
	make use of a smart rearrangement using logs. 
	We first have that
	\begin{align*}
		\nabla _ a V ^ a  & = \partial  _ \mu V ^ \mu + 
		\frac{1}{\sqrt{- g}  } V ^ \mu \partial  _ \mu \left( \sqrt{ - g }    \right) \\
				  &=  \partial  _ \mu V ^ \mu + V ^ \mu \partial  _ \mu \left( \log \sqrt{ - g }   \right) \\
				  &=  \partial  _ \mu V ^ \mu + \frac{1}{2 } V ^ \mu \partial  _ \mu \left( \log \left(  - \det g  \right)   \right)  
	\end{align*}
	Where in the last line we wrote out the determinant 
	explicitly. 
	Now the trick here is to use the identity which 
	relates the logarithm of the determinant to the trace of 
	the formal logarithm of a matrix.

	Observe that if $ A $ is a matrix, which we assume 
	to be positive definite (since our metric is), then 
	it is diagonalisable. Since it's diagonalisable, 
	we can write it in the appropriate basis such that 
	$ \exp A  = \text{diag} \left( e ^{ \lambda _ 1 }, \dots, e ^{ \lambda _ n } \right) $ where $ \lambda _i \in \mathbb{R }$ are the eigenvalues of the matrix $ A $. This means that the eigenvalues of $ \exp A $ are $ e ^{ \lambda _ i } $ for all $ i $. So, since the determinant of the matrix is equal to 
	the product of its eigenvalues, we have that 
	\[
		\det \left( \exp A  \right)   = \exp \left( \tr A  \right)  
	\] If we set $ B = \exp A $, then this identity reduces to 
	\[
		\det B = \exp \left( \tr \log B  \right)  
	\] Taking the logarithm of both sides once more, we 
	have that finally 
	\[
	 \log \det B = \tr \log B 
	\] We can now resume to 
	the question at hand. We rewrite the above using this identity 
	so that 
	\begin{align*}
		\nabla _ a V ^ a &=  \partial  _ \mu V ^ \mu 
		+ \frac{1}{2 } V ^  \mu \partial  _ \mu \left( \tr \log \left(  - g  \right)   \right)  \\ 
		&=  \partial  _ \mu V ^ \mu + \frac{1}{2 } V ^ \mu 
		\tr \partial  _ \mu \log \left(  -g  \right)  \\
		&=  \partial  _ \mu V ^ \mu 
		+ \frac{1}{2 } V ^ \mu \tr \left( g ^{ - 1 } \partial  _ \mu g  \right) 
	\end{align*}
	Where in this case we've put  $ g $ to denote 
	schematically the matrix $ g_{ \mu \nu }  $  and 
	\textbf{not } the determinant. Be careful 
	to observe that the minus signs when 
	differentiating the logarithm cancel out. 
	One can easily verify that $ \frac{1}{2 }g ^{ \alpha \beta } \partial  _ \mu 
	g _{ \alpha \beta }  = \Gamma ^{ \nu } _{ \mu \nu } $. 
	Thus, this completes the proof. 
\end{proof}
\end{thm}

