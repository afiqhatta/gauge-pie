\documentclass[11pt, oneside]{article}   	% use "amsart" instead of "article" for AMSLaTeX format
\usepackage[margin = 1.1in]{geometry}            		% See geometry.pdf to learn the layout options. There are lots.
\geometry{letterpaper}                   		% ... or a4paper or a5paper or ... 
\usepackage[parfill]{parskip}    		% Activate to begin paragraphs with an empty line rather than an indent
\usepackage{graphicx}				% Use pdf, png, jpg, or eps§ with pdflatex; use eps in DVI mode
								% TeX will automatically convert eps --> pdf in pdflatex	
\usepackage{adjustbox}	
\usepackage[section]{placeins}


%% LaTeX Preamble - Common packages
\usepackage[utf8]{inputenc}
\usepackage[english]{babel}
\usepackage{textcomp} % provide lots of new symbols
\usepackage{graphicx}  % Add graphics capabilities
\usepackage{flafter}  % Don't place floats before their definition
\usepackage{amsmath,amssymb}  % Better maths support & more symbols
\usepackage[backend=biber]{biblatex}
\usepackage{amsthm}
\usepackage{bm}  % Define \bm{} to use bold math fontsx
\usepackage[pdftex,bookmarks,colorlinks,breaklinks]{hyperref}  % PDF hyperlinks, with coloured links
\usepackage{memhfixc}  % remove conflict between the memoir class & hyperref
\usepackage{mathtools}
\usepackage[T1]{fontenc}
\usepackage[scaled]{beramono}
\usepackage{listings}
\usepackage{physics}
\usepackage{tensor}
\usepackage{simplewick} 
\usepackage{tikz} 
\usepackage{import}
\usepackage{xifthen}
\usepackage{pdfpages}
\usepackage{transparent}
\usepackage{pgfplots}
\usepackage[compat=1.1.0]{tikz-feynman}
\usepackage{subfiles}
\usepackage{simpler-wick}
\usepackage{slashed}
\usepackage{fancyhdr}

\pagestyle{fancy}
\fancyhf{}
\rhead{Notes by Afiq Hatta}
\lhead{Black Holes}
\rfoot{Page \thepage}

%% Commands for typesetting theorems, claims and other things.
\newtheoremstyle{slanted}
{1em}%   Space above
{.8em}%   Space below
{}%  Body font
{}%          Indent amount (empty = no indent, \parindent = para indent)
{\bfseries}% Thm head font
{.}%         Punctuation after thm head
{0.5em}%     Space after thm head: " " = normal interword space;
{}%         \newline = linebreak
{}%          Thm head spec (can be left empty, meaning `normal')

%% Commands for typesetting theorems, claims and other things. 

\theoremstyle{slanted}
\newtheorem{theorem}{Theorem}
\newtheorem*{thm}{Theorem}
\newtheorem*{claim}{Claim}
\newtheorem{example}{Example}

\newtheorem*{defn}{Definition}

\newcommand{\Lagr}{\mathcal{L}} 
\newcommand{\vc}[1]{\mathbf{#1}}
\newcommand{\pdrv}[2]{\frac{\partial{#1}}{\partial{#2}}}
\newcommand{\thrint}[1]{\int d^3 \vc{x} \left( {#1} \right)}

%% QFT specific macros 
\newcommand{\intp}{ \int \frac{ d^3 p }{ (2 \pi)^3 } \, }
\newcommand{\ann}[1]{a_{ \mathbf{ #1 }}}
\newcommand{\crea}[1]{a^\dagger_{ \mathbf{ #1 }}}
\newcommand{\ve}[1]{ \mathbf{ #1 } } 
\newcommand{\mode}[ 1]{ e^{ i \mathbf{ #1 } \cdot \mathbf{x} }}
\newcommand{\nmode}[1]{ e^{  - i \mathbf{ #1 } \cdot \mathbf{x} }}
\newcommand{\freq}[1]{\omega_\mathbf{ #1} } 
\newcommand{\scal}[1]{\phi ( \mathbf{ #1 })} 
\newcommand{\mom}[1]{ \pi (\mathbf{ #1 })} 
\newcommand{\arr}{\rightarrow} 

\newcommand{\incfig}[1]{%
\def\svgwidth{\columnwidth}
\resizebox{0.75\textwidth}{!}{\input{./figures/#1.pdf_tex}}
}

\newcommand{\anop}[2]{ #1_\mathbf{#2}}
\newcommand{\crop}[2]{#1_\mathbf{#2}^\dagger}

\usepackage{helvet} 

%tikz decoration commands 
\usetikzlibrary{decorations.pathmorphing}


\title{Notes on Black Holes}
\author{Afiq Hatta } 
\begin{document} 
\maketitle
\tableofcontents

\section*{Conventions and Housekeeping}
We will set $ G  = c = 1 $, 
and ignore the cosmological constant $ \Lambda $. 
For indices, greek letters $ \mu , \nu $ refer to 
a specific basis. 
$ a, b$ are abstract indices which refer to any basis, 
from Roger Penrose. 
For example, we have 
\[
\Gamma^ \mu _{ \nu \rho }  = \frac{1}{2 }g ^{ \mu \sigma } 
\left( g _{ \sigma \nu , \rho } + g _{ \sigma \rho , \nu }  - g _{ \nu \rho , \sigma }  \right), 
\quad R  = g ^{ ab } R _{ ab } 
\] Christoffel components are 
basis dependent so they are written with greek indices. 


As far as Black holes are concerned, 
the cosmological constant is incredibly small. 
Hence, we will ignore it from now. 
Check out Wald's book on General Relativity. It's great!

\section{Spherical Stars}
\subsection{Cold Stars}
If you think about a star like our sun, 
it's a ball of hot gas with nuclear reactions at its centre. 
Gravitational force makes the star contract, 
but nuclear reactions at the centre exert outward pressure 
to resist the contraction. 
If we wait long enough, th e star will exhaust the fuel 
it has and hence the star will contract. 
If we're interested in the final state of this star, 
we need to look at some source of pressure which is non-thermal in nature 
(so valid when the star is cold) to counter act gravity. 

There is another natural source of pressure to resist gravity. 
This is the Pauli principle - where a gas of Fermions 
resists compression. This is called degeneracy pressure. 
This is a purely quantum mechanical phenomenon. 

For example, a white dwarf is a star in which gravity is balanced 
by electron degeneracy pressure. 
A white dwarf is a very dense kind of star. If we 
had a white dwarf with the same mass of our sum $ M  = M _{ \sun } $. 
the radius would be $ R \sim \frac{R_{\sun } }{100 } $. 
A white dwarf is how the sun will end its life. 


Can all stars end their life this way? 
No. This is because there's a maximum mass for a white dwarf. 
This is called the Chandrasekhar limit, where 
\[
M _{ WD } \leq 1.4 M _{ \sun }
\]
What happens when we have a star more massive than this? 
When the star continues to get more and more dense, 
then inverse beta decay occurs where protons turn into neutrons. 
Neutrons are fermions and they also have 
degeneracy pressure. 
Thus, there's a second class of 
stars called Neutron stars where 
gravity is balanced by neutron degeneracy pressure. 
These stars are tiny. 

If we took a neutron star with $  M \sim M _{ \sun } $, $ R \sim 10 km $. 
Compare this to  $ R _{ \sun } \simeq 7 \times 10 ^ 5 km $. 
They are very dense! The gravitational field is very strong. 
If we have a Newtonian gravitational potential at the surface 
\[
| \phi | \sim 0.1 
\]  General relativity because important when $ | \phi | $ is order 1. 
Hence, GR is important here.
We can derive a maximum possible mass for neutrons stars as well. 
We'll derive this bound which is independent of our knowledge of dense matter. 

We need to make some 
simplifying assumptions. 
We assume spherical symmetry. 

\subsection{Spherical Symmetry}
Recall the unit round 
metric on a two dimensional sphere $ S ^ 2 $, 
\[
d \Omega ^ 2  = d \theta ^2 + \sin ^ 2 \theta d \phi ^ 2 
\] The isometries are diffeomorphisms 
which preseve the metric. 
They form a group. 
The isometry group of a two sphere with that metric is $ SO ( 3 ) $. 
Essentially, we define a space time as spherically symmetric 
if it has this as its isometry group. 
\begin{defn}{Spherically Symmetric Spacetime}
A spacetime is spherically symmetric if its isometry group 
contains an $ SO ( 3 ) $ subgroup, whose orbits are
$ 2  $-spheres. 
In other words, if we pick a point and act on it with all 
$ SO ( 3 ) $ elements, it will fill out a sphere. 
\end{defn}
\begin{defn}{Area Radius function}
In a spherically symmetric spacetime $ ( M , g )$, 
the area radius function is 
\[
	r : M \to \mathbb{ R } , r ( p ) = \sqrt{ \frac{ A ( p ) }{ 4 \pi } } 
\] where $ A ( p ) $ is the area of the $ S^ 2 $ orbit 
through the point $ p $. 
So we take a point $ p $, since its spherically symmetric, we 
can make a sphere out of it, and we take the area. 
This is linked to our understanding of the $ r $ coordinate, 
but this definition doesn't require a preferred origin. 
In other words, 
the $ S ^ 2 $ has induced metric $ r ( p ) ^ 2 d \Omega^ 2 $. 
\end{defn}
\subsection{Time independence}
\begin{defn}{Stationarity}
$ \left(  M , g  \right)  $ is stationary if there exists a Killing vector field
$ k ^ a $
(KVF) which is everywhere timelike. 
We're saying that $ g _{ ab } k ^ a k ^ b < 0 $. We can introduce adapted coordinates. 
Let's pick some hypersurface $ \Sigma $ which is transverse to a vector field 
$ k $. We can then pick coordinates  $ x ^ i $ on $ \Sigma $, where $ i =  1, 2, 3 $. 
This gives us coordinates on our surface. 
We assign coordinates $ \left(  t, x ^ i  \right)  $ to point parameter 
distance $ t $ along the integral curve of  $ k ^ a $ through point on 
$ \Sigma $ with coords $ x ^ i $. This implies 
\[
 k  = \frac{\partial }{\partial t} 
\] This implies the metric is independent of $ t $, since $ k ^ a $ is a 
Killing vector field. Therefore, the metric looks like 
\[
	ds ^ 2  = g _{ 00 } \left( x ^ k  \right) dt ^ 2  + 2 
	g _{ 0i } dt d x ^ i + g _{ ij  }\left( x ^ k  \right)  dx ^ i dx ^ j 
\] This means that $ g _{ 00 } < 0 $. Thus, 
given a stationary spacetime, we can construct coordinates which 
the metric is time independent. 
Conversely, any metric of this form is stationary. 
\end{defn}
There's a more refined version of time independence which 
we can use. 
Imagine we have a hypersurface $ \Sigma $ where $ f = 0 $, $ f : M \to \mathbb{ R } $
which is smooth, and $ df \neq 0  $ on $ \Sigma $. 
Then, $ df $ is normal to $ \Sigma $. 
If we let $ t ^ a $ tangent to $ \Sigma $, then we have that 
\[
df ( t )  = t ( f )  = t ^ \mu \partial  _ \mu f = 0 \text{ since } f  \text{ is constant}
\] A normal to a surface is 
not unique. 
Normals are not unique. What's the most general 
form of a covector field which is normal to $ \Sigma $. 
If $ n _ a$  also normal to $ \Sigma $, them 
\[
n = g df + f n ' 
\] where $ g $ is smooth and not equal to $ 0 $ on $ \Sigma $,
and  $ n ' $ is a smooth 1-form. 
Let's look at the exterior 
derivative of this vector field. Using the rules 
for the exterior derivative, 
\[
dn = dg \wedge  df + df \wedge  n ' + f dn ' 
\] Let's evaluate this 
on $ \Sigma $. We have that 
\[
\left. dn \right\vert_{ \Sigma }  = \left(  dg - n '  \right)  \wedge  df
\] Hence, $ n \wedge  dn - 0 $   on $ \Sigma $. This is because 
$ n \propto df $ on $ \Sigma $. The wedge product vanishes. 
\begin{thm}{Frobenius}
If $ n \neq 0 $ is a one form such that $ n \wedge  dn  = 0 $  everywhere, 
then there exist functions $g, f $ such that 
$ n   = g df $. So $ n $ is normal 
to surfaces of constant $ f $, so 
$ n $ is 'hypersurface orthogonal'. 
$ n$ is orthogonal to all surfaces 
of constant $ f $. 
\end{thm}

\begin{defn}{Static spacetimes}
A spacetime  $ \left( M , g  \right)  $ is static 
if there exists a hypersurface orthogonal, timelike $ KVF$. 
In particular, static implies stationary. 
Why is this useful? Returning to the adapted coordinates, 
how does hypersurfaece orgonality help? 
We choose $ \Sigma $ orthogonal to $ k ^ a $ when defining 
adapted coordinates $ \left(  t, x ^ i  \right)   $. 
But, $ \Sigma $ is $  t = 0$, therefore the normal 
to $ \Sigma $ is $ dt $. 
So
\[
	k _ \mu \mid _{ t = 0 } \propto \left( 1, 0 , 0 , 0  \right) 
\] In particular, we have that \[
 k _i \mid _{ t = 0 }  =0 
\] but, $ k _ i  = g _{ 0i } \left( x ^ k  \right)  $. 
Hence, $ g _{ 0i }  = 0$.
Thus, if we write down the metric, 
we have that 
\[
 ds ^ 2  = g _{ 00 } \left( x ^ k  \right)  dt ^ 2 + 
 g _{ ij} \left( x ^ k  \right)  dx ^ i dx ^ j , \quad g _{ 00 } < 0 
\] Thus we have a discrete time reversal isometry $ \left( t , x ^ i  \right)  \to 
\left(  - t , x ^ i  \right)  $. 
Thus,  
\[
\text{static} \iff \text{time independent and invariant under time reversal}
\] For example, for a rotating star, the metric 
may be time independent but not static since they're is no 
time reversal symmetry since it changes the sense of rotation. 
\end{defn}

\subsection{Static, Spherically Symmetric Spacetimes}
The isometry group is $ \mathbb{ R } \times SO ( 3 )$. 
We can show that this implies the metric must be static. 
If it wasn't static, it would be rotating, but that breaks spherical symmetry. 
On $ \Sigma $ choose coordinates $ x^ i  = \left(  r, \theta , \phi   \right)  $, 
where $ r $ is our area radius function. 
If we do this, then the metric 
is 
\[
ds ^ 2 \mid_{ \Sigma }  = e ^{ 2 \psi ( r )  } dr ^ 2 + r ^2 d \Omega ^ 2 
\]  where due to spherical symmetry, everything depends on $ r $. 
If we had $ dr d \phi $ or $ dr d \theta $ terms, 
this would break spherical symmetry. 
If we then go on to write down the full spacetime 
metric, we 
get that 
\[
ds ^ 2   =  - e ^{ 2 \Phi ( r ) } ds ^ 2 + e ^{ 2 \Psi ( r ) } dr ^ 2 + r ^ 2 d \Omega ^ 2 
\] At the moment, there's no origin  - $ r $ is not the distance from 
the origin! Because of spherical symmetry 
and stacity, the functions only depend on $ r$. 

We model the matter in the star 
as a perfect fluid. This is a pretty good approximation, 
where the stress tensor looks like 
\[
T _{ \alpha \beta  }  = \left( \rho  + p  \right)  u _ \alpha u _ \beta 
+ p g _{ \alpha \beta }
\]  where $ u _ \alpha $ is the velocity of the fluid, and $ \rho , p $ are
the energy desnity and pressure in the fluid's rest frame. 
We have that $ g _{ ab } u ^ a u ^ b  = - 1$. 
Time independence means that 
\[
u ^ a  = e ^{  - \Phi } \left( \frac{\partial  }{\partial  t }   \right)  ^ a 
\] So the fluid velocity is fixed by 
symmetry assumptions. 
Symmetry implies that $ \rho  = \rho ( r ) $ and 
$ p = p ( r  ) $. 
Our final observation is 
that outside the star, there is no fluid. 
This means that $ p , \rho   = 0 $ for $ r > R $, 
where $ R $ is the radius of the star. 

\subsection{Tolman-Oppenheimer-Volkoff Equations}
The equation of motion of a perfect fluid 
is the conservation law for the energy momentum tensor, 
which is guaranteed to hold from the Einstein equations 
(from the Bianchi identity). The Einstein tensor inherits 
the symmetries from the metric. 
There are essentially only three independent 
components, so only three independent equations 
to solve. 
We define a new function $ m ( r ) $ by 
\[
e ^{ 2 \Psi }  = \left( 1 - \frac{2m}{r }  \right) ^{ - 1 }
\] In particular, the LHS is positive, which means that 
$ m ( r ) < \frac{r}{2 } $. 
We can then look at the $ t t $ component of the Einstein equation, 
which gives 
\[
\frac{d m }{ dr }  = 4 \pi r ^ 2 \rho \quad \text{ TOV 1 }
\] we also get 
\begin{align*}
\frac{ d \Phi }{ dr } &=  \frac{ m + 4 \pi  r^  3 \rho }{ r ( r - 2m ) } \quad 
\text{TOV 2 }\\
\frac{dp  }{dr } & = - \left( \rho + p  \right)  
\frac{ m + 4 \pi r ^ 3 p }{ r \left( r - 2m  \right)  } \quad \text{TOV 3 }
\end{align*}

The last equation comes from imposing $ \nabla _ \mu T ^{ \mu r }  = 0 $. 
We have 3 equations in 4 unknowns 
$ \left( m , \Phi , p \rho  \right)  $. 
But, we get extra information from 
thermodynamics. 
In the case of a cold star, we have $ T = 0 $, 
but $ T = T \left( p , \rho  \right)  $, which 
means that this implier $ p = p \left( \rho   \right) $, 
which we call a 'barotropic' equation of state. 
We assume that $ \rho , p > 0 $ and also that 
$ \frac{ d p }{ d \rho } > 0 $. If this was not true, 
then we would have an unstable fluid. 
For example, if we had some region where 
the energy density went up, the the pressure would go down, 
which means that stuff flows into the region, and the density 
of that region rises up even more. 

\subsection{Outside the Star: Schwarzchild Solution}
In the case where  $ r > R $, $ p = \rho = 0$. 
This implies  by TOV 1 that $ m ( r ) = M $ which is 
a constant. 
Then TOV 2 implies that 
\[
\Phi ( r ) = \frac{1}{2 } \log \left( 1 - \frac{2M}{r }  \right) + \Phi _ 0 
\] where $ \Phi _ 0 $ is a constant. This constant is not 
physical. Since $ \Phi ( r ) \to \Phi _ 0 $ as  $ r \to \infty $, 
then 
\[
g _{ t t } \to  - e^{  - 2 \Phi_0 } \text{ as } r \to \infty 
\] We can eliminate $ \Phi _ 0 $ by a coordinate transform 
$ t' = e^{ \Phi _ 0 } t$. 
Putting this all together, we have that 
\[
ds ^ 2 =   - \left( 1 - \frac{2M}{r }  \right)  dt ^ 2 + \left( 
1 - \frac{2M}{r } \right)  ^{ - 1 } dr ^ 2 + r ^ 2 d d\Omega ^ 2 \text{ Schw solution}
\] What is $ M $ - a free parameter. 
We discovered at large $ r $ this solution is Minkwoski space. If we 
add a small correction, we can think of $ M $ as the mass of the star. 
We will see later how we properly define mass in general relativity. 
In particular, $ M > 0 $. 
One thing we notice with this solution is that something 
funny happens at $ r = 2M $. The metric $ g _{ \mu \nu } $ in 
these coordinates is singular. This is 
called the Schwarzchild radius. 
This means that the star must have a radius bigger than $ 2M $, so $ R > 2M $. 
For example, for the sun, $ 2M _{ \sun } \simeq 3km $, where as 
$ R _{\sun} \simeq 7 \times 10 ^{ 5 } km $. 

\subsection{Inside the Star - the Interior Solution}
TOV 1 implies that, by integrating, 
\[
m ( r ) = 4 \pi \int_ 0 ^{ r }\rho ( r ' ) r ^{ '  2} dr ' + m _{ * }
\] Let $ \Sigma _ 1 $ be a $ t  = const $ surface. 
The pullback 
\[
ds ^ 2 \mid_{ \Sigma _ 1 }  = e ^{ 2 \Psi ( r ) } dr ^ 2 + r ^ 2 d \Omega
\] This is smooth at $ r = 0$. This means that 
close enough to this point, the metric is locally flat 
at $ r  =0 $. So if we consider a sphere of radius  $ r $, 
if that value of $ r $ is small enough, it should 
be Euclidean space. A point on $ S ^ 2 $ of small radius $r $ 
must be distance $ r $ from $ r = 0 $. 
This is saying that 
\[
r \simeq  = \int_ 0 ^{ r } e ^{ \Psi ( r ' ) } dr ' \simeq e ^{ \Psi ( 0 ) } r  
\] for small $r  $. So, smoothness at the origin tells us that 
$ \Psi ( 0 )  = 0 $. 
If we go back to the definition of 
$ m $, this means that $ m ( 0 ) = 0 $. 
This means that from the above, $ m _ *  =0 $. 

By continuity, 
we have that $ m ( R ) = M  $. This means 
that $ M = 4 \pi \int_ 0 ^ R \rho ( r ) r ^ 2 dr $. 
This is exactly the same equation we'd get from Newtonian physics. 
This is just a coincidence. 
Let's just think about GR here. 
We should be integrating things with respect to the 
appropriate volume form. 
The volume form on $ \Sigma _ 1 $ is 
\[
e^{ \Phi ( r ) } r ^2 \sin \theta dr \wedge  d \theta \wedge  d \psi 
\] The energy of matter on $ \Sigma _ 1 $ is the integral 
of energy density with respect to the volume form. 
This is 
\[
E = 4 \pi \int_ 0 ^ R e ^{ \psi ( r ) } \rho ( r ) r ^ 2 dr 
\] The fact that $ m > 0 $ implies that $ e ^{ \Psi  } > 1 $. 
That therefore means that $ E > M  = \text{total energy of star}$. 
So, the energy of the matter of the star is made from 
is larger than that of M. The total energy includes gravitational 
binding energy. $ E -  M $ is gravitational binding energy. 

Coming back to the bound $ R > 2M $, if we reinsert units, 
the bound is 
\[
\frac{ G M }{ c ^2 R } < \frac{1}{ 2}
\] To get to Newtonian theory, we take the speed of light $ c \to \infty $. 
This equation becomes trivial and there is no Newtonian analogue of 
this bound. This is an intrinsically GR effect. 

Let's continue solving the equations. 
If we look at TOV 3, 
the right hand side is negative, so 
\[
\frac{ dp }{ dr } < 0 
\] But since $ \frac{dp}{dr } > 0 $, we have that $ \frac{ d \rho }{ dr  } < 0 $. 
In example sheet 1, we are asked to show 
\[
\frac{m ( r ) }{ r } < \frac{2}{9 } \left[  
1 - 6 \pi r ^ 2 p ( r ) + \left(  1 + 6 \pi r ^2 p ( r )  \right)  ^{ \frac{1}{2 } }\right] 
\] If we set $ r  = R $ at the surface of the star, 
the pressure $ p = 0 $, we get that 
\[
R > \frac{9}{4 } M 
\] So we've improved the previous bound. This 
is called the Buchdahl inequality.
This also holds for a hot star. 
Can we improve on this bound? No, since we 
can get close as we like to this for a star of constant 
density. 

Going back to solving our equations, 
we can solve TOV 1 and TOV 3 as they 
are coupled ordinary differential equations 
using the fact that $ p = p (\rho ) $ is known 
from the equation of state, given $ m ( 0 ) = 0 $, 
and a specified parameter $ \rho ( 0 )  = \rho _ c $. 

From TOV 3, $ p $ decreases as $ r $ increases, 
so we define the radius of the star where $ p ( R )  = 0$. 
This is how we define $ R $. This means that $ R =  R( \rho _ c ) $. 
This implies that $ M = M ( \rho _ c )$. 
The last thing we need to fix is $ \Phi $ by solving TOV 2 
in $ r < R $, with the initial condition 
$ \Phi ( R )  = \frac{1}{2 } \log \left( 1 - \frac{2M}{R }  \right)  $. 
So we've shown that for a given equation of state, 
cold stars form a 1 parameter family labelled by $ \rho _ c $. 

\subsection{Maximum Mass of a cold star}
If we plot the mass against $ \rho _ c $ (insert diagram), 
we have that the maximum mass $ M _{ \text{ max } } $ depends on 
the equation of state. 
Experimentally, we only 
know the equation of state up to 
nuclear density $ \rho _ 0 $. 
We will show that the maximum mass $ M _{ \text{max } } \leq 5 M_{ \sun } $, 
whatever happens for $ \rho > \rho _ 0 $. 

We have that $ \rho $ decreases with $ r$. 
We define the core to be the region where $ \rho > \rho _ 0$, 
or $ r < r_ 0 $. 
We define the envelope as where $ \rho < \rho _0 $, where $ r_0 < r < R $. 

Our core mass is $ m_0 = m ( r_0 ) $. 
Our equation for mass previously
implies that $ m_0 \ge  \frac{4}{3 }  \pi r_0 ^ 3 \rho _ 0 $. 

The other thing we have is that 
if we set $ r = r_0  $ , then we have 
\[
\frac{m_0}{r_0 }  = \frac{2}{9}  \left[  
1 - 6 \pi r_0 ^ 2 p_0 + \left( 1 + 6 \pi r_0 ^ 2 p_0  \right)  ^{ \frac{1}{2} } \right] 
\] where $ p_0 = p ( r_0 ) $ is known 
from the equation of state. 

The RHS of the equation above 
decreases with $ p_0 $. So, at $ p_0  =0 $, we get the Buchdahl bound
\[
m_0 < \frac{4}{9 } r_0 
\] so the core on its own satisfies the Buchdahl bound. 
We have two inequalities which we can use to constrain 
the parameters. 

(Insert plot of $ m_0  $ against $ r_0 $  ). 

This shows that the mass of the 
core satisfies 
\[
m_0  < \sqrt{ \frac{ 16 }{ 243 \pi \rho _ 0 } }  \rho_0 \sim \text{ nuclear density }
\le  5 M _{ \sun }
\] for any $ \left( m_0 , r_0  \right)  $ in allowed region, 
we can solve TOV 1 and TOV 3 in the envelope region with 
$ \rho = \rho _ 0 , m = m_0 $  at $ r= r_0 $. 
This fixes $ M $ in terms of $ \left( m_0 , r_0  \right)  $. 
We find numerically that $ M $ is maximised when $ m_0 $ is 
maximised. Furthermore, at that point, it turns out 
the envelope is very small. So, the upper bound of the 
total 
\[
M_{ \text{max } } \le  5 _{ \sun }
\]

\section{The Schwarzchild Solution}
If we're given a Schwarzchild solution 
with no matter present, what does this solution describe?
In this section, we'll explore this 
question and also look at the nature 
of singularities in this spacetime. 
\subsection{Birkhoff's Theorem}
In Schw coordinates $ \left( t , r, \theta, \phi  \right)  $
the Schwarzchild solution is 
\[
ds ^ 2  =  - \left( 1 - \frac{2M}{r }  \right) dt ^ 2 + 
\left( 1 - \frac{2M}{r }  \right)  ^{ - 1 } dr ^ 2 + r ^ 2 d \Omega
\] the only free parameter here is $ M $.
The location $ r  = 2M $ is the Schwarzchild radius, 
and we want to understand what happens here.
To derive this metric, 
what we assumed before was static and spherically symmetric.

\begin{thm}{Birkhoff's Theorem}
Birkhoff's theorem says that 
any spherically symmetric solution 
of the vacuum Einstein equations 
is isometric to the Schwarzchild solution. 
In other words, you don't need the static condition on

So
\[
\text{Spherically symmetric}  + \text{Vacuum} \to \text{static}
\] This is interesting. 
If we have any star, even if it's time independent, 
if it's spherically symmetric then the metric 
outside the star will have a time independent solution. 
Conversely, we can't have 
spherically symmetric gravitational waves. 


\end{thm}
\subsection{Gravitational Redshift}
(Diagram of Alice and Bob light rays, two vertical lines 
with coordinate time interval $ \Delta t $ ). 
So assume we have $ A, B $ at fixed $ \left( r, \theta, \phi  \right)  $. 
Suppose sends $ A $ sends 2 photons to $ B $ separated 
by $ \Delta t $. By time translation symmetry, 
we have that the receiving of these photons 
by Bob also has coordinate time difference
$ \Delta t $. 
Alice measures proper time. 
We find that 
\[
\Delta \tau _ A  = \sqrt{ 1 - \frac{2M}{r_A } }  \Delta t 
\] If $ r _ B > r _ A  $, the proper time 
received by $ B $ is 
\[
\Delta \tau _ B  = \sqrt{ 1 - \frac{2M}{r _ B } }  \Delta t 
\] Taking the ratio, 
we have that 
\[
\frac{ \Delta \tau  _ A }{ \Delta \tau _ B  } = 
\frac{ \sqrt{ 1 - 2M / r _ B } }{\sqrt{  1 - 2M / r _ A } } > 1 
\]  We can also look at this 
from the perspective of 
successive wave crests. 
The time interval in natural units between 
two successive wave crests is 
\[
\Delta \tau = \lambda, \quad \lambda _ B > \lambda _ A \text{ redshift }
\] The light has to crawl out 
of a potential well. 
A useful case to consider is 
when Bob is very far from the Schwarzchild radius. 
If Bob has radius $ r _ B \gg 2M $, then the 
numerator is just $ 1 $. 
We then have the redshift $ z $ as 
\[
1 + z  = \frac{ \lambda _ B }{ \lambda _ A }  = \frac{1}{ 1 - 2M / r _ A }
\] which diverges as $ r _ A \to 2M $. 
If we had a spherical star with $ A $ on the surface, 
so that  $ r _ A  = R  > \frac{9}{4 } M $, due to the 
Buchdahl bound this implies 
that $ z < 2 $ for light on the surface of a spherical 
star. 

Let's explore the geometry of 
the solutions. 

\subsection{Geodesics}
Let's assume we have 
affinely parametrised geodesics where 
$ x ^ \mu \left( \tau  \right)  $ , 
and we define $ u ^ \mu  = \frac{ dx ^ \mu }{ d \tau } $. 
If we let $ k  = \frac{\partial  }{\partial  t }  $, 
this is one killing vector field, 
and we have another one $ m   = \frac{\partial  }{\partial  d \phi }  $. 

These give us conserved quantities 
\[
E =  - k \cdot  u  = \left( 1 - \frac{2M}{ r }  \right)  \frac{dt}{d \tau }, 
\quad h = m \cdot  u  =  r^ 2 \sin ^ 2 \theta \frac{ d \phi }{ d \tau } 
\]  These are constant along an affinely 
parametrised geodesic. 
On a timelike geodesic,  
we normalise so that $ \tau $ is propertime. 
At large r, the Schw spacetime becomes 
Minkowski spaecetime, so we get special relativity. 
Using special relativity to understand what these constants are, 
we have that $ E $ represents the energy per unit rest mass, 
and $ h $ represents the angular momentum per unit rest mass. 

In the null case, these two quantities are not physical, 
and rescaling $ \tau $ allows use to rescale $ E $ and $ h$ . 
However, $ h / E $ is invariant. 
This is the impact parameter $ b  = |\frac{ h  }{E} | $  

What does this mean? 
If we consider a light ray far away approaching a 
star, we have that the impact parameter 
is the closest distance of the ray to 
the object. 
From the Euler Lagrange equations, 
we have that 
\[
r ^ 2 \frac{ d }{ dr } \left( r ^ 2 \frac{d \theta }{ dr }  \right)  
 - h ^ 2 \frac{ \cos \theta }{ \sin ^ 2 \theta }  =0 
\]  Rotate $  S^ 2 $ such that $ \theta \left( 0  \right)   = \sin \frac{\pi}{2 }$. 
We can also rotate our coordinates again such that $ \dot{ \theta } \left( 0  \right)   = 0  $ . In other 
words, our geodesic lies in, and moves 
tangentially, to the equatorial plane $ \left( \theta  = \frac{\pi }{ 2 }  \right)  $. 
By uniqueness of solutions, this must 
be the unique solution 
to the geodesic equation above. 
So, the geodesic stays in the equatorial plane. 

As an exercise, the normalisation 
of our parameter means that 
\[
g _{ \mu \nu } u ^ \mu u ^ \nu  =  - \sigma , \quad 
\sigma = \begin{cases}
  1 & \text{timelike } \\
  0 & \text{null } \\
   -1 & \text{spacelike } 
\end{cases}
\] we can show that 
we get an equation for $ r $ 
which is 
\[
 \frac{1}{2 } \left( \frac{dr}{d \tau }  \right)  ^ 2 + V ( r ) 
 = \frac{1}{2 } E ^ 2 , \quad V ( r )  = \frac{1}{2 } 
 \left(  1 - \frac{2M}{r }  \right)  \left( \sigma + \frac{h ^ 2 }{ r ^ 2 } \right) 
\] This looks like the equation 
moving in a classical potential. 
By looking at qualitative properties 
of the potential, we 
can figure out qualitative 
properties of the potential. 

\subsection{Eddington Finkelstein Coordinates}
Let's start in the region $ r > 2M $. 
The simplest geodesics are the 
radial  null geodesics, where  $ \theta, \phi $ are constant. 
This implies that $ h = 0 $. We can rescale $ \tau $ to 
set $ E  = 1 $. 
This gives us the equation 
\[
 \frac{ dt }{ d \tau }  =\left( 1 - \frac{2M}{2 }  \right) ^{  -1 } , \quad 
 \frac{ dr }{dt }  = \pm 1 
\] We have $ \frac{dr}{dt }  = 1 $ corresponding to 
outgoing coordinates, and $ \frac{dr}{dt }  =  -1  $ 
corresponding to ingoing. 

Ingoing reaches  $ r  = 2M $ in finite time. 
Dividing the equations above 
\[
\frac{ dt }{ dr }  = \pm \left( 1 - \frac{2M}{r }  \right)^{  -1 } \implies
t \to \mp \infty \text{ as } r \to 2M $
\] However, if we let $ dr _ *  = \frac{dr}{1 - \frac{2M}{r } } $, 
then we get that 
\[
r _ *  = r + 2M \log | \frac{r}{2M }  -1  | 
\] (Draw a diagram of this function). 

Why have we defined this coordinate 
as such? Well, if we 
divide through, 
we have that $ \frac{dt}{dr_ * }  = \pm 1 $. 
This means that $ t \mp r _ *  = C $. 
Let's define $ v  = t + r _  * $. 
This is a constant on ingoing radial null geodesics. 
Let's now introduce new coordinates $ (v, r, \theta, \phi )$, 
which are called ingoing EF coordinates. 
If we invert this transformation, 
we get 
\[
t  = v - r _ *  \implies dt   = dv - \frac{ dr }{ 1 - \frac{2M}{r } }
\] Thus, our new metric looks like 
\[
ds ^ 2  =  - \left( 1 - \frac{2M}{r }  \right)  dv ^ 2 + 2 dv dr + r ^ 2 d \Omega ^ 2 
\] The 
metric looks like 
\[
g _{ \mu \nu } = \begin{pmatrix}   - \left( 1  - \frac{2M}{r }  \right)  
& 1 & 0 & 0 \\ 
1 & 0 & 0 & 0 \\
0  & 0 & r ^ 2 & 0 \\ 
0 & 0 & 0 & r ^ 2 \sin ^ 2 \theta \end{pmatrix} 
\] This is smooth $ \forall r > 0 $ ! 
We get that $ \det g_{ \mu \nu }  =  - r ^  4 \sin ^ 2 \theta $. 
Thus, this is non-degenerate $ \forall r > 0 $. 
Thus, $ g _{ \mu \nu } $ is Lorentzian $ \forall r > 0 $. 

We can extend this spacetime through $ r = 2M $ to 
a new region $ 0 < r < 2M $. 
(Look at analyticity). 
For $ r < 2M $, we now define $ r _ * $ by 
going back to the modulus sign, 
and $ t$, by $ t = v - r _ * $. 
We can transform the metric 
to coordinates $ \left( t, r, \theta, \phi  \right)  $. 
Our exercise is to show that it is the 
$ r < 2M $ Schwarzchild solution. 

An ingoing radial null geodesic 
has $ \frac{dr }{ d \tau }  = - 1 $ and $ v = \text{const } $. 
This reaches $ r = 0 $ at finite $ \tau $. 
What happens here? Our curvature diverges. 
The simplest scalar which 
we can build from the Riemann tensor is 
\[
R _{ abcd } R ^{ ab cd }  \propto \frac{ M ^ 2 }{ r ^ 6 } \to \infty, \quad \text{ as } 
r \to 0 
\]  This scalar diverges in all charts. 
We can't smoothly extend like we did with the change of coordinates. 
The metric cannot be smoothly extended through $ r = 0 $. 
At $ r = 0 $, 
we get what is called a curvature singularity. 
We get $ \infty$ tidal forces. 
Strictly speaking, $ r   = 0$ is not defined 
to be part of the spacetime since $ g _{ ab } $ is not smooth there. All we should be talking 
about is the limiting behaviour as we approach $ r = 0 $. 

Another important thing is to look at the 
Killing vector field $ k  =\frac{\partial  }{\partial  t }  $ with $ \left( r > 2M   \right)  $. Converting this to EF coordinates. 
This becomes 
\[
k = \frac{\partial  x ^ \mu }{\partial  t }  \frac{\partial  }{\partial  x ^ \mu }  
= \frac{\partial  }{\partial  v } 
\]  We can use this to define $ k $ for $ r \leq 2M $. 
If we look at the norm of the Killing field in new coordinates, 
\[
k ^ 2 = g _{ v v }  =  - \left( 1 - \frac{2M}{r }  \right)  
\] If we cross the region $ r  = 2M $, $ k $ is null. 
And, inside the surface, it's spacelike. 
Since the definiton of being static requires 
the Killing vector field to be timelike, 
this means that the interior is not static. 
Only the $ r > 2M $ region is static. 

\subsection{The Black Hole Region}
In this section, 
we attempt to make rigorous the 
definition of a 'black hole region'. 

\begin{defn}{Causal vector}
A vector is causal if 
it is either timelike or null. 
A curve is causal if
the tangent vector is everywhere causal. 
\end{defn}

At any given point in spacetime, 
the set of causal vectors allows 
us to define two lightcones, 
the future lightcone and past lightcone. 

\begin{defn}{Time Orientable Spacetime}
A spacetime is time-orientable 
if it 
admits a time orientation, which is 
some causal vector field $ T ^ a  $. 
We say that a causal vector $ X ^ a  $
is future-directed if, at that given 
point, it lies in the same lightcone as $ T ^ a $. 
We call it past directed otherwise. 

Note that any other 
time orientation on the spacetime $ U ^ a $
necessarily lies everywhere in the 
past or future lightcone of $ T ^ a $. 
This means that any given 
spacetime has two inequivalent time-orientations. 
\end{defn}

For ingoing Eddington-Finkelstein coordinates, 
we have that $ k  = \partial  / \partial v  $  	 
can't be a valid time orientation 
since it's spacelike for the 
region $ r < 2M $. 
But, $ \pm \partial  / \partial  r $ is a 
globally null vector field. 

We would however, like to 
choose the correct sign based on 
making it have a consistent orientation 
with the Killing vector field $ \partial  / \partial  t  = 
\partial   / \partial  v $. 

We find that $ k \cdot  \left(   - \partial  / \partial  r  \right)   = 
- g_{ vr }  = -1 $. Thus, 
the vectors lie 
in the same light-cone, so $  - \partial   / \partial  r $ 
is the time-orientation we'll use. 

(Insert missing lecture notes here )
We can see that the energy of a 
circular orbit is 
\[
E  = \frac{ r -  2M }{ r ^{ \frac{1}{2 }} \left( r - 3M  \right)  ^{ \frac{1}{2 } } } 
\] If we have  
$ r \gg 2M $, we get that 
\[
E \simeq 1 - \frac{M}{2 r } 
\] This implies that 
the energy is given by 
\[
\text{energy }  = m  - \frac{ m M }{ 2 r}  
\] where $ m $ is our initial rest mass, 
and the second term is our 
gravitational binding energy. 
If we're talking about a 
solar mass black hole, with $ M \leq 100 M_{ \sun} $, 
these are the black holes 
formed by gravitational collapse of a star. 

How do we see this? 
The first way that black holes 
were observed is by looking at 
binary systems. 
(Diagram of star and black holes)
When matter from the star is stripped 
off due to tidal forces, the 
matter surrounds the black hole 
in an accretion disk. 
Let's try to understand the properties of one of 
these disks with a crude model. 

Let's approximate the disc as particles following circular orbits. 
Each particle in the disc has some energy. 
There's friction 
amongst the particles. Thus, energy is lost 
due to friction, and hence the radius 
decreases, until 
it reaches the ISCO.
The ISCO gives the disc an 
inner edge of $ r = 6M $. 
When it reaches this and loses more energy, it 
just falls into the black hole. 
The energy of the ISCO is $ E  = \sqrt{ 8 / 9 }   $. 
The remaining energy fraction 
is $ 1 - \sqrt{ 8 / 9 }   $, which is the energy 
of the rest mass lost to friction. 
This is carried away from the disc 
in the form of electromagnetic radiation. 
This is about $ 6 \pc $, which is a huge amount 
of energy converted. This accretion discs 
around black holes are very luminous. 
The first detections are from X-rays in the 1980s. 

Now there's another way 
to detect black holes, which are gravitational waves. 


\subsection{White Holes}
Taking a step back, if 
we define $ r > 2 M $ the coordinate $ u : = t - r _ * $, 
constant along outgoing radial null geodesics, 
we have the set of coordinates $ \left( u ,r  \theta, \phi  \right)  $, 
which are outgoing EF coordinates. 
$ ds^ 2  =  - \left( 1 - 2M  / r  \right)   - 2 du d r + r ^ 2 d \Omega ^ 2 	  $. 
We can smoothly extend this 
through $ r  = 2 M $ to  $ r \leq 2 M $, 
with a curvature singularity at $ r  = 0$. 
This is not the same as previous $ r  = 2 M $ region. 
If we look at the outgoing radial null geodesics, 
, $ u   = \text{const } $, and $ \frac{dr}{ d \tau }  =  1 $. 
Here, $r  $  can only increase in $ r \leq 2 M $, 
so it can't be the same region as we saw before 
since it has different properties. 

As an exercise, we can show 
$ k  = \partial   / \partial  u $. 
In outgoing EF coordinates, we can show $ \partial   / \partial  r $ , 
is the time orientation equivalent to $ k $ in 
$ r > 2M $.  
The $ r < 2 M $ is a white hole. 
This is a region which cannot receive a 
signal from infinity. 
To understand what a white hole is, 
it is just the time reverse of a black hole. 
So, $ u  = - v $ is an isometry which maps the white 
hole to the black hole, and reverses the time orientation. 

White holes are generally regarded as unphysical, 
because there is no mechanism for 
forming them. 
To form a white hole, we'd 
have to start with a singularity. 
White holes are also unstable due to time reversal 
since black holes are stable. 

\subsection{The Kruskal Extension}
Again, we start off 
with $ r > 2 M $, the Kruskal-Szekeres 
coordinates are 
\[
U  =   - e^{   - u  / 4 M }, U < 0 , \quad V  = e ^{ v / 4 M }, V > 0 
\]  This gives $ \left( U , V , \theta, \phi  \right)  $. 
Taking the product, we get that 
\[
U V  =  - e ^{  r _ *  / 2 M }  =   - e ^{ r / 2 M} \left( r / 2M  - 1  \right)  
\] The right hand side 
is monotonic, 
which determines $ r \left( U , V  \right)  $ uniquely. 
We $ V / U  = e ^{ t / 2 M } $ , fixes $ t \left( U , V  \right)  $. 
Transforming coordinates, we 
have $ ds ^ 2  =  - \frac{32 M ^ 3 }{ r \left( U , V  \right)  } e ^{ 
- r \left( U , V  \right)   / 2 M } d U d V r \left( U , V  \right)  ^ 2 d \Omega^ 2  $. 
To extend 
to a larger range of $ U $ and $ V $, 
we can use the equation above 
to define $ r \left( U , V  \right)  $ for $ U \geq 0 $, 
or $ V \leq 0 $. 
Metric can then be analytically extended with 
$ \det g _{ \mu \nu } \neq   0 $, to new regions $ U > 0 $ or $ V < 0 $. 

At $ r = 2M $, $ U V  = 0 $. 
This corresponds to $ U =  0 $ or $ V = 0 $. 
At  $ r = 0 $, we require $ U V  = 1 $, 
which are equations of hyperbola.

The shaded regions are not apart of the diagram. 
Radial null geodesics are 
constant $ U $ or $ V $, 
depending on ingoing or outgoing. 
If we pick $ r > 2 M $, and constant $ r $, 
this has hyperbolae on either side. 
We have four regions. 
Region I has $ r > 2 M $, 
which is Schwarzchild. We have region II, which is 
the black hole region, 
region III, the white hole region, 
and region IV, which is new. 

Region IV with $ R > 2M $, 
is Isometric to $ I $. 
We should have $ \left( U , V  \right)  \to \left(  - U , - V  \right)  $. 
The interior of a black hole is quite 
similar to the collapse of a universe. 
This is somewhat akin to the big crunch.
You shouldn't think of a singularity as a place in space. 
It's more like a region in time. 

If we have a star which undergoes collapse, 
the space time outside the star corresponds to this diagram. 
(Draw diagram of surface of the star)

Let's 
now think about the symmetries of this. 
As an exercise, 
we can show that 
\[
k  = \frac{1}{4 M } \left( V \frac{\partial  }{\partial  V }   - 
U \frac{\partial  }{\partial  U }   \right) , \quad k ^ 2 =  - 
\left(  1- 2M / r  \right) 
\] Thus, $ k $ is timelike in I and IV, 
null at $ U  = 0  $ and $ V  = 0 $, 
and spacelike in II or III. 
Drawing the integral curves of $ k $, 
we have the diagram below. 

Note that the lines $ \left\{  U = 0  \right\}  , \left\{  V  = 0  \right\}  $, 
are mapped to themselves, 
and so are fixed by $ k $. $ k  = 0 $ 
is on a bifurcation 2 -sphere 
$ U  = V  =0 $ which is also fixed by $ k $.


\subsection{Einstein Rosen Bridges}
For $ t $ constant in region $ I $, 
this happens if and only if $ V /  U $ is constant. 
This manifold extends in to $ I V $. 
If we 
let $ r  = \rho + M + \frac{ M ^ 2  }{4  \rho } $, 
the graph looks like below. 

For a given value of $ r $, there are 
two possible values of $ \rho $. 
We choose $ \rho > M / 2 $ in region I, 
and $ 0 < \rho < M / 2 $ in region $ I V $. 

It is an exercise to show that 
in isotropic coordinates, 
we have that $ \left( t, \rho , \theta, \phi  \right)  $ 
is 
\[
ds ^ 2 =  - \frac{\left(  1- \frac{M}{2 \rho }  \right)  ^ 2 }{ 
\left(  1+ \frac{M}{2 \rho }  \right)  ^ 2 } dt ^ 2 
+ \left( 1 + \frac{M}{2 \rho }   \right)  ^ 4 
\left(  d \rho ^ 2 + \rho ^ 2 d \Omega^ 2  \right)  
\] We said earlier 
that region $ I  $ and $ I V $ are 
isometric. 
Thus, we should have some isometry which 
relates regions I and $ I V $. This is 
\[
\rho \to \frac{M^2}{4 \rho } 
\] There is still a singularity in 
these coordinates at the Schwarzchild radius. 
If we set $ t  $ to be a constant, 
then the induced metric 
is 
\[
ds ^2 = \left(  1 + \frac{M}{2 \rho }  \right)  ^ 4 
\left( d \rho ^ 2 + \rho ^ 2 d \Omega^ 2  \right) 
\] This is smooth for $ \rho > 0 $. 
We can embed this in four dimensional 
Euclidean space. 
If we surpress the theta direction, 
we get that we have 
two asymptotically flat hypersurfaces 
connected by a throat region. 

We're suppressing the theta directions 
so it looks like cirles where we have the throat. 
It is flat as $ r \to  0 $ since the space 
is isometric to $ r \to \infty $. 
This is called an Einstein-Rosen bridge. 

\subsection{Extendibility}
\begin{defn}{Extendibility}
We define $ \left( M , g  \right)   $ to be 
extendible if it is isometric 
to a proper subset of another spacetime $ \left( M ' , g ' \right)   $. 
This is called an extension of $ \left( M , g  \right)  $. 
For example, take $ \left( M , g  \right)  $ to be 
the $ r > 2M $ section of the Schwarzchild manifold. 
Then, we can take $ \left( M ' , g'  \right)  $ 
to be the extension of $ \left( M , g  \right)  $. 
\end{defn}
If we can make the spacetime bigger, 
then we do. But, 
the Kurskal spacetime for example is 
inextendible.
The Kruskal spacetime is an example 
of a maximal analytic extension of 
$ \left( M , g  \right)  $. 

\subsection{Singularities}
In general, we say that the metric 
$ g _{ \mu \nu } $ is singular if it is not 
smooth or $ \det g _{ \mu \nu }  = 0 $ somewhere. 
A coordinate singularity 
is a singularity which we can eliminate 
via a change of coordinates. For example, 
$ r = 2M $ in Schwarzchild coordinates. 
These are unphysical. 

We also have a scalar curvature 
singularity, where a scalar build from $ R \indices{ ^ a _{ bcd  } }   $ 
diverges. For example, $ r = 0$ in 
Schwarchild. 

We can also have examples of singularities 
in which the curvature does not blow up. 
These are called non-curvature singularities. 

For example, take $ M  = \mathbb{ R  }^ 2 $, 
with polar coordinates $ \left( r, \phi  \right)  $, 
with $ \phi \sim \phi + 2 \pi $. 
Take $ g  = dr ^ 2 + \lambda ^ 2 r ^ 2 d \phi ^ 2 $. 
For $ \lambda > 0 $, the determinant 
$ \det g _{ \mu \nu }  = 0 $ at $ r = 0   $. 
At $ \lambda  = 1 $, 
we have that this is Cartesian coordinates, which 
implies $ r = 0 $ is a coordinate singularity. 

For $ \lambda \neq 1 $, 
set $ \phi  ' = \lambda \phi $, which implies $ g  = dr ^ 2 
+ r ^ 2 d \phi ^ 2 $. 
This is locally isometric to Euclidean space, 
so we have that $ R \indices{ ^ a _{ bcd } }  = 0  $. 
This isometry is only local 
because we've forgotten the period of $ \phi $. 
Since $ \phi  ' \sim \phi ' + 2 \pi \lambda $, 
this is not globally isometric. 
At $ r = 0 $, 
let's consider the circle $ r  = \epsilon $. 
The circumference divided by the radius is 
\[
\frac{2 \pi \lambda \epsilon }{ \epsilon } = 2 \pi \lambda \neq 2 \pi , \quad 
\text{as  } \epsilon \to 0 
\] Taking the limit as $ \epsilon \to 0 $, 
This is not 
locally flat at $ r  = 0 $, so the metric can't be smoothly 
extended to $ r = 0$. This is 
an example of a conic singularity. 
This is not a curvature singularity since 
curvature is zero. 


Singularities are not points in the manifold 
since they're not included by construction. 

\begin{defn}{Future endpoints}
We say that $ p \in \mathcal{ M } $ is a future 
endpoint of a future directed causual curve $ \gamma : \left( a,b  \right)  
\to \mathcal{  M } $, if, for any neighbourhood
$ \mathcal{ O } $ of $ p $, there 
exists some $ t_0 $ such that $ \gamma \left(  t  \right)  
\in \mathcal{ O } $ for all $ t > t_0  $. 
We say that $ \gamma $ is future inextendible 
if it has no future endpoint.
\end{defn}

For example, take our spacetime $ \left( M , g  \right)  $ 
to be Minkwoski, and the curve 
$ \gamma : \left(   - \infty, 0   \right)  \to M $, 
$ \gamma \left( t  \right)   = \left( t, 0 , 0 , 0  \right)  $. 
Then, $ \left( 0 , 0 , 0 , 0   \right)  $ is 
a future endpoint. 
However, if we define $ \left( M , g  \right)   = \text{ Mink } 
\ \left\{  \left( 0 , 0 , 0 , 0  \right)   \right\}   $, 
then the curve is future inextendible.

\begin{defn}{Completeness}
A geodesic is complete 
if an affine parameter 
extends to $ \pm \infty.$.
We say that $ (M , g ) $
is geodesically complete if all inextensible causal 
geodesics are complete. 
\end{defn}
For example, 
in Miknowski space, and the spacetime of a
static, spherical star, these are geodesically complete. 

In the Kruskal spacetime, 
this is geodesically 
incomplete since 
some geodesics reach $ r =  0 $ in finite affine 
parameter. 

An extendible spacetime is trivially geodesically incomplete. 
We say that 
a spacetime is singular if and only if 
$ \left( M , g  \right)  $ is inextendible and geodesically incomplete. 
For example, the Kruskal spacetime. 


\pagebreak
\section{The Initial Value Problem}


\subsection{Predictability}
\begin{defn}{Partial Cauchy Surface}
Suppose $ \left( M , g  \right)  $ is a 
time orientable spacetime. A partial Cauchy surface 
$ \Sigma $ is a hypersurface such that no 
two points are connected by a causal curve in $ M $. 
So, this could look like a $ t  = \text{const} $
curve in Minkwoski spacetime, since 
all points are spacelike separated. 
The future domain of dependence of $ \Sigma $
is defined as 
\[
  D^ + \left( \Sigma  \right)   = 
  \left\{  p \in M : \text{every past-inextendible causal curve through }  p 
  \text{ which intersects } \Sigma  \right\} 
\] The past domain of 
dependence $ D ^  - \left( \Sigma  \right)  $ is defined 
similarly. The entire domain of 
dependence is the union of the 
future and past domains of dependence
\[  D \left( \Sigma  \right)   = 
D^ + \left( \Sigma  \right)  \cup D ^  - \left( \Sigma  \right)  \] 
\end{defn}

A causal geodesic in $ D \left( \Sigma  \right)  $ must intersect $ \Sigma $, 
which is determined uniquely by tangent vectors on $ \Sigma $. 
A causal geodesic is a timelike or null curve, 
and is either future or past inextendible. 
It is 
in the domain of dependence since 
geodesics are determined 
by the tangent vector at $ p \in \Sigma $, 
where curve intersects. 

The solutions to hyperbolic 
partial differential equations 
in $ D \left( \Sigma  \right)  $ 
can be uniquely determined from 
initial data defined on $ \Sigma $.

By hyperbolic partial differential equations, 
we mean equations in the tensor field 
$ T \indices{ ^{ \left( i  \right) a b \dots } 
_{ cd \dots } }  $, with equations 
of motion, $ i  = 1 , \dots N $
equations of motion 
\[
g ^{ ef } \nabla _ e \nabla _ f T \indices{ ^{ i ab \dots } 
_{ cd \dots } }  = \dots  
\] where the right hand side depends on $ g $ and its derivatives, 
and depends linearly on $ T  $ and it's first derivatives.
These equations are satisfied for example, 
by Maxwell's equations in the Lorentz gauge. 


\begin{example}{Minkowski Spacetime with the positive 
	$ x  $ axis as the partial Cauchy surface}

	For example, let's have $ \left( M ,g  \right)  $ taken to 
be 2 dimensional Minkwoski space, and 
$ \Sigma $ to be the positive $ x $ axis. 
Then, we can draw the domain of 
dependence as shown in the figure.
We draw the domain of dependence 
by taking a point, and seeing 
if there exists a timelike 
curve which doesn't intersect the Cauchy surface. 
If there is, it's not in the domain of dependence. 

\begin{figure}[htpb]
	\centering
	\input{domain_of_dependence.pdf_tex}
	\caption{The domain of dependence 
	for the positive x-axis in Minkowski spacetime}%
	\label{fig:}
\end{figure} 

If we have a look at the wave equation 
\[
 \nabla ^ a \nabla _ a \psi  =  - \partial  _ t ^ 2 \psi + 
 \partial _ x ^ 2 \psi  = 0 
\] the solution in $ D \left( \Sigma  \right)  $ 
us uniquely determined by the data 
$ \left( \psi , \partial  _ t \psi  \right)   $ on 
$ \Sigma $. 
\end{example}


Generally, 
if $ D \left( \Sigma  \right)  \neq M $, then physics 
in $ M / D \left( \Sigma  \right) $ is not 
determined by data on $ \Sigma $. 

\begin{defn}{(Globally Hyperbolic)}
	$ \left( M , g  \right)  $ is 
	globally hyperbolic if there is a Cauchy surface. 
	A Cauchy surface is a partial Cauchy surface 
	such that $ D \left( \Sigma  \right)   = M $. 
	The Cauchy Horizon is a boundary of $ D \left( \Sigma  \right)  $ 
	in M. A space is globally hyperbolic if there is no Cauchy hoirzon for 
	$ \Sigma  $. 
\end{defn}

Some examples of globally hyperbolic spacetimes 
is the Minkowski metric ($ t = const  $  )
are Cauchy surfaces. 
Kruskal spacetime is also a globally hyperbolic 
spacetime. 

(Insert diagram here)

We can also look 
at spherical gravitational 
collapse, 
which also is a globally hyperbolic spacetime. 

The Minkowski metric 
with the origin removed is not since 
there is a future inextendible curve. 

\begin{thm}
	Let $ \left( M , g  \right)  $ be globally hyperbolic.
	Then, (i), there exists a global time function 
	$ t : M \to \mathbb{ R } $ such that 
	$  - \left( dt  \right)  ^ a $ is future directed and timelike. 
	Secondly, $ t = \text{const}$ surfaces are Cauchy, 
	and all have the same topology $ \Sigma $. 
	Finally, $ M $ has topology $ \mathbb{ R } \times \Sigma $. 
\end{thm}
As an exercise, show that $ U + V $ is a global
time function for Kruskal. 
Note that the surface $ U + V  = 0 $ is an 
Einstein-Rosen bridge. 
Topologically, $ \Sigma \simeq \mathbb{ R } \times S ^ 2 $. 
In this case, $ M \simeq \mathbb{ R  }^ 2 \times S ^ 2 $. 

We can also introduce coordinates on the spacetime. 
Let  $ x ^ i $ be coords on $ t =0 $ surface, $ \Sigma $. 
Let $ T ^ a $ be a timelike vector field with $ p \in \mathcal{ M  } $. 
The integral curve of $ T ^ a  $ through $ p $ 
intersects $ \Sigma $ at a unique point. Let $ x ^ i(p) $ 
be the coordinates of this. 
This defined three maps $ x ^ i: M \to \mathbb{ R  }$. 
Use $ \left( t, x ^ i  \right)   $ as 
coordinates on $ M $. 
So, we have a global 
set of coordinates. 
If we write down the most general 
metric in these coordinates, 
\[
 ds ^ 2  =  - N ^ 2 dt ^2 + h _{ ij } 
 \left(  d x^ i + N ^ i dt  \right)  \left( dx ^ j  + N ^ j dt  \right)  
\] $ N \left( x, t  \right)  $ 
is called the lapse function, $ N ^ i \left( t, x  \right)   $ 
is called the shift vector, 
and $ h _{ ij  } \left( t, x  \right)  $ 
is a metric on surfaces of constant $ t $.

\subsection{Extrinsic Curvature}
\begin{defn}
	We say the hypersurface $ \Sigma $ is spacelike 
	if the normal $ n _ i $ everywhere is timelike. 
	If $ X ^ a $ is tangent, then $ n _ a X ^ a  = 0 $ implies 
	that $ X ^ a $ is spacelike.

	Assume $ n _ a  n ^ a   = - 1 $. 
	We now define the quantity $ h \indices{ ^ a _ b }  = 
	\delta \indices{ ^  a _ b }  + n ^ a n _ b  $. 
	This implies that $ h _{ ab }  =g _{ ab } + n _ a n _ b $. 
\end{defn}
Thus, if $ X ^ a, Y ^ a $ are tangent vectors, 
then $ h _{ ab } X ^ a Y ^ b   = g _{ ab } X ^ a Y ^ b  $. 
$ h _{ ab } $ is the induced metric of $ \Sigma $, 
which is the pullback of $ g _{ ab } $. 
Then $ h \indices{ ^ a _ b } n ^ b  = 0 $. 
This means that $ h \indices{ ^ a _   c } h \indices{ ^ c _b } 
 = h \indices{ ^ a _ b }  $. 
 $ h \indices{ ^ a _ b }  $ is a projection 
 onto $ \Sigma $.
 We can decompose the vector as follows
 \[
  X ^ a  = \delta \indices{ ^ a _ b } X ^ b 
   = h \indices{ ^ a _ b  } X ^ b   - n ^  a n _ b X ^ b  = 
   X ^ a _{ \parallel} + X ^ a _{ \bot }
 \] We have that $ N _ a $ is perpendicular to 
 $ \Sigma $ at $ p $. 
 Parallel transport $ N _ a $ along $ C : X ^ b \nabla _ b N _ a =0  $.
 Does $N _ a 	 $ remain $ \bot $ to $ \Sigma $? 
 Suppose that $ Y ^ a $ is tangent to $ \Sigma $. 
 Then
 \[
	 X \left( N \cdot  Y  \right)  = X ^ b \nabla _ b 
	 \left(  Y ^ a N _ a   \right)   = N _ a X ^ b \nabla _ b Y ^ a
 \] If $ N \cdot  Y  = 0 $, 
 then $ ( \nabla _ X Y )_{ \bot  } = 0 $

 \begin{defn}
 Extend $ n _a $ to neighbourhood $ \Sigma $,  $ n _ a n ^ a  = - 1 $. 
 The Extrinsic curvature tensor $ K _{ ab } $ is defined 
 at $ p \in \Sigma $ by $K \left( X, Y  \right)   =  - n _ a 
 \left( \nabla _{ X _{ \parallel } } Y _{ \parallel  }  \right) ^ a $. 
 \end{defn}

 \begin{thm}
 	We have that, independent of the 
	extension of $ n_ a $, 
	we have that 
	\[
	 K _{ ab }  = h \indices{ _ a ^ c } h \indices{ 
	 _ b ^ d  } \nabla_ c n _ d   
	\]
\begin{proof}
	We have that 
	\begin{align*}
 - n _ d X _{ \parallel  } ^ c \nabla _ c Y_{ \parallel } ^ d &=   - X_{ \parallel } 
 ^ c \nabla _ c \left( n _ d Y ^ d _{ \parallel }  \right)  + 
 X _{ \parallel } ^ c Y _{ \parallel }^ c \nabla _ c n _ d  \\
							      &=  \left( n \indices{ _ a 
							      ^ c  } h \indices{_ b 
					      ^ d } \nabla _ c n _ d     \right)  
 X^ a Y ^ b \\
	\end{align*}
	Note that $ n ^ b \nabla _ c n _ b  = \frac{1}{2 } 
	\nabla _ c \left( n _ b n ^ b   \right)   = 0 $. 
	This implies that $ K _{ ab }  = h \indices{ _ a ^ c  } 
	\nabla _ c n _ b $. 
\end{proof}
 \end{thm}

 \begin{thm}
 	Our extrinsic curvature 
	tensor is symmetric, with $ K_{ ab }  = K _{ b a } $. 
	Let $ \Sigma  $ be a surface where 
	$ f   $ is constant, with $ df \mid _{ \Sigma } \neq 0 $. 
	Therefore, $ n _ a \mid _{ \Sigma }  = g (df )_ a $ 
	for some $ g $, fixed by $ n _ a n ^ a  = - 1 $. 
	We can use this to extend $ n _ a  $  off $ \Sigma $. 
	\[
		\nabla _ c n _ d  = g \nabla _ c \nabla _ d f 
		+ \nabla _ c g \nabla _ d f 
	\] this implies $ K _{ ab } = g h \indices{ _ a 
	^ c  } h \indices{ _ b ^  d  } \nabla  c \nabla _ d f      $, 
	which is symmetric. 
	A lemma that we have is $K _{ ab }  = \frac{1}{2 } \mathcal{ L }  _n 
	h _{ ab }  $. 
 \end{thm}


\subsection{Gauss-Codecci equations}
The tensor at $ p \in \Sigma $ is 
invariant under projection $ h \indices{ ^ a _ b }  $ 
if 
\[
	T \indices{ ^{ a_1 \dots a_ r  } _{ b_1 \dots b_ s }  } 
	 = h \indices{ ^{ a_1 } _{c_1 }  }  \dots 
	 h \indices{^{ a _ r } _{ c _ r } } h \indices{ 
	 ^{ d _ 1  } _{ b _  1}} \dots h \indices{ ^{ d _ s } _{ b _ s } } 
	 T \indices{ ^{ c_1 \dots c _ r } _{ d_1 \dots d _ s }} 
\]  can be identified 
with tensors on $ \Sigma $. 
Covariant derivatives on $ \Sigma $
are defined as 
\[
 D _ a T \indices{ ^{ b_1 \dots b _ r } _{ c_1 \dots c _  s } } 
  = h \indices{ _ a ^ d } h \indices{^{ b_1 } _{ c_1  }  } \dots 
  h \indices{ ^{ b _ r  } _{ e _ r }  } 
  h \indices{ ^{ f_1  } _{ c_1 } } \dots h \indices{ ^{ f _ s } 
  _{ c _ s }  } \nabla _ d T \indices{ ^{ e_1 \dots e_ r }  _{ 
  f_1 \dots f _s }}    
\] 
In actual fact, $  D_ a $ is the Levi-Civita connection 
of $ \mathcal{ L } _{ ab }  $.
The Riemann tensor of 
$ D _ a $ is $ R ' \indices{ ^ a _{ bcd  } }  $. 
We have that 
\[
 R  ' \indices{ ^ a _ { bcd  } }  = 
 h \indices{ ^  a_ e } h \indices{ ^ f _ b } h \indices{ ^ g _ c } 
 h \indices{ ^ h _ d } R \indices{ ^ e _{ fgh } }  - 2 K _{ [  c} ^ a  
 K _{ d ] } b 
\] 

It is a lemma that the Ricci scalar of $ D _ a $ 
is $ R   '  = R + 2 R _{ ab } n ^ a n ^ b  - K ^ 2 + K ^{ ab } K _{ ab } $. 
We have that $ K  = K \indices{ ^ a _ a }  = g ^{ ab } K _{ ab }  = 
h ^{ ab } K _{ ab } $. 

\begin{thm}
	\[
	 D _a K _{ bc }  - D _ b K _{ ac  } = h \indices{ _ a ^ d } 
	 h \indices{ _b ^  e } h \indices{ _ c ^ f  } n ^ g 
	 R _{ de fg  }
	\] This is Codavvi's equation. 
\end{thm}

\begin{thm}
	\[
		D _ a K \indices{ ^ a _ b  }  - D _b K 
		 = h \indices{ ^ c _ b  } R _{ dc } n ^ d  
	\] 
\end{thm}

\subsection{The Constraint Equations}
If we set $ G _{ ab }  = 8 \pi T _{ab }$, 
then 
\[
 G _{ ab } n ^ a n ^ b  = R _{ ab } n ^ a n ^ b + \frac{1}{2 } R 
 = \frac{1}{2 } \left( R ' - K ^{ ab } K _{ ab } + K ^ 2  \right) 
\] Thus, 
\[
 R '  - K ^{ ab } K _{ ab }  +   K ^ 2  = 16 \pi \rho 
\] where $ T _{ ab } n ^ a n ^ b  $ is 
energy density seen by the observer with velocity $ n ^ a $. 
This is a Hamiltonian constraint. 
\[
 8 \pi h \indices{ _ a ^ b } T _{ b c } n ^ c  = 
 h \indices{ _ a ^ b } G _{ b c } n ^ c  = h \indices{ _ a ^ b  } R _{ b c } 
 n ^ c 
\] The last proposition above 
tells us that 
\[
 D _ b K \indices{ ^ b _ a  }  - D _ a K  = 8 \pi h \indices{ _ a 
 ^ b  } T _{ b c } n ^ c   
\] The last quantity is minus the momentum density seen by the observer. 
This involved the spatial derivative of $ K $. 
\subsection*{Summary}

\subsubsection*{Spherical symmetry}
\begin{itemize}
\item A spacetime is spherically symmetric if 
the isometry group has an $ SO ( 3 ) $ subgroup. 
\item The orbit of $ p \in M $ is a $  S ^ 2 $ sphere. 
\item We define the area radius function
\[
	r ( p )  = \sqrt{ \frac{ A( p ) }{ 4 \pi }} 
\] This has the interpretation of radius. 
\end{itemize}
\subsubsection*{Properties of Spacetimes}

To help us, we can construct hyper-surfaces as follows 
\begin{itemize}
\item We can define a hypersurface $ \Sigma $ to be the surface where $ f ( x )  = 0$
\item $df $ is normal to this surface as $ df ( t ) = 0 $ for any tangent vector. 
\item Normals $ n $ to the surface can be written as 
\[
 n = g df + f n' \implies n \wedge  dn \mid_{ \Sigma }  =0 
\] 
\item Frobenius says that if $ n \wedge  dn  =0  $  everywhere, then 
there exist $ f, g $ such that $ n = g d f$ so $ n $ is 
normal to surfaces of constant $ f $. 


\end{itemize}
We can classify different spacetimes as below. 
\begin{itemize}
\item A spacetime is symmetric in a 
variable $ s $ if $ s $ is a coordinate
but the metric doesn't depend on $ s $. 
\item A spacetime is stationary if there exist coordinates $ x ^ \alpha $ 
such that $  x^ 0$ is timelike at infinity, and our metric 
doesn't depend on $ x^ 0 $ (equivalent to saying that there's a Killing vector
which is timelike at infinity). 
\item A spacetime is static if there are no cross terms in the metric like 
$ g _{ 0i } $. 
\item Construct stationary spacetimes by defining a Killing vector, 
then construct a hypersurface $ \Sigma $ nowhere tangent to 
that vector. Assign spatial coordinates $ x ^ i $ for positions in 
$ \Sigma $. Then, construct 
the coordinate $ t $ by moving a distance $ t $ in the parameter
orthogonal to the hypersurface. Then, the killing vector is 
\[
 k = \frac{\partial   }{\partial  t } 
\] 
\item Our final metric, including spherical symmetry is 
\[
	- e ^{  2 \Psi ( r ) } dt ^ 2 + e^{ 2 \Phi ( r ) } dr ^ 2 + r ^ 2 d \Omega ^ 2 
\] 
\end{itemize}

\subsection{The Schwarzchild Solution}

\subsubsection{Conserved quantities and Geodesics}
\begin{itemize}
\item The Schwarzchild solution is 
	\[
		ds ^ 2  =  - \left( 1 - 2M  /r  \right)  dt ^ 2 
		+ \left( 1 - 2M  /r   \right)  ^{ - 1 } dr ^ 2  + dr ^ 2 + 
		r^ 2 d \Omega ^ 2 
 \] 
\item \textbf{Birkhoff's Theorem.} A spacetime which is spherically symmetric, 
	and a solution of the vacuum Einstein 
	equations is isometric to the Schwarzchild solution. 
	The Schwarzchild solution is static. So spherical symmetry and 
	vacuum matter implies staticity!
\item Redshift is the difference between two proper times 
	\[
	 \frac{ \Delta \tau _ A }{ \Delta \tau _ B }  = \sqrt{ \frac{
	 1 - 2M  / r _ B  }{ 1 - 2M  / r _ A }} 
	\] 
\item Conserved quantities along a geodesic $ x ^ \mu \left( \tau  \right)  $ 
	and a Killing vector field are given by $ k \cdot  u $, where 
	$ u ^ \mu  = \frac{d x ^ \mu }{ d \tau } $. These conserved 
	quantities are $ E $ and $ h $. With null geodesics, 
	we are free to rescale $ E $ by reparametrisation. 
\item Geodesic equations from here 
	allow rotations such that $ \theta = \pi / 2 $. 
\end{itemize}

\subsubsection{Eddington-Finkelstein Coordinates}

\begin{itemize}
\item The black hole 
	is the region where a 
	signal can't be sent to infinity - for any 
	future directed causal curve, if $ r \left( \lambda _ 0  \right)  
	\leq 2M $, then $ r \left( \lambda  \right)  \leq 2M  $
	for all $ \lambda \geq \lambda _ 0 $. 
\item We define white holes 
	with the new coordinate $ u = t - r _ * $, 
	which is constant on outgoing radial, 
	null geodesics. 
\item The metric in these new 
	coordinates is 
	\[
	 ds ^ 2 =  - \left( 1 - 2M / r  \right)  du ^ 2 - 2 du dr 
	 + dr ^ 2 + r ^ 2 d \Omega ^ 2 
	\] 
\item Setting $ v =  - u $ recovers ingoing 
	Eddington-Finkelstein coordinates. The 
	physical interpretation 
	of doing this is time-reversal. 
\end{itemize}

\subsubsection{The Kruskal Extension}

\begin{itemize}
	\item We define Kruskal coordinates 
		as 
		\[
		 U  =  - e ^{  - u  / 4M }, \quad V  = e ^{ v  / 4 M }
		\]
	\item Metric looks like 
		\[
		 ds ^ 2  =  - 32 M ^ 3 \frac{e ^{  - r / 2M } }{ r } dU d V 
		 + r \left( U , V  \right)  ^ 2 d \Omega ^ 2 
		\] 
	\item A Kruskal diagram is a 
		diagram showing lines of constant $ r $ and constant 
		$ t $ on a 45 degree rotated $ \left( U , V  \right)  $ 
		axis - there are four regions here. 

\end{itemize}
\[
\frac{a}{b } \vec{v} \vec{v} \alpha \sigma \delta ` f
\frac{a}{b} 
\frac{a}{b} a_1 a_2 a_3 a_4 $ ^{ }$ 
\begin{pmatrix}  
\end{pmatrix} 
\] 

\subfile{formation.tex}
\subfile{useful_identities.tex}

\section*{Example Sheet 1}

\subsection*{Question 0}
We have two causal vectors $ X , Y $ 
at $ p $. 
We want to show that $ X \cdot  Y  \leq 0 $  
if and only if they lie in the 
same lightcone. 
This is when we set 
\[
	g _{ \mu \nu } \left( p  \right)   = 
	\eta _{ \mu \nu }  = \text{diag}\left( - 1 , 1 , \dots 1  \right) 
\] 
In the first case, where $ X $ is timelike, 
we can always do a local Lorentz transformation 
so that the time coordinate is 
\[
 X ^ \mu  = \left( X ^ 0 , 0 , \dots 0  \right)  
\] Since 
$ X , Y $ are causal, the time component $ X ^ 0 $
and  $ Y ^ 0  $ are non zero. 
This means 
that $ X \cdot  Y  = - X ^ 0 Y ^  0 \leq 0 $ . 
Thus, $ X ^ 0 Y ^  0 \leq 0 $. 
So, they necessarily lie in 
the same light-cone. 

In the case where $ X $ is null, 
we can write 
\[
 X ^ \mu  = \left(  X^ 0 , X ^ 0 , 0 , \dots 0  \right) 
\] This means that 
\[
 X \cdot  Y  =  - X ^ 0 Y ^  0 + X ^ 1 Y ^ 1  = 
 X ^ 0 \left( Y ^1  - Y ^ 0  \right)  \leq 0 
\] If 

\subsection*{Question 2}
We want to show 
Cartan's magic formula. 
\[
\mathcal{ L } _{ X} Y   = \iota_{ X } d Y + d \left(\iota _ X Y   \right) 
\] For a $ p$-form $ Y $ the Lie derivative 
is 
\[
\mathcal{ L } _{ X} Y = X ^ \alpha \partial  _ \alpha Y_{ 
\mu_1 \mu_1 \dots \mu_{ p } } + \left( \partial_{ \mu_1  } X^ \alpha \right) Y_{ 
\alpha \mu_2 \dots \mu_ p } + \dots + \left( \partial_{ \mu_{ p } } X ^ \alpha  \right)  
Y _{ \mu_1 \dots \alpha }
\]  
The basic thing to show here is 
that 
\[
d \left( \iota_ X Y  \right) _{ 
\mu_1 \dots \mu_ p }  = \sum_{ i  = 1, \dots n } \partial _{ \mu _ i } 
\left(  X ^{ \alpha } Y_{ \mu_1 \dots \alpha \dots \mu_{ p } } \right) 
\] In addition, we have that 
\[
\iota_X  d Y  = X ^{ \alpha } \partial  _ \alpha Y_{ 
\mu_1 \dots \mu _ p  }   - X^ \alpha \partial _{ \mu_1 } Y_{ \alpha \dots \mu_ p  }
- \dots  - X ^ \alpha \partial  _{ \mu_ p } Y_{ \mu _ 1 \dots \alpha }
\] 
We can use small cases 
to work out the right signs here. 

\subsection{Question 3}
The conformal Killing equation 
reads 
\[
	\nabla _ a k _ b + \nabla _ b k _ a = \phi g _{ ab } 
\] Consider 
the quantity $ X _ a V ^ a $, where 
$ V ^ a  $ is the tangent vector 
field associated with a null geodesic. 
\begin{align*}
	\frac{d }{ d \tau} \left( X _ a V ^ a  \right)  
	&=   V ^ b \nabla  _ b \left(  X_ a V ^ a  \right)   \\ 
	&=  V ^ b V ^ a \nabla _ b X_ a  \\
	&=  V ^ b V ^ a  \phi g _{ ab }  \\ 
	&=   0 
\end{align*} 
Going into the second line, 
we use the geodesic equation after the product rule. 
Going into the third line, 
we used the conformal Killing equation 
after symmetrisation. 
Finally, we use the fact that it's a null geodesic to 
take it to zero. 

Using the Leibniz property 
of the Lie derivative, 
if we have $ \Omega ^ 2 g _{ ab } $, 
\[
	\mathcal{ L } )_ k \left( \Omega ^ 2 g _{ ab }  \right)  
	= \Omega^ 2 \phi g _{ ab }  + 2 g _{ ab } \Omega k ^ \mu \partial  _ \mu 
	\Omega 
\] But this is just a scalar multiple of $ g _{ ab } $, 
so we're good.

\subsection{Question 4}
I can't 
find any other convincing argument 
that $ \left( \frac{m}{ r^ 3 }  \right) '  \leq 0 $ other 
than the fact that at large $ r $, we have that 
$\left( \frac{m}{r ^ 3 }  \right)  '  \leq 0 	$. 


Using the fact that $ \frac{m\left( r_1  \right)  }{ r_1 } \geq m\left( r  \right)  
\frac{r_1 ^ 3 }{ r ^ 3 } $, 
we bound the integral from below 
using this inequality. 
\begin{align*}
	\int_ 0 ^ r dr_1 \, r_1 \left( 1 -\frac{2m\left( r_1 \right)  }{ r_1} \right)
	^{  - \frac{1}{2 } } & \geq \int_ 0 ^  r 
	r_1 \left( 1 - \frac{2m\left( r )  \right)  r_1 ^ 2 }{ r ^ 3 }  \right)  \\ 
			     &=  \frac{r ^ 3  }{ 2 m \left( r  \right)   } 
			     \left[  1  - \left( 1 - \frac{2m}{ r }  \right)  ^{ \frac{1}{2}} \right]
\end{align*} 

\subsection{Question 5}
With constant density, we 
have that $ m  = \frac{4}{3 } \pi r ^ 3 \rho $. 
Substituting this 
in our differential equation for $ p $, gives us 
\[
	\frac{dp }{ dr }  = - \left( p + \rho  \right)  
	\left( \rho + 3p  \right)  \frac{4 \pi r ^ 3 }{ 3 } 
	\frac{1}{r \left( r - 2m  \right)  }
\] Integrating this between 
the centre and the surface of the star, 
and rearranging, we arrive at the equation 
\[
	p_ c  = \rho \frac{\left( 
	1 - \frac{2M}{R }  \right)^{ \frac{1}{ 2} }  - 1 \right) }{
1 - 3 \left( 1 - \frac{2M}{R }  \right) ^{ \frac{1}{2 } } }
\] 
This value diverges at $ M / R  = \frac{4}{9  } $, 
which means that the inequality is saturated. 

If the inequality was not 
saturated, then $ p _ c $ would not have 
diverged there, which suggests 'room to grow'. 


\subsection{Question 7}

Starting from the 
expression for a particle moving in the 
Schwarzchild potential, we have our 
equation as 
\[
	\frac{1}{2 } \dot{ r } ^ 2 + \frac{1}{2 } \left( 1 - 
	\frac{2M}{r } \right)   \frac{b ^ 2 }{ r ^ 2 }  = \frac{1}{2 } 
\] Here, we've rescaled $ E = 1 $ since 
we're dealing with null geodesics. 
At the closest point to the 
origin, where $ \dot{ r  }  = 0  $, 
we have that 
\[
 b  = \sqrt{ \frac{r ^ 3 }{ r - 2 M }} 
\] If we seek a stationary point for $ b$, 
this is at $ r _ *  = 3M $ for $ r _ *   >  2M $. 
This corresponds to 
$ b_{ \text{max } }  = 3 \sqrt{ 3 }  M $. 

$ b $ is the distance 

Hence, if you fire a photon with a 
parameter smaller than this radius, it will 
be 'captured'. So, we can 
view this as equivalent to an absorbing
disk of area $ \pi 27 M ^ 2 $. 

\subsection{Question 9}
The region II 
in a Kruskal diagram is enclosed by 
$ r = 0 $ on the top, 
and the positive axes for $ U $ and $ V $. 
In this region, $ 0 < r < 2M $. 

For any time-like geodesic, 
we have the equation of motion as
\[
 \dot{ r } ^ 2  = E ^ 2  - 1 + \frac{2M}{r }  
\] Now, we can bound $ E ^ 2 $ below as $ E ^ 2 \leq 0 $. 
Thus, we have that 
\[
	\dot{ r  }  = \sqrt{ \frac{\left( E ^ 2  - 1  \right)  r + 2 M }{ r } }  
	\geq \sqrt{\frac{ - r + 2 M }{ r } } 
\] Hence, 
integrating over this inequality, we have 
\[
 0 \leq \int_ 0 ^ r dr \, \sqrt{ 
 \frac{r }{ 2 M  - r } }   = \int_ 0 ^{ \tau_{ \text{total }  } } 
 \leq \int _ 0 ^{ 2 M } ^2 \, \sqrt{\frac{r }{ 2M  - r   }}   = M \pi 
\] Thus, we 
get a bound on the total proper time. 
The right-hand expression 
can actually be evaluated 
using the substitution 
$ r  = 2 M \sin ^ 2 \theta $. 

A curve which yields $ E ^ 2  =0 $
has $ \frac{dr }{ d\tau  }\mid_{ \tau  =0 }  = 0  $ 
and $ r_0  = 2 M $. 

\subsection{Question 10}
Setting $ R  = R \left( \rho  \right)  $ and $ z  = z \left( \rho  \right)  $, 
changing variables in our 
metric gives 
\[
 ds ^ 2  = d \rho ^ 2 \left( 
 \left( \frac{dR }{ d \rho }  \right) ^ 2 + \left( \frac{dz }{ d \rho }  \right)  
 ^ 2  \right) + R \left( \rho  \right)  ^ 2 d \Omega ^ 2  
\] Comparing coefficients, 
we have that 
\[
	\frac{R ^ 2 }{ \left(  R '  \right)  ^ 2 + \left( 
	z ' \right)  ^ 2 }  = \rho ^ 2 , \quad \left( R '  \right)  ^2 
	+ \left( z '  \right)  ^ 2  = \left( 1 +  \frac{M}{2 \rho}  \right)  ^4 
\] This implies that 
\[
 R  = \rho \left(  1 + \frac{M}{2 \rho }   \right)  ^2, 
 \quad ( R ' )^ 2  = 1 - \frac{M^ 2 }{ 2 \rho ^ 2 } + \frac{M ^ 4 }{16 \rho ^ 4  }
\]  Substituting this into the above, 
we get that 
\[
	\left( z '  \right) ^2  = \frac{M }{ 2 \rho } \left( 2 
	+ \frac{M}{\rho } \right)  ^2 , \quad 
	\implies  z ' = \sqrt{ \frac{M }{ 2 \rho } }  (2 + \frac{M}{\rho }  )
\] 
We just integrate this expression to find $ z $ as a function 
of $ \rho $. 
\end{document} 
