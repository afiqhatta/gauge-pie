\documentclass[11pt, oneside]{article}   	% use "amsart" instead of "article" for AMSLaTeX format
\usepackage[margin = 1.1in]{geometry}            		% See geometry.pdf to learn the layout options. There are lots.
\geometry{letterpaper}                   		% ... or a4paper or a5paper or ... 
\usepackage[parfill]{parskip}    		% Activate to begin paragraphs with an empty line rather than an indent
\usepackage{graphicx}				% Use pdf, png, jpg, or eps§ with pdflatex; use eps in DVI mode
								% TeX will automatically convert eps --> pdf in pdflatex	
\usepackage{adjustbox}	
\usepackage[section]{placeins}


%% LaTeX Preamble - Common packages
\usepackage[utf8]{inputenc}
\usepackage[english]{babel}
\usepackage{textcomp} % provide lots of new symbols
\usepackage{graphicx}  % Add graphics capabilities
\usepackage{flafter}  % Don't place floats before their definition
\usepackage{amsmath,amssymb}  % Better maths support & more symbols
\usepackage[backend=biber]{biblatex}
\usepackage{amsthm}
\usepackage{bm}  % Define \bm{} to use bold math fontsx
\usepackage[pdftex,bookmarks,colorlinks,breaklinks]{hyperref}  % PDF hyperlinks, with coloured links
\usepackage{memhfixc}  % remove conflict between the memoir class & hyperref
\usepackage{mathtools}
\usepackage[T1]{fontenc}
\usepackage[scaled]{beramono}
\usepackage{listings}
\usepackage{physics}
\usepackage{tensor}
\usepackage{simplewick} 
\usepackage{tikz} 
\usepackage{import}
\usepackage{xifthen}
\usepackage{pdfpages}
\usepackage{transparent}
\usepackage{pgfplots}
\usepackage[compat=1.1.0]{tikz-feynman}
\usepackage{subfiles}
\usepackage{simpler-wick}
\usepackage{slashed}
\usepackage{fancyhdr}

\pagestyle{fancy}
\fancyhf{}
\rhead{Notes by Afiq Hatta}
\lhead{Black Holes}
\rfoot{Page \thepage}

%% Commands for typesetting theorems, claims and other things.
\newtheoremstyle{slanted}
{1em}%   Space above
{.8em}%   Space below
{}%  Body font
{}%          Indent amount (empty = no indent, \parindent = para indent)
{\bfseries}% Thm head font
{.}%         Punctuation after thm head
{0.5em}%     Space after thm head: " " = normal interword space;
{}%         \newline = linebreak
{}%          Thm head spec (can be left empty, meaning `normal')

%% Commands for typesetting theorems, claims and other things. 

\theoremstyle{slanted}
\newtheorem{theorem}{Theorem}
\newtheorem*{thm}{Theorem}
\newtheorem*{claim}{Claim}
\newtheorem{example}{Example}
\newtheorem*{defn}{Definition}

\newcommand{\Lagr}{\mathcal{L}} 
\newcommand{\vc}[1]{\mathbf{#1}}
\newcommand{\pdrv}[2]{\frac{\partial{#1}}{\partial{#2}}}
\newcommand{\thrint}[1]{\int d^3 \vc{x} \left( {#1} \right)}

%% QFT specific macros 
\newcommand{\intp}{ \int \frac{ d^3 p }{ (2 \pi)^3 } \, }
\newcommand{\ann}[1]{a_{ \mathbf{ #1 }}}
\newcommand{\crea}[1]{a^\dagger_{ \mathbf{ #1 }}}
\newcommand{\ve}[1]{ \mathbf{ #1 } } 
\newcommand{\mode}[ 1]{ e^{ i \mathbf{ #1 } \cdot \mathbf{x} }}
\newcommand{\nmode}[1]{ e^{  - i \mathbf{ #1 } \cdot \mathbf{x} }}
\newcommand{\freq}[1]{\omega_\mathbf{ #1} } 
\newcommand{\scal}[1]{\phi ( \mathbf{ #1 })} 
\newcommand{\mom}[1]{ \pi (\mathbf{ #1 })} 
\newcommand{\arr}{\rightarrow} 

\newcommand{\incfig}[1]{%
\def\svgwidth{\columnwidth}
\resizebox{0.75\textwidth}{!}{\input{./figures/#1.pdf_tex}}
}

\newcommand{\anop}[2]{ #1_\mathbf{#2}}
\newcommand{\crop}[2]{#1_\mathbf{#2}^\dagger}

\usepackage{helvet} 

%tikz decoration commands 
\usetikzlibrary{decorations.pathmorphing}


\title{Notes on Black Holes}
\author{Afiq Hatta } 
\begin{document} 
\maketitle
\tableofcontents

\section*{Conventions and Housekeeping}
We will set $ G  = c = 1 $, 
and ignore the cosmological constant $ \Lambda $. 
For indices, greek letters $ \mu , \nu $ refer to 
a specific basis. 
$ a, b$ are abstract indices which refer to any basis, 
from Roger Penrose. 
For example, we have 
\[
 \Gamma^ \mu _{ \nu \rho }  = \frac{1}{2 }g ^{ \mu \sigma } 
 \left( g _{ \sigma \nu , \rho } + g _{ \sigma \rho , \nu }  - g _{ \nu \rho , \sigma }  \right), 
 \quad R  = g ^{ ab } R _{ ab } 
\] Christoffel components are 
basis dependent so they are written with greek indices. 


As far as Black holes are concerned, 
the cosmological constant is incredibly small. 
Hence, we will ignore it from now. 
Check out Wald's book on General Relativity. It's great!

\section{Spherical Stars}
\subsection{Cold Stars}
If you think about a star like our sun, 
it's a ball of hot gas with nuclear reactions at its centre. 
Gravitational force makes the star contract, 
but nuclear reactions at the centre exert outward pressure 
to resist the contraction. 
If we wait long enough, th e star will exhaust the fuel 
it has and hence the star will contract. 
If we're interested in the final state of this star, 
we need to look at some source of pressure which is non-thermal in nature 
(so valid when the star is cold) to counter act gravity. 

There is another natural source of pressure to resist gravity. 
This is the Pauli principle - where a gas of Fermions 
resists compression. This is called degeneracy pressure. 
This is a purely quantum mechanical phenomenon. 

For example, a white dwarf is a star in which gravity is balanced 
by electron degeneracy pressure. 
A white dwarf is a very dense kind of star. If we 
had a white dwarf with the same mass of our sum $ M  = M _{ \sun } $. 
the radius would be $ R \sim \frac{R_{\sun } }{100 } $. 
A white dwarf is how the sun will end its life. 
 

Can all stars end their life this way? 
No. This is because there's a maximum mass for a white dwarf. 
This is called the Chandrasekhar limit, where 
\[
 M _{ WD } \leq 1.4 M _{ \sun }
\]
What happens when we have a star more massive than this? 
When the star continues to get more and more dense, 
then inverse beta decay occurs where protons turn into neutrons. 
Neutrons are fermions and they also have 
degeneracy pressure. 
Thus, there's a second class of 
stars called Neutron stars where 
gravity is balanced by neutron degeneracy pressure. 
These stars are tiny. 

If we took a neutron star with $  M \sim M _{ \sun } $, $ R \sim 10 km $. 
Compare this to  $ R _{ \sun } \simeq 7 \times 10 ^ 5 km $. 
They are very dense! The gravitational field is very strong. 
If we have a Newtonian gravitational potential at the surface 
\[
 | \phi | \sim 0.1 
\]  General relativity because important when $ | \phi | $ is order 1. 
Hence, GR is important here.
We can derive a maximum possible mass for neutrons stars as well. 
We'll derive this bound which is independent of our knowledge of dense matter. 

We need to make some 
simplifying assumptions. 
We assume spherical symmetry. 

\subsection{Spherical Symmetry}
Recall the unit round 
metric on a two dimensional sphere $ S ^ 2 $, 
\[
 d \Omega ^ 2  = d \theta ^2 + \sin ^ 2 \theta d \phi ^ 2 
\] The isometries are diffeomorphisms 
which preseve the metric. 
They form a group. 
The isometry group of a two sphere with that metric is $ SO ( 3 ) $. 
Essentially, we define a space time as spherically symmetric 
if it has this as its isometry group. 
\begin{defn}{Spherically Symmetric Spacetime}
	A spacetime is spherically symmetric if its isometry group 
	contains an $ SO ( 3 ) $ subgroup, whose orbits are
	$ 2  $-spheres. 
	In other words, if we pick a point and act on it with all 
	$ SO ( 3 ) $ elements, it will fill out a sphere. 
\end{defn}
\begin{defn}{Area Radius function}
	In a spherically symmetric spacetime $ ( M , g )$, 
	the area radius function is 
	\[
		r : M \to \mathbb{ R } , r ( p ) = \sqrt{ \frac{ A ( p ) }{ 4 \pi } } 
	\] where $ A ( p ) $ is the area of the $ S^ 2 $ orbit 
	through the point $ p $. 
	So we take a point $ p $, since its spherically symmetric, we 
	can make a sphere out of it, and we take the area. 
	This is linked to our understanding of the $ r $ coordinate, 
	but this definition doesn't require a preferred origin. 
	In other words, 
	the $ S ^ 2 $ has induced metric $ r ( p ) ^ 2 d \Omega^ 2 $. 
\end{defn}
\subsection{Time independence}
\begin{defn}{Stationarity}
	$ \left(  M , g  \right)  $ is stationary if there exists a Killing vector field
	$ k ^ a $
	(KVF) which is everywhere timelike. 
	We're saying that $ g _{ ab } k ^ a k ^ b < 0 $. We can introduce adapted coordinates. 
	Let's pick some hypersurface $ \Sigma $ which is transverse to a vector field 
	$ k $. We can then pick coordinates  $ x ^ i $ on $ \Sigma $, where $ i =  1, 2, 3 $. 
	This gives us coordinates on our surface. 
	We assign coordinates $ \left(  t, x ^ i  \right)  $ to point parameter 
	distance $ t $ along the integral curve of  $ k ^ a $ through point on 
	$ \Sigma $ with coords $ x ^ i $. This implies 
	\[
	 k  = \frac{\partial }{\partial t} 
	\] This implies the metric is independent of $ t $, since $ k ^ a $ is a 
	Killing vector field. Therefore, the metric looks like 
	\[
		ds ^ 2  = g _{ 00 } \left( x ^ k  \right) dt ^ 2  + 2 
		g _{ 0i } dt d x ^ i + g _{ ij  }\left( x ^ k  \right)  dx ^ i dx ^ j 
	\] This means that $ g _{ 00 } < 0 $. Thus, 
	given a stationary spacetime, we can construct coordinates which 
	the metric is time independent. 
	Conversely, any metric of this form is stationary. 
\end{defn}
There's a more refined version of time independence which 
we can use. 
Imagine we have a hypersurface $ \Sigma $ where $ f = 0 $, $ f : M \to \mathbb{ R } $
which is smooth, and $ df \neq 0  $ on $ \Sigma $. 
Then, $ df $ is normal to $ \Sigma $. 
If we let $ t ^ a $ tangent to $ \Sigma $, then we have that 
\[
 df ( t )  = t ( f )  = t ^ \mu \partial  _ \mu f = 0 \text{ since } f  \text{ is constant}
\] A normal to a surface is 
not unique. 
Normals are not unique. What's the most general 
form of a covector field which is normal to $ \Sigma $. 
If $ n _ a$  also normal to $ \Sigma $, them 
\[
 n = g df + f n ' 
\] where $ g $ is smooth and not equal to $ 0 $ on $ \Sigma $,
and  $ n ' $ is a smooth 1-form. 
Let's look at the exterior 
derivative of this vector field. Using the rules 
for the exterior derivative, 
\[
 dn = dg \wedge  df + df \wedge  n ' + f dn ' 
\] Let's evaluate this 
on $ \Sigma $. We have that 
\[
	\left. dn \right\vert_{ \Sigma }  = \left(  dg - n '  \right)  \wedge  df
\] Hence, $ n \wedge  dn - 0 $   on $ \Sigma $. This is because 
$ n \propto df $ on $ \Sigma $. The wedge product vanishes. 
\begin{thm}{Frobenius}
	If $ n \neq 0 $ is a one form such that $ n \wedge  dn  = 0 $  everywhere, 
	then there exist functions $g, f $ such that 
	$ n   = g df $. So $ n $ is normal 
	to surfaces of constant $ f $, so 
	$ n $ is 'hypersurface orthogonal'. 
	$ n$ is orthogonal to all surfaces 
	of constant $ f $. 
\end{thm}

\begin{defn}{Static spacetimes}
	A spacetime  $ \left( M , g  \right)  $ is static 
	if there exists a hypersurface orthogonal, timelike $ KVF$. 
	In particular, static implies stationary. 
	Why is this useful? Returning to the adapted coordinates, 
	how does hypersurfaece orgonality help? 
	We choose $ \Sigma $ orthogonal to $ k ^ a $ when defining 
	adapted coordinates $ \left(  t, x ^ i  \right)   $. 
	But, $ \Sigma $ is $  t = 0$, therefore the normal 
	to $ \Sigma $ is $ dt $. 
	So
	\[
		k _ \mu \mid _{ t = 0 } \propto \left( 1, 0 , 0 , 0  \right) 
	\] In particular, we have that \[
	 k _i \mid _{ t = 0 }  =0 
 \] but, $ k _ i  = g _{ 0i } \left( x ^ k  \right)  $. 
 Hence, $ g _{ 0i }  = 0$.
 Thus, if we write down the metric, 
 we have that 
 \[
	 ds ^ 2  = g _{ 00 } \left( x ^ k  \right)  dt ^ 2 + 
	 g _{ ij} \left( x ^ k  \right)  dx ^ i dx ^ j , \quad g _{ 00 } < 0 
 \] Thus we have a discrete time reversal isometry $ \left( t , x ^ i  \right)  \to 
 \left(  - t , x ^ i  \right)  $. 
 Thus,  
 \[
  \text{static} \iff \text{time independent and invariant under time reversal}
 \] For example, for a rotating star, the metric 
 may be time independent but not static since they're is no 
 time reversal symmetry since it changes the sense of rotation. 
\end{defn}

\subsection{Static, Spherically Symmetric Spacetimes}
The isometry group is $ \mathbb{ R } \times SO ( 3 )$. 
We can show that this implies the metric must be static. 
If it wasn't static, it would be rotating, but that breaks spherical symmetry. 
On $ \Sigma $ choose coordinates $ x^ i  = \left(  r, \theta , \phi   \right)  $, 
where $ r $ is our area radius function. 
If we do this, then the metric 
is 
\[
	ds ^ 2 \mid_{ \Sigma }  = e ^{ 2 \psi ( r )  } dr ^ 2 + r ^2 d \Omega ^ 2 
\]  where due to spherical symmetry, everything depends on $ r $. 
If we had $ dr d \phi $ or $ dr d \theta $ terms, 
this would break spherical symmetry. 
If we then go on to write down the full spacetime 
metric, we 
get that 
\[
	ds ^ 2   =  - e ^{ 2 \Phi ( r ) } ds ^ 2 + e ^{ 2 \Psi ( r ) } dr ^ 2 + r ^ 2 d \Omega ^ 2 
\] At the moment, there's no origin. 
\subsection*{Summary}

\subsubsection*{Spherical symmetry}
\begin{itemize}
	\item A spacetime is spherically symmetric if 
		the isometry group has an $ SO ( 3 ) $ subgroup. 
	\item The orbit of $ p \in M $ is a $  S ^ 2 $ sphere. 
	\item We define the area radius function
		\[
			r ( p )  = \sqrt{ \frac{ A( p ) }{ 4 \pi }} 
		\] This has the interpretation of radius. 
\end{itemize}
\subsubsection*{Properties of Spacetimes}

To help us, we can construct hyper-surfaces as follows 
\begin{itemize}
	\item We can define a hypersurface $ \Sigma $ to be the surface where $ f ( x )  = 0$
	\item $df $ is normal to this surface as $ df ( t ) = 0 $ for any tangent vector. 
	\item Normals $ n $ to the surface can be written as 
		\[
		 n = g df + f n' \implies n \wedge  dn \mid_{ \Sigma }  =0 
		\] 
	\item Frobenius says that if $ n \wedge  dn  =0  $  everywhere, then 
		there exist $ f, g $ such that $ n = g d f$ so $ n $ is 
		normal to surfaces of constant $ f $. 


\end{itemize}
We can classify different spacetimes as below. 
\begin{itemize}
	\item A spacetime is symmetric in a 
		variable $ s $ if $ s $ is a coordinate
		but the metric doesn't depend on $ s $. 
	\item A spacetime is stationary if there exist coordinates $ x ^ \alpha $ 
		such that $  x^ 0$ is timelike at infinity, and our metric 
		doesn't depend on $ x^ 0 $ (equivalent to saying that there's a Killing vector
		which is timelike at infinity). 
	\item A spacetime is static if there are no cross terms in the metric like 
		$ g _{ 0i } $. 
	\item Construct stationary spacetimes by defining a Killing vector, 
		then construct a hypersurface $ \Sigma $ nowhere tangent to 
		that vector. Assign spatial coordinates $ x ^ i $ for positions in 
		$ \Sigma $. Then, construct 
		the coordinate $ t $ by moving a distance $ t $ in the parameter
		orthogonal to the hypersurface. Then, the killing vector is 
		\[
		 k = \frac{\partial   }{\partial  t } 
		\] 
	\item Our final metric, including spherical symmetry is 
		\[
			- e ^{  2 \Psi ( r ) } dt ^ 2 + e^{ 2 \Phi ( r ) } dr ^ 2 + r ^ 2 d \Omega ^ 2 
		\] 
\end{itemize}

\subfile{formation.tex}
\subfile{useful_identities.tex}

\section*{Example Sheet 1}

\subsection*{Question 2}
We want to show 
Cartan's magic formula. 
\[
	\mathcal{ L } _{ X} Y   = \iota_{ X } d Y + d \left(\iota _ X Y   \right) 
\] For a $ p$-form $ Y $ the Lie derivative 
is 
\[
 \mathcal{ L } _{ X} Y = X ^ \alpha \partial  _ \alpha Y_{ 
 \mu_1 \mu_1 \dots \mu_{ p } } + \left( \partial_{ \mu_1  } X^ \alpha \right) Y_{ 
 \alpha \mu_2 \dots \mu_ p } + \dots + \left( \partial_{ \mu_{ p } } X ^ \alpha  \right)  
 Y _{ \mu_1 \dots \alpha }
\]  
The basic thing to show here is 
that 
\[
 d \left( \iota_ X Y  \right) _{ 
 \mu_1 \dots \mu_ p }  = \sum_{ i  = 1, \dots n } \partial _{ \mu _ i } 
 \left(  X ^{ \alpha } Y_{ \mu_1 \dots \alpha \dots \mu_{ p } } \right) 
\] In addition, we have that 
\[
 \iota_X  d Y  = X ^{ \alpha } \partial  _ \alpha Y_{ 
 \mu_1 \dots \mu _ p  }   - X^ \alpha \partial _{ \mu_1 } Y_{ \alpha \dots \mu_ p  }
  - \dots  - X ^ \alpha \partial  _{ \mu_ p } Y_{ \mu _ 1 \dots \alpha }
\] 
We can use small cases 
to work out the right signs here. 

\end{document} 
