\subsection{The Weyl equations} 
Let's go back to our Dirac equation, but this time let's view it in the context of our spinor which we split up. We have that \[ 
\left( i \gamma^0 \partial_0 - \gamma^i \partial_i - m  \right) \psi  = 0 \] 

Now, we can write this in the form of a matrix equation, bearing in mind that $m$ acts as a scalar multiple of the identity. Thus, we have the matrix equation \[ 
\begin{pmatrix} 
-m & i (\partial_0 + \sigma \cdot \nabla ) \\
i(\partial_0 - \sigma \cdot \nabla) & -m 
\end{pmatrix} 
\begin{pmatrix} 
\psi_L \\ \psi_R \end{pmatrix} = 0 
\]
In this case, we've mixed the components $\psi_R$ and $\psi_L $. However, we can separate out the mixing by exploring the massless case with $m = 0$, which reduces us to the equations 
\begin{align*}  
( \partial_0 + \sigma \cdot \nabla ) \psi_R & = 0 \\ 
( \partial_0 - \sigma \cdot \nabla ) \psi_L & = 0 
\end{align*} 
We can simplify this notation even further by 'extending' out $\sigma$, and defining the objects 
\[ \sigma^\mu = (1, \sigma ), \quad \bar{ \sigma}^\mu = ( 1,  - \sigma ) \] 
Which gives us a condensed form of the equations to read \begin{align*}  
\sigma^\mu \partial_\mu \psi_R & = 0 \\
\bar{\sigma}^\mu \partial_\mu \psi_L & = 0 
\end{align*} 

