\subsection{Cross sections and decay rates} 

\subsubsection{Kinematics} 
Let's use handy variables called Mandelstam variables. 

Consider $ 2 \to 2 $ scattering. 
Due to conservation of momentum, we have that 
\[
 p_1 ^ \mu + p_2 ^ \mu = q_ 1 ^ \mu + q_2 ^ \mu 
\] if we define the Mandelstam variables 
\begin{align*}
	s  & := (p_1 + p_2 ) ^ 2 \\
	t & : = ( p_1 - q_1 ) ^ 2 \\
	u & : = ( p_1 - q_2 ) ^ 2 
\end{align*} 
Exercise: show that 
\[
	s + t + u = m_1 ^ 2 + m_2 ^ 2 + ( m_1' )  ^ 2 + ( m_2' )  ^ 2
\] (Insert diagram of scattering here) 
\begin{align*}
	i \mathcal{ M } &=  ( - i g ) ^ 2 \left\{  \frac{1}{ 
	( p_1 - q_1 ) ^2 - m ^ 2 } + \frac{1}{ ( p_1 - q_2) ^ 2 - m^ 2 } \right\}  \\
			&=  ( - i g ) ^ 2 \left\{  \frac{1}{t - m ^ 2 } + \frac{1}{ u - m ^ 2 } \right\}  \\
\end{align*}

\subsubsection{Decay rates} 
Our probability of scattering 
is expressed in terms of 
\[
	\bra{ \phi } S- 1 \ket{ i } = i \mathcal{ M } ( 2 \pi ) ^ 4 
	\delta ^ 4 ( \sum_ i p_ i - \sum_i q_i ) 
\] Our initial and final states 
are in terms of definite momenta. 
Our probability is given by 
\[
	P = \frac{ \| \bra { f} ( S -  1 ) \ket { i } \| ^ 2 }{
	\bra { f} \ket { f} \bra{ i } \ket{ i } }
\] The language of probability in this space is 
given by 
\[
	d \sigma = \frac{ ( 2 \pi ) ^ 4 \delta ^ 4 ( p_1 + p_2 - \sum_ i q _ i ) | \mathcal{M } | ^ 2 }{
	\mathcal{ F }}
\] Here, $ \mathcal{ F } $ is our flux factor with 
\[
	\mathcal{ F } = 4 \sqrt{ ( p_1 \cdot  p_2 ) ^ 2 - m_1^ 2 m_2 ^ 2} 
\] To find our total cross section from $ i \to f $, 
we integrate over our final state momentum in the usual 
Lorentz invariant way. 
 \[
 \sigma = \int \frac{1}{ \mathcal{ F } } d p_{ f } | \mathcal{ M } | ^ 2 
\] Our measure is given by 
\[
	\int d p _{ f } = ( 2 \pi ) ^ 4 \int \left( 
	\prod_{ r = 1 } ^ n \frac{ d^ 3 q }{ ( 2 \pi ) ^ 3 2 E_{ r } }\right) 
	\delta ^ 4 ( \sum_ i p _ i - \sum_ i q_i ) 
\]
\subsubsection{$ 2 \to 2 $ scattering} 
We write our mandelstam variable 
\[
	t =  ( p_1 - q_1) ^ 2 = m_1 ^ 2 + m_2 ^ 2 - 2 E_{ p _ 1 } E _{ q _ 1 } + 2 
	\vec{p} _ 1 \cdot  \vec{q} _ 1 
\] We have that 
\[
\frac{ d t }{ d \cos \theta } = 2 | \vec{p} _ 1 | | \vec{q} _ 1 | 
\] where  $ \cos \theta $ is the angle between $ \vec{q}  _ 1 $ and $ \vec{p} _ 1 $. 
It's convinient for us to work in the center of mass frame, 
the frame where our total momenum is $ 0 $. 
Our second Mandelstam variable is 
$ s= ( p_1 + p_2 ) ^ 2 $, which we can work out to be the centre 
of mass energy. This quantity is a constant of the scattering. 
We write 
\[
	\frac{ d ^ 3 q_2 }{ 2 E _{ q_ 2 } } = d ^ 4 q _ 2 \delta ( q_2 ^ 2 - ( m_2 ') ^ 2 ) \theta ( q_2 ^ 0 ) 
\] where we've defined $ \theta $ as the Heaviside step function. 
For the other variable, we write
\[
	\frac{ d ^ 3 q_1 }{ 2 E _{ q_1 } } = \frac{ | \vec{q} _ 1 | ^ 2 d | \vec{q} _1 | d ( \cos \theta ) 
	d \phi }{ 2 E_{ q_{1 } }} = \frac{1}{ 4 | \vec{p} _ 1 | } d E_{ q_ 1 } d \phi dt 
\] Now, putting this together, 
we have an expression 
\[
 \frac{ d \sigma }{ dt } = \frac{1}{8 \phi \mathcal{ F } | \vec{p}_ 1  | } 
 \int d E_{ q_ 1 } | \mathcal{ M } | ^ 2 \delta ( s - m_2 ^ 2  -  ( m_1')  ^ 2 - 2 \vec{q} _ 1 
 \cdot  ( \vec{p} _ 1 + \vec{p} _ 2 ) ) 
\] Now, if we boost back to the centre of mass fram, we have that 
\begin{align*}
	p_1 ^\mu & = ( \sqrt{| \vec{p} _ 1 | ^  2 - m_1 ^ 2 }, \vec{p} _ 1 )  \\
	p_2 ^ \mu &=  ( \sqrt{ | \vec{p} _ 1 | ^ 2 + m_2 ^ 2 }, - \vec{p} _ 1 )   \\
\end{align*}
This means we compute th $ s $ mandelstam variable 
\[
	s = ( \sqrt{ | \vec{p} _1 | ^ 2 + m_1 ^ 2  } + \sqrt{ | \vec{p} _ 2 |^ 2 + m_2 ^ 2  } ) ^ 2  
\] This implies that we can rewrite 
\[
	| \vec{p} _ 1 | = \lambda ^{ \frac{1}{2 } } ( S, m_1 ^ 2 , m_2 ^ 2 ) / 2 \sqrt{ s }  
\] where we have defined 
\[
 \lambda ( x, y , z ) = x^ 2 + y ^ 2 + z ^ 2 - 2xy - 2xz - 2yz
\] From this, we have that 
\[
	\mathcal{ F } = 2 \sqrt{ \lambda ( s , m_1 ^ 2 , m_2 ^ 2 ) } \implies 
	\frac{ d \sigma }{ d t } = \frac{ | \mathcal{ M } | ^ 2 }{ 16 \pi \lambda ( s , m_1 ^ 2 , m_2 ^ 2 ) }
\] 
\subsubsection{Decay rates} 
We defime a partial decay rate for $ \ket{ i } \to \ket{ f } $, 
or partial width, as 
\[
\Gamma _f = \frac{1}{ 2 E_ p } \int d p _ f | \mathcal{ M } | ^ 2  
\] Note that this is not a Lorentz invariant 
quantity due to time dilation. 
Hence, our convention is to quote this in terms 
of the res frame of the initial particle, 
where the energy $E _{ p _ i } $ is equal to 
the mass $ m _ i $. 
We define the total decay with whcih we write as  $ \Gamma $. 
This is the sum of all possible states 
 \[
\Gamma = \sum _ f  \Gamma _ f  
\]
The branching ratio for $\ket{ i } \to \ket { f} $ is denoted as 
\[
 BR ( i \to f )  = \frac{\Gamma_ f }{ \Gamma}
\] If we look at the units, we notice that the average 
lifetime is 
\[
	\tau = 6.6 \times 10 ^{ - 25 } \times ( 1 \text{GeV } ) / \Gamma \text{ seconds }
\]

\pagebreak
