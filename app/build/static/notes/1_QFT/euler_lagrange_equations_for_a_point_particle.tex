\subsection {Euler Lagrange equations for a point particle}  
We'll start off by reviewing some ideas in classical mechanics to motivate our approach to quantum field theory. In particular, the simple harmonic oscillator is a good place to start. In classical mechanics, we describe the motion of a particle $ \mathbf{x}  $ as a function of time, so $\mathbf{x} = \mathbf{x}( t ) $. In high school, we learnt that Newton's laws dictate that in one dimension, mass times acceleration is equal to the force on the particle. So, in the case of the simple harmonic oscillator, \[ m \ddot{x} = - k x \], where $k$ is our spring constant. We then learnt as undergraduates that this is a more specific case of the motion of a particle in a potential $V$, where in the case of a simple harmonic osccilator, $V (x) = \frac{1}{2} k x^2 $. In this formalism, our equations of motion are governed by the equation \[ m \ddot{x} = - \nabla V \] You can check that this recovers the above equation of motion.      

Generalising one step further, we see that we can encode all of this infomation into a single useful quantity, the Lagrangian, which is the quantity $L = T - V$, where $T$ is a kinetic energy term. The Lagrangian should always be a scalar function, and in the case of our harmonic oscilator in 3 dimensions it is \[ 
L = 	\frac{1 } {2} m \dot{ \mathbf{x}}^ 2 - \frac{1}{ 2} k \mathbf{x}^2 \]   
To derive the equations of motion, we want to apply the principle of least action. This principle is that the particle will move along a path which minimises the integral of the Lagrangian over time. In other words, to find out the path of the particle between to points in time, say $t_1$ and $t_2$, we would like to minimise the action.
\[ S = \int_{t_1}^ {t_2} L( \mathbf{ x} , \dot{\mathbf{x} } )   dt \] 
A condition that we impose here is that the endpoints should be fixed, so $\mathbf{x}( t_1 ) = \mathbf{ x} _1$, and $ \mathbf{x}( t_2 ) = \mathbf{x}_2 $.  
To minimise the action, we vary the curve a tiny bit by sending $\mathbf{x} \rightarrow \mathbf{x} + \delta \mathbf{ x} $. This also induces a variation in $\dot{\mathbf{x} } $. We use the chain rule with respect to the variables $\mathbf{x} $ and $\dot{\mathbf{x}} $ and to give \[ 
\delta S = \int_{ t_1}^{t_2} \delta \mathbf{x} \cdot \frac{ \partial L}{ \partial \mathbf{x} } + \delta \dot{\mathbf{x}}\cdot  \frac{ \partial L }{ \partial \dot{\mathbf{x}}} \] 
But, we can perform integration by parts on the second term, by integrating the $\delta \dot{\mathbf{x}} $ and differentiating in the integrand. This gives us a surface term which we can push to zero.
\[ \delta S = \big [ \delta \mathbf{ x} \cdot \frac{ \partial L } { \partial \dot{\mathbf{x}}}   \big ]^{t_2}_{t_1 }  + \int_{t_1}^{t_2} dt \, \, \delta \mathbf{x} \cdot \left( \frac{ \partial L }{ \partial \mathbf{x}} - \frac{d}{dt} \left( \frac{ \partial L }{\partial \dot{\mathbf{x}}} \right) \right)   \] 
Now, since our variation was arbitrary, we have that the integrand needs to be zero, so we have the Euler-Lagrange equation \[ \frac{ d} {dt } \left( \frac{ \partial L } { \partial \dot{ \mathbf{x}} } \right) = \frac{ \partial L }{ \partial \mathbf{x} } \] 

