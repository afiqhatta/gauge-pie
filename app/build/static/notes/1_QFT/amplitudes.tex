\subsection{Amplitudes} 
In the above, we will always have a factor 
to impose momentum conservation 
between final and initial states. Hence, 
this fact helps us simply things a bit. 
We will define the matrix element $ \mathcal{ M } $ by 
\[
\bra { f} ( S - 1 ) \ket{ i }  = i \mathcal{ M } ( 2 \pi ) ^ 4 \delta ( \sum_{ \text{ final state particles }}  p_i  - \sum_{ \text{ initial state particles }} p_ i ) 
\] Our factor of $ i  $ in our expression 
is done to match conventions with non-relativistic quantum mechanics. 
Our delta function expression follows from translation invariance, and 
is common to all S-matrix elements we compute. 

We now can define a new set of Feynman rules to compute $ i \mathcal{ M  }$, 
which makes things slightly easier. 
\begin{enumerate}
\item We draw all possible diagrams with appropriate external legs, 
and impose 4 momentum conservation at each vertex with appropriate use of delta functions. 
\item We assign a factor of $ ( - ig) $ at each vertex to take into account 
the order of our expansion. 
\item We integrate over closed loops in the diagram, since they give rise
to dummy variables which we can integrate over. So, for closed loops, we 
do the integral $ \int \frac{d^ 4 k }{ ( 2 \pi ) ^ 4 }$. 
\end{enumerate}

We'll now do an example with meson scattering. 

For example, for meson-meson scattering $ \phi \phi \to \phi \phi $, 
we need to draw diagrams. Our lowest order diagram is 
a bit tricky and our requirement that our vertex has to be a 
meson, nucleon and anti-nucleon trio 
means that there'll be a loop. This is shown in figure \ref{fig:mesonLoop}. 

\begin{figure}[htpb]
\centering
\includegraphics[width=0.5\linewidth]{figures/feyn6.png}
\caption{Our lowest order scattering term for $ \phi \phi \to \phi \phi $ decay}%
\label{fig:mesonLoop}
\end{figure}
In the loop, we build up our momentum labels by picking a side as $ k$, 
then just working around whilst using conservation 
of momentum.
This is
\begin{align*} 
\mathcal{ M }  &= \int \frac{ d^ 4 k  }{ ( 2 \pi ) ^ 4  }( - i g ) ^ 4 \frac{i ^ 4 }{ ( k ^ 2 - \mu ^ 2 + i \epsilon ) ( ( k + p_1 ) ^ 2  - \mu ^ 2 + i \epsilon )} \\
	       & \times \frac{1}{( ( k + p_1  - p_1' ) ^ 2 - \mu ^  2+ i \epsilon ) ( k + p_1 ) ^ 2 - \mu ^ 2 + i \epsilon  }
\end{align*} The integrand goes as $ \frac{1}{ k ^ 8 } $, so 
we're sure that this thing converges. 

\pagebreak
