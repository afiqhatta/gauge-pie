\subsection{ Computing the two point correlation function for the ground state of our perturbed Hamiltonian } 
Let's call $\ket {\omega } $ our ground state for our Hamiltonian. A natural question to ask would be what $\bra {\omega } \phi(x) \phi( y ) \ket{\omega} $ is in terms of the spectrum we already know for the free Hamiltonian $H_0$.

One way to do this would be to take the ground state of the free Hamiltonian which we call $\ket{ 0 } $ and expand it in terms of the energy eignestates of the full Hamiltonian. with the full Hamiltonian, the state evolves as \begin{align*} e^{- iHt}\ket{0} &= e^{ - iHt } \ket{\omega}\bra{ \omega }\ket{0} + \sum_{ n \geq 1 }  e^ { -iHt } \ket{n}\bra{n}\ket{ 0 } \\
&= e^ { - iE_0 t } \ket{ \omega }\bra{\omega}\ket{0} + \sum_{n \geq 1} e^{ - i E_n t }\ket{n}\bra{n}\ket{0}   \end{align*}
but by construction we've assigned these energy eignestates in terms of the magnitudes of their energy eigenvectors, so $E_0 < E_1 \dots E_n < \dots $. Thus, to make the terms in the sum disappear, we employ a clever trick. This trick is to basically make the $e^{ - i E_0 t} $ term decay slower than the $e^ { - iE_n t} $ terms, by taking the limit of $t$ to $t \rightarrow (1 - i\epsilon )\infty$, where $\epsilon$ is chosen sufficiently small as to make the terms in the sum decay, but not the exponential term in front of the ground state. Hence our final result is that, our ground state for the full hamiltonian $\ket{\omega} $ can be written as \[ \ket{ \omega} = lim_{t \rightarrow (1  -i \epsilon)\infty  }e^{ - i Ht } \left( e^{ - i E_0 t}\bra{\omega}\ket{0} \right)^{- 1} \]

