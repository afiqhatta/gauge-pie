\section{Causality in QFT} 
We still have that potentially, a hint of Lorentz variance because 
the commutation relations of $ \phi$ and $\pi $ are 
equal time relations. If we shift to a differed time, this
may not hold. Our previous fixed time commutation relations involved expressions of the form 
\[
[ \mathcal{ O}_2( \vec{x}, t ) , \mathcal{O}_1 ( \vec{y}, t ) ]
\] but, if we transform the coordinates with a Lorentz transformation, we would get this turning into
\[
[ \mathcal{ O }_1 ( \vec{x}' , t' ) , \mathcal{ O }_2 ( \vec{y}', t'' ) ]  
\] where $ t' ,  t'' $ are not necessarily the same since $ \vec{x}, \vec{y}$ are different. 
What about arbitrary space time separations? 
The real requirement of causality is that all 
space like separated operators commute 
\[ 	[ \mathcal{ O }( x_1 ) , \mathcal{ O }( x_2) ] = 0 , \quad \forall ( x- y)^2 < 0 
\] The reason why commutation is synonymous 
with the fact that operators don't affect one another, 
comes from the fact that they have a shared eigenbasis. 
Suppose we have $A, B  $, hermitian operators which commute. 
Then, we have that they share an eigenbasis which we label as $ \left\{  \ket{ i }  \right\} $, with the property that 
\begin{align*}
A \ket{ i }  & = \lambda_i \ket{ i }, \quad \lambda_i \in \mathbb { R} \\
B \ket{i } &=  \mu_i \ket{ i }, \mu_i \in \mathbb{ R}  
\end{align*} In particular, this means that 
if we expand a given quantum state in terms of the eigenbasis, we get 
\[
\ket{ \psi }  = \sum_{ i } a_{ i } \ket{ i } \implies \bra{ \psi }BA\ket{\psi}  = \sum_i a_{ i } \lambda_{ i } \mu_{ i}
\] where in the last sum we used the orthogonality property 
of the eigenstates. However, notice that this is the same by symmetry of 
calculating $ \bra{ \psi }AB\ket{ \psi}$. Hence, applying an operator 
in the process of measuring the other one doesn't change our 
outcomes. Thus, we have no causality.

If we define $ \Delta( x -y )  = [ \phi ( x), \phi ( y ) ] $, we'd like to check if this holds for space-time separation.  
\[ [ \phi ( x), \phi ( y ) ]  = 
\int \frac{d^{ 3 } p \, d^ 3 p' }{ ( 2 \pi )^6 \sqrt{ 4 E_p E_p' } } \left\{ [a_{\vec{p}}, a_{\vec{p}'}^\dagger ] e^{ i ( p \cdot  x  - p' \cdot  y) } + [ a_{\vec{p}}^\dagger , a_{\vec{p}'} ] e^{ i ( p \cdot  x - p ' \cdot  y) }   \right\} 
\] which is just 
\[ [ \phi ( x) , \phi ( y ) ] 
= \int \frac{ d^3 p }{ 2 E_p ( 2 \pi )^3 } \left\{ e^{  - i p \cdot  ( x - y ) } - e^{ i p \cdot  ( x - y ) } \right\} 
\] Well, what do we know about this function? 
We know that this is Lorentz invariant because the measure is, and so is the 
integrand. 
If  $ x - y $ is space-like, it vanishes because we can take
a Lorentz transformation to $ y - x$ in the first term, 
giving 0. Another way to say this is that if we have space-like separated events, 
then we don't care about breaching causality in time, so we can freely 
transform events in this way. We can't however do this for 
time like events because that would mean time reversal. 
This integral doesn't vanish, however, 
for timelike separations. If we have time like separated events, we 
can Lorentz transform them such that they're constant in space. 
\[
[ \phi ( \vec{x}, 0 ) , \phi ( \vec{x}, t ) ] = \int \frac{d^3 p }{ ( 2\pi ) ^3 2 E_p }\, ( e^{  - i E_p t }  - e^{ i E_p t } ) \neq 0 
\]
We assert that this is non-zero.
To show that this is non-zero, we can evaluate this expression explicitly. Let's do the 
first term. 
\begin{align*}
\int \frac{d^ 3 p }{( 2 \pi )^  3 2E_p }e^{  - i E_p t } & \sim \int dp \, \frac{p^ 2 }{ \sqrt{ p ^2 + m^2 }  }e^{  -i \sqrt{ p^2 + m^ 2 }  t} \\
& = \int_m^\infty dE \sqrt{ E^2  - m^ 2 } e^{  - i E t }
\end{align*}
In the second line, we expressed the integral  in polar form using $ p $ as a radial coordinate. 
In the second line, we used a change of variables and 
set $ E = \sqrt{ p^ 2 + m^2 } $. To evaluate the final integral, 
we need to analytically continue it to $ \mathbb{ C} $. 
Since we have a square root, this induces a branch cut at 
$ E = m $. By the residue theorem, our whole contour evaluates to zero. 
However, the part on the x-axis is what we'd like to calculate. 
To do this, we calculate the arc, as well as the contour which is show as the vertical line. 
Our integral along the arc is given by 
\[
C_R  = 	\int dz \sqrt{ z^2  - m^2 }  e^{ - i z t }, z = m + Re^{ i \theta }  
\] If we do this integral in polars, for large $ R $ this 
integral goes like
\begin{align*}
C_R  & \sim i \int d \theta e^{ i \theta } R \sqrt{ R^ 2 } e^{  - i mt } e^{ R  e^{ i \theta } t } \\
& \sim  e^{  - i m t } i \int d \theta R^2 e ^{ i R \cos \theta t } e ^{ R \sin \theta t } 
\end{align*} Now, since we're taking the contour in 
$  -  \frac{\pi}{2 } <  \theta < 0$, and ignoring
our oscillating $ e ^{  i R \cos \theta t }$ term, 
this integral goes like 
\[
\sim e^{  - i m t } i  \int  d  \theta e^{ - \theta } R^2 e^{  - \epsilon R t }
\] The negative epsilon comes from the fact that 
we're taking a negative contour. Due to 
the negative in the exponential, this term is bounded. So, 
$ C_R \sim e^{  - i m t}$. A similar analysis of the 
downwards vertical contour shows that it grows in the same way. Hence, we have that 
\[
\int \frac{ d^ 3 p }{ ( 2 \pi )^2 2 E_{ p }} e^{  - E_{ p } t } \sim e^{ - i m t } 
\] Thus, our time-like separated integral goes like 
\[
[ \phi ( \vec{x}, 0), \phi ( \vec{x}, t ) ] \sim e^{  - i m t }  - e^{ i m t }
\] This is non-zero. 
\begin{figure}[h]
\centering 
\begin{tikzpicture}

% arcs for tikz 
\draw[help lines, ->]  ( - 0.5, 0) -- ( 4, 0)  node [above left]  {$ Re ( E) $}; 
\draw[help lines, ->] ( 0, - 4) -- ( 0, 1) node [below left] { $Im ( E) $} ;
\draw[blue] (3.5, 0) arc (0:-90:2.5);  
\draw[blue] (1, -2.5) -- (1, -0.5);
\draw[blue] (1.5, 0) -- (3.5, 0); 
\draw[blue] (1, -0.5) arc (-90:0:0.5);
\draw[dotted] (1, 0) node [below] {$m$}; 
\draw [->,decorate,decoration=snake] (1, 0) -- (1,1);

\end{tikzpicture} 
\caption{Our contour integral which we use to calculate our timelike integrals}
\end{figure}

At equal times, we have 
\[
[ \phi ( \vec{x}, t ) , \phi ( \vec{y}, t ) ] = \int \frac{d^ 3 p }{ ( 2 \pi )^3 2E_p }\, ( e^{ i \vec{p} \cdot  ( \vec{x} - \vec{y}) }  - e^{  - i \vec{p} \cdot  ( \vec{x} - \vec{y}) } )  = 0  
\] where in the second term we apply a Lorentz transformation put $\vec{x - y } \to  - \vec{y - x}$. We can do a Lorentz transformation from within this integral because the whole 
integral was Lorentz invariant in the first place. 
Thus, the second term cancels, which agrees with equal time
commutation relations, since it goes to zero. 

