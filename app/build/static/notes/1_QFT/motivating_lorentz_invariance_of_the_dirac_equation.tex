\subsection{Motivating Lorentz invariance of the Dirac equation}
In this section, we'll present the Dirac equation and show that it's Lorentz invariant. The way to do this is slightly more convoluted than showing the Lorentz invariance of the Klein-Gordon equation, like we did in the previous section. It involves exploring how $\gamma^\mu$ transforms under both the $S_{ \mu \nu } $ and $ J_{ \mu \nu} $ representations of the Lorentz transformations, and how these representations relate to one another. 

But first, we present the Dirac equation. A field $\psi(x)$ satisfying the Dirac equation satisfies the equation \[ ( i \gamma_\mu \partial^\mu  - m ) \psi( x) = 0  \]
Note that, as opposed to the Klein-Gordon equation, we have a $m$ as the constant term instead of $m^2$. Also notice the fact that we're not taking two derivatives in this equation, we instead contract the derivative $\partial^\mu$ with $\gamma_\mu$. Now, under the Lorentz transformation where $x^\mu \rightarrow x^\nu \Lambda\indices{^\mu_\nu} $, recall our field transforms as \[ 
\psi(x)  \mapsto \Lambda_{ \frac{ 1 }{2 }} \psi( \Lambda^{ -1} x ) \] 
To show that the Dirac equation is invariant under this specific transformation, we require that (for reasons we will show later) \[ 
\Lambda_{ \frac{ 1}{ 2}}^{ -1} \gamma^\mu \Lambda_{ \frac{ 1}{2}}  = \Lambda\indices{ ^\mu_\nu } \gamma^\nu \] 
One can see that if this condition is satisfied, then, upon transforming the equation
\begin{align*} 
( i \gamma^\mu \partial_\mu  -m) \psi & \rightarrow ( i \gamma^\mu \partial_\mu  - m) \Lambda_{ \frac{ 1}{2} } ( \phi \Lambda^{ -1} x )  \\ 
& = \Lambda_{ \frac{1}{ 2} } \Lambda_{ \frac{1}{2} }^{ -1}  ( i \gamma^\mu ( \Lambda^{-1 })\indices{_\mu^\nu} \partial_\nu  - m ) \Lambda_{ \frac{ 1}{2} } \psi( \Lambda^{-1} x ) \\ 
& = \Lambda_{ \frac{ 1}{ 2} } ( i \Lambda_{ \frac{ 1}{2}}^{ -1} \gamma^\mu \Lambda_{ \frac{ 1}{2} } ( \Lambda^{ -1} )\indices{_\mu^\nu} \partial_\nu - m ) \psi( \Lambda^{ -1} x) \\
&= \Lambda_{ \frac{ 1}{2}} ( i \gamma^\mu \partial_\mu  -m ) \psi( \Lambda^{-1} x)    \\ 
&= 0  
\end{align*} 
Let's review carefully what we've done here. In the first line, we've transformed $\psi( x) $ appropriately according to our Lorentz transformation. In the second line, we've done two things. First, we multiply on the left by $\Lambda_{ \frac{1}{2}} \Lambda_{\frac{1}{2}}^{ -1} $ since it's just the identity. In addition, we also pull out a factor of $\Lambda^{-1} $ due to the chain rule. Now, in the third line, we used the identity we've derived above to switch out our expression for $\Lambda_{ \frac{ 1}{2}}^{ -1}  \gamma^\mu \Lambda_{ \frac{1}{2} } $ with just $\Lambda \gamma^\nu $. In doing this, the factors of $\Lambda $ have cancelled each other out. 
The final expression is zero, which means that we've successfully shown that the Dirac equation is Lorentz invariant. 

Now, we'll prove the expression above, by using the infinitesimal generators instead. Observe that the above condition is equivalent to \[ 
( 1 + \frac{ i \omega_{ \rho \sigma} }{ 2} S^{ \rho \sigma} ) \gamma^\mu (1 - \frac{ i \omega_{ \rho \sigma }}{ 2} S^{ \rho \sigma} = ( 1 - \frac{ i \omega_{ \rho \sigma}}{ 2} ( J^{ \rho \sigma} )\indices{ _\mu^\nu}\gamma_nu \] So in turn, if we can show that \[ [\gamma^\mu, S^{\rho \sigma}] = \gamma^\nu (J^{\rho \sigma})\indices{^\mu_\nu} \] we've proven the identity above. We do this as follows 
\[ (J^{\mu \nu} )\indices{_\alpha_\beta}  = i \left( \delta\indices{^\mu_\alpha} \delta\indices{^\nu_\beta} - \delta\indices{^\mu_\beta}\delta\indices{^\nu_\alpha} \right)\]  
When we raise the index on a delta function, we get our metric back. So our expression is 
\[ (J^{\mu\nu})\indices{_\alpha^\beta} = i \left( \delta\indices{^\mu_\alpha} g^{\nu\beta} - g^{\mu\beta}\delta\indices{^\nu_\alpha} \right) \]
Thus \[ \gamma^\alpha (J^{\mu\nu} )\indices{_\alpha^\beta} = i \left( \gamma^\mu g^{\nu\beta} - \gamma^\nu g^{\mu\beta} \right) \]


As for the left hand side \begin{align*} [\gamma^\beta, S^{\mu\nu}] &= [\gamma^\beta, \frac{i}{4} [\gamma^\mu, \gamma^\nu]] \\ &= \frac{i}{4}\left( \gamma^\beta \gamma^\mu \gamma^\nu - \gamma^\beta \gamma^\nu \gamma^\mu - \gamma^\mu \gamma^\nu \gamma^\beta + \gamma^\nu \gamma^\mu \gamma^\beta \right) \end{align*} but we can use the identity that \[ \{ \gamma^\mu , \gamma^\nu \} = - g^{\mu\nu} \] and commute index pairs which are either $( \nu, \beta) $ or $ ( \mu, \beta ) $, to give 
\[ \frac{i} { 4} \left( - \gamma^\mu \gamma^\beta \gamma^\nu - 2g^{\mu \beta } \gamma^\nu  + \gamma^\nu \gamma^\beta \gamma^\mu + 2 g^{\beta \nu} \gamma^\mu + \gamma^\mu \gamma^\beta\gamma^\nu + 2 g^{ \nu \ beta } \gamma^\mu - \gamma^\nu \gamma^\beta \gamma^\mu - 2 \gamma^\nu g^{\mu \beta} \right) \] 
but the terms above cancel to give the correct expression. Hence we've proven the identity above, and have shown that Dirac's equation is Lorentz invariant.  

So we've done all this work - but let's remind ourselves what it's for. This commutation relation is important because it gives us an equation relating the generators $J^{\mu\nu}$ and $S^{ \mu \nu } $. This commutation relation is equivalent to saying, with $\omega_{\rho \sigma} $ as a parameter, that infinitesimally we have \[ \left(  1 + \frac{ i \omega_{ \rho \sigma } S^{ \rho \sigma} }{ 2} \right) \gamma^\mu \left( 1 - \frac{ i \omega_{ \rho\sigma} S^{ \rho \sigma }}{2} \right)  = ( 1 - i ( J^{ \rho \sigma } )\indices{^\mu_\nu})  \gamma^\nu \implies \Lambda_{ \frac{1}{2}}^{ -1} \gamma^\mu \Lambda_{ \frac{1}{2}}  = \Lambda\indices{^\mu_\nu}\gamma^\nu \]  
And with this identity, we've shown that the Dirac equation is Lorentz invariant. 

