\subsection{Lorentz tranformations and spinors}  
To motivate the derivation of the Dirac equation, we need to first discuss in general, objects which transform sensibly with Lorentz transformations. If we have a coordinate system denoted by $x^\mu$, we Lorentz transform it with the map $x^\mu \rightarrow \Lambda\indices{^\mu_\nu} x^\nu$. If we had a scalar field $\psi(x)$, then this passive transformation would induce the transformation \[ 
\psi(x) \rightarrow \psi(\Lambda^{ -1} x) \] 
Now, instead of a scalar field, we might wish to consider what happens to, say, a vector field when we induce a Lorentz transformation. In the case that our vector field transforms linearly in response to Lorentz transformations, we have a spinor. More specifically, we have a complex vector field $\psi^a (x) $ which obeys the transformation law 
\[ \psi^a (x) \rightarrow \Lambda\indices{^a_b}\psi^b(x ) \] 


Suppose we had some eaution which a scalar field $\phi$ satisfies. For example, we have the Klein-Gordon equation 
\[ \left( \partial_\mu\partial^\mu - m^2 \right) \phi = 0 \]. If we were to perform a Lorentz boost which shifts our frame of reference, then it only makes physical sense that the equation still holds, because nothing about the actual physical system has changed. Suppose that our Lorentz boost is written as 
\[ x^\mu \rightarrow x'^\mu  = \Lambda\indices{^\mu_\nu}x^\nu \]
Then our corresponding scalar field should transform as 
\[ \phi(x) \rightarrow \phi(\Lambda^{-1} x) \] 
We can chack that, under this transformation, the Klein-Gordon equation still holds. 


We can generalise this kind of transformation to a field with multiple components which we index as $\phi_a(x) $. As inspiration, we explore how some vector field $V^i$ in three dimensions transforms under a rotation.
\[ 
V^i \rightarrow V'^i  = R^{ij}V^j 
\]
and so we'd expect a Lorentz boost to have a scalar field transform like 
\[
\phi_a(x) \rightarrow \phi'(x)_a = M(\Lambda)\indices{^b_a}\phi_b(\Lambda^{-1}x)
\] 
In the above, we're representing the Lorentz transformation as a matrix $M(\Lambda)$. We call this a representation of our Lorentz transformation. Representations should obey the rule that 
\[ M(\Lambda)M(\Lambda') = M(\Lambda \Lambda')\] or in other words the representation should respect the group structure of Lorentz transformations. Proverbially, we are taking a methematical 'encoding' of our transformation. How do we go about finding the representation $M$? 

Again, we seek inspiration from rotation groups in 3 dimensions. A rotation in 3 dimensions is represented in real space by the matrices in $SO(3)$, which is isomorphic to $SU(2)$. This $SU(2)$ has a basis, which are the Pauli matrices $\sigma_i$. Given a quantum state $\ket{\psi}$, how do we represent its transformation?

Let $\ket{\psi}$ be a wavefunction representing a $\frac{1}{2}$ spin particle. Then, we can represent rotations with thefollowing transformation \[ \ket{\psi} \rightarrow \ket{\psi '}  = \exp{i \mathbf{n} \cdot \mathbf{J}} \ket{\psi} \]
In this case, $J_i$ are the generators of this transformation. We say that the Lie group of rotations in three dimensions are generated by a Lie algebra with commutation relations 
\[ J_k  = i \epsilon_{ijk} [J_i, J_j] \]   
In three dimensions for our rotation group, this operator is given by 
\[ \mathbf{J}  = \mathbf{x} \times \mathbf{p} = \mathbf{x} \times ( - i \nabla) \] 
Where we can index this angular momentum object as 
\[ J^{\nu \mu} = i (x^{\nu} \partial^\mu  - x^\mu \partial^nu) \]
This will precisely be our generalisation for our generatorfor our Lorentz group, except now the indices $\mu, \nu$ are taken over $\mu, \nu = 0, 1, 2, 3$. This object obeys the Lorentz algebra.
\[ [J^{\mu \nu}, J^{\rho \sigma}] = i( - g^{\mu \rho} J^{\nu \sigma} + g^{\mu \sigma}J^{\nu \rho}  - g^{\nu \sigma}J^{\mu \rho} + g^{\nu \rho}J^{\mu \sigma}) \]
And in general, any expression for $J^{\mu \nu}$ which satisfies this equation is a valid representation of our Lie algebra. We denote the components of $J^{\mu \nu}$ by writing down the object with two additional indices $\alpha \beta$. Thus we denote our object by $(J^{\mu \nu})_{\alpha_beta}$. A simple representation that we pull out of the hat is \[ (J^{\mu \nu})_{\alpha \beta} = i \left( \delta\indices{^\mu_\alpha} \delta\indices{^\nu_\beta}  - \delta\indices{^\mu_\beta}\delta\indices{^\nu_\alpha} \right) \] 

Why is this choice intuitive? Well, our Lorentz transformation is obtained by exponentiating our generator that we have here. Thus, for an anti-symmetric 2nd rank tensor, our full transformation is given by 
\[ 
\exp(i \frac{\omega_{\mu \nu}}{2} J^{\mu \nu})
\]
and infinitesimally applied to a 4-vector, our transformation is \[ V^\mu \rightarrow \left( \delta\indices{^\mu_\nu} - \frac{i}{2} \omega_{\alpha\beta}(J^{\alpha\beta})\indices{^\mu_\nu} \right) V^\mu \]

