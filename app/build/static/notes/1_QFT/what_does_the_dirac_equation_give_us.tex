\subsection{What does the Dirac equation give us?} 
The Dirac equation is important because it provides the framework for quantization of spin-$\frac{1}{ 2}$ particles, like electrons and other fermions. Another startling prediction that the Dirac equation gives us is the existence of anti-matter. This equation is an amazing combination of the geometry of spinors in Minkowski space-time, and the wave-function formalisation of quantum mechanics that we all know and love. 
In this section, we'll provide some motivation for constructing the Dirac equation, solve it, then quantise this object to give rise to particles (fermions). 

Right now we'll try to analyse how fields with \textbf{multiple}
components change under an underlying Lorentz transformation. 
Recall that when we have a Lorentz transformation in our 
coordinates, we induce a transform on our scalar field. 
\[
	x ^ \mu \to ( x ') ^ \mu = \Lambda \indices{ ^ \mu _ \nu } x  ^ \nu, \quad 
 \phi ( x) \to \phi ' ( x) = \phi ( \lambda ^{ -1 } x ) 
\] Most particles have an intrinsic angular momentum-spin however, 
and so transform in a non-trivial way 
under boosts and rotations. 
For example a spin-1 vector field $ A ^ \mu $ (for example, 
our familiar electromagnetic vector potential) transforms under a 
Lorentz boost as 
\[
	A ^ \mu ( x) \to (A ') ^ \mu ( x) = \Lambda \indices{ ^ \mu _ \nu } A ^ \nu ( \Lambda ^{ - 1 } x )  
\] There's no 
reason a priori why vector fields with 
multiple components need to transform with $ \Lambda \indices{ ^ \mu _ \nu } $
multiplying it out front. In general, we have that our field transforms 
as 
\[
	\phi ^ a ( x) \to D \indices{ ^ a _ b  } ( \Lambda ) \phi ^ b ( \Lambda ^{ - 1} x )  
\] where our matrices $ D $ form a representation of our Lorentz group. Let's 
explain why this is. We treat $ \psi ^ a $ as a vector which 
is acted on by a symmetry. However, for this symmetry to  \textbf{act} 
on an object like a vector, it needs to be represented in 
the same space (for example, as a matrix). 
Up until now, we've been representing a Lorentz transformation group
in it's standard form as a matrix, but this isn't
the only representation (recall the Lorentz group isn't 
defined in terms of matrices in the first place, but 
as members of the orthogonal group $O(1, 3 )$ ). 
Representations obey the properties that the group structure is preserved.
This means that, if $ D $ is a representation, then
\begin{align*}
	D ( \Lambda _ 1 ) D ( \Lambda _ 2) & = D ( \Lambda _  1 \Lambda _ 2 ) \\
	D ( \Lambda ^{ - 1} ) &= D ( \Lambda ) ^{ - 1} \\
	D ( I ) &= 1 
\end{align*} To find representations 
we look at the Lorentz algebra, considering infinitesimal transformations
\[
 \Lambda \indices{ ^ \mu _ \nu } = \delta \indices{ ^ \mu _ \nu } + 
 \epsilon \omega \indices{ ^ \mu _ \nu } + O ( \epsilon ^ 2 )  
\] Working to infinitesimal order, the find that 
the generators obey the relation 
\[
	\Lambda \indices{ ^ \mu _ \sigma  } \Lambda \indices{ ^ \nu _  \rho }  =  \eta ^{ \mu \nu }
\]  Substituting this into our relation we have that 
\[
	( \delta \indices{ ^ \mu _ \sigma } + \epsilon \omega \indices{ ^ \mu _ \sigma } ) ( 
	\delta \indices{ ^ \nu  _ \rho }   + \epsilon \omega \indices{ ^ \nu  _ \rho }  ) \eta ^{ \sigma \rho }  =
	\eta ^{ \mu \nu } + O ( \epsilon ^ 2 ) 
\]  This implies that our generator satisfies 
\[
 \omega _{ \mu \nu } = - \omega _{ \nu \mu } \implies 
 \omega _{  \mu \nu } \text{ is anti symmetric}
\] This generator has $ 6 $ components, split up into 
3 rotations and 3 boosts. Thus, we can introduce a basis of 
six $ 4 \times 4 $ matrices 
\[
	( \mathcal{ M } ^{ \rho \sigma } ) ^{ \mu \nu }  = \eta ^{ \rho \mu } \eta ^{ \sigma \nu } 
	- \eta ^{ \sigma \mu } \eta ^{ \rho \nu }
\] We can lower the index here to give 
\[
	( \mathcal{ M } ^{ \rho \sigma } ) \indices{ ^ \mu _ \nu } = 
	\eta ^{ \rho \mu } \delta \indices{ ^ \sigma  _ \nu }  - 
	\eta ^{ \sigma \rho } \delta \indices{ ^ \rho _ \nu } 
\]  For example, we have that 
\[
	( \mathcal{ M } ^{ 01  } ) \indices{ ^ \mu_ \nu } = 
	\begin{pmatrix}  0 & 1 & 0 & 0 \\ 
	1 & 0 & 0 & 0 \\
        0 & 0 & 0 & 0 \\ 0 & 0 & 0 & 0 \end{pmatrix} 
\] This specific generator generates a boost in the $x ^ 1 $ 
direction. We can hence write our generator as 
a linear sum of these basis matrices. 
\[
	\omega \indices{ ^ \mu _ \nu } = \frac{1}{2 } \left(  \Omega_{ \rho \sigma } \mathcal{ M } ^{ \rho \sigma } \right)
	\indices{ ^ \mu _ \nu } 
\]  We call $ \mathcal{ M }^{ \rho\sigma } $ the generators 
of our Lorentz group, and our corresponding $ \Omega ^{ \rho\sigma } $ 
the anti symmetric parameters of the Lorentz transformation.  
Our Lorentz algebra, one can calcuate, is 
\[
	[ \mathcal{ M } ^{ \rho \sigma } , \mathcal{ M  } ^{ \tau \nu  } ] 
	= \eta ^{ \sigma \tau } \mathcal{ M } ^{ \rho \nu } - \eta ^{ \rho \tau } \mathcal{ M} ^{ \sigma \nu }
	- \eta ^{ \rho \nu } \mathcal{ M }^{ \sigma \tau } - \eta ^{ \sigma \nu } \mathcal{ M } ^{ \rho \tau }
\] What we've 
done here is that we've calculated the Lie bracket of two 
elements in our representation. This means that any other representation 
of the Lorentz group necessarily satisfies this relation, since 
representations of a Lie algebra needs to preserve the Lie bracket. 
Our corresponding boost is $ \exp ( \frac{1}{2 } \Omega_{ \rho \sigma  } ) \mathcal{ M } ^{ \rho \sigma }$.
This is because when we are returning 
the infinitesimal transformation back into its full form 
by exponentiating. 

\subsection{The Spinor Representation} 

Now, the aim of the game is to try and find 
other representations which satisfy the 
Lorentz algebra.
To help us find representations that satisfy the 
Lorentz algebra, we start by defining 
something called the Clifford algebra. This is an algebra whose 
elements satisfy the anti-commutation 
relations 
\[
 \left\{  \gamma ^ \mu , \gamma ^ \nu  \right\}  = 2 \eta ^{ \mu \nu } I , 
 \quad \left\{  \gamma ^ \mu , \gamma ^  \nu  \right\}   = \gamma ^ \mu \gamma ^ \nu 
 + \gamma ^ \nu \gamma ^ \mu 
\] where $ \gamma ^ \mu $ are a set of four matrices. 
We have that our matrices in the algebra obey 
the properties 
\begin{align*}
	\gamma ^ \mu \gamma ^ \nu & = - \gamma ^ \nu \gamma ^ \mu , \quad \mu \neq \nu \\
	( \gamma ^ 0 ) ^ 2 & = I, \quad ( \gamma ^ i ) ^ 2 =  - I , \quad i \in \left\{ 1, 2, 3  \right\} 
\end{align*}
Our simpliest representation which obeys these commutation relation 
are a set of $ 4 \times 4 $ matrices, which we call the Chiral representation. 
These are given in block matrix form as 
\[
	\gamma ^ 0 = \begin{pmatrix}  0 & I _ 2 \\ I _ 2 & 0  \end{pmatrix} , \quad 
	\gamma ^ i = \begin{pmatrix}  0 & \sigma ^ i \\ - \sigma ^ i & 0  \end{pmatrix} 
\] Here, $ \sigma ^ i  $  are the 
Pauli matrices given by 
\[
	\sigma ^ 1 = \begin{pmatrix}  0 & 1 \\ 1 & 0  \end{pmatrix}  , \quad 
	\sigma ^ 2 = \begin{pmatrix}  0 & - i \\ i & 0  \end{pmatrix} , \quad
	\sigma ^ 3  = \begin{pmatrix}  1 &  0 \\ 0 & 1  \end{pmatrix} 
\] 

These Pauli matrices 
obey the commutation and anti commutation relations 
\[
 \left\{  \sigma ^ i , \sigma ^ j  \right\}  = 2 \delta ^{ ij } I _ 2 , 
 \quad [ \sigma ^ j , \sigma ^ k ] =  2i \epsilon ^{ jkl } \sigma ^ l 
\] Given this representation, we can also 
generate a new representation by conjugating by a constant 
invertible matrix $ U$, so $ U \gamma ^ \mu U ^{ - 1} $  works 
also as a representation. 

If we define 
\[
	S ^{ \rho\sigma } : = \frac{1}{4} [  \gamma ^ \rho ,  \gamma ^ \sigma ] 
	= \begin{cases}
		0 & \rho = \sigma \\ 
		\frac{1}{2  } \gamma ^ \rho \gamma ^ \sigma &  \rho \neq \sigma 
	\end{cases}
	 = \frac{1}{2 } \gamma ^ \rho \gamma ^ \sigma  - \frac{1}{2 } \eta ^{ \rho \sigma } I 
\] we aim to show that this satisfies 
the Lorentz algebra, whilst being philosphically different 
to the algebra we defined before. 
Now, expanding this whole thing out explicitly is difficult to 
do, so we prove a handful of useful relations to show this. We can 
 
\begin{claim}
We have that 
	\[
		[ S ^{ \mu \nu } , \gamma ^ \rho ]  = \gamma ^ \mu \eta ^{ \nu \rho } 
		  - \gamma ^ \nu \eta ^{ \rho \mu }
	\]  
\begin{proof}
There's some steps we have to follow and it's not as straightforward 
as it seems from first glance. 
The basic idea is to commute things out 
so that we only have one $ \gamma $ factor per term. 

\begin{align*}
	[S ^{ \mu \nu } , \gamma ^{ \rho } ] &=  
	\frac{1}{2 } [ \gamma ^ \mu \gamma ^ \nu  - \eta ^{ \mu \nu } , \gamma ^{ \rho }  ] \\
					     &=  \frac{1}{2 } [ \gamma ^ \mu \gamma ^ \nu , \gamma ^ \rho ]  \\
					     &=  \frac{1}{2  } ( \gamma^ \mu \gamma ^ \nu \gamma ^ \rho - 
					     \gamma ^ \rho \gamma ^ \mu \gamma ^ \nu ) 
\end{align*}
Now, this is a bit tricky to manage. The key is to anti-commute things 
in the right order. We anti-commute the last two 
matrices in the first term, and the first two matrices in 
the last term. 
The above is then equal to 

\begin{align*}
	\dots & = \frac{1}{2} \gamma ^ \mu ( \left\{  \gamma ^ \nu , \gamma ^ \rho  \right\}   - \gamma ^ \rho \gamma ^ \nu )
	 - \frac{1}{2 }\left(  \left\{  \gamma ^ \rho , \gamma ^ \mu  \right\}  - \gamma ^ \mu \gamma ^ \rho  \right)  \gamma ^ \nu \\
	      &=  \frac{1}{2 } \gamma ^ \mu \left(  
	      2 \eta ^{ \nu  \rho }  - \gamma ^ \rho \gamma ^ \nu \right)   - \frac{1}{2 } 
	      ( 2 \eta ^{ \rho \mu } - \gamma ^ \mu g \gamma ^ \rho ) \gamma ^ \nu \\
	      &=  \gamma ^ \mu \eta ^{ \nu \rho } - 
	      \gamma ^ \nu \eta ^{ \rho \mu }
\end{align*}
\end{proof} 
\end{claim}
Using this, we can then prove that $ S^{ \rho \sigma} $
satisfies the Lorentz algebra! This means we 
would then be able to use this 
as a valid representation of the Lorentz group. 
\begin{claim}
	\[
		[ S ^{ \rho\sigma } , S ^{ \tau \nu } ] = 
		\eta ^{ \sigma \tau } S ^{ \rho \nu }  - 
		\eta ^{ \rho \tau } S ^{ \sigma \nu } 
		+ \eta ^{ \rho \nu } S ^{ \sigma \tau } 
		 - \eta ^{ \sigma \nu } S ^{ \rho \tau }
	\]

\begin{proof}
We can expand out one side and appeal to the relation 
we proved earlier. 
\begin{align*}
	[ S ^{ \rho \sigma }, S^{ \tau \nu } ] &=  
	\frac{1}{2 } [ S ^{ \rho \sigma } ,  \gamma ^ \tau 
	\gamma ^ \nu - \eta ^{ \tau \mu } I ] \\
	&=  \frac{1}{2 } [ S^{ \rho \sigma } , \gamma ^{ \tau } 
	\gamma ^{ \nu } ] \\
	&=  \frac{1}{2 } [ S ^{ \rho \sigma } , \gamma ^ \tau ] 
	\gamma ^{ \nu } + \frac{1}{2 } \gamma ^ \tau [ 
	S ^{ \rho \sigma } , \gamma ^ \nu ] \\
	= \frac{1}{2 } ( \gamma ^ \rho \eta ^{ \sigma \tau}  - \gamma ^ \sigma 
	\eta ^{ \rho \tau } ) \gamma ^ \nu + \frac{1}{2 } \gamma ^ \tau 
	( \gamma ^ \rho \eta ^{ \sigma \nu }  - \gamma ^ \sigma \eta ^{ \rho \nu 
	} ) 
\end{align*}
Now, to recover what we had earlier, we need to sub out our expressions 
with two gammas for an expression in $S$. To do this, 
we invert our relation for $ S $ so that 
\[
 \gamma ^ \mu \gamma ^ \nu  = 2 S ^{ \mu \nu } + \eta ^{ \mu \nu } I 
\] We then 
get that the above is equal to 
\begin{align*}
	\dots & = \frac{1}{2 } \left(  2 ^{ \rho \nu } \eta ^{ \sigma \tau }
	 + \eta ^{ \sigma \tau } \eta ^{ \rho \nu } I \right) \\ 
	      &  - \frac{1}{2 } \left(  
	      2 S^{ \sigma \nu } \eta ^{ \rho \tau } + \eta ^{ \sigma \nu } 
      \eta ^{ \rho \tau } I  \right) \\
	      & + \frac{1}{2 } \left(  2 S^{ \tau \rho } \eta ^{ \sigma \nu }
	      + \eta ^{ \rho \tau } \eta ^{ \sigma \nu } I \right) \\
	      & - \frac{1}{2 } \left(  2 S ^{ \tau\sigma }
	      \eta ^{ \rho \nu } + \eta ^{ \rho \nu } \eta ^{ \tau \sigma }\right)
	      \\
	      &=  S ^{ \rho \nu } \eta ^{ \sigma \tau } 
	       - S^{ \sigma \nu } \eta ^{ \tau \rho } + 
	       S ^{ \tau \rho } \eta ^{ \sigma \nu } 
	        - S ^{ \tau \sigma } \eta ^{ \rho \nu }\\
\end{align*}
The terms with two $ \eta $ terms cancel out. 
The object that we're left with is precisely the Lorentz algebra. 

\end{proof}
\end{claim} Thus, $ S $ provides a representation of 
the Lorentz group. 
We now define a Dirac spinor, $ \psi _ \alpha ( x) $which is a four component 
vector that satisfies the following transformation 
law 
\[
	\psi ^ \alpha ( x) \to  S[ \Lambda ] \indices{ ^ \alpha _ \beta } \psi ^ \beta ( 
	\Lambda ^{ - 1} x ) 
\] where we've defined $ \Lambda = \exp \left(  \frac{1}{2 } 
\Omega _{ \rho \sigma } M ^{ \rho \sigma } \right)  $, and 
our spinor representation $ S [ \Lambda ] = \exp \left(  
\frac{1}{2 } \Omega _{ \rho \sigma } S ^{ \rho \sigma } \right)  $. 
We need to check that the Spinor representation is not 
equivalent to the usual vector representation, in the
sense that we need to check what we've
written down gives something \textbf{different} than 
our standard Lorentz boost. 

To check that what we have is different, we trial a particular Lorentz transformation (in this case, a rotation by $ 2 \pi$). 

\begin{example}{(The Spinor representation is different)} 
If we choose 
\[
 S^{ ij } = \frac{1}{4 }\left[  \begin{pmatrix}  
 0 & \sigma ^ i \\ - \sigma ^ i & 0 \end{pmatrix}  , 
 \begin{pmatrix}  0 & \sigma ^ j \\ 0 & \sigma^ k   \end{pmatrix}  \right]  
 =  - \frac{i \epsilon ^{ ijk } }{ 2 } \begin{pmatrix}  \sigma ^ k & 0 
 \\ 0 & \sigma^k \end{pmatrix}  \text{ from } \sigma ^ i \text{ algebra}
\] If we write as a paramater 
$ \Omega_{ ij } =  - \epsilon _{ ijk } \phi ^ k $ . 
Exponentiating, we get that 
\[
	S [ \Lambda ]  = \exp \left(  \frac{1}{2 } \Omega _{ \rho \sigma } 
	S ^{ \rho\sigma } \right)  = \begin{pmatrix}  
	e ^{ i \phi \cdot  \sigma / 2  } & 0 \\ 0 & 
e^{ i \phi \cdot  \sigma / 2 } \end{pmatrix}  
\]   Now, if we consider a rotation of 
$ 2 \pi $ abou the $ x ^ 3 $ axis, we get that 
our corresponding representation is 
\[
	S [ \Lambda ] = \begin{pmatrix}  e ^{ i \pi \sigma ^ 3 }` & 0 \\ 
	0 & e ^{ i \pi \sigma ^ 3 } \end{pmatrix}   = - I _ 4 
\] In this notation, the exponential 
needs Thus, a rotation of $ 2 \pi  $  takes a 
spinor $ \phi _ \sigma ( x ) \to  - \phi _{ \sigma } ( x) $, not like a vector, 
which goes like $ \Lambda = I $ as expected (in our fundamental representation). 

Now, what happens with boosts of spinors 
\[
	S ^{ 0i  }  = \frac{1}{2 } \begin{pmatrix}  - \sigma ^ i & 0 \\ 0 & \sigma ^ i   \end{pmatrix}  
\]  If we write our boost parameter as 
$ \Omega _{ 0i }  = \chi _ i $, then our corresponding representation 
is 
\[
	S [ \Lambda ] = \begin{pmatrix}  e^{  - \chi \cdot \sigma / 2   } & 
	0 \\ 0 & e ^{ \chi \cdot  \sigma / 2 } \end{pmatrix}  
\] Note that for rotations, $ S [ \Lambda ] $ is unitary, since $ S [ \Lambda ] ^ \dagger S [ \Lambda ]  = I $, but for boosts, it's not.
\end{example} 

\begin{thm}{(No finite dimensional, unitary representations of the Lorentz group)} 
There are no finite dimensional unitary representations of the Lorentz group! 

\begin{proof} 
$ S [ \Lambda ]  = \exp \left[  \frac{1}{2 } \Omega_{ \rho \sigma } 
S ^{ \rho \sigma } \right]  $ is only unitary if $ S ^{ \rho \sigma } $ are 
anti hermitian. Now, when we take the Hermitian conjugate 
of a commutator, not only does each element pick up a 
hermitian conjugate, but the arguments in the commutator switch around. 
This means we have to pick up a minus sign. 
\[
	( S ^{ \rho \sigma } ) ^ \dagger  = -  \frac{1}{4 } [ ( \gamma ^ \rho ) ^ \dagger , 
	( \gamma ^ \sigma ) ^ \dagger ] 
\] can be anti-hermitian only if all $ \gamma ^ \mu $ are hermitian or are all anti hermitian. 
However, this can't be arranged, and is impossible to do. 
This can't be arranged, because
\[
 ( \gamma ^ 0 ) ^ 2 = 1 \] 
 This means in particular that $ \gamma ^ 0 $ has real eigenvalues, 
 which means it can't be hermitian since hermitian matrices 
 only have pure imaginary eigenvalues. 
\[
 ( \gamma ^ i ) ^ 2 = 1 \implies \gamma ^ i \text{ can't be hermitian }
\] In general, there's no way to pick $ \gamma ^ \mu $  such that $ S ^{ \mu \nu } $ 
are anti-Hermitian.
\end{proof} 
\end{thm} 

\subsection{Constructing a Lorentz Invariant action of $ \psi $ }
Now, something of immediate interest is to see whether we 
can construct some sort of action from this. To carry this out, we
should try to construct Lorentz invariant quantities. 
We write $ \psi ^ \dagger ( x )   = ( \psi ^ * ) ^ T ( x ) $. 
One question that we might have is to ask whether 
$ \psi ^ \dagger ( x) \psi( x) $ is a Lorentz scalar. 
However, under a Lorentz transformation, 
\[
	\psi ^ \dagger ( x) \psi  ( x) \to \psi ^ \dagger ( \Lambda ^{ - 1} x ) 
	S [ \Lambda ] ^ \dagger S [ \Lambda ] \psi( \Lambda ^{ - 1}x ) 
\] This is not invariant in general since $ S [ \Lambda ] $ is 
not unitary. 
In the chiral representation, we have that 
\[
	( \gamma ^ 0 ) ^ \dagger  = \gamma ^ 0 , \quad ( \gamma ^ i) ^ \dagger = 
	 - ( \gamma ^ i ) 
\] We can verify by checking for each index, 
that we can write out an identity to relate 
the hermitian conjugate of a gamma matrix to 
an expression which doesn't involve conjugates as 
\[
 ( \gamma ^ \mu ) ^ \dagger = \gamma ^ 0 \gamma ^ \mu \gamma ^ 0 
\] Thus, we can find the Hermitian conjugate of $ S $ as 
\[
	( S ^{ \mu \nu } ) ^ \dagger = - \frac{1}{4 } [ ( \gamma ^ \mu ) ^ \dagger, 
	( \gamma^ \nu ) ^ \dagger ]  =
	- \frac{1}{4 } [ \gamma ^ 0 \gamma ^ \mu \gamma ^ 0 , 
	\gamma ^ 0 \gamma ^ \nu \gamma ^ 0 ]  = - \gamma ^ 0 S ^{ \mu \nu } \gamma ^ 0 
\]  We do this because, since $ \gamma ^ 0 = ( \gamma ^ 0 ) ^{ -1 } $, 
when we exponentiate this object we can pull out the $ \gamma ^ 0 $ on
both sides. Hence, our exponential gives 
\[
 S [ \Lambda ] ^ \dagger = \exp \left(  \frac{1}{2 } 
 \Omega _{ \mu \nu } ( S ^{ \mu \nu } ) ^ \dagger \right)   = \gamma ^ 0 S [ \Lambda ] ^{ - 1} 
 \gamma ^ 0 
\]
\begin{defn}{(The Dirac adjoint)}
	The Dirac adjoint of  $ \psi $ is defined as 
	\[
	\overline{\psi } ( x) := \psi ^ \dagger  (x) \gamma ^ 0 
\] We define this object in the spirit of 
creating scalars, vectors and other objects which transform nicely 
under Lorentz transformations. 
\end{defn}

 
\begin{claim}{(Lorentz invariance of $ \overline{\psi } \psi $)}
Our claim is that $ \overline{ \psi } \psi  $ is indeed a Lorentz scalar. 
To prove this 
observe that 
\[
	\overline{ \psi  } \psi  = \psi ^ \dagger \gamma ^ 0 \phi \to \psi ^ \dagger ( 
	\Lambda ^{ - 1} x ) S [ \Lambda ] ^ \dagger \gamma ^ 0 S [ \Lambda ] \psi ( 
	\Lambda ^{ - 1 }x ) 
\] Using the identity above, we get that this is 
the same. 
\end{claim} 

\begin{claim}{($ \overline{\psi } \gamma ^ \mu \psi $ transforms as a Lorentz vector)}
Things that transform nicely under Lorentz boosts 
are called Lorentz vectors (or more 
familiarly, 4-vectors). This means that they 
transform much like how a position vector 
transforms under a Lorentz boost: 
\[
 x ^ \mu \to \Lambda \indices{^ \mu _ \nu } x ^ \nu  
\] 
We claim that the object $ \overline{\psi } \gamma ^ \mu \psi  $
transforms in this way to, where under a Lorentz 
transform we have that 
\[
 \overline{ \psi } \gamma ^ \mu \psi \to 
 \Lambda \indices{ ^ \mu _ \nu } \overline{ \psi } \gamma ^ \nu 
 \psi 
\]
\begin{proof} 
	Our vector transforms as follows (note that 
	the $ \gamma ^ \mu $ matrix doesn't change 
	since it's a constant object - only the spinors change.
\[
	\overline{ \psi } \gamma ^ \mu \psi \to \overline{ \psi } S [ \Lambda ] ^{ - 1} 
	\gamma ^ \mu S [ \Lambda ] \psi 
\]  For this to be a Lorentz vector, we require that 
\[
	S [ \Lambda ]^{ - 1} \gamma ^ \mu S [ \Lambda ] = \Lambda \indices{ ^ \mu _ \nu } \gamma ^ \nu  
\] To go from here, we need to 
write out the transformations in terms of the 
exponents of their generators. In particular, 
we require that the coefficients of the 
generators $ \Omega_{ \rho \sigma  }  $ are 
the same on both sides  - it's just the matrices $ \mathcal{ M } $
and $ S $ that are different since we're dealing with distinct 
representations. 
\[
	\Lambda \indices{ ^ \mu _ \nu } = \exp \left(  
	\frac{1}{2 } \Omega _{ \rho \sigma  }\mathcal{ M } ^{ \rho \sigma } \right)  , 
	\quad S[ \Lambda ] = \exp \left(  \frac{1}{2 } \Omega _{ \rho \sigma } 
	S ^{ \rho \sigma } \right) 
\] So, we need to prove that 
\[
	( \mathcal{ M  }^{ \rho \sigma } ) \indices{ ^ \mu _ \nu } \gamma ^ \nu 
	=- [ S ^{ \rho \sigma } , \gamma ^ \mu ] 
\] This can be shown to be true by observing that 
\[
 \left(  \eta ^{ \rho \mu }  \delta \indices{ ^ \sigma _ \nu }
  - \eta ^{ \sigma \mu } \delta \indices{ ^ \rho  _ \nu }  \right)  \gamma ^ \nu 
    = \eta ^{\rho \mu } \gamma ^{ \rho }  - \gamma ^{ \rho } \eta ^{ \sigma \mu } = 
    - [ S ^{ \rho \sigma } , \gamma ^{ \mu } ] 
\] 
\end{proof} 
\end{claim} 
Now, the upshot of doing all this 
work is that we can now write down 
an action (which by right should be Lorentz invariant). 
We know that $ \overline{\psi }  \gamma ^ \mu \psi $ is 
a Lorentz four vector, which implies that 
$ \overline{ \psi } \gamma ^ \mu \partial  _ \mu \psi $ 
is a Lorentz invariant quantity. We also already know that 
$ \overline{ \psi } \psi $  is also Lorentz invariant on its 
own. Finally, 
we can construct a Lorentz invariant action from these objects, 
by setting 
\[
	S = \int  d^ 4 x \, \overline{ \psi  } \left(  
	 \gamma ^ \mu \partial  _\mu  - m \right) \psi ( x) 
\] 
\subsection{The Dirac equation} 
Varying $ \overline{ \psi } $ independently 
in the Euler Lagrange equations gives us the Dirac equation.
Because there is no dependence on $ \dot{ \overline{ \psi } }  $, 
we have that our equation of motion is given by 
\[
 \frac{\partial  \mathcal{L } }{\partial  \overline{ \psi } }   = 0 
\] 
Specifically, this is the equation 
\[
 \left(  i \gamma ^ \mu \partial  _ \mu - m  \right)  \psi  =0 
\] This is first order, not second order like KG.
To save some ink, we write 
objects which are contracted with $ \gamma ^ \mu $ with 
a slash. For example, we write $ A _ \mu \gamma ^ \mu = \slashed{ A } $. In this notation, 
the Dirac equation reads  
 \[
	 \left(  i \slashed{\partial } - m  \right)  \psi = 0 
\] Note that each component of $ \psi $ solves
the KG equation. 
\begin{align*}
	( i \slashed{\partial} + m ) ( i \slashed {\partial} - m ) \psi &=  0  \\
	( \gamma ^ \mu \gamma ^ \nu \partial  _ \mu \partial  _ \nu + m ^ 2 ) \psi &=  0  \\ 
	( \frac{1}{2 } \left\{  \gamma ^ \mu , \gamma ^ \nu  \right\}  \partial  _ \mu \partial  _ \nu + m^ 2 ) \psi &  = 0 \\
	- ( \partial  ^ 2 + m ^ 2 ) \psi & = 0 
\end{align*}

\subsection{Chiral Spinors} 
Since $ S [ \Lambda ] $ is block diagonal in our 
Chiral representation, we say that it's reducible (since 
we can write the representation as a direct sum of 
two different representations). 
It decomposes into 2 irreducible representations. 
We write our spinor as 
\[
 \psi = \begin{pmatrix} u _ L \\ u _ R  \end{pmatrix} 
\] where $ u_L , u _ R $ are 2 $ \mathbb{ C } $ component objects
which we call Weyl, or chiral, spinors. 
These objects transform identically under rotations, so that 
\[
 u _{ L, R } \to e ^{ \phi \cdot  \sigma / 2 } u _{ L , R }
\] but under boosts, they transform in the opposite fashion. This 
can be seen by just multiplying the matrix with the vector 
as $ \psi \to S[ \Lambda ]_{ \text{ rot } } \psi $, which gives us 
\[
 u _{ L } \to e ^{ - \chi \cdot  \sigma / 2 } u_L  , \quad u _{ R } \to e ^{ \chi \cdot  \sigma / 2 } u _ R
\] Heuristically we write 
\[
	u _{ L } \sim ( \frac{1}{2 } , 0 )\quad u _{ R } \sim ( 0 , \frac{1}{2 } ) \in ( SU ( 2) , SU ( 2) ) 
\] we have that $ \psi $ is in $ ( \frac{1}{2 } , 0 ) \oplus ( 0 , \frac{1}{2 } ) $. 

\subsubsection{The Weyl equation} 
If we take our Dirac Lagrangian 
\[
	\mathcal{L}_D  = \overline{\psi} \left(i\slashed{\partial} - m\right)\psi 
\] which is equal to 
\[
	\dots = i u_L ^ \dagger \overline{\sigma} ^ \mu \partial  _ \mu u _ L + 
 i u _ R ^ \dagger  \sigma ^ \mu \partial  _ \mu u _{ R } - m \left(  
 u _ L ^ \dagger u _ R + u _ R ^ \dagger u _ L \right) 
\] where we define $ \sigma ^ \mu = ( I , \sigma ) , \overline{\sigma } ^ \mu = ( I , - \sigma ) $.
This is derived simply by writing out 
the components explicitly, then multiplying
by the required matrices. 
A massive fermion requires both $ u _L $ and $ u _ R $ , but 
the massless limit $ m =0 $ requires only $ u _ L $ or $ u _ R $. 
The mass term is mixing up left handed and right handed spinors. 
In the case  $ m = 0$, we get the following EL equations
\[
 i \sigma ^ \mu \partial  _ \mu u _ L = 0 , \quad i \overline{ \sigma } ^ \mu \partial  _ \mu u _ R  = 0 
\] These are called Weyl's equations. 
In classical particle mechanics, the number of degrees of freedom is given by 
half the dimension of phase space. in field theory, we need to do things different
and discuss the number of degrees of freedom per spacetime point. 
In field theory, we have a scalar $ \phi $ and $ \Pi_{ \phi } = \dot { \phi } $, so our 
real degrees of freedom is $ \frac{1}{2 } \times 2  = 1$. 
However, for a spinor $ \psi _{ \alpha } $ we have 8 degrees of freedom since 
it's complex. For our conjugate momenta, 
\[
 \Pi _{ \psi } = i \psi ^{ i } 
\] which doesn't add any degrees of freedom. This means 
that our real degrees of freedom amounts to $\frac{1}{2 } \times 8  = 4$. 
We have 2 for the spin up and spin down particle, and 2 for the 
spin up and spin down anti-particle.

\subsection{Dirac Field Bilinears}  
Throughout this section, 
we've only been using a specific representation 
of $ S $ (which we dubbed the chiral representation). 
However, note that in a different basis,  $ S [ \Lambda ] $ is not necessarily block diagonal! We can transform to a different 
basis by applying the map  
\[
 \gamma ^ \mu \to U \gamma ^ \mu U ^{ - 1 }, \psi \to U \psi 
\] We use 
\[
 \gamma ^{ 5 }  : = i \gamma ^ 0 \gamma ^ 1 \gamma ^ 2 \gamma ^ 3 
\]  to define Weyl spinors in a basis independent way. 
This object has very nice properties which relate 
agonistically with $ \gamma ^ \mu $ for $ \mu  = 0 , 1, 2,3 $. 
\begin{claim}{(Commutation relations for $ \gamma ^ 5 $)}
	The $ \gamma ^ 5$ matrix satisfies the 
	relations 
\[
	\left\{  \gamma ^ \mu , \gamma ^ 5 \right\}   = 0 , \quad  \left(  \gamma ^ 5  \right)  ^ 2 = I \] 

\begin{proof}
	To prove the first 
relation, it's instructive to set $ \mu  =0  $ just to 
see what's going on. We have that the anti-commutator is 
\begin{align*}
	\left\{ \gamma ^ \mu , \gamma ^ 5  \right\}  &=  i ( 
	\gamma ^ 0 \gamma ^ 0 \gamma ^ 1 \gamma ^ 2 \gamma ^ 3  + 
	\gamma ^ 0 \gamma ^ 1 \gamma ^ 2 \gamma ^ 3 \gamma ^ 0 ) \\
						     &=  i ( \gamma ^ 1 \gamma ^ 2 \gamma ^ 3 - \gamma ^ 1 \gamma ^ 0 \gamma ^ 2 \gamma ^ 3 \gamma ^ 0 )  \\
						     &=  i ( \gamma ^ 1 \gamma ^ 2 \gamma ^ 3 + \gamma ^ 1 \gamma ^ 2 \gamma ^ 0 \gamma ^ 3 \gamma ^ 0 )  \\
						     &=  i ( \gamma ^ 1 \gamma ^ 2 \gamma ^ 3 - \gamma ^ 1 \gamma ^ 2 \gamma ^ 3 ( \gamma ^ 0 ) ^ 2 )   = 0
\end{align*}
We've used liberally here the fact that 
\[
 \gamma ^ \mu \gamma ^ \nu = - \gamma ^ \nu \gamma ^ \mu \text{ when } \nu \neq \mu 
\] For the second relation, we use this fact as well. 
Note that when we pass a $ \gamma ^ j  $ matrix
past $ n $ matrices $ \gamma ^ i $ where $ i $ never equals
$ j  $, we pick up a factor of $ ( - 1) ^ n $. 
Thus, note that $ ( \gamma ^ 5 ) ^ 2$ is just 
\begin{align*}
	( \gamma ^ 5) ^ 2 & = i^ 2 \gamma ^ 0 \gamma ^ 1 \gamma ^ 2 \gamma ^ 3 \gamma ^ 0 \gamma ^ 1 \gamma ^ 2 \gamma ^ 3 \\
			  &=  ( - 1) ^ 4 \gamma ^ 1 \gamma ^ 2 \gamma ^ 3 \gamma ^ 1 \gamma ^ 2 \gamma ^ 3  \\
			  &=  ( - 1) ^ 3 \gamma ^ 2 \gamma ^ 3 \gamma ^ 2 \gamma ^ 3  \\
			  &=  ( - 1) ^ 4 I  = I 
\end{align*} 
\end{proof}
\end{claim}
Now, we can write out $ \gamma ^ 5 $ explicitly 
depending on the representation we choose. 
We check that in our chiral representation, 
\[
	\gamma ^ 5 = \begin{pmatrix}  - I & 0 \\ 0 & I  \end{pmatrix} 
\] We define projection operators in the spirit of 
finding a way to extract the left-handed and right-handed spinors 
from the entire spinor. We define  
\[
	P _ L = \frac{1}{2 } ( I _ 4 - \gamma ^ 5 ) , \quad P _ R = \frac{1}{2 } ( 1 + \gamma ^ 5 ) 
\] These obey the properties that come with what 
we usually know as a projection operator, namely, 
that projecting twice is the same as projecting once, 
and that $ P_ L $ and $ P _ R $ project onto 'orthogonal' 
spaces 
\[
 P_ L ^ 2 = P_L , \quad P _ R ^ 2 = P _ R , \quad P _ L P _ R = 0 
\] We then define 'left handed' and 'right handed' spinors
$ \psi _ L $ and $ \psi_ R $, 
which are the projection spinors with these operators 
\[
 \psi _ L = P_L \psi , \quad \psi _ R = P _ R \psi 
\] for left handed and right handed spinors respectively. 
Under a Lorentz transformation, 
\[
	\overline{ \psi } ( x) \gamma ^ 5 \psi ( x) \to \overline{ \psi } ( \Lambda ^{ - 1} x ) 
	S [ \Lambda ] ^{ - 1} \gamma ^ 5 S [ \Lambda ] \psi 
\] Checking that $ [ S_{ \mu\nu } , \gamma ^ 5 ] $, we have that 
this is 
\[
	\dots = \overline{ \psi } ( \Lambda ^{ -1 } x ) \gamma ^ 5 \psi ( \Lambda ^{ - 1} x ) 
\] This is called a pseudoscalar since it transforms under 
parity transformation. We can also define the axial vector, $ \overline{ \psi } \gamma ^ 5 \gamma ^ \mu 
\psi $ which transforms in the same way. 

\subsection{Parity} 
We'll now explore how spinors transform under discrete 
transformations. We'll look at a specific class of transformations 
which flips either our time coordinate or our spatial coordinate. 
The transformations are called parity transformations. 
Ultimately, we'll find that 
$ \psi _ L $ and $ \psi _ R $ are related by a parity trasnformation. 
In the Lorentz group $ O ( 1, 3 ) $, we've 
dealt with boosts and rotations, which are a continuous 
group of transformations. However, there are other discrete Lorentz transformations 
which are not part of the Lorentz group's component connected to the identity. 
These transformations are  
\[
 \text{Time Reversal } T : x ^ 0 \to  - x ^ 0 , \quad x ^ i \to x ^ i 
\] The other one is parity. 
\[
 \text{Parity } P : x ^ 0 \to x ^ 0 , \quad x ^ i \to  - x ^ i 
\] Other transformations like  $x ^ 2 \to - x ^ 2 $
are connected to the identity, because something like this can 
be written as a rotation. 
Under $ P $, rotations don't change sign, but 
boosts do. 
\[
 u_{ L , R } \to e^{ \phi \cdot  \sigma / 2} u_{ L , R }, \quad 
 u _{ L, R } \to e^{  \mp \chi \cdot  \sigma / 2} u_{ L, R }
\] This is equivalent to writing, 
in our new notations with $ \psi _{ L , R }  = \frac{1}{2 } ( I \mp \gamma ^ 5 )  \psi $, 
that  $ P $ exchanges LH and RH spinors. 
\[ 
	P : \,  \psi_{ L , R } ( \vec{x} , t ) \to \psi_{ R, L } ( - \vec{x} ,  t) 
\] Since our parity transformation switches around 
the left and right handed components of the spinor, it is 
completely equivalent to applying the $ \gamma ^ 0 $ matrix to the spinor, 
(but also we have to remember to change the $ \vec{x} $ sign) 
\[
	P : \,  \psi ( \vec{x}, t ) \to \gamma ^ 0 \psi ( - \vec{x}, t ) 
\]
How do our terms in our $\mathcal{ L } $ density change? We look at each field one 
by one 
Firstly, 
\[
	\overline{ \psi } \psi ( \vec{x}, t) \to \overline{ \psi } \psi ( - \vec{x}, t ) 
\] Also, 
\[
	\overline{\psi } \gamma ^ \mu \psi ( \vec{x}, t ) \to \begin{cases}
		\mu = 0 & \overline{ \psi } \gamma ^ 0 \psi ( - \vec{x}, t ) \\
		\mu = i & \overline{ \psi } \gamma ^ 0 \gamma ^ i \gamma ^ 0 \psi ( -\vec{x}, t ) 
		= - \overline{ \psi } \gamma ^ i \psi ( - \vec{x}, t ) 
	\end{cases}
\] On the other hand, we have that under parity, 
\[
	\overline{ \psi } \gamma ^ 5 \gamma ( \vec{x}, t ) \to \overline{ \psi } \gamma ^ 0 \gamma ^ 5 \gamma ^ 0 
	\psi ( - \vec{x}, t ) = - \overline{ \psi } \gamma ^ 5 \psi 
\] In addition, 
\[
 \overline{ \psi } \gamma ^ 5 \gamma ^ \mu \psi \to \overline{ \psi } \gamma ^ 0 \gamma ^ 5 \gamma ^ 0 \psi 
  = \begin{cases}
	  \mu = 0 &  - \overline{ \psi } \gamma ^5 \gamma ^ 0 \psi \\
	  \mu = i & + \overline{ \psi } \gamma ^5 \gamma ^ i \psi 
  \end{cases}
\] To summarise, 
we write down the number of components as
\begin{equation}
\begin{aligned}
	\overline{ \psi } \psi &1\\
	\overline{ \psi } \gamma ^ \mu \psi &4\\
	\overline{\psi } S^{ \mu \nu } & 6\\
	\overline{\psi } \gamma ^  5 \psi & 1\\
	\overline{\psi } \gamma ^ 5 \gamma ^ \mu \psi & 4\\ 
\end{aligned} 
\end{equation}this has a total of $ 16 $ components. 
Now we add extra terms to $ \mathcal{ L } $ involving $ \gamma ^ 5 $. This 
can break $ P $ invariance. For example, 
\begin{equation*}
		\mathcal{ L }  = g  W_ \mu \overline{\psi} \frac{ \gamma ^ \mu ( 1 - \gamma ^ 5 ) \psi }{ 2}
\end{equation*}
This represents a $ W $ boson vector field coupling only to LH $\psi$ ' s. 
If $\mathcal{ L }$ treats $\psi _ L$ and $\psi _ R$ equally it is called vector like. 
If they appear differently they're called chiral. 

\subsection{Symmetries and Currents of Spinors} 
If we have a space time translation $x ^ \mu \to x ^ \mu - \epsilon ^ \mu $, 
then  $ \Delta \psi  = \epsilon ^ \mu \partial  _ \mu \psi $. This 
gives us the stress energy tensor 
\[
 T ^{ \mu \nu }  = i \overline{ \psi } \gamma ^ \mu \partial  ^ \nu \psi - \eta ^{ \mu \nu } \mathcal{ L }
\] We get a conserved current when equations of motion 
are obeyed, do we can impose them in $ T ^{ \mu \nu } $. 
\[
	\mathcal{ L } _ D = \overline{ \psi } \left(  
	i \slashed{\partial} - m \right)  \psi  = 0 
\] This implies $ \mathcal{ L }  = 0 $ in $ T ^{ \mu \nu } $ thus 
\[
 T ^{ \mu \nu } = i \overline{ \psi } \gamma ^ \mu \partial  ^ \nu \psi 
\] 
\[
	S = \int  d^ 4 x \frac{1}{ 2 } \overline{ \psi } \left(  i \slashed{\partial}^{ \rightarrow}
	- m \right)  \psi + \frac{1}{2 } \overline{ \psi } ( - i \slashed{\partial} ^{ \leftarrow } - m ) \psi 
\] This is 
\[
	\int d ^ 4 x \frac{1}{2 } \overline{ \psi } ( i \slashed {\partial} ^{ \leftrightarrow} - 2m ) \psi 
\]  where $\slashed{\psi} ^{ \leftrightarrow }  = \slashed{\partial} ^ \rightarrow - \slashed{\partial} ^ 
\leftarrow $. So
\[
	T ^{ \mu \nu }  = \frac{ i }{ 2} \overline{ \psi } ( \partial  ^ \mu \partial  ^ \nu - \partial  ^ \nu \partial  ^ \mu ) \psi 
\]
\subsection{Lorentz transformations} 
We know how a spinor is supposed to transform. 
We have that infinitesimally, 
\[
	\psi \to S [ \Lambda ] \indices{ ^ \alpha _ \beta }  \psi ^ \beta  ( x ^ \mu - \omega  \indices{ ^ \mu _ \nu} x ^ \nu )  
\] where we have that $ \omega  $ is generated as 
\[
\omega \indices{ ^ \mu _ \nu } =  \frac{1}{2 } \Omega_{ \rho \sigma } ( M ^{ \rho \sigma } )\indices{ ^ \mu _ \nu }  
\] Since we defined earlier that 
\[
( M ^{ \rho \sigma } ) \indices{ ^ \mu _ \nu } = \eta ^{ \rho \mu } \delta \indices{ ^ \sigma _ \nu }
 - \eta ^{ \sigma \mu } \delta \indices{ ^ \rho _ \nu } 
\] This implies that $ \omega ^{ \mu \nu } = \Omega ^{ \mu \nu } $. 
Our infinitesimal change in our spinor is 
\[
\delta \psi ^ \alpha    = i \omega ^{ \mu \nu }  \left[ 
x ^ \nu \partial  _ \mu \psi ^ \alpha  - \frac{1}{2 }  ( S ^{ \rho \sigma } ) \indices{ ^ \alpha 
_ \beta } \psi ^ \beta  \right] 
\] Our corresponding conjugate change is 
\[
\delta \overline{ \psi } _ \alpha  = 
- \omega ^{ \mu \nu } \left[  x _ \nu \partial  _ \mu \overline{\psi }_ \alpha 
+ \frac{1}{2 } \overline{\psi } ( S_{ \mu \nu } ) \indices{ ^ \beta _ \alpha }  \right]  
\] where the last term comes from $ \overline{\psi }_ \beta  S [ \Lambda ] ^{ - 1} $. 
Our conserved current 
\[
( J ^ \mu ) ^{ \rho \sigma }  = x ^ \rho T ^{ \mu \sigma } - x ^ \sigma T ^{ \mu \rho }  - i 
\overline{ \psi } \gamma ^ \mu S ^{ \rho \sigma } \psi 
\] THe first term looks like a real scalar stress energy tensor. 
The new piece will give us spin $ \frac{1}{2 } $ after we quantise the
field. 
For example, looking at the last term $ ( J ^ 0 ) ^{ ij }  =  -i \overline{ \psi } \gamma ^ 0 S ^{ ij } \psi$. 
Using the representation in Pauli matrices, 
we get that the above is equal to 
\[
( J ^ 0 ) ^{ ij }  = \frac{1}{2 } \epsilon ^{ ijk  } \psi ^ \dagger 
\begin{pmatrix}  \sigma ^ k & 0 \\ 0 & \sigma ^ k  \end{pmatrix} \psi  
\] 
\subsection{Internal vector symmetry} 
We get that under the phase rotation 
\[
\psi \to e^{ i \alpha } \psi \implies \delta \psi  = i \sigma \psi 
\] This gives us a conserved quantitiy 
\[
j ^{ \mu _ \nu }  = \overline{ \psi } \gamma ^ \mu \psi \text{ a vector current }
\] This gives us the conserved charge 
\[
Q = \int d^ 3 x \, \overline{\psi } \gamma^ 0 \psi  = \int d ^ 3 x \, \psi ^ \dagger \psi 
\]  in other words, an electric charge. 

\subsection{Axial Symmetry} 
For massless spinors in the $ m = 0 $ limit, 
we transform 
\[
\psi _{ \alpha } \to \left(  e ^{ i \alpha \gamma ^ 5 }  \right) \indices{ _ \alpha ^ \beta }
\psi _ \beta 
\] This rotates left handed and right handed spinors in the 
opposite direction. This leads to 
the conserved axial vector current 
\[
j _ A ^ \mu  = \overline{ \psi  } \gamma ^ \mu \gamma ^ 5 \psi 
\] 

\subsection{Plane-Wave Solutions}
Now, let's actually start writing down solutions to the 
Dirac equation. 
We want to solve $ ( i \slashed{\partial}  -m ) \psi = 0 $. 
Even though this is a first order differential equation 
and typically we wouldn't think of using wave solutions, 
we try this anyway and make the ansatz 
\[
\psi = u_{ \vec{p} } e ^{ - i p \cdot  x }
\] where $ u _{ \vec{p} }  $ is a constant spinor 
depending on $ \vec{p} $ . 
Substituting this in, using the chiral representation
of $ \gamma ^ \mu $, we have that 
\[
( \gamma ^ \mu p _ \mu - m ) u _{ \vec{p} }  = 0  = 
\begin{pmatrix}  - m & p _ \mu \sigma ^ \mu \\ p _ \mu \overline{ \sigma } ^ \mu 
& - m  \end{pmatrix} u _{ \vec{p} }  =0 
\] Here, we are again using 
our notation that $ \sigma^ \mu = ( 1 , \sigma ) $, and $ \overline{ \sigma } ^ \mu  = ( 1 , - \sigma ) $. 

\begin{claim}{(The Spinor plane-wave solution)}
	
\end{claim}
Our claim is that we have a solution 
given by 
\[
u _{ \vec{p} }  = \begin{pmatrix}  \sqrt{ p \cdot  \sigma }  \xi \\
\sqrt{ p \cdot  \overline{ \sigma } }  \xi \end{pmatrix} 
\] for any two component spinor $ \xi $ such that $ \xi ^ \dagger \xi = 1 $. 
To prove this, we let  $ u _ { \vec{p} } = ( u _ 1, u _ 2 ) $, and 
substitute into the above such that 
\[
( p \cdot  \sigma ) u _ 2 = m u _1 , \quad ( p \cdot  \overline{ \sigma } ) u _ 1 
 = m u _ 2 
\] Either of these equations implies the other, since 
\begin{align*}
( p \cdot  \sigma ) ( p \cdot  \overline{ \sigma } ) &=  
p_0^ 2 - p _ i p _ j \sigma ^ i \sigma ^ j \\
&=   p_0 ^ 2  - p _ i p_ j \frac{1}{2 } \left\{  \sigma ^ i , \sigma ^ j  \right\}  \\
&=  p _\mu p ^ \mu   = m ^ 2  
\end{align*}

Now, we try the ansatz $ u _ 1 = ( p \cdot  \sigma ) \xi $ for the 
two component spinor $ \xi ' $. 
Substituting this into the above, 
we have that 
\[
u _ 2 = \frac{1}{m} ( p \cdot  \overline{ \sigma } ) ( p \cdot  \sigma ) \xi = m \xi ' 
\] Hence, 
any vector of the form $ u _{ \vec{p} }  = A ( ( p \cdot  \sigma ) \xi ' , m \xi ' ) $. 
To make this look more symmetric, choose $ A = \frac{1}{m } $  and 
$ \xi '  = \sqrt{ p \cdot  \overline{ \sigma } }  \xi $  with $ \xi $ const. 
Then, we have that 
\[
u _{ \vec{p} }  = \begin{pmatrix}  \sqrt{ p \cdot  \sigma }  \xi \\
\sqrt{ p \cdot  \overline{ \sigma }   } \xi  \end{pmatrix}  
\]  
Examples. Let's take $ \mathbb{p } = 0$, then we have 
that our solution looks like 
\[
u_{ \vec{p} = 0 }  = \sqrt{ m }  \begin{pmatrix}  \xi \\ \xi  \end{pmatrix} \text{ for any } \xi 
\] Under spatial rotations, we have that 
\[
\xi \to e ^{ i \sigma \cdot  \phi  / 2 } \xi 
\] After quantisation, $ \xi $ describes the spin of the spinor. 
For example, we have that 
\[
\xi = \begin{pmatrix}  1 \\ 0  \end{pmatrix}  \text{ represents spin up along } x ^ 3 \text{ axis }
\] Let's consider a particle boosted along the $ x ^ 3 $ direction. 
Now, the momentum is not going to be zero. 
Now what we have is that 
\[
u_{ \vec{p} }  = \begin{pmatrix}  \sqrt{ E - p ^ 3 }  \begin{pmatrix}  1 \\ 0  \end{pmatrix} 
\\ \sqrt{ E + p_3 }  \begin{pmatrix}  1 \\ 0  \end{pmatrix}  \end{pmatrix} 
\] which, in the $ m \to 0 $ or $ E \to p^ 3 $ limit, converges to 
\[
\sqrt{ 2 E }  \begin{pmatrix}  0 \\ 0 \\ 1 \\ 0  \end{pmatrix} 
\] Now, with instead $ \xi  = \begin{pmatrix}  0 \\ 1  \end{pmatrix}  $, 
when we act on this we instead get that 
\[
u _{ \vec{p} }  = \begin{pmatrix}  \sqrt{ E + p^ 3 }  \begin{pmatrix}  0 \\ 1   \end{pmatrix}  \\
\sqrt{ E - p ^ 3 }  \begin{pmatrix}  0 \\ 1  \end{pmatrix}  \end{pmatrix} 
\] which tends to 
\[
\sqrt{2 E }  \begin{pmatrix}  0 \\ 1 \\ 0 \\ 0  \end{pmatrix} 
\]
\subsection{Helicity}

The Helicity operator projects angular momentum 
along the direction of motion. 
If we define 
\[
h = \hat{\vec{p} } \cdot  \vec{s}  = \frac{1}{2 } \hat{ p } _ k 
\begin{pmatrix}  \sigma ^ k & 0 \\ 0 & \sigma ^ k  \end{pmatrix} 
\] A massless spin up operator has $ h = \frac{1}{2 } $, and a massless spin down 
particle has $ h = - \frac{1}{2 } $. 

\subsubsection{Negative frequency solutions} 
If we define $ \psi  = v _{ \vec{p} } e ^{ i p \cdot  x } $. 
The Dirac equation then gives 
\[
v_{ \vec{p} }  = \begin{pmatrix}  \sqrt{ p \cdot  \sigma }  \eta \\
- \sqrt{ p \cdot  \overline{ \sigma }  } \eta  \end{pmatrix} 
\] where we have as well that $ \eta ^ \dagger \eta  = 1 $ and $ \eta $ 
is a constant 2 -spinor. 

\subsection{A canonical Basis for normalised Spinors} 
In what follows, we'll construct a basis 
for the two component spinors in a way such that 
some nice identities will hold when we're doing calculations. 
It will be convenient for us to choose the basis 
\[
	\xi ^ 1  = \eta ^ 1 = \begin{pmatrix}  1\\ 0  \end{pmatrix} , 
	\quad \xi ^ 2 = \eta ^ 2 = \begin{pmatrix}  0 \\ 1  \end{pmatrix} 
\] This gives us the sweet property that these 
things are normalised, so 
\[
 ( \xi ^ r ) ^ \dagger \xi ^  s =  \delta ^{ r s }, \quad 
 ( \eta ^ r ) ^ \dagger \eta ^ s = \delta ^{ r s }
\] In this basis, we then get 
two independent solutions for both $ u(\vec{p} ) $ 
and $ v ( \vec{p} ) $, which 
we will index as $ u ^ s ( \vec{p} ) $ and $ v ^ s ( \vec{p}  ) $. 
We get these solutions by just plugging in $ \xi ^ s $ or $ \eta ^ s $
into the respective spinors -  these linearly independent solutions are given by 
\[
	u ^ s (\vec{p} )  = \begin{pmatrix}  
	\sqrt{ p \cdot  \sigma }  \xi \\ \sqrt{ p \cdot  \overline{ \sigma } }  
\xi \end{pmatrix}, \quad v ^s ( \vec{p} ) = \begin{pmatrix}  
\sqrt{ p \cdot  \sigma }  \eta \\  - \sqrt{ p \cdot  \overline{\sigma}}  \eta \end{pmatrix}
\]  The neat thing is that, now with our 
choice of basis for the two component spinors, 
we get that $ u ^ s $ and $ v ^ s $ are themselves 
normalised depending on how we contract them. 
We can contract them in two ways. 
The first thing that we can do is take the hermitian 
dot product, so something like $ u^{ s \dagger  } u ^ r $. 
Secondly, we could've also equally defined $ \overline{ u } ^ s  = u^{ s \dagger } \gamma ^ 0 $, and we also check the normalisation 
$ \overline{ u } ^ s u ^ r $. It turns out, both 
will be useful. 

Let's do the first contraction. 
\begin{align*}
	u^{ s \dagger } u ^ r &=  \begin{pmatrix}  \xi^{ s \dagger } \sqrt{ p \cdot  \sigma } & \xi ^{ s \dagger } \sqrt{ p \cdot  \overline{ \sigma } }   \end{pmatrix} 
	\begin{pmatrix}   \sqrt{ p \cdot  \sigma }  \xi ^{ r  } \\
	\sqrt{ p \cdot  \overline{ \sigma } }  \xi ^{ r } \end{pmatrix} \\
	&=  \xi ^{ s \dagger } ( p \cdot  \sigma ) \xi ^{ r } 
	+ \xi ^{ s \dagger } ( p \cdot  \overline{ \sigma } ) \xi ^{ r } \\
	&=  2p_0 \xi ^{ s \dagger } \xi ^ r  \\ 
	&=  2 p_0 \delta ^{ rs  } 
\end{align*}
We can see clearly that this is 
not a necessarily Lorentz invariant 
contraction since $ p_0 $ changes under boosts. 
Now, lets do the other type of contraction which 
is indeed Lorentz invariant! We have that 
\begin{align*}
	\overline{u }^{ s \dagger } u^ r &= \begin{pmatrix}  
		\xi ^{ s \dagger } \sqrt{ p \cdot  \sigma }  & 
\xi ^{ s \dagger } \sqrt{ p \cdot  \overline{ \sigma } }  \end{pmatrix} 
\begin{pmatrix}  0 & I \\ I & 0  \end{pmatrix}  \begin{pmatrix} 
\sqrt{ p \cdot  \sigma }  \xi ^ r \\ \sqrt{ p \cdot  \overline{ \sigma } }  \xi ^ r 
\end{pmatrix} \\
		   &=  2 \xi ^{ s \dagger }  \sqrt{ ( p \cdot  \sigma   ) 
		   ( p \cdot  \overline{ \sigma   } ) }   \\
		   &=  2m \delta ^{ r s }  
\end{align*}
In the above we've used the relation we proved 
earlier that $ ( p \cdot  \sigma ) ( p \cdot  \overline{ \sigma } )  = m ^ 2 $. 

Now, what about the 'negative frequency' spinors 
$ v ^ s ( \vec{p} ) $? 
Well, it is easy to also check that these spinors
obey similar normalisation conditions, up to 
a minus sign. 
\[
 v ^{ s \dagger} v ^ r = 2 p_0 \delta ^{ rs } , 
 \quad \overline{ v }^{ s } v ^{ r }  =  - 2m \delta ^{ rs }
\] 
Finally, something we will also have to use 
frequently in the coming section is 
the outer product 

\subsection{Quantising the Dirac field}
In conclusion, we've found two 
plane wave solutions of the Dirac equation, which, for 
cleanliness, we will write as 
$ u ^ s (\vec{p} )  $ and $ v ^s ( \vec{p} )$. 
Recall, the $ s $ index let's us choose between 
$ \xi ^ 1  $  or $ \xi ^ 2 $ in the definition $ u^ s ( \vec{p} )$, 
or $ \eta ^ 1 $ or $ \eta ^ 2 $ in the definition of $  v ^ s ( \vec{p} ) $.  
When we take these solutions and write them out 
in Fourier modes, we have that 
\begin{align*}
\psi ( \vec{x} ) &=  \sum _{ i = 1 } ^ 2 
\int \frac{ d^ 3 p }{ ( 2 \pi ) ^ 3 \sqrt{ 2 E_{ \vec{p} } }  } \left[  
b_{ \vec{p} } ^ s u _{ \vec{p} } ^ s e ^{ i \vec{p} \cdot  \vec{x} } 
+ c_{ \vec{p} } ^{ s \dagger } v_{ \vec{p} } ^ s e ^{ - i \vec{p} \cdot  \vec{x}}\right] \\
\psi ^ \dagger ( \vec{x} ) &=  \sum _{ i = 1 } ^ 2 
\int \frac{ d ^ 3 p }{ ( 2 \pi ) ^ 3 \sqrt{ 2 E _{ \vec{p} } }  } \left[  
	b_{ \vec{p} } ^ { s \dagger }  u_{\vec{p} } ^{ s \dagger } e ^{ - \vec{p} \cdot  \vec{x} } + 
c_{ \vec{p} } ^ s v_{ \vec{p}} ^{ s \dagger } e ^{ i \vec{p} \cdot  \vec{x} } \right] \\ 
\end{align*}
Analogous to our previous case in 
nucleon anti-nucleon quantisation, we have that
$ b_{\vec{p} }^ s $  and $ b_{\vec{p} } ^{ s \dagger } $ 
represent the annihilation and creation 
operators for $ u( \vec{p} ) ^ s $, and that $  c_{ \vec{p} } ^ s , 
c_{ \vec{p} } ^{ s \dagger } $  represent annihilation 
and creation for $ v ^ s ( \vec{p} ) $. 
In spinor quantisation, it turns out that we require 
anti commutations relations instead of commutation relations, with 
\[
\left\{  A, B  \right\}   = AB + B A 
\] We'll go into more detail about why we 
need anti-commutation relations later, 
but for now just acknowledge that this is due to the fact that 
if we were to impose commutation relations, 
we'll come across the problem of unbounded negative energy. 
The specific anti-commutation relations are:  
\[
\left\{  \psi _ \alpha ( \vec{x} ) , \psi _ \beta ( \vec{y} )  \right\}   = 0 
= \left\{  \psi _{ \alpha } ^ \dagger ( \vec{x} ) , \psi _{ \beta } ^ \dagger 
( \vec{y} ) \right\}  , \quad \left\{  \psi _ \alpha ( \vec{x} ) , 
\psi _{ \beta }^\dagger ( \vec{y} ) \right\}  = \delta _{ \alpha \beta } \delta ^ 3 ( \vec{x} - \vec{y} ) 
\]
In the same vein as the case we 
had with scalar fields and nucleons, 
we construct a set of commutation relations on 
the creation and annihilation operators
which are logically equivalent to our anti-commutation 
relations. We can prove that these are equivalent to 
\[
\left\{  b_{ \vec{q} } ^ r , b _{ \vec{q} } ^{ s \dagger }  \right\}  = 
( 2 \pi ) ^ 3 \delta ^ 3 ( \vec{p} - \vec{q} ) \delta ^{ rs }  = \left\{  c_{ \vec{p} } ^ r , 
c _{ \vec{q} } ^{ s \dagger }\right\} 
\] We do the Legendre transform to 
get our Hamiltonian density. It's easy to 
see that our conjugate momenta $ \pi $ is 
$ \pi = i \psi^{ \dagger } $. This means that 
our Hamiltonian looks like 
\begin{align*}
\mathcal{ H } &=  \pi \dot{ \psi }  - \mathcal{ L }   \\
&=  i \psi ^ \dagger \dot{ \psi }  - i \overline{ \psi } \gamma ^ 0 \partial  _ 0 
\psi - i \overline{ \psi } \gamma ^ i \partial  _ i \psi + m \overline{ \psi } \psi \\
&=  \overline{ \psi }( - i \gamma ^ i \partial  _ i + m ) \psi -  
\end{align*}
If we plug in $ \psi , \overline{ \psi } $ from the above, 
we use anti commutation relations and some reuslts on inner 
products of spinors to get that 
\[
u _{ \vec{p} } ^{ r \dagger } u _{ \vec{p} } ^ s  = 
v_{\vec{p} } ^{ r \dagger } v _{ \vec{p} } ^ s  = 2 p_0 \delta ^{ rs } , \quad 
u_{\vec{p} } ^{  r\dagger } v_{\vec{p} } ^ s = 0 = v_{ \vec{p} } ^{ r \dagger  } u_{ \vec{p} } ^ s 
\] 
This means that for our Hamiltonian, 
we fortunately get a positive definite quantity
\[
H = \int \frac{ d^ 3 p }{ ( 2 \pi ) ^ 3 } E _ p \sum_{ s = 1 } ^ 2 
\left(  b_{ \vec{p} } ^{ s \dagger  } b _{ \vec{p} } ^ s 
+ c _{ \vec{p} } ^{ s \dagger } c _{ \vec{p} } ^ s \right) 
\] If we used commutation relations, we get a minus sign which means that we 
have unboudned lower energy and therefore unstable theory.

\subsection{The Dirac Hole Interpretation}
We can write the Dirac equation as 
\[
i \frac{\partial  \psi }{\partial  t }   = 
( - i \alpha \cdot  \nabla + m \beta ) \psi , \quad \alpha =  - \gamma ^ 0 \gamma , \beta = \gamma ^ 0 
\] The term in the brackets is considered as 
the 1-particle Hamiltonian $ \hat{ H } $. 
This has positive and negative energy solutions. 
There's no consistent way of realising the 
Dirac equation to one particle states.

\subsection{Fermi-Dirac Statistics} 
We expanded our spinor field with operators $ b  $ and $ c $, which 
had a spin index. These operators annihilate the vacuum 
\[
b_{ \vec{p} } ^ s \ket{ 0 }  = 0 =  c_{ \vec{p} } ^ s \ket{ 0 } 
\] Recall that these operators obey anti-commutation relations. 
We can check that these commute with the Hamiltonian as follows 
\begin{align*}
[ H , b_{ \vec{p} } ^{ r \dagger } ] &=  E _p b _{ \vec{p} } ^{ r \dagger }  \\ 
\left[  H , b_{ \vec{p} } ^{ r  }  \right]  &=   - E _ p b_{ \vec{p} } ^ r  \\ 
\end{align*}
If we index this state, for example, as $ \ket{ \vec{p} _ 1, r _ 1 } : = b_{ \vec{p} _ 1 } ^{ r _ 1 \dagger } \ket{ 0 }  $. 
Then the anti-commutation relations give us a sign change when 
we swap two things around. In particular, 
we have that 
\[
\ket{ \vec{p} _ 1 , r _ 1 ; \vec{p} _ 2 , r _ 2 } =  - \ket{ \vec{p} _ 2 , r _ 2 ; \vec{p} _ 1 , r _ 1 } 
\]   

\subsubsection{Going into the Heisenberg picture}
To study propagators in the theory, 
we make the operators $ \psi $ and $ \overline{ \psi } $ time 
dependent by moving into the Heisenberg picture. 
Now, we have a time dependent operator $ \psi ( x) $ satisfying 
$ \frac{\partial  \psi }{\partial  t }  = i \left[  H , \psi  \right]   $ 
which is solved by the Heisenberg picture expansion 
\[
\psi _ \alpha ( x ) = \sum_{ s = 1 } ^ 2 \int \frac{ d ^ 3 p }{ ( 2 \pi ) ^ 3 \sqrt{ 2 E _ p }  } \, 
\left(  b^ s _{\vec{p} } u_{ \vec{p}_ \alpha } ^s e^{ - i p \cdot  x } 
+ c _{\vec{p} } ^{ s \dagger } v_{\vec{p} _ \alpha } ^ s e ^{ i p \cdot  x } \right)  
\] with an analogous expression for $ \psi_ \alpha ^ \dagger $. Honestly, 
the only thing we're doing here is 
attaching a time dependence to the exponential 
previously. Now, we are in a 
position to define the Feynman propagaator 
but in the case of these fermionic fields. 
We then define, in analogy with $ \Delta ( x -y )  = \left[  \phi ( x) , \phi ( y )  \right]  $, 
that for $ \mathbb{ R }$ scalars, 
\[
i S_{ \alpha \beta } ( x  - y )  = \left\{  \psi _ \alpha ( x ) , \overline{ \psi } _ \beta ( y )  \right\} 
\] where $ S _{ \alpha \beta } $ is a four by four matrix. 
For brevity, we'll now drop the $ \alpha , \beta $ for brevity. 
Substituting in our results from previously, 
we have that 
\[
i S ( x - y ) = \sum_{ r, s } \int \frac{ d ^ 3 p d ^ 3 q  }{ ( 2 \pi ) ^ 6 \sqrt{ 4 E_ p E _ q }  } 
\left[  \left\{  b_{ \vec{p} } ^ s , b_{ \vec{q} } ^{ r \dagger }  \right\} e ^{  - i ( p \cdot  x  - q \cdot  y ) }
u_{ \vec{p} } ^ s \overline{u } _{ \vec{q} } ^ r + 
\left\{  c_{\vec{p} } ^{ s \dagger } , c _{\vec{q} } ^{ r } \right\}  v _{ \vec{p} } ^ s 
\overline{ v } _{\vec{q} } ^{ r } e ^{ i ( p \cdot  x - q \cdot  y ) } \right] 
\]
The anti commutators in this expression yield 
delta functions. So, this simplifies to the expression 
\[
i S_{ \alpha \beta } ( x -y )  = \int \frac{d ^ 3 p }{ ( 2 \pi ) ^ e 2 E_ p } 
\left[  \sum _s u_{ \vec{p} _ \alpha } ^ s \overline{ u } _{ \vec{p} _ \beta } ^ s 
e ^{ - i p \cdot  ( x - y ) } + \sum _ s v_{ \vec{p} _ \alpha } ^ s 
\overline{ v } _{ \vec{p} _ \beta } ^ s e ^{ i p \cdot  ( x  -y ) }\right] 
\] Earlier we showed that the outer products 
of the spinors $ u $ and $ v $ summed 
over the spin indices give $\slashed{p} + m $ and $ \slashed{p} - m $. 
This thing then simplifies to 
\[
= \left(  i \slashed{\partial } _ x + m   \right)  D ( x - y )  - \left(  i 
\slashed{\partial} _ x + m \right)  D ( y - x )  = \left(  i \slashed{\partial} _ x 
+ m \right)  \left(  D ( x - y ) - D ( y  - x)  \right) 
\] 
Note that for $ ( x - y ) ^ 2 < 0 $, we have that $ D ( x - y ) - D ( y - x) =0  $. 
We now have $ \left\{  \psi _ \alpha ( x) , \overline{ \psi  }_ \beta ( y )  \right\}   = 0\forall ( x - y ) ^ 2 < 0 $. 
So what about causality? Our observables are bilinear in fermions. 
They do commute at space-like separations so the theory is causal. 

Away from singularities, we have that 
\begin{align*}
( i \slashed{p} _ x - m ) i S ( x - y ) &=  0  \\ 
					&=  ( i \slashed{\partial} _ x - m ) 
					( i \slashed{\partial}  + m ) \left[ D ( x - y )  - D ( y - x )  \right]  \\
					&=  - ( \partial _ x ^ 2 + m ^ 2 ) \left[  D ( x -y )  - D( y - x)  \right]  \\
					&= 0 \text{ using } p ^ \mu p _ \mu  = m ^ 2
\end{align*}

\subsubsection{The Feynman Propagator} 
A similar calculation gives us us that 
the two point propagator between $x $ and 
$ y $ are 
\begin{align*}
\bra{ 0 } \psi _ \alpha ( x) \overline{ \psi } _ \beta ( y ) \ket{ 0 } &=  \int \frac{ d ^ 3 p }{ ( 2 \pi ) ^ 3 
2 E _ p } ( \slashed{p} + m ) _{ \alpha \beta } e ^{  - i p \cdot  ( x - y ) } \\
\bra{ 0 } \overline{\psi } _{ \beta } ( y ) \psi _ \alpha ( x ) \ket{ 0  } &=  
\int \frac{ d ^ 3 p }{ ( 2 \pi ) ^ 3 2 E _ p } ( \slashed{p}  - m )_{ \alpha \beta } e ^{ i p \cdot  ( x - y ) }\\
\end{align*}
If we define $ S _{ F \alpha \beta }  = \bra{ 0 } T \psi _{ \alpha } ( x) \overline{ \psi } _{ \beta } ( y ) \ket{ 0 } $
as the object 
\[
\begin{cases}
  \bra{ 0 } \psi _ \alpha ( x) \overline{ \psi } _ \beta ( y ) \ket{ 0 } & x ^ 0 >  y ^ 0 \\
	  - \bra{ 0 }{ \overline{ \psi } _ \beta ( y ) \psi _ \alpha ( x ) \ket{ 0 }  & y ^ 0 } x ^ 0 
\end{cases} 
\]
The negative sign in the second term is required 
 for Lorentz invariance when $ ( x  -y ) ^ 2 < 0 $, since 
 there exists no Lorentz invariant way to determine whether $x ^ 0 < y ^ 0 $ 
 or the other way around. 
 In this case, $ \left\{  \psi ( x) , \overline{ \psi } ( y )  \right\}   = 0 $ and 
 so $ T $ as defined is lorentz invariant. 
 This is the same for strings of fermionic operators in $T $  - they anti commute. 
 We see the same behaviour for normal ordering, we get that 
 \[
  : \psi _ 1 \psi _ 2 : = - : \psi _ 2 \psi _ 1 : 
 \]
 We can define our Feynman propagator as 
 \begin{equation*}  
  \wick{ \c \psi ( x) \c \psi^*( y )} 
 \end{equation*} 
	From this we get that 
	\[
	T ( \psi ( x) , \overline{ \psi  }( y )  = : \psi ( x) \overline{ \psi ( y ) } 
	\]
\subsection{Momentum Space Feynman Rules for Fermion Amplitudes} 
Let's review the diagrams for 
the Yukawa interaction, but with the view of
Fermionic quantisation. 
We have the following diagrams.   
The important thing 
is that operators need to remain the same order, unless 
we decide to anti-commute them and pick up a minus sign. 
Hence, for a closed fermionic loop which looks like 
$ :  \overline{ \psi } _ \alpha ( x) 
\wick{ \c \psi_ \alpha ( x) \c \psi ^ * _ \beta ( y )  } \psi _ \beta ( y )   : $
where we contract the middle two fields, 
to contract the other two fields we need to pick up a 
minus sign (show here). 

Question: what if $ \mathcal{ L } _{ \text{ int } }  = - \lambda \phi  \overline{ \psi }_ \alpha ( \gamma ^ 5 )_{ \alpha \beta } \psi _{ \beta } $  ? 
This interaction preserves parity if $ \phi $ is a pseudo-scalar, 
in other words 
\[
	P \phi ( \vec{x} , t )   = - \phi ( - \vec{x} , t  )
\] 
This interaction looks like the below (diagram here) 
this contributes $ ( - i \lambda ) ( \gamma ^ 5 )_{ \alpha \beta } $. 

Question: how do we deal with spin and $ | \mathcal{ M } | ^ 2 $, 
in the cross section calculation? 
In most experiments $ \ket{ i } $ spin states are random 
and so we average over them. For example, for 
$ \psi \psi $, it would be $ \frac{1}{4 } \sum_{ r , s = 1 } ^ 2 $, 
Also spins in $ \ket{ f } $, aren't observed - they're summed over. 
Specifically 
\[
 \mathcal{ M } = B - A , \text{ in } \psi \psi \to \psi \psi 
\] where $ B , A $ are the different terms. 
We wrote appropriate spin sums averages with a line on top. 
\begin{align*}
	\overline{ | \mathcal{ M } | } ^ 2 &=  | \overline{ A } | ^ 2 
	+ | \overline{ B } | ^ 2 - \overline{ A ^ \dagger B } - 
	\overline{ B ^ \dagger A } \\
	A &=  \frac{\lambda ^ 2 \left[  \overline{u  }_{ \vec{p} ' } ^ s 
	\cdot  u_{ \vec{q} } ^ r  \right] \left[  
\overline{ u } _{ \vec{q} ' } ^{ r ' } \cdot  u_{ \vec{p} } ^ s \right]}{
\mu ^ 2- \mu ^ 2+ i \epsilon } \\
\end{align*}
Note that we have $ ( \gamma ^ 0 ) ^ \dagger  = \gamma ^ 0 $. 
 \[
	 | \overline{ A } | ^ 2 = \frac{ | \lambda | ^ 4  }{ 4 ( \mu ^ 2 - m ^ 2 ) ^ 2  } 
 \sum_{ r , s , s' , r ' } \overline{u }_{ \vec{p} ' _ \alpha } ^{ s ' } 
 u_{ \vec{q} _ \alpha } ^ r \overline{ u } _{ \vec{q} _ \beta } ^ r 
 u_{ \vec{p} ' _ \beta } ^{ s ' } \overline{u }_{ \vec{q} ' _ \beta } ^{ r ' } 
 u_{ \vec{p} _ \gamma } ^{ s } \overline{ u }_{ \vec{p} _ \delta } ^ s 
 u_{ \vec{q}' _ \delta } ^{ r ' }
\]  This 
is equal to 
\[
	\frac{ | \lambda |  ^ 4 }{ 4 } \frac{( \slashed{p}' + m)_{ \alpha \beta } 
	( \slashed{q} + m )_{ \beta \alpha } \tr \left[  
( \slashed{q}' + m ) (\slashed{p} + m )  \right] }{ ( \mu ^ 2 - m ^ 2 ) ^  2}
\] Often, we are in the high energy limit, 
where we may wish to neglect particle masses. In this 
case, we then find that 
\[	| \overline{ A } | ^ 2 = \frac{ | \lambda | ^ 4 \tr ( 
	\slashed{p} ' \slashed{q} ) \tr (\slashed{q} ' \slashed{p} ) }{4 \mu ^ 2 }
\] Similarly, we have that for $ | \overline{ B  } | ^ 2 $, 
we get that 
\[
 | \overline{ B } | ^ 2 = \frac{  | \lambda | ^ 2 }{ 4 t ^ 2 }
 \tr ( \slashed{q} ' \slashed{q} ) \tr ( \slashed{p} ' \slashed{p} ) 
\] We also want $  - \overline{ A ^ \dagger B  }  - \overline{ B ^ \dagger A }  = 
- 2 Re \overline{ ( A ^ \dagger B ) } $. 
We calculate this as 
\begin{align*}
	\overline{ A ^ \dagger B  } &  = \frac{ | \lambda | ^ 4 }{ 4 u t} 
	\sum_{ r , r' , s, s ' } \overline{u } _{ \vec{q} _ \beta } ^ r 
	u _{ \vec{p} ' _ \beta }^{ s' } \overline{ u } _{ \vec{p} _ \alpha } ^ s 
	u _{ \vec{q} ' _ \alpha } ^{ r '} \overline{ u } _{ \vec{q}' _ \gamma } ^{ r ' } 
	u_{ \vec{q} _ \gamma } ^ r \overline{ u } _{ \vec{p} ' _ \delta } ^{ s' } 
	u _{ \vec{p} _ \delta } ^ s \\
				    &=  \frac{|\lambda | ^ 2 }{ 4 ut } \tr \left(  
				    \slashed{p} \slashed{p} ' 
			    \slashed{q} \slashed{q}' \right)  \\
\end{align*} 

This gives us the Feynman rules 
for spin summed $ \mathcal{ M } ^ 2 $ diagrams. 
We have that complex conjugation switches $ \ket{ i } $ and 
$ \ket{ f } $ in the diagram. 
Fermion lines are joined with identical momentum on the LHS and RHS. 
Adter a spin sum, a closed fermion line in the $| \mathcal{ M } | ^ 2 $ diagram 
is given by a trace over $ \gamma $ matrices, with appropriate $\gamma ^ 5 $ ' s
etc in vertices at the correct posiition in the trace. trace follows 
fermion arrows backwards. 
