\subsection{What does the Dirac equation give us?} 
The Dirac equation is important because it provides the framework for quantization of spin-$\frac{1}{ 2}$ particles, like electrons and other fermions. Another startling prediction that the Dirac equation gives us is the existence of anti-matter. This equation is an amazing combination of the geometry of spinors in Minkowski space-time, and the wave-function formalisation of quantum mechanics that we all know and love. 
In this section, we'll provide some motivation for constructing the Dirac equation, solve it, then quantise this object to give rise to particles (fermions). 

LECTURE CONTENT
Recall that when we have a Lorentz transformation in our 
coordinates, we induce a transform on our scalar field. 
\[
	x ^ \mu \to ( x ') ^ \mu = \Lambda \indices{ ^ \mu _ \nu } x  ^ \nu, \quad 
 \phi ( x) \to \phi ' ( x) = \phi ( \lambda ^{ -1 } x ) 
\] Most particles have an intrinsic angular momentum-spin however, 
for example a spin 1 vector field which transforms our vector as 
\[
	A ^ \mu ( x) \to (A ') ^ \mu ( x) = \Lambda \indices{ ^ \mu _ \nu } A ^ \nu ( \Lambda ^{ - 1 } x )  
\] In general, we have that our field transforms 
as 
\[
	\phi ^ a ( x) \to D \indices{ ^ a _ b  } ( \Lambda ) \phi ^ b ( \Lambda ^{ - 1} x )  
\] where our matrices $ D $ form a representation of our Lorentz group.
Representations obey the properties that 
\begin{align*}
	D ( \Lambda _ 1 ) D ( \Lambda _ 2) & = D ( \Lambda _  1 \Lambda _ 2 ) \\
	D ( \Lambda ^{ - 1} ) = D ( \Lambda ) ^{ - 1} \\
	D ( I ) = 1 
\end{align*} To find representations 
we look at the Lorentz algebra, considering infinitesimal transformations
\[
 \Lambda \indices{ ^ \mu _ \nu } = \delta \indices{ ^ \mu _ \nu } + 
 \epsilon \omega \indices{ ^ \mu _ \nu } + O ( \epsilon ^ 2 )  
\] Working to infinitesimal order, the find that 
the generators obey the relation 
\[
	\Lambda \indices{ ^ \mu _ \sigma  } \Lambda \indices{ ^ \nu _  \rho }  =  \eta ^{ \mu \nu }
\]  Substituting this into our relation we have that 
\[
	( \delta \indices{ ^ \mu _ \sigma } + \epsilon \omega \indices{ ^ \mu _ \sigma } ) ( 
	\delta \indices{ ^ \nu  _ \rho }   + \epsilon \omega \indices{ ^ \nu  _ \rho }  ) \eta ^{ \sigma \rho }  =
	\eta ^{ \mu \nu } + O ( \epsilon ^ 2 ) 
\]  This implies that our generator satisfies 
\[
 \omega _{ \mu \nu } = - \omega _{ \nu \mu } \implies 
 \omega _{  \mu \nu } \text{ is anti symmetric}
\] This generator has $ 6 $ components, split up into 
3 rotations and 3 boosts. Thus, we can introduce a basis of 
six $ 4 \times 4 $ matrices 
\[
	( \mathcal{ M } ^{ \rho \sigma } ) ^{ \mu \nu }  = \eta ^{ \rho \mu } \eta ^{ \sigma \nu } 
	- \eta ^{ \sigma \mu } \eta ^{ \rho \nu }
\] We can lower the index here to give 
\[
	( \mathcal{ M } ^{ \rho \sigma } ) \indices{ ^ \mu _ \nu } = 
	\eta ^{ \rho \mu } \delta \indices{ ^ \sigma  _ \nu }  - 
	\eta ^{ \sigma \rho } \delta \indices{ ^ \rho _ \nu } 
\]  For example, we have that 
\[
	( \mathcal{ M } ^{ 01  } ) \indices{ ^ \mu_ \nu } = 
	\begin{pmatrix}  0 & 1 & 0 & 0 \\ 
	1 & 0 & 0 & 0 \\
        0 & 0 & 0 & 0 \\ 0 & 0 & 0 & 0 \end{pmatrix} 
\] This specific generator generates a boost in the $x ^ 1 $ 
direction. We can hence write our generator as 
a linear sum of these basis matrices. 
\[
	\omega \indices{ ^ \mu _ \nu } = \frac{1}{2 } \left(  \Omega_{ \rho \sigma } \mathcal{ M } ^{ \rho \sigma } \right)
	\indices{ ^ \mu _ \nu } 
\]  We call $ \mathcal{ M }^{ \rho\sigma } $ the generators 
of our Lorentz group, and our corresponding $ \Omega ^{ \rho\sigma } $ 
the anti symmetric parameters of the Lorentz transformation.  
Our Lorentz algebra, one can calcuate, is 
\[
	[ \mathcal{ M } ^{ \rho \sigma } , \mathcal{ M  } ^{ \tau \nu  } ] 
	= \eta ^{ \sigma \tau } \mathcal{ M } ^{ \rho \nu } - \eta ^{ \rho \tau } \mathcal{ M} ^{ \sigma \nu }
	- \eta ^{ \rho \nu } \mathcal{ M }^{ \sigma \tau } - \eta ^{ \sigma \nu } \mathcal{ M } ^{ \rho \tau }
\]  Our corresponding boost is $ \exp ( \frac{1}{2 } \Omega_{ \rho \sigma  } \mathcal{ M } ^{ \rho \sigma }$.

\subsection{The Spinor Representation} 
To help us find representations that satisfy the 
Lroentz algebra, we start with the 
Clifford algebra. This is an algebra whose 
elements satisfy the anti-commutation 
relations 
\[
 \left\{  \gamma ^ \mu , \gamma ^ \nu  \right\}  = 2 \eta ^{ \mu \nu } I , 
 \quad \left\{  \gamma ^ \mu , \gamma ^  \nu  \right\}   = \gamma ^ \mu \gamma ^ \nu 
 + \gamma ^ \nu \gamma ^ \mu 
\] where $ \gamma ^ \mu $ are a set of four matrices. 
We have that our matrices in the algebra obey 
the properties 
\begin{align*}
	\gamma ^ \mu \gamma ^ \nu & = - \gamma ^ \nu \gamma ^ \mu , \quad \mu \neq \nu \\
	( \gamma ^ 0 ) ^ 2 = I, \quad ( \gamma ^ i ) ^ 2 =  - I , \quad i \in \left\{ 1, 2, 3  \right\} 
\end{align*}
Our simpliest representation which obeys these commutation relation 
are a set of $ 4 \times 4 $ matrices, which we call the Chiral representation. 
These are given in block matrix form as 
\[
	\gamma ^ 0 = \begin{pmatrix}  0 & I _ 2 \\ I _ 2 & 0  \end{pmatrix} , \quad 
	\gamma ^ i = \begin{pmatrix}  0 & \sigma ^ i \\ - \sigma ^ i & 0  \end{pmatrix} 
\] Here, $ \sigma ^ i  $  are the 
Pauli matrices given by 
\[
	\sigma ^ 1 = \begin{pmatrix}  0 & 1 \\ 1 & 0  \end{pmatrix}  , \quad 
	\sigma ^ 2 = \begin{pmatrix}  0 & - i \\ i & 0  \end{pmatrix} , \quad
	\sigma ^ 3  = \begin{pmatrix}  1 &  0 \\ 0 & 1  \end{pmatrix} 
\] These Pauli matrices 
obey the commutation and anti commutation relations 
\[
 \left\{  \sigma ^ i , \sigma ^ j  \right\}  = 2 \delta ^{ ij } I _ 2 , 
 \quad [ \sigma ^ j , \sigma ^ k ] =  2i \epsilon ^{ jkl } \sigma ^ l 
\] Given this represetnation, we can also 
generate a new representation by conjugating by a constant 
invertible matrix $ U$, so $ U \gamma ^ \mu U ^{ - 1} $  works 
also as a representation. 

If we define 
\[
	S ^{ \rho\sigma } : = \frac{1}{4} [  \gamma ^ \rho ,  \gamma ^ \sigma ] 
	= \begin{cases}
		0 & \rho = \sigma \\ 
		\frac{1}{2  } \gamma ^ \rho \gamma ^ \sigma &  \rho \neq \sigma 
	\end{cases}
	 = \frac{1}{2 } \gamma ^ \rho \gamma ^ \sigma  - \frac{1}{2 } \eta ^{ \rho \sigma } I 
\] then, we can 
prove that 
\begin{claim}
	\[
		[ S ^{ \mu \nu } , \gamma ^ \rho ]  = \gamma ^ \mu \eta ^{ \nu \rho } 
		  - \gamma ^ \nu \eta ^{ \rho \mu }
	\]  
\end{claim}
Using this, we can then prove that 
\begin{claim}
	\[
		[ S ^{ \rho\sigma } , S ^{ \tau \nu } ] = 
		\eta ^{ \sigma \tau } S ^{ \rho \nu }  - 
		\eta ^{ \rho \tau } S ^{ \sigma \nu } 
		+ \eta ^{ \rho \nu } S ^{ \sigma \tau } 
		 - \eta ^{ \sigma \nu } S ^{ \rho \tau }
	\] 
\end{claim} Thus, $ S $ provides a representation of 
the Lorentz group. 
We now define a Dirac spinor, $ \psi _ \alpha ( x) $which is a four component 
vector that satisfies the following transformation 
law 
\[
	\psi _ \alpha ( x) \to  S[ \Lambda ] \indices{ ^ \alpha _ \beta } \psi ^ \beta ( 
	\Lambda ^{ - 1} x ) 
\] where we've defined $ \Lambda = \exp \left(  \frac{1}{2 } 
\Omega _{ \rho \sigma } M ^{ \rho \sigma } \right)  $, and 
our spinor representation $ S [ \Lambda ] = \exp \left(  
\frac{1}{2 } \Omega _{ \rho \sigma } S ^{ \rho \sigma } \right)  $. 
We need to check that the Spinor representation is not 
equivalent to the usual vector representation, which we 
can see via a specific lorentz transformation 
\[
 S^{ ij } = \frac{1}{4 }\left[  \begin{pmatrix}  
 0 & \sigma ^ i \\ - \sigma ^ i & 0 \end{pmatrix}  , 
 \begin{pmatrix}  0 & \sigma ^ j \\ - \sigma ^ j & 0  \end{pmatrix}  \right]  
 =  - \frac{i \epsilon ^{ ijk } }{ 2 } \begin{pmatrix}  \sigma ^ k & 0 
 \\ \sigma ^ k & 0 \end{pmatrix}  \text{ from } \sigma ^ i \text{ algebra}
\] If we write as a paramter 
$ \Omega_{ ij } =  - \epsilon _{ ijk } \phi ^ k $ . 
Exponentiating, we get that 
\[
	S [ \Lambda ]  = \exp \left(  \frac{1}{2 } \Omega _{ \rho \sigma } 
	S ^{ \rho\sigma } \right)  = \begin{pmatrix}  
	e ^{ i \phi \cdot  \sigma / 2  } & 0 \\ 0 & 
e^{ i \phi \cdot  \sigma / 2 } \end{pmatrix}  
\]   Now, if we consider a rotation of 
$ 2 \pi $ abou the $ x ^ 3 $ axis, we get that 
our corresponding representation is 
\[
	S [ \Lambda ] = \begin{pmatrix}  e ^{ i \pi \sigma ^ 3 }` & 0 \\ 
	0 & e ^{ - i \pi \sigma ^ 3 } \end{pmatrix}   = - I _ 4 
\] Thus, a rotation of $ 2\Pi  $  takes a 
spinor $ \phi _ \sigma ( x ) \to  - \phi _{ \sigma } ( x) $, not like a vector, 
which goes like $ \Lambda = I $ as expected (in our fundamental representation). 

Now, what happens with boosts of spinors 
\[
	S ^{ 0i  }  = \frac{1}{2 } \begin{pmatrix}  - \sigma ^ i & 0 \\ 0 & \sigma ^ i   \end{pmatrix}  
\]  If we write our boost parameter as 
$ \Omega _{ 0i }  = \chi _ i $, then our corresponding representation 
is 
\[
	S [ \Lambda ] = \begin{pmatrix}  e^{  - \chi \cdot \sigma / 2   } & 
	0 \\ 0 & e ^{ \chi \cdot  \sigma / 2 } \end{pmatrix}  
\] Note that for rotations, $ S [ \Lambda ] $ is unitary, since $ S [ \Lambda ] ^ \dagger S [ \Lambda ]  = I $, but for boosts, it's not. 
There are no finite dimensional unitary representations of the Lorentz group! 
$ S [ \Lambda ]  = \exp \left[  \frac{1}{2 } \Omega_{ \rho \sigma } 
S ^{ \rho \sigma } \right]  $ is only unitary if $ S ^{ \rho \sigma } $ are 
anti hermitian. Now, 
\[
	( S ^{ \rho \sigma } ) ^ \dagger  = -  \frac{1}{4 } [ ( \gamma ^ \rho ) ^ \dagger , 
	( \gamma ^ \sigma ) ^ \dagger ] 
\] can be anti-hermitian only if all $ \gamma ^ \mu $ are hermitian or are all anti hermitian. 
However, this can't be arranged, and is impossible to do. 
This can't be arranged, because
\[
 ( \gamma ^ 0 ) ^ 2 = 1 \implies \gamma ^ 0 \text{ has real eigenvalues, so can't be anti-hermitatian 
 since these have imaginary eigenvalues }
\] Similarly, 
\[
 ( \gamma ^ i ) ^ 2 = 1 \implies \gamma ^ i \text{ can't be hermitian }
\] In general, there's no way to pick $ \gamma ^ \mu $  such that $ S ^{ \mu \nu } $ 
are anti-Hermitian.

\subsection{Constructing a Lorentz Invariant action of $ \psi $ } 
We write $ \psi ^ \dagger ( x )   = ( \phi ^ * ) ^ T ( x ) $. 
One question that we might have is to ask whether 
$ \psi ^ \dagger ( x) \phi ( x) $ is a Lorentz scalar. 
However, under a Lorentz transformation, 
\[
	\psi ^ \dagger ( x) \psi ^ ( x) \to \psi ^ \dagger ( \Lambda ^{ - 1} x ) 
	S [ \Lambda ] ^ \dagger S [ \Lambda ] \psi( \Lambda ^{ - 1}x ) 
\] This is not invariant in general since $ S [ \Lambda ] $ is 
not unitary. 
In the chiral representation, we have that 
\[
	( \gamma ^ 0 ) ^ \dagger  = \gamma ^ 0 , \quad ( \gamma ^ i) ^ \dagger = 
	 - ( \gamma ^ i ) 
\] We learn that we can write out
\[
 ( \gamma ^ \mu ) ^ \dagger = \gamma ^ 0 \gamma ^ \mu \gamma ^ 0 
\] Thus, we can find the Hermitian conjugate of $ S $ as 
\[
	( S ^{ \mu \nu } ) ^ \dagger = - \frac{1}{4 } [ ( \gamma ^ \mu ) ^ \dagger, 
	( \gamma & \nu ) ^ \dagger ]  =  - \gamma ^ 0 S ^{ \mu \nu } \gamma ^ 0 
\]  hence, our exponential gives 
\[
 S [ \Lambda ] ^ \dagger = \exp \left(  \frac{1}{2 } 
 \Omega _{ \mu \nu } ( S ^{ \mu \nu } ) ^ \dagger \right)   = \gamma ^ 0 S [ \Lambda ] ^{ - 1} 
 \gamma ^ 0 
\] In this spirit, we can 
define our Dirac adjoint of $ \psi $, as 
\[
	\overline{\psi } ( x) := \psi ^ \dagger  (x) \gamma ^ 0 
\] Our claim is that $ \overline{ \psi } \psi  $ is indeed a Lorentz scalar. 
To prove this 
observe that 
\[
	\overline{ \psi  } \psi  = \psi ^ \dagger \gamma ^ 0 \phi \to \psi ^ \dagger ( 
	\Lambda ^{ - 1} x ) S [ \Lambda ] ^ \dagger \gamma ^ 0 S [ \Lambda ] \psi ( 
	\Lambda ^{ - 1 }x ) 
\] Using the identity above, we get that this is 
the same. 
In addition, we have that $ \overline{\psi } \gamma ^ \mu \psi ( x) $ is 
a Lorentz vector. We have that 
\[
	\overline{ \psi } \gamma ^ \mu \psi \to \overline{ \psi } S [ \Lambda ] ^{ - 1} 
	\gamma ^ \mu S [ \Lambda ] \psi 
\]  For this to be a Lorentz vector, we require that 
\[
	S [ \Lambda ]^{ - 1} \gamma ^ \mu S [ \Lambda ] = \Lambda \indices{ ^ \mu _ \nu } \gamma ^ \nu  
\] We have that, infinitesimally, 
\[
	\Lambda \indices{ ^ \mu _ \nu } = \exp \left(  
	\frac{1}{2 } \Omega _{ \rho \sigma  }\mathcal{ M } ^{ \rho \sigma } \right)  , 
	\quad S[ \Lambda ] = \exp \left(  \frac{1}{2 } \Omega _{ \rho \sigma } 
	S ^{ \rho \sigma } \right) 
\] So, we need to prove that 
\[
	( \mathcal{ M  }^{ \rho \sigma } ) \indices{ ^ \mu _ \nu } \gamma ^ \nu 
	=- [ S ^{ \rho \sigma } , \gamma ^ \mu ] 
\] This can be shown to be true by observing that 
\[
 \left(  \eta ^{ \rho \mu }  \delta \indices{ ^ \sigma _ \nu }
  - \eta ^{ \sigma \mu } \delta \indices{ ^ \rho  _ \nu }  \right)  \gamma ^ \nu 
    = \eta ^{\rho \mu } \gamma ^{ \rho }  - \gamma ^{ \rho } \eta ^{ \sigma \mu } = 
    - [ S ^{ \rho \sigma } , \gamma ^{ \mu } ] 
\] Finally, 
we can construct a Lorentz invariant action from these objects, 
by setting 
\[
	S = \int  d^ 4 x \, \overline{ \psi  } \left(  
	0 \gamma ^ \mu \partial  _\mu  - m \right) \psi ( x) 
\] 
\subsection{The Dirac equation} 
Varying $ \psi $ and $ \overline{ \psi } $ independently 
in the Euler Lagrange equations gives us the Dirac equation. 
\[
 \left(  i \gamma ^ \mu \partial  _ \mu - m  \right)  \psi  =0 
\] This is first order, not second order like KG.
We write $ A _ \mu \gamma ^ \mu = \slashed{ A } $. 
The equation is 
 \[
	 \left(  i \slashed{\partial } - m  \right)  \psi = 0 
\] Note that each component of $ \psi $ solves
the KG equation. 
\begin{align*}
	( i \slashed{ \partial  } + m ) ( i \slashed { \partial  } - m ) \psi &=  0  \\
	( \gamma ^ \mu \gamma ^ \nu \partial  _ \mu \partial  _ \nu + m ^ 2 ) \psi &=  0  \\ 
	( \frac{1}{2 } \left\{  \gamma ^ \mu , \gamma ^ \nu  \right\}  \partial  _ \mu \partial  _ \nu 
	+ m^ 2 ) \psi = 0 
	- ( \partial  ^ 2 + m ^ 2 ) \psi  = 0 
\end{align*}
