\subsection{Constructing the Chiral representation for the Dirac equation}
You can show that if we have a set of matrices $\{ \gamma^\mu \}_\mu$ which obey the identity 
\[ \{ \gamma^\mu , \gamma^\nu \} =  - 2 g^{\mu\nu}  \] then the object 
\[ S^{ \mu \nu } = \frac{i}{4} [\gamma^\mu, \gamma^\nu] \] obeys the Lorentz algebra which we derived above.  

Whilst this choice of representation is not unique, we choose a canonical one called the Weyl representation which is given by
\[ \gamma^0 = \begin{pmatrix} 0  & I \\ I & 0 \end{pmatrix} \quad \gamma^i = \begin{pmatrix} 0 & \sigma^{i} \\ - \sigma^i & 0 \end{pmatrix} \]  
I'll touch on this a bit more later, but the important this is that this object represents boosts in 3 different dimensions, given by $S^{0i}$ and rotations in 3 dimensions given by $S^{ij}$. 
Thus, our field $\phi(x)$ transforms as, when we exponentiate $S_{ij}$, as \[ \phi(x) \rightarrow \phi'(x) = \exp(i \frac{\omega_{\mu\nu}}{2} S^{\mu\nu}) \phi(\Lambda^{-1}) \] where we've done and inverse transform to represent our change in frame of reference, and also done a map on the field itself.  

