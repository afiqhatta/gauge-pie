\subsection{Green's function vacuum} 
Let's delve into the 
nuts and bolts of the vacuum state in our interacting 
theory. We denote vacuum state of our interacting 
theory as $ \ket{ \Omega } $. This is a 
different state from our vacuum state in our free theory, 
where we have that 
\[
 H _ 0 \ket{ 0 } = 0 
\] However when we move to our interaction picture, 
we get that $ H = H_0 + H_{ \text{ int } } $ satisfies 
\[
	H \ket{ \Omega } = 0 , \quad \bra { \Omega } \ket { \Omega } 
\] We define 'Green's functions' as 
\[
	G ^{ ( n )  } =  \bra{ \Omega } T \left\{  \phi _ H ( x_ 1 ) \dots 
	\phi _ H ( x_ n ) \right\} \ket{ \Omega } 
\] which can be interpreted as our n point correlator 
but this time, in our interaction picture. Notice that 
this is time dependent, and that our states are in the Heisenberg 
picture.
Now, interaction picture vacuum states are difficult 
to deal with. The goal of this section is to be able to 
write them as a function of our free theory vacuum state 
$ \ket{ 0 } $.  
Our claim is that 
\[
	\bra{ \Omega } T \left\{  \phi_{ 1H  } \dots \phi _{ m H }  \right\} \ket{ \Omega } 
	= \frac{ \bra{ 0 } T \left\{  \phi_{ 1I  } \dots \phi_{ m I }  S \right\} \ket{ 0 }   }{ \bra{ 0 } S \ket{ 0 } }
\] Let's explain some of the notation here. Recall, that 
$ S $ is merely our time evolution operator and since 
we are considering initial and final state scattering this 
is 
\[
	S = \lim _{ T \to \infty } U ( T , -T ) = \lim _{ T \to \infty }
		\exp \left( \int _{ - T } ^ T  dt H _ I ( x )  \right)  
\] 
We denote states in the Heisenberg picture (on the left hand side), 
as $ \phi _{ i H } $, and states in the interaction 
picture as $ \phi _{ i I  }$. From the previous section, 
we showed that the numerator on the right has side is 
\[
  \bra{ 0 } T \left\{  \phi_{ 1I  } \dots \phi_{ m I }  S \right\} \ket{ 0 }  = 
  \sum \left(  \text{ connected diagrams with m external points } \right) 
  \bra{ 0 } S \ket{ 0 } 
\] 
In other words, our Green's function 
is \textbf{the sum of connected diagrams} with $ m $ external points. 
To prove this, we take, without loss of generality, 
that $ x_1 ^ 0 > x_2 ^ 0 > \dots > x_{ m } ^ 0 $. This means, 
we can employ a trick to parition the integral and reorder things in 
the expression. 
\begin{align*}
	\mathcal{T } \big (  \phi _{ 1I  } \phi _{ 2 I  } \dots \phi _{ m I }  & \exp 
	\left(   - i \int_{ - \infty } ^ \infty dt H _ I ( t)  \right) \big ) =  
	\mathcal{ T } \big \{ \phi _{ 1I  } ( x _ 1 ^ 0 ) \phi _{ 2 I  } ( x _ 2 ^ 0 ) \dots 
	\phi _{ m I  } ( x_ m ^ 0 ) \\ 
	 & \exp \left(  
- i \int_{ - \infty } ^{ x_ m ^ 0   } dt H _I ( x )  - i \int_{  x_ m ^ 0 } ^{ x _{ m - 1 } ^ 0  } 
dt H _ I ( x )  - \dots  -i \int _{ x_ 1 ^ 0  } ^{ \infty } dt H_ I ( t)  \right) \big \} 
\end{align*}
The uposhot of writing this that we can now we apply to time 
ordering operator, which shuffles around the terms 
in the exponential. Thus, the expression above reads 
\begin{align*} 
	\dots  & = \exp \left(   - i \int _{ x _ 1 ^ 0 } ^ \infty dt H _ I ( t)  \right)  
	\phi _{ 1 I } ( x_ 1 ) \exp \left(  - i \int _{ x_2 ^ 0 } ^{ x_ 1 ^ 0 } dt H _ I ( t)  \right)
	\dots \\
	       & ... \exp \left(  -i  \int _{ x ^ 0 _{ n - 1 } } ^{ x _ n ^ 0 } dt H _I ( t) \right) 
	\phi _{ I m } \exp \left(  - i \int_{ - \infty } ^{ x _ n ^ 0 } dt H _ I ( t)   \right)  
\end{align*}  Now, from our definition of $ U ( t, t' ) $, we have that finally
\[
	\mathcal{ T } \left\{  \phi _{ 1 I } \dots \phi _{ m I }  \right\} = 
	U ( \infty, t _ 1 ) \phi _{ 1 I } U ( t_1 , t_ 2 ) \phi _{ I 2 } \dots
	U ( t _{ m - 1 } , t _ m ) \phi _{ I m } U ( t_ m , - \infty ) 
\] where we've cleaned up our notation above by writing
$ t _ 1 = x_{ i } ^ 0 $.

Recall that $U ( t, t_0 ) = e ^{ i H_0 t } e^{ i H (  t- t_0 ) } e ^{  - i H_0 t_0 }$. 
From our discussion, we sandwich the above expression with 
$ \ket{ 0 } $, so we have that 
\[
\bra{ 0 } T \left\{  \phi_{ 1 I } \dots \phi _{ m I } S  \right\} \ket{ 0 } =  
\bra{ 0 } U ( \infty, t_1 ) \phi_{ I 1 } U ( t_1, t_2 ) \phi _{ 2 I } \dots U ( t_{ m-1 } , t _{ n } ) 
\phi _{ m I } U ( t_{ m } , - \infty ) \ket{ 0 } 
\] Now, we split up the unitary operator terms 
to wrangle this expression to a form 
which is in the Heisenberg picture. The above term 
is 
\[
	\dots = \bra{ 0 } U ( \infty, 0 ) U ( 0 , t_1 ) 
	\phi _{ 1 I } U ( t_1, 0 ) U ( 0 , t_2 ) \phi _{ 2 I } \dots 
	U ( 0 , t_m ) \phi _{ m I  } U ( t_ m , 0 ) U ( 0 , \infty ) \ket{ 0 } 
\]
Now, we use this to switch to the Heisenberg picture 
by observing that the above expression is equal to 
\[
	\bra{ 0 } T \left\{  \phi_{ 1 I } \dots \phi _{ m I } S  \right\} \ket{ 0 }  = \bra{ \psi} U ( 0 , - \infty ) \ket{0} = \bra{ 0 } U ( \infty, 0 ) \phi _{ 1H } \dots \phi _{ m H } U ( 0 ,  - \infty ) \ket{ 0 }
\] Here, we've defined 
the state $ \bra{ \psi } $ as
\[
	\bra{ \psi } = \bra{ 0 } U ( \infty, 0 ) \phi _{ 1H } \dots \phi _{ m H }
\] From the right hand side, this 
is just 
\begin{align*}
	\dots &=  \lim_{ t_0 \to - \infty } \bra{ \psi } U ( 0 , t_0 ) \ket{ 0 }  \\
	      &=  \lim_{ t_0 \to - \infty } \bra{ \psi } e ^{ -  i H t_0 } \ket{ 0 }, \text{ since } 
	      U ( 0 , t_0 ) \ket{ 0 } = e ^{ - i H t_0 } e ^{ - i H_0 t_0 } \ket{ 0 } = e ^{ - i H t_0 } \ket{ 0 }  \\
	&=  \lim_{ t_0 \to - \infty } \bra{ \psi } e^{ i H t_0 } \left[  \ket{ \Omega } 
	\bra{ \Omega } + \sum_{n=1}^{ \infty} \prod_{ j = 1 } ^ n \int \frac{d ^ 3 p_{ j} }{
2 E_{ p_{ j } } ( 2 \pi ) ^ 3  } \ket{ p_1 , \dots p_n } \bra{ p_1, \dots p_n }  \right] \ket{ 0 }  \\
	&=  \bra{ \psi }\ket{ \Omega } \bra{ \Omega }\ket{ 0 }  + \lim_{ t_0 \to - \infty } \sum_n \int \left( 
	\prod_ j \frac{ d^ 3 p _ j }{ ( 2 \pi ) ^ 3 2 E_{ p _{ j } } }  \right) 
	e^{ - i \sum_{ j = 1 } ^ n E_{ p _{ j } }  t_0 } \bra{ \psi }\ket{ p_1, \dots p_n }\bra{ p_1 , \dots p_n }\ket{ 0 } \\
\end{align*}
Going into the third line, we've applied our completeness 
relation for Fock space, where we sum over all momentum 
eigenstates for every number of particle we can have. 
Now, the second term in the last line goes to zero due to the Riemann-lebesgue 
lemma
\[
	\lim_{ \mu \to \infty } \int _ a ^ b f( x) e ^{ i \mu x } dx  =0 
\] Hence, the above expression reads 
\[
	\dots = \bra{ 0 } U ( \infty, 0 ) \phi_{ 1H } \dots \phi_{ m H } \ket{ \Omega }\bra{ \Omega }\ket{ 0 } 
\] Similar to the above, we haev that 
\begin{align*}
	\bra{ 0 } U ( \infty, 0 ) \ket{ \psi } &=  \lim_{ t_0 \to \infty } 
	\bra{ 0 } e ^{ i H t_0 } \ket{\psi }  \\
	&=  \bra{ \Omega } \phi_{ 1H } , \dots \phi_{ m H } \ket{ \Omega } 
	\bra{ \Omega }\ket{ 0 } \bra{  0 }\ket{ \Omega } \\ 
\end{align*}
Our denominator is $ \bra{ 0 } S \ket{ 0 } = \bra{ 0 } \ket{ \Omega } \bra{ \Omega } \ket{ 0 } $, 
by setting $ m = 0 $. 
This means we ( insert previous example ) discounting connected diagrams 

\subsubsection{LSZ Reduction} 
To describe scattering in interacting theory, with external states (eg 
 $ \ket{ p_1 , p_2 } $, should be from interacting theory. This means we 
 exclude loops on external legs. 
THe above diagram is banned since it's absorbed into 
the definition of an initial interacting state. See AQFT or look up 
amputated diagrams. 
