\subsection{ Scattering } 
The time evolution used in scattering theory is called 
the 'S-matrix', where
\[
S= \lim_{t  \to \infty, t_0 \to  - \infty}U ( t, t_0 )   
\] The initial state $ \ket{i } $ and final state $ \ket{ \phi }  $ 
are some sense 'far away' form each other and the interaction. 
We assume that $ \ket{ i } , \ket{ f  } $ behave like free particles; 
they're eigenstates of $ H_0 $ . 
THe amplitude is 
\[
\lim_{ t  \to \infty, t_0 \to  - \infty } \bra{  f} U ( t, t_0 )\ket { i }  = \bra{ f} S \ket{ i }   
\] The latter expression is called the 'S-matrix' matrix element.

\subsubsection{An example with Yukawa theory} 
Going back to scalar Yukawa theory, the interaction Hamiltonian is
\[
H_ I = g \psi _ I ^ * \psi _ I \phi _ I 
\] We model the creation of mesons with $ \phi $, and the creation of nucleons or anti nucleons with $ \psi $. 
Concentrating on the creation and annihilation operators in the 
Fourier mode expansion,
\[
\phi \sim a_{ \vec{p} } + a_{ \vec{p} }^ \dagger 
\] These things destroy and create mesons respectively. 
We have that 
\begin{itemize}
\item $ \psi \sim b_{ \vec{p} } + c_{ \vec{p} }^\dagger $ which destroys a nucleon and create an anti nucleon respectively. 
\item $ \psi ^ *  \sim b_{ \vec{p} } ^ \dagger + c_{ \vec{p} } $ which creates a nucleon and destroys an anti nucleon. 
\end{itemize}
With this interaction, we have the commutation 
relations 
\[
[ a_{ \vec{p} } , a_{ \vec{p} ' }^\dagger ] = [ b_{ \vec{p} }, b_{ \vec{p} ' }^ \dagger ] = [ c_{\vec{p} } , c_{ \vec{p}' }^\dagger ]  = ( 2 \pi ) ^ 3 \delta ( \vec{p} - \vec{p} ' ) 
\] All other commutation relations are zero. 
To our first order our interaction, expanding out 
our interaction Hamiltonian in $ g$, we might have an 
integral term like 
\[
\int dp' dp'' dp'''  b_{\vec{p}}^\dagger c_{\vec{p} ' }^\dagger a_{ }
\] How we we interpret this? 
Well, we could interpret this as a term which destroys 
a meson, then creates a nucleon and anti-nucleon. 

If we expand to second order in $ g$, 
we find more involved terms which stem from our integral. 
For example, we could have the term 
\[
\int ( c^ \dagger b^\dagger a ) ( b c  a^ \dagger ) 
\] This is a two stage process. 
First, from our first bracket term we annihilate a nucleon and an anti-nucleon, 
and then create a scalar meson. 
Then, we annihilate that scalar meson an then create a nucleon and anti-nucleon. 
Thus, this is nucleon and anti-nucleon scattering, and is 
represented by the process
\[
\psi \overline{\psi } \to \phi \to \psi \overline{\psi }
\] Thus, this term \textbf{specifically } contributes 
to nucleon and anti-nucleon scattering. 

\subsubsection{First order term in meson decay}
In this part, we will be focusing on probability 
amplitudes that arise from certain processes. 
Let's focus specifically on the case 
of our first order interaction with a meson  $ \phi $ 
decaying into a nucleon and anti-nucleon
\[
\phi \to \psi \overline{ \psi }
\] Let's give a momentum to our meson, $ \vec{p}$. 
Treating this as an initial state, we assume that 
we can use the ground state of our free theory, $ \ket{ 0 } $, 
as a springboard to excite states from. Thus, we write down 
the momentum eigenstate for our initial state as 
\[
\ket{ i } = \sqrt{ 2E_{ p   } } a_{ \vec{p} }^ \dagger \ket{ 0 }  
\] If we assign our final state $ \ket{ f  } $ to be  
a nucleon and anti nucleon with momenta $ \vec{q}_1,  \vec{q}_ 2 $, 
then we have 
\[
\ket{ f  }  = \sqrt{ 4 E_{ q_1 } E_{ q _ 2 } }  b_{ \vec{q}_ 1 }^ \dagger c_{ \vec{q}_ 2 }^\dagger \ket{ 0 } 
\]  
Our scattering amplitude is given by 
\[\bra{ f } S \ket{ i } = \bra{ 0 } bc a ^ \dagger \ket{ 0 } - ig \bra{ f }  \int d^4 x \, \psi_I^* ( x) \psi_I  ( x) \phi_I ( x) \ket{ 0 }  + O ( g ^ 2 ) \] The zeroth order term is just zero, because $ a^ \dagger , c $ commute. 
Once we commute the $c $ past $ a ^ \dagger $, it hits our vacuum 
state and sends it to zero. 
The form of the first order term deserved some explanation. 
Since we're taking the limit as $ t \to \infty $ and $ t_0 \to  - \infty $, 
the form of our first order term is 
\[
\bra{ f} \int_{ - \infty }^\infty dt \int d^ 3 x \psi^ *_ I  ( x) \psi_ I  ( x) \phi_I  ( x)  \ket{ i } 
\] we can compose the integrals together to get $ \int d^4 x $.
Let's go slowly. We expand on the right hand side our $ \ket{ i } $ term 
to get that 
\begin{align*}
\bra { f} S \ket{ i } & = - i g \bra{f} \int d^ 4 x \, \psi ^ * ( x) \psi ( x) \int \frac{ d^ 3 k }{ ( 2 \pi ) ^ 3  \sqrt{ 2 E_{ k  } }  	 } \left(  a_{ \vec{k} } a_{ \vec{p} } ^ \dagger e ^{  - i k \cdot  x } + a_{ \vec{k} } ^ \dagger a_{ \vec{p} } ^ \dagger e^{ - k \cdot  x } \right) \ket{ 0 }  \\
      & =    \int d^ 4 x \, d^ 3 k  \, \bra{f} \psi ^ * ( x) \psi ( x) \frac{ 1 }{ ( 2 \pi ) ^ 3  \sqrt{ 2 E_{ k  } }  	 } \left(  a_{ \vec{k} } a_{ \vec{p} } ^ \dagger e ^{  - i k \cdot  x } + a_{ \vec{k} } ^ \dagger a_{ \vec{p} } ^ \dagger e^{ - k \cdot  x } \right) \ket{ 0 }
\end{align*} Now, since our $ a_{ \vec{k} } ^ \dagger $ commutes with $ b, c, b ^ \dagger, c^ \dagger $, 
we have that the second term trivially commutes past the $ \psi $ and $ \psi ^ * $ term, and it 
the part goes to zero. As for the first term, 
we commute $ a_{ \vec{k} } $ past and $ a_{ \vec{p} }^ \dagger $, to pick up 
a $ ( 2 \pi ) ^ 3 \delta ( \vec{k} - \vec{p} ) $. 
Thus we replace 
\[
a_{ \vec{k} } a_{ \vec{p}  }^\dagger \ket{ 0 } = [ a_{ \vec{k} }, a_{ \vec{p} } ^ \dagger ] \ket{ 0 }  = ( 2 \pi ) ^ 3 \delta ( \vec{p}  - \vec{k} ) \ket{ 0 }  
\] This means that our term above is 
\begin{align*}
\bra{ f} S \ket{ i } &= - ig \int \frac{ d^ 4 x d^  3 k_1 d^ 3 k_ 2  }{ ( 2 \pi ) ^ 6} \sqrt{ 4 E_{ q_1 } E_{ q_2 }} \bra{ 0 }\frac{1}{ \sqrt{ 4 E_{ k_1 } E_{ k_2 } }} c_{ \vec{q}_2 } b_{ \vec{q}_1 } \left(  b_{\vec{k}_ 1 } ^ \dagger e^{ i k_1 \cdot  x } + c_{ \vec{k}_1 } e^{  - i k_1 \cdot  x } \right) \\
& \left( b_{ \vec{k}_ 2  } e^{ - i k_2 \cdot  x  } + c_{\vec{k}_ 2 }^\dagger e^{ i k_2 \cdot  x } \right) e^{ - i p \cdot  x } \ket{ 0 }  
\end{align*}

We can create a caricature of this product by ignoring the 
momenta and integrals. We get a string of operators 
that look like 
\[
\sim c b b ^\dagger b + c b b ^\dagger c^ \dagger + c  b c b + c b c c^\dagger 
\] Now, all strings ending with $ b $ vanish since they 
operate on the $ \ket{ 0 } $ term to the right. 

We're left with the terms 
\[
\sim c b b ^ \dagger c ^ \dagger + c b c c^ \dagger 
\] Now, in the second term, if we commute $ c $ past  $ c ^ \dagger$, 
we pick up a delta function. But this leaves a term of the form  $ \delta c b $, 
and the  $b  $ still annihilates. So, our only surviving term is 
\[
\sim c b b ^ \dagger c ^ \dagger = c_{ \vec{q} _ 2 } b_{ \vec{q}_ 1 }b_{ \vec{k} _ 1}^\dagger c_{ \vec{k} _ 2 }^\dagger  
\] Commuting $ b $ past $ b^ \dagger$ gives the following term in our integral!
\[
( 2 \pi ) ^ 3 \delta ( \vec{q}_ 1 - \vec{k}_ 1 ) c_{ \vec{q}_ 2 } c_{ \vec{k} _ 2 } ^ \dagger e^{ i x\cdot   ( k_1 + k_2 - p)  }
\] Once again, commuting $ c $ past $ c ^ \dagger $ gives us another delta function to include. 
So the only term preserved in our integral is 
\[
( 2 \pi ) ^ 6  \delta ( \vec{q}_ 1 - \vec{k}_ 1 ) \delta( \vec{q}_ 2 - \vec{k}_ 2 )  e^{ i x\cdot   ( k_1 + k_2 - p }
\] Thus, performing our integrals over $ \vec{k}_ 1 $ and $ \vec{k}_ 2 $ gives us our only term 
at first order which contributes to our scattering. 
\[
\bra{ f} S \ket{ i } =  - i g \bra{ 0 } \int d^ 4 x \, e^{ i ( q_1 + q_2 - p ) \cdot  p } \ket{ 0 }  = - ig \delta ^ 4 ( q_ 1 + q_ 2 - p ) 
\] But, this gives a final amplitude which is a delta function in 4 space. 
This means that for a non zero scattering amplitude, we require our four momentum to be conserved!

