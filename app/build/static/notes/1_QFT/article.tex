\documentclass[11pt, oneside]{article}   	% use "amsart" instead of "article" for AMSLaTeX format
\usepackage[margin = 1.1in]{geometry}            		% See geometry.pdf to learn the layout options. There are lots.
\geometry{letterpaper}                   		% ... or a4paper or a5paper or ... 
\usepackage[parfill]{parskip}    		% Activate to begin paragraphs with an empty line rather than an indent
\usepackage{graphicx}				% Use pdf, png, jpg, or eps§ with pdflatex; use eps in DVI mode
								% TeX will automatically convert eps --> pdf in pdflatex	
\usepackage{adjustbox}	
\usepackage[section]{placeins}


%% LaTeX Preamble - Common packages
\usepackage[utf8]{inputenc}
\usepackage[english]{babel}
\usepackage{textcomp} % provide lots of new symbols
\usepackage{graphicx}  % Add graphics capabilities
\usepackage{flafter}  % Don't place floats before their definition
\usepackage{amsmath,amssymb}  % Better maths support & more symbols
\usepackage[backend=biber]{biblatex}
\usepackage{amsthm}
\usepackage{bm}  % Define \bm{} to use bold math fontsx
\usepackage[pdftex,bookmarks,colorlinks,breaklinks]{hyperref}  % PDF hyperlinks, with coloured links
\usepackage{memhfixc}  % remove conflict between the memoir class & hyperref
\usepackage{mathtools}
\usepackage[T1]{fontenc}
\usepackage[scaled]{beramono}
\usepackage{listings}
\usepackage{physics}
\usepackage{tensor}
\usepackage{simplewick} 
\usepackage{tikz} 
\usepackage{import}
\usepackage{xifthen}
\usepackage{pdfpages}
\usepackage{transparent}
\usepackage{pgfplots}
\usepackage[compat=1.1.0]{tikz-feynman}
\usepackage{subfiles}
\usepackage{simpler-wick}
\usepackage{slashed}

%% Commands for typesetting theorems, claims and other things.
\newtheoremstyle{newline}
{\topsep} 
{\topsep} 
{} 
{} 
{\bfseries} 
{}
{.5em}
{\newline}
{}

\theoremstyle{newline} 
\newtheorem{theorem}{Theorem}

\theoremstyle{newline} 
\newtheorem*{thm}{Theorem}

\theoremstyle{newline} 
\newtheorem*{claim}{Claim}

\theoremstyle{newline} 
\newtheorem{example}{Example}

\theoremstyle{newline} 
\newtheorem*{defn}{Definition}

\newcommand{\Lagr}{\mathcal{L}} 
\newcommand{\vc}[1]{\mathbf{#1}}
\newcommand{\pdrv}[2]{\frac{\partial{#1}}{\partial{#2}}}
\newcommand{\thrint}[1]{\int d^3 \vc{x} \left( {#1} \right)}

%% QFT specific macros 
\newcommand{\intp}{ \int \frac{ d^3 p }{ (2 \pi)^3 } \, }
\newcommand{\ann}[1]{a_{ \mathbf{ #1 }}}
\newcommand{\crea}[1]{a^\dagger_{ \mathbf{ #1 }}}
\newcommand{\ve}[1]{ \mathbf{ #1 } } 
\newcommand{\mode}[ 1]{ e^{ i \mathbf{ #1 } \cdot \mathbf{x} }}
\newcommand{\nmode}[1]{ e^{  - i \mathbf{ #1 } \cdot \mathbf{x} }}
\newcommand{\freq}[1]{\omega_\mathbf{ #1} } 
\newcommand{\scal}[1]{\phi ( \mathbf{ #1 })} 
\newcommand{\mom}[1]{ \pi (\mathbf{ #1 })} 
\newcommand{\arr}{\rightarrow} 

\newcommand{\incfig}[1]{%
\def\svgwidth{\columnwidth}
\resizebox{0.75\textwidth}{!}{\input{./figures/#1.pdf_tex}}
}

\newcommand{\anop}[2]{ #1_\mathbf{#2}}
\newcommand{\crop}[2]{#1_\mathbf{#2}^\dagger}

\usepackage{helvet} 

%tikz decoration commands 
\usetikzlibrary{decorations.pathmorphing}


\title{Part III Quantum Field Theory}
\author{Afiq Hatta } 
\begin{document} 
\maketitle
\tableofcontents


\pagebreak 

\subfile{why_do_we_care_about_quantum_field_theory_.tex}
\subfile{the_problem_of_locality.tex}
\subfile{particle_indistinguishability_and_particle_conservation.tex}
\subfile{problems_with_qft.tex}
\subfile{scaling_in_qft.tex}
\subfile{classical_field_theory.tex}
\subfile{euler_lagrange_equations_for_a_point_particle.tex}
\subfile{promotion_of_point_particles_to_the_classical_field.tex}
\subfile{lorentz_invariance.tex}
\subfile{symmetries_and_noether's_theorem.tex}
\subfile{switching_over_to_the_hamiltonian_formalism.tex}
\subfile{promoting_discrete_operators_to_fields.tex}
\subfile{reviewing_the_harmonic_oscillator_in_one_dimension_in_quantum_mechanics.tex}
\subfile{promoting_fields_to_operators.tex}
\subfile{creating_particle_states.tex}
\subfile{quantising_the_free_complex_scalar_field.tex}
\subfile{introducing_time_with_the_heisenberg_picture.tex}
\subfile{causality_in_qft.tex}
\subfile{propagators.tex}
\subfile{the_feynman_propagator.tex}
\subfile{interacting_fields_and_the_approach_with_perturbation_theory.tex}
\subfile{why_do_we_need_interacting_fields_and_what_are_'valid'_conditions_for_perturbation_expansions.tex}
\subfile{the_interaction_picture.tex}
\subfile{_scattering_.tex}
\subfile{wick's_theorem.tex}
\subfile{feynman_diagrams.tex}
\subfile{scattering_revisited.tex}
\subfile{amplitudes.tex}
\subfile{_computing_the_two_point_correlation_function_for_the_ground_state_of_our_perturbed_hamiltonian_.tex}
\subfile{wick's_theorem.tex}
\subfile{applying_wick's_theorem_for_the_interaction_term.tex}
\subfile{feynman_phi_rules.tex} 
\subfile{greens_function_vacuum.tex}
\subfile{cross_sections.tex} 
\subfile{the_dirac_equation.tex}
\subfile{what_does_the_dirac_equation_give_us.tex}
\subfile{warming_up_with_spinors_from_the_$so(3)$_rotation_group.tex}
\subfile{lorentz_tranformations_and_spinors.tex}
\subfile{constructing_the_chiral_representation_for_the_dirac_equation.tex}
\subfile{motivating_lorentz_invariance_of_the_dirac_equation.tex}
\subfile{constructing_boosts_and_rotations_in_block_diagonal_form.tex}
\subfile{the_weyl_equations.tex}
\subfile{constructing_plane-wave_solutions_to_the_weyl_equation.tex}
\subfile{example_sheet_1.tex}
\subfile{question_1.tex}
\subfile{question_2.tex}
\subfile{question_3.tex}
\subfile{question_4.tex}
\subfile{question_5.tex}
\subfile{question_6.tex}
\subfile{question_7.tex}
\subfile{question_8.tex}
\subfile{example_sheet_2.tex}
\subfile{question_1.tex}
\subfile{question_2.tex}
\end{document} 
