\subsection{Why do we need interacting fields and what are 'valid' conditions for perturbation expansions?}
\textit{This section follows David Tong's notes, and some of Peskin and Schroesder}



\par 
So far our Lagrangian $ \mathcal{ L }$ has only been quadratic in the field
$ \phi $. We've referred to this as our free theory; it gives rise to 
the second order, linear PDE 
\[
\partial _\mu \partial ^\mu \phi + m^2 \phi = 0 
\] which we can solve and quantise exactly. 
However, even though our analysis of this has considered multi particle states, 
we have yet to consider theories in which there particles interact. 
For example, scattering scenarios. 

Interaction terms can be written as higher order terms to our 
free Lagrangian; 
\[
\mathcal{ L } = \frac{1}{2 } \partial _\mu \phi \partial ^\mu \phi  - \frac{1}{2 }m^ 2 \phi ^ 2 - \sum_{n=3 } \frac{ \lambda _n \phi ^ n }{ n !}
\] The $ \lambda _{ n }  $ coefficients are called coupling constants, 
because they couple our free theory with interaction terms. 
For example, we could choose our constants to give rise to what we call 
$ \phi ^ 4 $  theory, with $ \lambda_ 4 = \lambda  $ and  $ \lambda _ n $ zero otherwise: 
\[
\mathcal{ L } = \frac{1}{2 } \partial _\mu \phi \partial  ^ \mu \phi - \frac{1}{2 } m ^ 2 \phi ^ 2 - \frac{ \lambda }{ 4 ! } \phi ^ 4 
\] However, if we have hopes of trying to solve our 
theory for expansions like this in a perturbative fashion, 
we then need to ensure that the $ \lambda_ n $ terms are small. 
In other words, $ \lambda _ n  \ll 1 $. But what does this even mean? 
For something to be smaller than a number, it better first be dimensionless. 
So, let's try figure out what the dimensions of our objects in question are. 

Our action is 
\[
S = \int d^ 4 x \, \mathcal{ L }, \quad [ S]  =0 
\] Our action should be dimensionless. We've denoted our 
\textbf{mass dimension} of a quantity $ Q $ as $ [ Q] $,
which is why we wrote $ [ S] = 0 $ as the above. 
$ d^4 x $ is an integral, which means it has dimensions of $ 4 $
in length terms. But, in natural units, we have that 
$ [ M ] = [ L ] ^ { - 1} $. 
So, we have that  $ [ d^ 4 x ] = - 4$. 
This implies that our Lagrangian must have dimensions  $ [ \mathcal{ L } ] = 4$
so compensate. 

So, what is $ [ \phi ] $ ? To find this out, 
we look at our kinetic term $ \partial _\mu \phi \partial  ^\mu \phi$. 
Since we have two length differentials which contribute a $ - 2$ to the dimension, we 
can then infer that $ [ \phi ] = 1 $. 

This means that since $ [ \mathcal{ L } ]  = 4$, our dimension of $ [ \lambda_ n ] $
must satisfy  $ [ \lambda_n ] + n = 4 \implies $
\[
[ \lambda _ n ] = 4 - n
\] Let's go case by case 
\begin{itemize}
\item  If we have $ n = 3$, our dimension of  $ \lambda_3  $ is 1. 
This means, to force a dimensionless parameter what we actually require is that 
$ \lambda_3 / E  \ll 1 $, since $ E $ has the same dimensions as mass. 
This means that a perturbation of this form is only valid at high energies! 
High energy scattering is one case where we could apply this theory. This is 
called a \textbf{relevant} perturbation, 
because the perturbation is significant at low energies.
\item In the case where $ d = 4$, our mass dimensions $ [ \lambda_4 ]  =0 $, so we 
require that the perturbation is small when $ \lambda_4 \ll 1 $. 
This, the boundary case, is called a \textbf{marginal} operator.
We can deal with the infinities that come from this perturbation, 
so we call it a renormalisable theory. 
\item The perturbation $ \lambda_{ n } \phi^ n / n! $ are called \textbf{irrelevant} 
operators. This is because their associated dimensionless parameter is 
$ \lambda E^{ n - 4}$, so the perturbation expansion is only valid at low energies. 
Their effect is only significant at high energies. These are non re normalisable theories.   
\end{itemize}
Let's look at some of the properties of $ \phi^ 4 $ theory. 
This has Lagrangian 
\[
\mathcal{ L } = \frac{1}{2 } \partial _\mu \phi \partial  ^\mu \phi - \frac{1}{2 } m ^2 \phi ^ 2 - \frac{ \lambda \phi^ 4 }{4 !}, \lambda \ll 1 
\] We can already see from first glance that our Hamiltonian is going to contain 
a bunch of extra terms which are inherited from the last term. 
This means that our number operator 
\[
N = \int \frac{ d^ 3 p }{ ( 2 \pi ) ^ 3 } a_{ \vec{p} }^ \dagger a_{ \vec{p}}
\]  will satisfy $ [ H , N ] \neq 0 $. 
This already tells us that particle number will not be conserved. 
We can see this explicitly by expanding the last term, which will yield combinations
of the form 
\[
\dots \int a^\dagger_{ \vec{p}} a^\dagger_{\vec{p}'} a^\dagger_{\vec{p}''}a^\dagger_{ \vec{p}'''}
\] and other terms which create or destroy particles. 

Another example is scalar Yukawa theory. 
THis is what we get when we lump together a 
a $ \mathbb{ C} $ scalar field, a real scalar field, as well as an interaction term. 
\[
\mathcal{ L } = \partial _\mu\psi \partial ^ \mu \psi - \mu ^ 2 \psi \psi ^ * + \frac{1}{2 } \partial _\mu \phi \partial  ^ \mu \phi - \frac{m^2}{ 2 } \phi ^ 2 - g \psi \psi ^ * \phi  
\] This Lagrangian has been used for example in 
the interaction of mesons.
Now, our $ g $ term here is combined with three fields. 
This means that it's like a $ \lambda _ 3  $ term. 

This means it's a relevant operator. Hence, working perturbatively works 
for low energies. In a relativistic setting, 
$ E > m$. So, we can make perturbations small by taking 
$ g \ll m $. 
We can start to make preliminary statements about this Lagrangian by 
observing that  its invariant under the transformation 
\[
\psi \to e^{ i \theta } \psi, \quad \psi^ * \to e^{ - i \theta }\psi^ *  , \quad \phi \to \phi 
\] This means that our Noether current 
for this symmetry exists. This Noether current $ Q $ is our
particle minus anti particle number, and we have that $ [ Q, H ] = 0 $. 
Thus, with this Lagrangian the number of particles minus the number of antiparticles
stays constant. 

We have no conservation however, for scalar particles generated by  $ \phi$. 

\pagebreak 

