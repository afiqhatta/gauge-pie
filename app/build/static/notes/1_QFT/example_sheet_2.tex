\section{Example Sheet 2} 

In general, save ink by writing exponents 
with either $ e ^{ i x \cdot  ( p - q ) } $ together, 
or either $ e ^{ i x \cdot  ( p + q ) } $ together.

Save ink by not bothering to write the whole measure, 
and writing out just the operators we're concerned with. 
We can write out. 

\subsection{Question 1} 
Our stress energy tensor for $ \mu  = 1, 2, 3 $
is given by $ T^{  0 \mu }  = \partial  ^ \mu \phi \dot{ \phi }  = \eta^{ \mu \nu  }$. 

We can write out time evolution 
for $ a_{ p } $ explicitly 
by using the Heisenberg equation
of motion 
\[
	\frac{ d a_{ p } }{  dt }  = i [ H , a_ p ] = i E_ p a_ p 
\]

\subsection{Question 4} 
Notes on this question. 

We need to use the energy momentum tensor expansion. 
Tske the derivatives out. 

Write out $ x ^ j  $ by differentiating in the 
very beginning, then only differentiate by parts after that. 

Use liberally the argument involving odd and even. 


\subsection{Question 7} 

Be careful with counting! Especially with the double 8 diagram.  
In this question, we wish to verify the bubble diagram expansion for 
$ \phi ^ 4 $ theory, for vacuum to vacuum scattering. 
Specifically, we're interested 
in the expansion to $ \lambda ^ 2 $ order for 
$ \bra{ 0 } S \ket{ 0 } $, where, due to Dyson's formula, 
we have that 
\[
	S = 1 - \frac{i \lambda }{ 4 ! } \int_{ - \infty } ^ \infty dt H_ I ( x) 
	+ \frac{1}{2 } \frac{\lambda ^ 2 }{ (4 ! ) ^ 2  } \int_{ - \infty } ^ \infty 
	 dt_ 1 \int_{ - \infty} ^ \infty dt _ 2 \mathcal{ T } \left\{  
	 H _I (t_1, \vec{x} ) H _ I (t_2, \vec{x}) \right\} 
\] Writing this in terms 
of our Hamiltonian density instead, we have that this simplifies 
down to 
\[
	S = 1 - \frac{i \lambda }{  4 ! } \int d ^ 4 x \, \mathcal{ H } _ I ( x ) 
	+ \frac{1}{2 } \frac{\lambda ^ 2 }{ ( 4 ! ) ^ 2 } \int d^ 4 x_ 1  d^ 4 x_ 2 
	\mathcal{ T } \left\{  \mathcal{ H } _ I ( x _ 1 ) H _ I ( x _ 2 )  \right\} 
\] where we have our interaction picture Hamiltonian given 
by $ \mathcal{ H } _ I ( x) = \phi _ I ^4  ( x) $. 
Now,let's look at the contribution 
to the first order in $ \lambda $. This contribution 
is given by 
\[
  \frac{ - i \lambda }{ 4 ! } \int d ^ 4 x \,  \bra{ 0 } 
  \mathcal{ T  } \left\{  \phi ( x) \phi ( x) \phi ( x) \phi ( x)  \right\} \ket{ 0 } 
\] Now, due to Wick's theorem, the integrand 
can be expanded out in terms of all possible contractions 
of the two fields. There are $3$ ways to do this. 
This is because we could've 
contracted the first field with $ 3$ others, 
which leaves the last two fields to contract 
on their own. This means
that our total contribution from first order from this field 
is 
\[
	- \frac{ i \lambda }{ 8 } \int d^ 4 x \, D ( x - x ) D ( x -x) 
\] This is represented by a figure $ 8 $ diagram. 
Now, we aim to show that this matches up with the 
diagram shown in the exponential. 
To first order in $ \lambda $, our 
exponential is given by 
\begin{equation*}
	\feynmandiagram[small]{ 
		a [particle = \( x \) ]  -- [ghost, half left] c 
		-- [ghost, half left] a 
		-- [ghost, half left] b 
		-- [ghost, half left] a, 
	}; 
\end{equation*}
We know that from this diagram 
that, by assigning propagators to each line, that 
this is proportional to $ \int d ^ 4 x \, D ( x- x ) D ( x - x) $. 
But, what about the symmetry factors? This diagram 
has a symmetry factor of $ 8 $ since we assign a factor 
of two for being able to switch the ends of the bottom and top loops. 
In addition, we assign another factor of $2 $ since we 
can rotate the diagram $ 180 $ degrees.

Now, looking to second order in $ \lambda ^ 2 $, we'd
like to find the amplitude given by the expression 
\[
	\frac{\lambda ^ 2 }{2 ( 4 ! ) ^ 2 }
	\int d ^ 4 x \, d ^ y \mathcal{ T } \left\{  \phi_x \phi _ x \phi_ x \phi_ x \phi _ y \phi _ y \phi _ y \phi _ y  \right\} 
\] Let's examine the possible contractions that we get from this. 
We could have that the $ \phi _ x $ and $ \phi _ y $ terms are 
contracted internally amongst themselves. There are $3 $ ways 
to contract the $ \phi _ x $ amongst themselves and $ 3 $ ways to 
contract the $ \phi _ y $ amongst themselves, so we have $ 9 $ ways in total. 
This means that our contribution is given by 
\[
	\frac{1}{2 } \frac{\lambda ^2 }{ 8 ^ 2 } \int d ^ 4 x d ^ 4 y D ( x - x)  D( x - x ) D ( y - y ) D ( y - y) 
\] The factor of $ \frac{1}{2 } $ comes from Dyson's formula. 
As a diagram, this is represented by the diagram 
\begin{equation*}
	\feynmandiagram[small]{ 
		a [particle = \( x \) ]  -- [scalar, half left] c 
		-- [scalar, half left] a 
		-- [scalar, half left] b 
		-- [scalar, half left] a, 
		}; 
	\feynmandiagram[small]{ 
		a [particle = \( y \) ]  -- [scalar, half left] c 
		-- [scalar, half left] a 
		-- [scalar, half left] b 
		-- [scalar, half left] a, 
		};
\end{equation*}
Our symmetry factor for this diagram is $ 2 \times 8 ^ 2 $, since, 
as we discussed before, we have a symmetry of 8 for each diagram. 
In addition, we could've swapped the diagrams as well. 
This agrees with the above. 

The next thing to do is that we could have contracted 
each of the $ \phi_ x  $  terms with $ \phi _ y $. Since 
we have four choices to contract the first  $ \phi _ x $
with, then 3, then 2, we have that our contribution 
to this term is \begin{align*} 
	\frac{\lambda ^ 2 4 ! }{ 2 ( 4 ! ) ^ 2 } &  \int d^ 4 x \, d ^ 4 y\, 
	D ( x - y ) D ( x - y ) D ( x - y ) D ( x - y )  = \\ & 
	\frac{ \lambda ^ 2 }{  2 ( 4 ! ) }  \int d^ 4 x \, d ^ 4 y\, 
	D ( x - y ) D ( x - y ) D ( x - y ) D ( x - y ) 
\end{align*} This corresponds to the diagram 
\begin{equation*}
	\feynmandiagram[small]{ 
		a [particle  = \( x \) ] -- [scalar, half left, looseness = 1.5] b [particle = \( y \) ], 
		a -- [scalar, half right, looseness = 1.5] b, 
		a -- [scalar] b, 
		a -- [scalar] b, 
	}; 
\end{equation*}

Finally, our last possible type of contraction is 
what happens when we contract just two of the $ \phi _ x $ 
fields with two of the $ \phi _ y $ fields, 
and loop the rest. 
There are $ \begin{pmatrix}  4 \\ 2  \end{pmatrix}   = 6 $ ways 
to choose which to $ \phi _ x $ fields we contract with $ \phi _ y $, 
and once we chose the $ \phi _ x $ fields, we have $ 12  = 4 \times 3 $ 
options to which we can contract them. This means that 
the contribution gained from this is 
\begin{align*} 
	- \frac{ \lambda ^ 2 6 \times 4 \times 3}{2 ( 4 ! ) ^ 2  } & 
	\int d ^ 4 x \,    d ^ 4 y D ( x - x ) D ( y -x ) D ( y - x ) D ( y - y )  
	=   \\  & - \frac{\lambda ^ 2 }{16  }	\int d ^ 4 x \,    d ^ 4 y D ( x - x ) D ( y -x ) D ( y - x ) D ( y - y )  
\end{align*} This is associated with the 
diagram 
\begin{equation*}
	\feynmandiagram[small]{ 
	a -- [scalar, half left] b [particle = \( x\) ] -- [scalar, half left] a , 
	b -- [scalar, half left] c [particle = \( y \) ] -- [scalar, half left] b, 
	c -- [scalar, half left] d -- [scalar, half left] c, 
}; 
\end{equation*}

We can switch around the two loops as well as the two propagators in the centre.
Hence, we've showed that to second order in $ \lambda$, 
the combinatoric and symmetry factors work 
out for each diagram when we exponentiate the bubble 
diagrams.

\pagebreak 
\subsection{Question 8} 
For the Yukawa term, we need only attach 
p> 
$ i g $ to each vertex. 
We need to take into account combinatoric factors for 
each vertex if the 
same time of propagator is going into them. 



To find the mass dimensions of $ g, h, k, l $, we 
need to first find the mass dimensions of the fields $ \phi $ and $ \psi $. 
Our scalar action needs to be dimensionless, so 
\[
\int d ^ 4 x \, \frac{1}{2 } \partial  _ \mu \phi \partial  ^ \mu \phi 
\] has mass dimension $ 0 $. This implies that $ [ \phi ] = 1 $. 
Similarly, we find that  $ [ \psi ] = 1 $. 
This means that the mass dimensions of $ g, k$ are  $ [ g ] = [ k]  = 1$. 
We have that $[ h ] = 0 $, and $ [ l ] = - 1$. 
This theory contains both relevant, marginal and irrelevant operators.
Thus, the theory is non-renormalisable since the $ l $ dimension 
is irrelevant.

We have four interaction terms, 
which means that we get 4 distinct types 
of diagrams, each with their own set of Feynman rules. 
Let's tackle the Yukawa interaction first. 
We know the momentum rules for this. 
Now, for the interaction $ h | \psi | ^ 4  = h \psi \psi ^ * \psi \psi ^ * $, 
we have that our interaction vertex is given by 
\begin{equation*}
 \feynmandiagram[small]{ 
	 a -- [fermion] b, 
	 c -- [fermion] b, 
	 b -- [fermion] d, 
	 b -- [fermion] e
	}; 
\end{equation*}
Now, with this kind of diagram, our propagator 
comes from contracting just $ \psi , \psi ^ *$ together, 
which gives us a propagator in momentum space which looks 
like 
\[
\frac{i}{p ^ 2 - \mu ^ 2 + i \epsilon } 
\]
For the term $ k \phi ^ 3 $, we have that our vertices 
look like 
\begin{equation*}
\feynmandiagram[small]{ 
	 a -- [scalar] b, 
	 c -- [scalar] b, 
	 d -- [scalar] b, 

}; 
\end{equation*}
We need to also add in combinatoric factors, so we attach a 6 to this. 
Each of these objects come with a propagator which comes from 
contracting two scalars together 
\[
\frac{i }{ p ^ 2 - m ^ 2 + i \epsilon }
\] 
Finally, the non-trivial one is the propagator that comes 
from interactions with $ \phi \partial  _ \mu \psi \partial  ^ \mu \psi ^ *  $. 
In this case, we have a different kind of propagator! 
Non-zero contributions come from contractions of the form 
\[
\wick{ \c \phi \c \phi  \c \psi ^ * \partial _ \mu \c \psi 
\c \partial  ^ \mu \c \psi ^ * \c \psi }
\] Differentiating the contraction of $ \psi $ and $ \psi ^ * $
pulls out a factor of  $ i \pi ^ \mu $. Thus, since 
we'll always have an even number of propagators involving the anti-nucleons, 
for each pair we attach a factor of 
\[
\frac{ - p \cdot  q  }{ p ^ 2-m ^ 2 + i \epsilon }
\] 

\pagebreak 
\subsection{Question 9} 
We want to find the decay width 
for the scattering process $ \phi \to \psi \overline{ \psi } $. 
For studying the decay width up to order $ g ^ 2 $, 
we need only consider the trivial Feynman diagram with 
one vertex (since $ \Gamma \sim | \mathcal{ M } | ^ 2 $.  
\begin{equation*}
\feynmandiagram[small]{
	a -- [scalar] b, 
	c -- [fermion] b -- [fermion] d, 
}; 
\end{equation*}
Now, since this already imposes momentum conservation, 
our amplitude $\mathcal{ M } $ is just $  i g  $.
Thus, our decay width is given by the formula 
\[
\Gamma = \frac{1}{2m } g ^ 2 \frac{1}{( 2 \pi ) ^ 2}  \int 
\frac{d ^ 3 q_1  d ^ 3 q_2 }{ 2 E _{ q_1 } 2 E _{ q_2 } } \delta ^ 4 ( p_1- q_1 - q_2) 
\] The trick here is to 
rewrite the invariant measure $ \frac{ d ^ 3 q_2 }{ ( 2 E_{ q_2 } }$ as $  \delta^ 4 \left(  
( q_2 ^ 2 - \mu ^ 2 \right)  $ (notice we're using $ \mu ^ 2 $ since this 
comes from the nucleon. 
Substituting this in, then integrating over $ q_2 $, 
we get that our decay width is 
\[
\Gamma = \frac{g ^ 2 }{ ( 2 \pi ) ^ 2 2m } \int \frac{d ^ 3 q_1 }{ 2 E_{ q_1 } } \delta ^  4
\left(  ( p_1 - q_1 ) ^ 2  - \mu ^ 2  \right) 
\] Now, we'd like to find the set of $ q_1 $ such that $ ( p_1 - q_1 ) ^ 2  = \mu ^ 2 $. 
Our relativistic dispersion relation gives us that $ p_1 ^ 2 = m ^ 2 $, 
and $q_1 ^ 2 = \mu ^ 2 $. Also, in the convention of calculating 
decay widths, we work in the rest frame of the incoming particle (in this case, 
the meson). This means that we take $ p_1 = ( m , 0 ) $. 
\begin{align*}
p_1 ^ 2 + q_1 ^ 2 - 2 p_1 \cdot  q_1  &=  \mu ^ 2 \\
m ^ 2 + \mu ^ 2  - 2 m E_{ q_1 }  &= \mu ^ 2 \\
m ^ 2  - 2m \sqrt{ \mu ^ 2 + \vec{q} _ 1 ^ 2 } &=  0  \\
\| \vec{q} _ 1 \|  & = \pm \sqrt{ \frac{m ^ 2 }{ 4 }  - \mu ^ 2 } 
\end{align*}
Our decay width is the integral 
\[
\Gamma = \frac{g ^ 2 }{ 16 \pi ^ 2 m } \int \frac{d ^ 3 q_1 }{ E_{ q_1 } } \delta ^ 4 \left(  
m ^ 2- 2m \sqrt{ \mu ^ 2 + \| \vec{q} _ 1 \| ^ 2 } \right) 
\] Selecting the positive root in the delta function, using 
our standard identity for a delta function of a function, this 
gives that the integral is 
\begin{align*} 
\Gamma & = \frac{g ^ 2 }{ 16 \pi ^ 2 m } \int \frac{d ^ 3 q_1 }{ E_{ q_1 } } 
\frac{\delta \left(  | \vec{q} _ 1 |   - | \vec{q} _ 1 ^ * |   \right) }{
 \frac{2m | \vec{q} _ 1 ^ * | }{ \sqrt{ \mu ^ 2 + |\vec{q} _ 1 ^ *  | ^ 2 }  }} \\
 &=  \frac{ g ^ 2 }{ 16 \pi ^ 2 m } \int d |\vec{q} _ 1 | \frac{ 4 \pi | \vec{q} _ 1 | ^ 2 
 \delta ( | \vec{q} _ 1 |  - | \vec{q} _ 1 ^ * | ) }{2m E_{ q _ 1  } \frac{| \vec{q} _ 1 ^ * | }{
\sqrt{ \mu ^ 2 + | \vec{q} _ 1 ^ * | ^ 2 } }} \\
\end{align*} 
In the last two lines we converted this integral into spherical 
coordinates. 
Integrating out the delta function, this is just 
\begin{align*} 
 \Gamma & = \frac{g ^ 2 ( 2 \pi ) }{ 16 \pi ^2 m} \frac{1}{m } \frac{ |\vec{q} _ 1 ^ * | | E_{ q_ 1 ^ * }}{
 | E_{ q_ 1 ^ * } | } \\ 
 &=  \frac{ g ^ 2 }{ 8 \pi m ^ 2 } \sqrt{ \frac{m ^ 2 }{ 4 }  - \mu ^ 2 }   \\ 
 &=  \frac{g ^ 2 }{ 16 \pi m ^ 2 } \left(  1 -\frac{4 \mu ^ 2 }{ m ^ 2 }  \right)^{ \frac{1}{2 } }  \\
 &=  \frac{g ^ 2 }{ 16 \pi m } \sqrt{ 1 - ( 4 \mu ^ 2 / m ^ 2 ) }  \\
\end{align*} 
This has dimensions of mass since $ g  $ has mass dimension $ 1 $, thus 
decay make sense since this is a rate. 
\pagebreak 

\section{Example Sheet 4}

\subsection*{Question 1} 

\subsubsection{Part a}
For $\psi \overline{ \psi } \to \psi \overline{ \psi }$ scattering, 
the only two possible diagrams to second 
order in $ \lambda $ are 
\begin{figure}[htpb]
\centering
\input{e4_q1.pdf_tex}
\caption{E4_q1}%
\label{fig:e4_q1}
\end{figure}

Due to Feynman rules, since we have two vertices 
in each diagram, we have a factor of $ ( - i \lambda  ) ^  2 $. 
For the first diagram, 
our ingoing nucleon spinor contributes $ u\left( \vec{p} \right)^s $, 
and the ingoing anti-nucleon contributes $ \overline{v }\left(  \vec{q} \right)^ r $. 
Since they meet at a vertex, we need to contract them to give 
a factor of $ u \left( \vec{p}  \right)  ^ s \cdot  \overline{ v } \left( 
\vec{q} \right)  ^ r $. Similarly, for the outgoing nucleon and anti-nucleon, 
we have a factor of 
$\overline{ u } \left( \vec{p} '   \right) ^{ s '  }  \cdot  
v \left( \vec{q} '  \right)  ^{ r ' }  $. 
Our nucleon propagator in the centre contributes a factor of 
$ \frac{1}{\left( \left(  p + q  \right)^ 2   + \mu ^ 2  \right) }$. 
This gives a contribution of 
\[
\left(  - i \lambda  \right)  ^ 2 \frac{
\left[  \overline{ v } ^ r \left( \vec{q}  \right) \cdot  u ^{ s } \left( \vec{p}  \right)   \right]  \left[  \overline{ u }^{ s ' } \left( \vec{p} '  \right) \cdot  v ^{ r '  } \left( \vec{q} '  \right)    \right]  }{ \left(  p + q  \right)  ^  2 - \mu ^ 2  }
\] 
For the second diagram, the appropriate contractions 
give
\[
\left(  - i \lambda  \right)  ^ 2 
\frac{ \left[  \overline{ u } ^{ s ' } \left( \vec{p} '   \right) 
\cdot  u ^{ s } \left( \vec{p}  \right)  \right]  \left[  
\overline{v } ^{ r } \left( \vec{q}  \right)  \cdot  v ^{ r  ' } 
\left( \vec{q} '  \right)  \right]  }{ \left(   p -  p '  \right)  ^ 2 
- \mu ^ 2 }
\] But, due to fermi-dirac statistics, we need to 
add a minus sign relative to the first diagram since we're 
'switching' the out-going fermions. This 
gives the final contribution shown in the question. 

\subsubsection*{Part b}

\begin{figure}[htpb]
\centering
\input{e4_q1_b.pdf_tex}
\caption{$\psi \phi \to \psi \phi $}%
\label{fig:e4_q1}
\end{figure}

In the first diagram, we have an 
incoming nucleon with spinor $ u\left( \vec{p}  \right)  ^ s $ and 
and outgoing nucleon with spinor $ \overline{ u } \left( \vec{p '} \right) ^{ s ' } $. 
These spinors are connected with the nucleon propagator
with four momentum $  p + q$. 
The nucleon propagator contributes the factor 
\[
 \frac{ \gamma _ \mu \left(  (p ^ \mu +q ^ \mu) + m   \right)  }{ 
 \left(  p  + q  \right)  ^ 2  - m ^ 2  }
\] This is a matrix object. 
We need to contract the spinor objects on either side to 
give a contribution of 
\[
\left(   - i \lambda  \right)  ^ 2 \frac{
\overline{ u } ^{ s  ' } \left( \vec{p} '   \right) \gamma _ \mu 
\left(  (p ^ \mu + q ^ \mu) + m   \right)  u ^ s \left( \vec{p}  \right)  }{ 
\left(  p + q  \right)  ^ 2 - m ^ 2 }
\]
In the second diagram, we have the same 
story but we've switched around the 
meson lines. This means the 
4 momentum of our nucleon propagator is different. 
We \textbf{don't } add an extra minus sign here 
since mesons are bosons. We get the contribution 
\[
\left(   - i \lambda  \right)  ^ 2 \frac{
\overline{ u } ^{ s  ' } \left( \vec{p} '   \right) \gamma _ \mu 
\left(  ( p ^ \mu  - q ^{  ' \mu }) + m   \right)  u ^ s \left( \vec{p}  \right)  }{ 
\left(  p  - q'   \right)  ^ 2 - m ^ 2 }
\]
Adding these two amplitudes 
together gives our total 
contribution. 

\subsection*{Question 3}
According to Feynman rules, 
each vertex contributes a factor of $  - i e \gamma ^ \mu $. 
The ingoing and outgoing polarisation vectors of 
the photons are contracted along these 
vertex indices, and the 
spinor indices are contracted across the propagator as before. 
This means that we get, 
for the diagrams below

\begin{figure}[htpb]
\centering
\input{e4_q3.pdf_tex}
\caption{Nucleon and photon 
to nucleon and photon scattering}%
\label{fig:e4_q1}
\end{figure}

For the first diagram, 
we have the internal fermion propagator which 
contributes a matrix factor of 
\[
\frac{\slashed{p} + \slashed{q}  + m  }{\left(  p + q   \right)  ^ 2 
 - m^ 2 }
\] From our vertices, 
we attach a factor of $  - i e \gamma ^ \mu $ to each one, 
which we have to contract with the polarisation vectors $\epsilon_{ \text{out } }
\left( \vec{q} '   \right) $ and $ \epsilon_{ \text{ in } } \left(  \vec{q} \right) $. Putting this 
on either side of the matrix, and then contracting 
with our ingoing and outgoing spinors, 
means that we get our amplitude from this diagram as 
\[
i\mathcal{ A }_ 1  = i \left(  - i e  \right)  ^ 2 
\overline{ u } ^{ r ' } \left( \vec{p} '  \right)  
\slashed{\epsilon}_{ \text{out } } \frac{\slashed{p} + \slashed{q} + m }{
\left(  p + q  \right)  ^ 2  - m ^ 2 } \slashed{\epsilon}_{\text{in}}u ^ s \left(  \vec{p} \right) 
\] 
The contribution from the second diagram is exactly 
the same, except that for the momentum 
associated with the internal propagator, 
it's now $ p  - q' $. 
\[
i \mathcal{ A } _2 =	i \left(  - i e  \right)  ^ 2 
\overline{ u } ^{ r ' } \left( \vec{p} '  \right)  
\slashed{\epsilon}_{ \text{out } } \frac{\slashed{p}  -  \slashed{q} '  + m }{
\left(  p  - q'   \right)  ^ 2  - m ^ 2 } \slashed{\epsilon}_{\text{in}}u ^ s \left(  \vec{p} \right) 
\] The sum of these contributions 
give us our total amplitude for this scattering 
problem. 

Now, what happens which we replace our polarisation vector 
$ \epsilon_{ \text{ in } } \left(  \vec{q}  \right)  $ with the 
four vector $ q $? 
For our first amplitude $ i \mathcal{  A} _ 1 $, schematically we 
have that 
\begin{align*}
i\mathcal{ A } _ 1 & \propto \frac{\left(  \slashed{p} + 
\slashed{q} + m  \right) }{\left( p + q  \right)  ^ 2 - m ^ 2 } \slashed{q} \lu ^ s \left( \vec{p} \right) \\
&= \frac{\slashed{p}\slashed{q} + 
\slashed{q}\slashed{q} + \slashed{q}\slashed{p}}{
\left(  p + q  \right)  ^2   - m ^ 2 } u ^ s \left( \vec{p} \right)  \\
&=  \frac{ 2 p \cdot  q }{\left( p + q  \right)  ^ 2  - m ^ 2 }
u ^ s \left( \vec{p} \right) \\
&=  1 
\end{align*}
There are some important justifications
to be made with the calculations above. 
Firstly, we have that since $ q $ is a null vector, 
\[
\slashed{q} \slashed{q} = q ^ 2  = 0 
\] Secondly we subbed out the $ m $ factor 
by using the on-shell spinor condition 
\[
\left(  \slashed{p} - m   \right) u ^ s \left( \vec{p}  \right)   = 0 
\] Finally, 
we use the fact that 
\[
\slashed{p} \slashed{q} + \slashed{q} \slashed{p} = 2 p \cdot  q 
\] Similarly, for our second factor, we get a 
$  - 1 $ from cancellations. So adding the 
two together gives us a 0. 

Now, to calculate the cross-section, 
we need to take the spin-sum amplitude. 
To do this, we 
not need to only average over the incoming spins 
and sum over out-going spins, but 
we also need to average and sum over incoming 
and outgoing photons as well. 

Our expression for our 
differential scattering with respect to the $ t $ 
Mandelstam variable is 
\[
\frac{ d \sigma }{ dt }  = \frac{| A | ^ 2  }{ 16 \pi \lambda\left( 
s, m_1^ 2 , m_2 ^ 2 \right) }
\] where $ \lambda\left( x, y, z  \right)   = 
x ^ 2 + y ^ 2 + z ^ 2  - 2xy  - 2 yz  - 2 zx $. 
In this case, $ | A |^ 2 $ is the 
spin and polarised average. We decompose this into the 
separate amplitudes we calculated earlier, so 
that 
\[
| A | ^ 2 = | A _ 1 | ^ 2 + | A _ 2 | ^ 2 + 2 Re \left(  A_ 1 ^ * 
A _ 2 \right) 
\] 
\subsection{Question 4}
To introduce scalars into our Lagrangian, 
we need to add in covariant derivatives to couple 
the photons from $ A _ \mu $ to charged scalars $ \phi $.
This acts as 
\[
	D ^ \mu \phi  = \partial  ^ \phi + ie A ^ \mu \phi 
\] This enforces gauge invariance when we put 
it in  the Lagrangian 
\[
 \mathcal{ L }  = D ^ \mu \phi D _ \mu \phi ^\dagger = 
 \partial  ^ \mu \phi \partial  _ \mu \phi ^\dagger 
 + ie A ^ \mu \left(  \phi \partial  _ \mu \phi ^\dagger  - 
 \phi ^\dagger \partial  _ \mu \phi \right)  + e ^ 2 A ^ \mu A ^ \mu \phi ^\dagger
 \phi 
\] The first term is a kinetic term, 
and our two interaction terms give us two vertices 
for our Feynman diagrams which we need to calculate Feynman rules 
for. We get this from Fourier transforming. 

\begin{figure}[htpb]
\centering
\input{e4_q4.pdf_tex}
\caption{Nucleon and photon 
to nucleon and photon scattering}%
\label{fig:e4_q1}
\end{figure}

The first diagram contributes a factor of $ i e ( p^ \mu + q^ \mu )$ 
due to the derivative terms. 
The second diagram contributes a factor 
of $ 2 i e ^ 2 \eta _{ \mu \nu  } $. 
We have two identical photons from the second 
diagram, so we need to add an additional 
combinatoric factor of $ 2 $. 



\subsection{Question 5}
In QED, the only possible diagram which 
we can get is the one below

\begin{figure}[htpb]
\centering
\input{e4_q5.pdf_tex}
\caption{Nucleon and photon 
to nucleon and photon scattering}%
\label{fig:e4_q1}
\end{figure}

As before, each vertex in this diagram 
contributes one factor of $  -i e \gamma ^ \mu $. 
We contract the incoming fermions  $ e ^ + $ and $ e ^ - $. 
Our progagator is just a factor of $ \frac{i \eta_{ \mu \nu } }{ \left( p + q  \right)  ^ 2 }$, 
since it's a photon propagator, and this contracts the two 
indices for the $ \gamma ^ \mu $ matrices.  


\subsection*{Question 6}

Our amplitude from this 
scattering problem is 
\[
 i \mathcal{  A}  = \frac{i e ^ 2 }{q ^ 2 } 
 \left( \overline{v}_{ s ' } \left( p'  \right)  \gamma ^ \mu 
 u _ s \left( p  \right) \right) \left( 
 \overline{ u } _ r\left( k  \right)  \gamma _ \mu 
 v_{ r' } \left( k'  \right)  \right)    
\] 
We calculate the spin sum-average of $ | A | ^ 2 $. 
\[
 | \overline{ A } | ^ 2  = 
 \frac{1}{4 } \sum_{ r, r', s , s' } 
 \left[ \overline{v}_{ s'  }\left( p'  \right)  \gamma ^ \mu 
 u _ s \left( p  \right)  \right]  \left[  
 \overline{ u } _ s \left( p  \right)  \gamma ^ \nu v _ s \left( p'  \right)  \right]  \left[  
 \overline{ u } _{ r } \left( k  \right)  \gamma _ \mu v _{ r' } 
 \left( k '  \right)  \right]  \left[  
 \overline{ v } _{ r' } \left( k '  \right)  \gamma _ \nu 
 u _ r \left( k  \right) \right] 
\] 
In the massless case, we use the completeness 
relation for $ u  $ and $ v $ this simplifies to 
\[
 | A | ^ 2 =  \frac{1}{4 } \frac{ e ^ 4 }{ q ^ 4 } 
 \tr \left( \slashed{p} ' \gamma ^ \mu \slashed{p } \gamma ^ \nu  \right)  \tr \left( \slashed{k} \gamma _ \mu \slashed{k} ' \gamma _ \nu  \right) \] where $ q ^ 2 = \left( p + p'  \right)  ^ 2 
  = 4  E^ 2 $
 Now here, we take the $ p , p ' , k , k' $ indices 
 out and use the trace relation 
 \[
  \tr\left( \gamma ^ \alpha \gamma ^ \beta \gamma ^ \gamma 
  \gamma ^ \delta \right)   = 
  4 \left( \eta ^{ \alpha \beta } \eta 
  ^{ \gamma \delta }  - \eta ^{ \alpha \gamma } \eta ^{ 
  \beta \delta } + \eta ^{ \alpha \delta } \eta ^{ \beta \gamma } \right) 
 \] After some simplification of the above, 
 we arrive at the amplitude being 
 \[
  | A | ^ 2 = \frac{ e ^ 4 }{ 2 E ^ 4 } 
  \left( \left( p \cdot  k  \right)  \left( p ' \cdot  k '  \right) + \left( p ' \cdot  k  \right)  \left( p \cdot  k '  \right)    \right) 
 \] 
 We recognise that we can use the Mandelstam 
 variables to replace these momenta, 
 and so we get that 
 \[
  | A | ^ 2 = \frac{1}{8 } \frac{ e ^ 4 }{ E ^ 4 } 
  \left( t ^ 2 + u ^ 2  \right) 
 \] However, since we're scattering 
 identical particles, we use the fact that, 
 in this massless case, 
 \[
  t  =  - 2p ^ 2 \left( 1 - \cos \theta  \right) =  - 2 
  E ^ 2 \left(  1 - \cos \theta  \right) , \quad 
  u  = 2p ^ 2 \left(  1 + \cos \theta  \right) = 2 E ^ 2 
  \left(  1 + \cos ^ 2 \theta  \right) 
 \] where $ \theta $ is our scattering angle for this problem. 
 This means that 
 \[
	 | A | ^ 2 = e ^ 4  \left(  
	 1 + \cos ^ 2 \theta \right)  
 \] We substitute this into our expression for 
 our differential cross section 
 \[
 \frac{ d \sigma }{ d \Omega  }   = 
 \frac{ | A  | ^ 2 }{ 64 \pi ^ 2 s }  = \frac{
 e ^ 4 \left( 1 + \cos ^2 \theta  \right)  }{ 64 \pi ^ 2 s }
 \] Integrating along the whole solid angle, this gives 
 \[
  \sigma  = \frac{e ^ 4 }{ 12 \pi s }  = \frac{4 \pi \alpha ^ 2 }{ 3 s } 
 \] 

The centre of mass frame 
is the frame where our total three momentum is zero. 
This means that if we 
have to incoming 4-momenta $ p_1 , p_ 2 $ , say in a $ 2 \to 2 $ 
scattering problem, then 
we have that $ \vec{p}_ 1 = - \vec{p} _ 2 $. As a result, 
we also have that 
\[
	( p_1  + p_2  )^ 2  = \left(  E_1 + E_ 2  \right)^ 2  
\]  
We use this to our advantage in simplifying aspects of the 
problem. 

\pagebreak 
