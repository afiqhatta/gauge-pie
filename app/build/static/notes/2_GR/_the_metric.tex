\subsection{ The metric}

\begin{defn}{(The metric tensor). } 
We'll now do introduce a tensor object which 
turns our tangent space into an inner product space. 
We do this by introducing an object called a metric, 
which intuitively has been a way in which we define 
the notion of 'distance' in a space. 
A metric $ g $ is a $ ( 0, 2 ) $ tensor that is
\begin{itemize}
\item  symmetric $ g ( X, Y )  = g ( Y , X) $
\item non-degenerate: $ g ( X, Y )_ p = 0, \quad \forall Y_p \in T_p ( \mathcal{ M }) \implies X_p = 0$ 
\end{itemize}
\end{defn}
Notice that our condition for non-degeneracy is \textbf{not}
the same as having a point where  $ g ( X, X )_p = 0 $ as we
shall soon see. 
In a coordinate basis, $ g = g_{ \mu \nu } dx ^ \mu  \otimes dx^ \nu $.
The components are obtained by our standard way of 
subbing in basis vectors into our tensor. 
\[
	g _{\mu \nu  } = g \left( \frac{\partial }{\partial  x^ \mu } , \frac{\partial }{\partial x^ \nu } \right)  
\]  We often write this as 
a line element which we call 
\[
ds^ 2 = g_{ \mu \nu } dx^ \mu dx ^ \nu 
\] This is something that perhaps we're more 
familiar with. 
Since our metric is non-degenerate, it is 
a theorem in linear algebra that we can diagonalise this thing, and furthermore 
we have no zero eigenvalues. 
If we diagonalise $ g _{ \mu \nu }$, 
it has positive and negative elements (non are zero). 
The number of negative elements is called the 
signature of the metric. 
There's a theorem in linear algebra (Sylvester's law of inertia) 
which says that the signature is invariant, which means that 
it makes sense to talk about signatures in a well defined sense. 

\subsubsection{Riemannian Manifolds} 
A Riemannian manifold is a manifold with metric 
with signature all positive. 
For example, Euclidean space in $ \mathbb { R} ^ n $ endowed with 
the usual Pythagorean metric. 
\[
g = dx^ 1 \otimes dx ^ 1 + \dots + dx^ n \otimes dx^ n 
\]  A metric gives us a way to measure the
length of a vector $ X \in \mathcal{ X } ( \mathcal{ M } ) $. 
Since our signature is positive, we have that $ g ( X, X ) $ is a 
positive number, and hence we can take a square root to 
define a norm. 
\[
| X | = \sqrt{ g ( X, X ) } 
\] We can also find the angle between vectors, where
\[
g ( X, Y ) = | X| | Y | \cos \theta 
\] It also gives us a way to measure the distance between two points, 
$ p $, $ q $. Along the curve 
\[
\sigma : [ a, b ] \to \mathcal{ M }, \quad \sigma ( a) = p, \sigma ( b ) = q
\] our distance is given by the integral of the metric at 
that point where $ X $ is the tangent to the curve' 
\[
s = \int_ a ^ b dt \, \sqrt{ g ( X, X) } \mid_{ \sigma( t) }  
\] where at each point $ X $ is our tangent to the curve. 
If our curve has the coordinates $ x^ \mu ( t ) $, 
then our distance is 
\[
 s = \int_ a ^ b dt\, \sqrt{ g_{ \mu \nu } ( x) \frac{dx^ \mu }{ dt } \frac{ dx^ \nu }{ dt }} 
\]
Note that this notion of distance still makes sense since and 
is well defined since it's easy to check that 
$ s $ is invariant under re-parametrisations of our curve. 

\subsubsection{Riemannian Geometry} 
A Lorentzian manifold is a manifold equipped with 
a metric of signature $ ( - + + ... ) $. 
For example, Minkowski space is  $ \mathbb{ R} ^ n $
but our metric is 
\[
\eta = - dx^ 0 \otimes dx^ 0 + dx^ 1 \otimes dx^ 1 + \dots + dx^{ n-1 } \otimes dx^{ n- 1}
\] with components
\[
\eta_{ \mu \nu  } = diag ( - 1, 1 \dots, 1 )  
\]  This is slightly different to our Riemannian manifold 
case since now we can have vectors with negative or zero length. 
We classify vectors $ X_ p \in T_p ( \mathcal{ M } ) $
as 
\[
g ( X_p , X_p ) = \begin{cases}
 < 0 & \text{ timelike} \\
 = 0 & \text{null} \\
 > 0 & \text{spacelike}
\end{cases}
\] At each point $ p \in \mathcal{ M } $, we draw null tangent vectors called lightcones, 
and as we'll soon see, this region outlines our area of possible causality.

A curve is called timelike if it's tangent 
vector at every point is timelike. We can see this in the figure 
where we have two light-cones for future and past time. 
\begin{figure}[h]
	\centering
	\hspace*{3cm}\includegraphics[width=0.5\linewidth]{figures/lightcone.png}
	\caption{Here we show timelike vectors with negative norm!}%
	\label{fig:figures / lightcones}
\end{figure}
In this case, we can measure the distance between two points. 
\[
\tau = \int_{ a} ^ b dt \sqrt{  - g_{ \mu \nu } \frac{dx^ \mu }{ dt } \frac{ dx ^ \nu }{ dt }} 
\] This object $ \tau $ is called the proper time 
between two points. Philosophically, this is a parameter which is 
invariant in all frames. 
If we were to reparametrise this curve, our 
definition of $ \tau $ remains invariant. 

\subsubsection{The Joys of a metric}

\begin{claim}{Metrics induce a natural isomorphism from vectors to 1-forms.} 
The metric gives a natural (basis independent) isomorphism 
\[
g : T_p ( \mathcal{ M }) \to T_p ^ * ( \mathcal{ M } ) 
\] Given $ X \in \mathcal{ X } \left( M  \right) $, we can construct 
$ g ( X , \cdot  ) \in \Lambda^ 1 ( \mathcal{ M }) $. 
If $ X = X^ \mu \partial _ \mu $, our corresponding one form is 
\[
g_{ \mu \nu } X^ \mu dx^ \nu : = X_\nu dx^ \nu 
\] In this formula, we've written the index on $ X $ downstairs!
The metric provides a natural isomorphism between 
our vector space and our one-forms. Hence, 
this metric allows us to raise and lower indices ,
which means that our metric switches the mathematical space 
we are working in. 
Lowering an index is really the statement that there's a natural 
isomorphism. 
Because $ g $ is non-degenerate, there's an inverse
\[
g ^{ \mu \nu } g _{ \nu \rho } = \delta \indices{ ^ \mu _ \rho } 
\] This defines a rank $ ( 2, 0 ) $ tensor $ \hat{g} = g^{ \mu \nu } \partial  _\mu \otimes \partial  _ \nu  $, and 
we can use this to raise indices. We have 
\[
X^\mu = g ^{ \mu \nu } X_ \nu  
\]
\end{claim} 

\begin{claim}{Metrics induce volume forms to integrate with.}
There's something else that the metric gives us. 
We also get a natural volume form. On a Riemannian manifold, our 
volume form is defined to be 
\[
v = \sqrt{ det g_{ \mu \nu } }  dx^ 1 \wedge  \dots \wedge  dx^ n 
\] We write $ g = det g_{ \mu \nu }$
On a Lorentzian manifold, $ v = \sqrt{ - g }  dx^ 0  \wedge  \dots \wedge  dx^{ n- 1}$
This is independent of coordinates. 
In new coordinates, 
\[
dx ^ \mu = A \indices{ ^ \mu _ \nu } \quad, A \indices{ ^ \mu _ \nu } \frac{\partial x^ \mu }{\partial  x^ \nu }    
\] 
We see how they change. 
\begin{align*}
dx^ 1 \wedge  \dots \wedge  dx^ n & = A \indices{ ^ 1 _{ \mu _ 1 }} \dots A \indices{^ n _{ \mu _ n } } d \tilde{x} ^{ \mu _ 1} \wedge  \dots \wedge  d \tilde{x }^{ \mu _ n }   \\
				  &=  \sum_{ \text{ perms } \pi } A \indices{ ^ 1 _{ \pi ( 1)  } } \dots A \indices{ ^ n _{ \pi ( n ) } } d \tilde{x }^ 1 \wedge  \dots \wedge  d \tilde{x  }^ n      \\
		& = det ( A) d \tilde{x  }^ 1 \wedge  \dots \wedge  d \tilde{x } ^ n   
\end{align*} If we have that $ det A > 0 $ , coordinate change preserves 
orientation. 
Meanwhile, 
\begin{align*}
g_{ \mu \nu } &=  \frac{\partial  \tilde{x } ^ \rho  }{\partial  x ^ \mu }  \frac{\partial  \tilde{x } ^ \sigma  }{\partial  x^ \nu }  \tilde{ g }_{ \rho \sigma }  \\ 
       &=  ( A^{ - 1} ) \indices{ ^ \rho _  \mu } ( A ^{ - 1}) \indices{ ^ \sigma _ \nu } \tilde{g }_{ \rho \sigma }    \\
\end{align*} 
Hence, 
\[
det g _{ \mu \nu } = ( det A^{ - 1} ) ^ 2 det \tilde{ g } _{ \rho \sigma } 
\] Thus we have that 
\[
v = \sqrt{ | \tilde{ g } |  } d \tilde{x }^ 1 \wedge  \dots \wedge  d \tilde{ x }^ n    
\] in components, 
we have that 
\[
v = \frac{1}{ n ! } v_{ \mu_1 \dots \mu_ n } dx^{ \mu _ 1 } \wedge  \dots \wedge  dx^{ \mu _ n }
\] where our components are given by 
\[
v _{ \mu _ 1 \dots \mu _ n } \epsilon_{ \mu _ 1 \dots \mu _ n }
\] we can integrate functions as 
\[
\int_{ \mathcal{ M } } fv = \int_{ \mathcal{ M } } d^ n x \sqrt{ | g | } f (x)  
\] 

The metric provides a map from $ \omega  \in \Lambda^p ( \mathcal{ M } ) $
to $ ( * \omega ) \in \Lambda ^{ n - p } ( \mathcal{ M } ) $  defined by
\[
 ( * \omega )_{ \mu_1 \dots \mu_{ n - p }  } = \frac{1}{p! } \sqrt{  | g ! | }  \epsilon_{ \mu_1 \dots \mu_{ n - p } \nu_1 \dots \nu_{ p } } \omega^{ \nu_1 \dots \nu_{ p } }
\] This object is called Hodge dual. We can check that 
\[
	* ( * \omega ) = \pm ( -1) ^{ p ( n - p ) } \omega 
\] with $ + $ used in with a Riemannian metric, and $ - $ used 
with a Lorentzian metric. 
We can then define an inner product on forms. 
Given $ \omega, \eta \in \Lambda ^ p ( \mathcal{ M } ) $, let
\[
 \left< \eta , \omega  \right>  = \int_{ \mathcal{ M } } \eta \wedge  * \omega 
\] The integrand is a top form so this is okay. 
This allows us to introduce a new object. 
If we have a p-form $ \omega \in \Lambda ^ p ( \mathcal{ M } ) $, and a 
p-1 form $ \alpha \in \Lambda ^{ p - 1 }( \mathcal{ M } )  $, then 
\[
 \left< d\alpha, \omega    \right> = \left< \alpha, d^ \dagger \omega \right>
\] when $ d ^ \dagger : \Lambda ^ p ( \mathcal{ M } ) \to \Lambda ^{ p -1 } ( \mathcal{ M } ) $, is 
\[
 d^ \dagger  = \pm ( - 1)^{ np + n - 1 } * d * 
\] where again our $ \pm $ signs depend on whether we have 
a Riemannian or Lorentzian metric. 
To show this, on a closed manifold Stokes' theorem implies that 
\[
	0 = \int_{ \mathcal{ M } } d ( \alpha \wedge  * \omega )  = \int_{ \mathcal{ M } } d \alpha \wedge  * \omega + ( - 1) ^{ p - 1 }\alpha \wedge  d * \omega 
\] But the term on the right is just 
\[
	= \left< d\alpha , \omega  \right> + ( - 1 ) ^{ p - 1 } \text{sign} \left<\alpha, * d * \omega  \right>
\] When we fix our sign, we 
get the result. 
There's actually a close relationship between forms in differential 
Geometry and fermionic antisymmetric fields in quantum field theory.
\end{claim} 

