\subsection{Parallel Transport} 
Now that we have a connection $\nabla $, we will show
that we can use this
object to make vectors 'travel' along the surface of a manifold. 
This means that our connection has given us a map from the vector space at 
a point $p \in \mathcal{ M } $, to another point $ q \in \mathcal{ M } $. 
To set up this map, we take our favourite vector field $ X$ and 
then generate integral curves. At a specific point $p \in \mathcal{ M }$, 
this specifies a unique curve $ \mathcal{ C } $. This curve is given by 
\[
	X^\mu \mid_C = \frac{ d x^\mu ( \tau ) }{ d \tau } 
\] 
A vector field  $\mathcal{ X } ( \mathcal{ M } )  $ is said to 
be \textbf{parallel transported} along this curve if at every point, 
we have that the covariant derivative of $ Y $ with respect to 
$ X $ is zero. 
\[
 \nabla _ X Y = 0 , \quad \text{ at all } p \in \mathcal{ C } 
\] From this object 
we have an equation which specifies the components 
$ Y ^ \mu $ at each point on the curve. Our initial 
condition here is what $ Y ^ \mu ( \tau = 0  ) $ is. 
The above condition reads 
\[
 X^ \mu \left(  \partial  _ \mu Y ^ \nu + \Gamma^ \nu _{ \mu \rho } Y ^ \rho   \right) 
 = \frac{d Y ^ \nu ( \tau ) }{ d\tau } + X ^ \mu Y ^ \rho \Gamma^\nu _{ \mu \rho } = 0 
\] 
\subsubsection{Geodesics}
From our idea of parallel transport which 
we introduced above, we shall talk about a very special type 
of curve called a \textbf{geodesic}. Suppose we 
have a curve $ C $. Now, this curve $ C $ has a tangent vector 
$ X( \tau )  $ at every point, where $ \tau $ is some parameter 
we have used to construct the curve. If the tangent vector $ X $ 
is parallel transported along this curve, that is, we have that 
\[
 \nabla_ X X = 0 , \quad \text{for all } p \in C 
\] then, we call this object a geodesic. 
Our equation for our tangent vector component $ X^ \mu$
is 
\[
 \frac{d X^ \nu }{ d \tau } + X ^ \mu X ^ \rho \Gamma ^ \nu _{ \mu \rho } = 0 
\] However, we can go a bit further here 
since we're on an integral curve of $X ^ \mu $. 
 \[
	 X ^ \mu = \frac{d x ^ \mu ( \tau ) }{ d \tau } \implies \frac{d ^ 2 x ^ \mu ( \tau ) }{ d \tau ^ 2 } + \frac{d x ^ \nu }{ dt } 
	 \frac{d x ^ \rho  }{ d \tau } \Gamma^\mu_{ \nu \rho }  =0 
\] This specifies a unique geodesic given 
a set of initial conditions on $ x ^ \mu ( 0 ) $ and $ \dot{ x} ^ \mu ( 0 ) $, 
and is determined from our connection on our manifold. 


\subsection{Normal coordinates}
Working with metrics is difficult, because they can 
be verbose and involved. What if, there was some way to change our 
basis such that we can work with a simple (diagonal) and flat metric at 
a given point. Lucky for us, there is a way! 
This is called switching to normal coordinates. In this bit 
we'll show that at a point $ p \in \mathcal{ M } $, we 
can always choose coordinates $ \left\{  x ^\mu   \right\} $ such that 
\[
 g_{ \mu \nu }  = \delta_{ \mu \nu  }, \quad  g_{ \mu \nu, \rho }  = 0 
\] In this context our tensor $ \delta $ can be either 
the standard Euclidean or Minkowski metric depending on 
whether we're on a Riemannian or Lorentzian manifold. 

To show this, we count degrees of freedom (which in my opinion 
I think is a bit of a weird technique). Suppose we're in a coordinate system 
$ \left\{  \tilde{ x } ^ \mu   \right\}  $ at a point $ p $. We 
can, without loss of generality, set $ \tilde{ x } ^\mu ( p ) = 0 $. 
Our aim then is to find a change of basis which yields a new coordinate 
system $ \left\{  x^\mu   \right\} $ (without loss of generality 
we also have that $ x ^ \mu ( p ) = 0 $, such that in 
this basis, the above conditions on 
the metric are satisfied. 

This means that our change of basis satisfies
\[
 g_{ \mu \nu } = \frac{\partial  \tilde{x } ^ \alpha  }{\partial  x ^ \mu }  
 \frac{\partial  \tilde{ x } ^ \beta  }{\partial  x ^ \nu }  \tilde{ g } _{ \alpha \beta } 
\] Now, we employ a trick. 
We Taylor expand $ \tilde{ x } ^ \alpha  $ in terms of $ x ^ \mu $
about the point $ p$, which gives us the expansion 
 \[
 \tilde{ x } ^ \mu = \left. \frac{\partial \tilde{x } ^ \mu  }{\partial  x ^ \nu }  \right\vert_{ x = 0 }
	 x ^ \nu + \frac{ x ^ \rho x ^ \nu }{ 2 } \left. \frac{\partial  ^ 2 \tilde{x } ^ 2 }{\partial 
		 x ^ \rho   x ^ \nu } \right\vert_{ x= 0 } 
	 \] When we differentiate this 
Taylor expansion we get that, 
\[
 \frac{\partial  \tilde{ x } ^ \mu  }{\partial  x ^ \nu }  = \left.
	 \frac{\partial  \tilde{ x  } ^ \mu  }{\partial  x ^ \nu }  \right\vert_{ x =0 }
\] Now, when we substitute this into 
our change of basis formula and evaluate at the point 
$ p = 0 $, we find that we need to solve the system 
 \[
	 \delta _{ \mu \nu } = \left( \left. \frac{\partial  \tilde{x } ^ \alpha  }{\partial  x ^ \mu } 
			 \right\vert_{  x=0 }
	 \left. \frac{\partial  \tilde{ x } ^ \beta  }{\partial  x ^ \nu }
	 \right\vert_{ x =0 }  \tilde{g  }_{ \alpha \beta } ( p )  \right) 
\] This equation looks like 
\[
 I = A G A ^ T 
\] where $ A $ represents our change of basis, and $ G $ our
metric. From our change of basis, we are free to choose 
$n ^ 2 $ different components for each element. Now, since 
this equation is symmetric upon taking a transpose, 
we only have $ \frac{1}{2 } n ( n + 1)  $  constraints. 
Thus, it's possible to choose a change of basis such that this works. 
Now, to second order, we have to expand further.
We find that 
\[
	g_{ \mu \nu }  = x ^ \gamma \left( \left. \frac{\partial ^ 2 \tilde{x } ^ \alpha   }{\partial x ^ \mu x ^ \gamma  }
		\right\vert_{ x = 0 } \frac{\partial \tilde{x } ^ \beta  }{\partial  x ^ \nu }
		 + \left. \frac{\partial^ 2  \tilde{ x } ^ \beta  }{\partial x ^ \gamma x ^ \nu  }
		 \right\vert_{ x = 0} \frac{\partial  \tilde{x } ^ \alpha  }{\partial  x ^ \mu }
		 \tilde{g}_{ \alpha \beta }
		\right) 
\] Our condition for our metric to be flat means 
we need to differentiate this to get 
\[
	g_{ \mu \nu , \rho  } = \left(  \left. \frac{\partial ^ 2 \tilde{x } ^ \alpha  }{\partial x ^ \mu 
	x ^ \rho }  \right\vert_{ x= 0 }  \left. \frac{\partial  \tilde{ x } ^ \beta  }{\partial  x ^ \nu }
\right \vert_{  x=0}  + \left. \frac{\partial ^ 2 \tilde{x }^ \beta   }{\partial x ^ \mu x ^ \rho }
\right\vert_{ x =0  } \left. \frac{\partial  \tilde{ x }^ \alpha  }{\partial  x ^ \mu }  
\right\vert_{ x = 0 }\right)  
\] From this equation, we 
need to make sure we have enough components
to specify our change of basis. Now, let's do some counting of our available degrees of freedom. 
$ g_{ \mu \nu } $ is symmetric and hence has $ \frac{1}{2 } n ( n -1 ) $
free components. The addition of our derivative gives us a 
further $ n $ free components for each element of $ g $. Thus, 
we have $ \frac{1}{2 } n ^ 2 ( n - 1) $ components. 
Furthermore, our second derivative term $ \partial  ^ 2 \tilde{ x } ^ \alpha / \partial x ^ \mu 
\partial  x ^ \rho $, due to symmetry of mixed partials, has 
$ \frac{1}{ 2} n ( n -1 )  $ for each index $ \alpha$. Thus, 
we have in total also  $ \frac{1}{2 } n ^ 2 (  n- 1) $ 
free components to choose from. This is enough to specify our 
change of basis!

\subsubsection{Constructing Normal Coordinates} 
This section is going to 
be a short one. We basically will pull a cat 
of the bag and construct a set of normal 
coordinates with a map. 
The tangent space associated at a point on a manifold 
has a canonical map to recover points on the manifold. 
Consider the exponential map 
\[
	\text{Exp } : \mathcal{ T }_ p ( \mathcal{ M } ) \to \mathcal{ M }
\] This map is given by picking a vector, then following
the geodesic curve there for one unit of parameter 
distance $ \tau = 1$. This means we will arrive at some point $ q $. 
Now, that this point $ q $ , we define our coordinates to be 
\[
 x ^ \mu ( q) = X^ \mu 
\] where  $ X ^ \mu $ was the vector we had originally chosen. 

 
