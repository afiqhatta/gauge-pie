\subsection{The exterior derivative} 
Notice that given a function $ f$, we can construct a  1-form 
\[
df = \frac{\partial f }{\partial  x^{ \mu} } 
\] In general, there exists a map 
$ d : \Lambda^ p ( M ) \to \lambda^{ p + 1 } (  M ) $. 
this is the exterior derivative. In coordinates, 
we have that 
\[
dw = \frac{1}{ p ! } \frac{\partial \omega_{ \mu_1 \dots \mu_{ p  } }}{\partial x^\nu} dx^\nu \wedge  \dots \wedge  dx^{ \mu_{ p }} 
\] One should view this as a 
generalised curl of some vector field. 
In components, we have that 
\[
( d \omega  ) _{ \mu_1 \dots \mu_{ p + 1  }} = ( p +  1) \partial_{ [ \mu_1 } \omega_{ \mu_2 \dots \mu_{ p + 1 }]  }
\] Let's try to gain an intuition about 
why these two definitions are equivalent.
If we contract our component definition with the tensor 
product $ dx^ 1 \otimes \dots \otimes dx^{p +1}$, 
then we are summing over 
\[
	d \omega = \frac{1}{p ! } \sum_{ \sigma \in S_ n } \epsilon ( \sigma)  \partial_{ \sigma ( \mu_ 1 ) } w_{ \sigma ( \mu _ 2 )  \dots \sigma ( \mu _{ p + 1 ) }} dx^{ \mu _ 1 } \otimes \dots \otimes dx^{ \mu _{ p +1}}
\] However we can transfer our 
permutations to permutations on tensor product, 
but by definition this would just be the 
wedge product on our basis one-forms. 

By antisymmetry,  we have a very significant identity that 
\[
d ( d \omega  ) = 0 
\] We write this as $ d^ 2 = 0 $.
To show this, the easiest way is not to use our 
definition of $ d \omega  $ in components but 
rather to use our definition in terms of wedge product 
basis vectors. Let's think carefully about how the exterior 
derivative acts on some on $ p $ form but with 'components' 
in our wedge product basis $ \left\{  dx^1 \wedge  \dots \wedge  dx^p \right\}  $. 
In our wedge product basis, our components are $ \frac{1}{p !} \omega_{ \mu_ 1 \dots \mu _{ p } } $ since 
\[
	\omega = \frac{1}{p!} \omega_{ \mu_1 \dots \mu_{ p } }
\] We have that under the exterior derivative, we are mapping 
\[
 d : \frac{1}{p!} \omega_{ \mu_ 1 \dots \mu_ p } dx^{ \mu _ 1 } \wedge  \dots \wedge dx ^{ \mu_ p } \mapsto \frac{1}{p!} \partial_\nu \omega_{ \mu _1 \dots \mu _ p } dx^ \nu \wedge dx^{ \mu _ 1 } \wedge  \dots \wedge  dx^{ \mu _ p } 
\] So, in our new fancy wedge product basis, we are mapping 
 \[
 w_{ \mu_1 \dots \mu_{ p }} \mapsto \partial _ \nu \omega_{ \mu _ 1 \dots \mu _{ p } }
\] 
Hence, we have that, in our wedge product basis, our components of 
$ d ( d \omega ) $ are 
\[
d ( d \omega ) = \frac{1}{p ! } \partial _ \rho \partial  _ \nu w_{ \mu _ 1 \dots \mu _ p } dx^ \rho \wedge  dx ^ \nu \wedge  dx ^{ \mu _ 1 } \wedge  \dots \wedge  dx^{\mu _ p } = 0 
\] since we have symmetry of mixed 
partial derivatives in $ \rho , \nu $ contracted 
with the wedge product which is antisymmetric in those 
indices.

It's also simple to show that 
\begin{itemize}
\item $ d ( \omega  \wedge \epsilon ) = d \omega  \wedge  \epsilon + ( - 1)^ p \omega  \wedge  d \epsilon $
\item For pull backs, $ d ( \phi^ * \omega  ) = \phi^* ( d \omega  ) $
\item $ \mathcal{ L }_X ( d \omega  ) = d ( \mathcal{ L }_ X \omega $
\end{itemize}
A p-form is closed if $ d \omega   = 0 $ everywhere. 
A p form is exact if $ \omega   = d \epsilon$ everywhere for some $ \epsilon$. 
We have that  
\[
d^ 2 = 0 \implies \text{exact} \implies \text{closed} 
\] 
\textbf{Poincare's lemma} states that on $ \mathbb{ R}^n$, or locally on  $ \mathcal{  M }$, exact implies closed. 

\pagebreak 
\subsubsection{Examples} 
Consider a one form $ \omega = \omega_\mu( x) dx^\mu $. Using our formula for the exterior derivative: 
\[
( d \omega)_{ \mu \nu } = \partial _\mu \omega_\nu - \partial  _\nu \omega_\mu 
\] Or, in terms of our form basis, 
\[
d \omega = \frac{1}{2 } ( \partial _\mu \omega_\nu - \partial _\nu \omega _\mu ) dx^\mu \wedge  dx ^ \nu 
\] In three dimensions, 
\[
d \omega = ( \partial_1 \omega_2 - \partial_2 \omega_1 ) dx^ 1 dx ^ 2 + ( \partial_2 \omega_3 -\partial_3 \omega_2 ) dx^ 2 \wedge  dx^ 3 + ( \partial_3 \omega_1 - \partial_1 \omega_3  ) dx^ 3 \wedge  dx^ 1 
\] These are the components of $ \nabla \times \omega $. 
The exterior derivative of a 1 form is a 2 form, but 2 forms have just three components in $ \mathbb{ R} ^3$ by anti symmetry ( with the components shown there). So we think of it has another vector field, if we identify the basis vectors $ \left\{  dx^ i \wedge  dx^ j  \right\} $ with components in $ \mathbb{R}^ 3 $!. 
However, getting another 'vector field' by doing an exterior derivative on the 
same type of object we had before is not the case in general.

Consider  $ B \in \Lambda^ 2 ( \mathbb{ R} ^ 3 ) $. So, we have
a 2-from in a 3 dimensional manifold. Let's label the components out 
explicitly here. 
\[
 B = B_1 ( x) dx^ 2 \wedge  dx^ 3 + B_2 ( x) dx^ 3 dx^ 1 + B_3 ( x) dx^ 1 \wedge  dx^ 2 
\] Before we do any explicit calculation, 
we know that the exterior derivative pushes this up to a 3 -form, which only has one component
in a three dimensional manifold. We compute the exterior derivative 
explicitly by differentiating each component
and then adding on the wedge.
\begin{align*}
	d B &=  \partial _ 1 B_ 1 dx ^1 \wedge  dx ^ 2 \wedge  dx^ 3  + \partial _ 2 B_2 dx^ 2 \wedge  dx^ 3 \wedge  dx^ 1 + \partial  _ 3 B _ 3 dx^ 3 \wedge  dx^ 1 \wedge  dx^ 2  \\
	    &=   \left( \partial  _ 1 B _ 1 + \partial  _ 2 B _ 2 + \partial  _ 3 B _ 3 
	     \right) dx^ 1 \wedge  dx^ 2 \wedge  dx^ 3 
\end{align*} 
In the last step, we permuted the indices cyclically 
so that we don't have a sign change. 
Note that we get our components of a grad
operator acting on $ \mathbf { B } $! 

For our final example, we take something from
electromagnetism. The gauge field, or perhaps more commonly 
known as our vector potential $ A \in \Lambda ^ 1 ( \mathbb{ R} ^ 4 ) $, 
can be written out as a one-form in $ \mathbb{ R } ^ 4 $. 

If we expand this as a one form, we can write 
\[
 A = A_ \mu dx^ \mu 
\] 
What happens when we take the exterior 
derivative of this thing? 
We get that 
\begin{align*}
	dA &= \partial  _ \nu A _ \mu dx^ \nu \wedge  dx ^ \mu   \\
	   &=  \frac{1}{2 } \partial _{ [ \mu } A _{ \mu ] } dx^ \nu \wedge  dx ^ \mu \\ 
	   &=  \frac{1}{2 } \left(  \partial  _ \nu A _ \mu - \partial  _ \mu 
	   A _ \nu  \right) dx^ \nu \wedge dx ^ \mu \\
	   &=  \frac{1}{2 } F_{ \mu  \nu  } dx^ \mu  dx ^ \nu  \\
\end{align*} 
We identify here that $ F_{ \mu \nu} = \partial  _ \mu A _ \nu - \partial  _ \nu A _ \mu $ 
is our electromagnetic field strength tensor! 
We also have that since $ d F = d ^2 A = 0 $, 
we get for free what one may recognise as the Bianchi 
identities. 

We can also introduce gauge transformations
which act on our electromagnetic 
vector potential. These act as 
\[
A\to A+ d\alpha  \implies F \to d ( A d\alpha ) = dA \text{ invariant } 
\] where we treat $ \alpha \in \Lambda^ 0 ( \mathcal{ M } ) = C^\infty ( \mathcal{ M }) $
We also get Maxwell's equations for free! 
\[
F = dA \implies dF = d^ 2 A = 0
\] There's are two of Maxwell's equations. 

