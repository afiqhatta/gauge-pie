\subsection{The Levi-Civita Connection} 
The fundamental theorem of Riemannian 
Geometry is that there exists a unique, 
torsion-free connection with the property 
obeying 
\[
	\nabla _ X g = 0 , \forall X \in \mathcal{ X } ( \mathcal{ M } ) 
\]
To prove this, suppose that this object exists. 
Then, 
\begin{align*}
	X ( g ( Y , Z ) ) &=  \nabla _ X [ g ( Y , Z ) ]  \\
			  &=  \nabla _ X g ( Y , Z ) + g ( \nabla _ X Y , Z ) + g ( Y , \nabla _X , Z )  \\
			  &=  g ( \nabla _ X Y , Z ) + g ( Y , \nabla _ X  Z )  
\end{align*}
The fact that our torsion vanishes implies 
that 
\[
	\nabla  _ X Y - \nabla _ Y X = [ X, Y ] 
\] Hence, our equation on the left hand of 
our blackboard reads 
\[
	X ( g ( Y ,Z ) ) = g ( \nabla _ Y X , Z ) + g ( \nabla _ X Z , Y ) + g ( [ X, Y ] , Z ) 
\] Now we cycle $ X, Y , Z $, where we find that 
\begin{align*}
	Y ( g ( X, Z ) ) & = g ( \nabla _ Z , Y , X ) + g ( \nabla _ Y X , Z )  + g ( [ Y , Z] , X ) \\
	Z ( g ( X, Y ) ) &=  g ( \nabla _ X Z , Y ) + g ( \nabla _ Z Y , X ) + g ( [ Z, X] , Y ) 
\end{align*}
If add the first two equations and 
then subtract by the third one we get that 
\begin{align*} 
	g ( \nabla _ Y X , Z ) &= \frac{1}{2 } \big [  X g ( Y , Z ) + Y g ( X, Z ) - Z g ( X, Y )  \\
		 & - g ( [ X, Y ] , Z ) - g ( [ Y , Z ] , X ) + g ( [ Z, X ] , Y ) 
\end{align*} 
In a coordinate basis, we have
that $ \left\{  e_ \mu  \right\}   = \left\{  \partial _ \mu  \right\} $, we 
have that 
\begin{align*}
	g ( \nabla _ \nu e _ \mu , e _ \rho ) &=  \Gamma ^ \lambda _{ \nu \mu } g _{ \lambda \rho } = 
	\frac{1}{2 } ( \partial _ \mu g _{ \nu \rho } + \partial  _ \nu g _{ \mu \rho }  - \partial  _ \rho 
	g _{ \mu \nu } ) \\
\end{align*} 
Where we have that 
\[
	\Gamma ^ \lambda _{ \mu \nu } = \frac{1}{2 } g ^{ \lambda \rho } ( \partial  _ \mu g _{ \nu \rho } + \partial  _ \nu g _{ \mu \rho } - \partial  _ \rho g _{ \mu \nu } ) 
\] 
This is the Levi-Civita connection, 
and the $ \Gamma ^ \lambda _{ \mu \nu } $ are called 
the Christoffel symbols. 
We still need to show that it transforms as a connection, 
which we leave as an exercise.

\subsubsection{The Divergence Theorem} 
Consider a manifold $ \mathcal{ M } $ with 
metric $ g $, with 
boundary $ \partial  \mathcal{ M } $ , and
let $ n ^ \mu $ be an outward pointing 
vector orthogonal to $ \partial  \mathcal{ M } $. 
Then, for any $  X^ \mu $, our claim is that 
\[
 \int_{ \mathcal{ M } } d^ n x \sqrt{ g }  \nabla _ \mu X^ \mu = \int _{ \partial  \mathcal{ M } } d^{ n - 1} x \sqrt{ \gamma }  n_\mu 	 X ^ \mu 
\]   On a Lorentzian manifold, this also holds 
with $ \sqrt{ g  }  \to \sqrt{  - g}   $ and this 
also holds provided $ \partial   \mathcal{ M } $ is 
purely timelike or purely spacelike. 

First, we need a lemma, that $ \Gamma^{ \mu } _{ \mu \nu }  = \frac{1}{ \sqrt{ g }  } \partial_\nu \sqrt{ g } $. To prove this, we have that, writing out 
the definitions, that 
\[
	\Gamma ^ \mu _{ \mu \nu }  = \frac{1}{ 2  } g ^{ \mu \rho } \partial  _ \nu g _{ \mu \rho } = \frac{1}{2 }\tr ( \hat{ g } ^{ - 1 } \partial  _ \nu \hat{ g } ) 
\] But from this we have that 
\begin{align*}
	\dots &= \frac{1}{2 } \tr ( \partial  _\nu \log \hat{ g } )    \\
	      &=  \frac{1}{2 } \partial  _ \nu \log \det \hat{ g }  \\
	      &=  \frac{1}{2 } \frac{1}{ \det \hat{ g } } \partial  _ \nu \det \hat{ g }  \\
	      &=  \frac{1}{ \sqrt{ g } } \partial _ \nu \sqrt{ g }  	
\end{align*}

\pagebreak 
