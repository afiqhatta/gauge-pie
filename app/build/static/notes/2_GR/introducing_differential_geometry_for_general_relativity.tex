\section{Introducing Differential Geometry for General Relativity} 
Our main mathematical objects of interest in general relativity are manifolds. Manifolds are topological spaces which, at every point, has a neighbourhood which is homeomorphic to a subset of $\mathbb{R}^n$, where we call $n$ the dimension of the manifold. In plain English, manifolds are spaces in which, locally at a point, look like a flat plane. This can be made more rigourous by the creation of maps, which we call 'charts', that take an open set around a point (a neighbourhood), and mapping this to a subset of $\mathbb{R}^n$. 

Precisely, for each $p \in \mathcal{M}$, there exists a map $\phi : \mathcal{O} \rightarrow \mathcal{U} \subset \mathbb{R}^n$, where $p \in \mathcal{ O} \subset \mathcal{M}$, and $\mathcal{ O} $ is an open set of $M$ defined by the topology. Think of $\phi$ as a set of local coordinates, assigning a coordinate system to $ p$. We will write $\phi(p) = (x_1, \dots x_n) $ in this regard.   

This map must be a 'homeomorphism', which is a continuous, invertible map with a continuous inverse. In this sense, our idea of assigning local coordinates to a point in $\mathcal{M}$ becomes even more clear. 

We can define different charts to different regions, but we need to ensure that they're we'll behaved on their intersections. Suppose we had two charts and two open sets defined on our manifold, and looked at how we transfer from one chart to another. For charts to be compatible, we require that the map 
\[ 
\phi_\alpha \circ \phi_\beta^{ -1}: \phi_\beta ( \mathcal{ O}_\alpha \cap \mathcal{O}_\beta ) \rightarrow \phi_\alpha (\mathcal{ O}_\alpha \cap \mathcal{O}_\beta )
\] is also smooth (infinitely differentiable). 

A collection of these maps (charts) which cover the manifold is called an atlas, and the maps taking one coordinate system to another ($\phi_\alpha \circ \phi_\beta^{ -1} $), are called transition functions. 

Some examples of manifolds include 
\begin{itemize} 
\item $\mathbb{R}^n$ and all subsets of $\mathbb{R}^n$ are n-dimensional manifolds, where the identity map serves as a sufficent chart. 
\item $S^1, S^2$ are manifolds, with modified versions of polar coordinates patched together forming a chart (as we'll see in the case of $S^1$. 
\end{itemize}
Let's start simple and try to construct a chart for $S^1$. Our normal intuition would be to use a single chart $S^1 \rightarrow [0, 2 \pi) $, which indeed covers $S^1 $ but doesn't satisfy the condition that the target set is an open subset of $\mathbb{R}$. This yields problems in terms of differentiation functions at the point $0 \in \mathbb{R}$, because the interval is closed there, not open. One way to remedy this is to define two coordinate charts then path them together to form an atlas. Our first open set will be the set of points on the circle which exclude the rightmost point on the diameter, a set denoted by $\mathcal{O}_1$, and our second open set is the whole sphere excluding the leftmost point. We'll denote this $\mathcal{ O}_2 $. 

We assign the following charts which are inline with this geometry
\begin{align*} 
\phi_1 : \mathcal{O}_1 &\rightarrow \theta_1 \in (0, 2 \pi ) \\
\phi_2 : \mathcal{O}_2 &\rightarrow \theta_2 \in ( - \pi , \pi ) 
\end{align*} 

It's easy to verify that if we take a point on the manifold, our transition matrix reads that 
\[ 
\theta_2 = \phi_2 ( \phi_1^{ -1} ( \theta_1) )  = \begin{cases} 
\theta_1, \, \theta_1 \in (0 , \pi ) \\
\theta_1  - 2 \pi,  \, \theta_1 \in ( \pi, 2 \pi ) 
\end{cases} 
\] 
Now that we have coordinate charts, we can do things that we usually do on functions described in $\mathbb{ R}^n $, like differentiate. 
Furthermore, we can define maps between manifolds (which don't necessarily have the same dimension), where smoothness is defined via smoothness on coordinate charts. These are called diffeomorphisms. A function 
\[
f: \mathcal{ M} \rightarrow \mathcal{ N } 
\]
is a diffeomorphism if the corresponding map between $\mathbb{R}^{ dim \mathcal{M} } $ and $\mathbb{R}^{ dim \mathcal{N } } $ is smooth: 
\[ 
\psi \circ f \circ \phi^{ -1} : U_1 \rightarrow U_2 
\] for all coordinate charts $\phi : \mathcal{O}_1 \rightarrow U_1$ and $\psi: \mathcal{O}_2 \rightarrow U_2$ defined on the manifolds $\mathcal{M}$ and $\mathcal{N}$ respectively. 

