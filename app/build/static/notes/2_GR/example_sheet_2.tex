\subsection{Being careful about how we expand things} 
We need to be careful about trying to 
apply the covariant derivative to Christoffel symbols. 
For example, writing out 
\[
 \nabla _ \nu \Gamma ^ \nu _{ \rho \mu } 
\]  doesn't make sense since $ \Gamma $ is not a tensor!
$ \Gamma $ doesn't transform as a function. 
Thus, we need to expand out things the right way. 
Let's try to expand out the 
term 
\[
	\nabla _{ [ \mu } \nabla _{ \nu ] } f 
\] We need to work out from in, because 
we know that $ \nabla _{ \nu } f $ is a covector. 

\subsection{Tricks in the Levi-Civita connection} 
In the Levi-Civita connection, we have that 
raising indexes 

\subsection{Integrating over volume forms} 
Always integrate 'as is'  - we don't have to add
the metric if it's not there. 
When trying to calculate the hodge star, 
write things out in terms of the Vielbein basis. 

The minus sign when taking our hodge star comes
from our $ \hat{\theta } ^ 0 $ has the Minkowski metric. 

(Weird thought - 

\subsection{Showing that magnitude is constant} 
It's a lot easier to 
show things like the angle between two 
vectors being constant 

Sometimes we need to avoid coordinate singularities, 
a work around is to rotate the manifold. 

\subsection{Symmetries of the Riemann Tensor} 
We have 4 main symmetries of the Riemann Tensor to worry about. 
Antisymmetry in the 1 and 2 indices, anti-symmetry in the 3 and 4 indices, 
Symmetry in swtiching, and Bianchi identity.

We do the first steps by writing things out in
terms of an anti-symmetric basis. Then, adding 
on the Bianchi identity is a matter of counting the extra independent 
constraints we can get.

In 2D, this means that $G_{ \mu \nu} =0 $, so no
matter exists. 

\subsection{General techniques} 
Looking at the number of components 
can help simplify things greatly. 
