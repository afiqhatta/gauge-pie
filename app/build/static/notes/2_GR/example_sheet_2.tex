\section{Example Sheet 2}

\pagebreak 

\section{Example Sheet 4}

\subsection{Question 1}
We want to solve the linearised 
equation 
\[
	\nabla ^ 2 \overline{ h } _{ 00 }  = - 16 \pi M \delta \left( \vec{x} \right) 
\] Using the standard 
Green's function technique, 
the solution to this equation is 
\[
 \overline{ h } _{ 00 }  = - \frac{4M}{ r} 
\] This means that we have
\[
	h_{ 00 }  =  - \frac{2M }{r}, h _{ ii }  =  - \frac{2M}{ r }
\] Our gravitational field is hence 
\[
	ds ^ 2  =  - \left( 1 + \frac{2M }{ r }  \right)  dt ^ 2 + \left( 1 - \frac{2M}{r }  \right)  
	d\vec{x} \cdot  d \vec{x} 
\] Now, where is this valid? 
Well in comparison to the Scharwzchild metric, 
the spatial perturbation to the metric is the leading order expansion for 
$ \left( 1   - \frac{2G M }{r}  \right)^{  - 1 } $, 
thus we should require that $ r \ll 2 G M $
 
Our expansion is valid when 
$ | h | \ll 1 $, so all we need from 
this is that $ \overline{ h } _{ 00 } \ll 1 $, 
for example. This gives the condition 
that $ r \ll GM $ as well. 
\subsection{Question 2} 

The linearised Einstein equations are 
\[
 \Box \overline{h _{ \mu \nu } }  = - 16 \pi G T_{ \mu \nu } , \quad 
 \overline{h } _{ \mu \nu }  = h _{ \mu \nu }   - \frac{1}{2 } \eta _{ \mu \nu } h 
\] We choose to solve for $ \overline{ h } _{\mu \nu } $ 
first because it's easier. We then obtain $ h _{ \mu \nu } $ from 
$ \overline{ h } _{ \mu \nu }  $  by just inverting the above relation. 
This inverse is 
\[
 h _{ \mu \nu }  = \overline{ h } _{ \mu \nu } - \frac{1}{2 } \eta _{ \mu \nu } \overline{h } 
\] We are given the following time independent 
stress-energy tensor to solve for, which is 
\[
	T _{ \mu \nu }  = \mu \delta \left( x  \right)  \delta \left( y  \right)  
	\text{diag}\left( 1, 0 , 0 , - 1  \right)  
\] We proceed by solving component by component. 
Since our stress energy tensor is time-independent we 
can reduce the wave operator $ \Box $ to just the Laplacian 
$ \nabla ^ 2 $. For $ \overline{ h } _{ 00 } $, we wish to solve 
\[
	\nabla ^ 2 \overline{ h } _{ 00 }   = - 16 \pi G \mu \delta \left( x  \right)\delta\left( y \right) 
\] For now, let's work in cylindrical polar 
coordinates and isolate to the case where 
our solution for $ \overline{ h } $ and therefore 
$ \overline{ h }  $ relies solely on $ r $. 
We hence rewrite our equation as 
\[
	\nabla ^ 2 \overline{ h } _{ 00 }\left( r  \right)  = - 16 \pi G \mu \delta\left(r = 0)  
\] Now, if you already know the 
Green's function for this kind of problem then 
you can just solve this thing right here. 
In case you don't, we have that the radial part of the Laplacian 
gives us the equation 
\[
	\frac{1}{r } \frac{d }{ dr } \left( r \frac{ d \overline{ h } _{ 0 0 } }{ d r } 		   \right) = - 16 \pi G \mu \delta \left(  r  =  0  \right)  
\]  For the case $ r > 0 $, when the 
right hand side is zero, our solution is 
\[
	\overline{ h }_{ 0 0 }  = A \log \left( \frac{r}{ r_0} \right) 
\] We need 
to fix the constant  $ A $ consistently with the 
delta function contribution on the 
right hand side of the equation. 
Integrating both sides in cylindrical polars, to 
some radius  $ R $, and then taking the limit as  $ R  \to 0 $, 
we have that 
\begin{align*}
	2 \pi A \int _{ 0 } ^ R dr \, r \frac{1}{r } \frac{ d  }{ dr } 
	\left( r \frac{ df }{ dr}  \right) &  =  - 16 \pi G \mu \\
	2 \pi R \frac{ d \log \frac{ R }{ r_ 0} }{ d R }  & =  - 16 \pi G \mu \\
	A & = - 8 G \mu 
\end{align*}
Thus $ \overline{h } _{ 00 }  = - 8 G \mu \log \left( \frac{ r }{ r _ 0 }  \right) $ 
Additionally, we have that $ \overline{ h } _{ 33 }  = 8 G \mu \log \left( \frac{ r }{ r^0 } \right) $. All of the other components of $ \overline{ h} _{ ij } $  
can be set to zero, since this is a valid solution 
in the homogeneous case. We thus have $ \overline{ h }  = 16 G \mu 
\log \left( \frac{r}{ r_0 }  \right) $. Inverting with the formula above, 
to get $ h $, the only non-zero contributions are
$ h_{ 11 }  = h _{ 22 }  =  -  8 \mu G \log \left( \frac{r}{ r_0 } \right) $

Adding on the perturbation, our metric is thus 
\[
	ds ^ 2  = - dt ^ 2 + \left( 1 - \lambda  \right) \left( dx ^ 2 + dy ^ 2  \right)  + dz ^ 2 
\] In polar coordinates, the $ dx ^ 2 + dy ^ 2 $ is just $ dr ^ 2  + r ^ 2 d \phi ^ 2 $. 
This means that our resulting metric is 
\[
	ds ^ 2 =  - dt ^ 2 + dz ^  2 + \left( 1 - \lambda   \right)  \left( dr ^ 2 + r ^ 2 d \phi ^ 2  \right) 
\] Using the substitution $ \left( 1 - \lambda ^ 2   \right) r ^ 2  = \left( 1 - 8 \mu G  \right)  \tilde{r } ^ 2   $, our angular component of the 
metric reads $ \left( 1 - 8 \mu G  \right)  d \tilde{ r } ^ 2    $. 
To first order, we have that 
\begin{align*}
	\tilde{ r } &  = \frac{\left( 1 - \lambda  \right)  ^{ \frac{1}{2 } } }{ \left( 1 - 8 \mu G  \right)  ^{ \frac{1}{2 } }  } r  \\
		    &\simeq   r \left( 1 + 4 \mu G \log r   \right) \left(  1 + 4 \mu G   \right)    \quad \text{to first order in } 8 \mu G  \\ 
		    &=  \left( 1 + 4 \mu G  - 4 \mu G \log r  \right)  r  
\end{align*}  
Differentiating, this 
gives 
\[
	d \tilde{ r }  = \left( 1 - 4 \mu G \log r   \right)   dr 
\]  Squaring, we get that this is 
\[
	d \tilde{ r } ^ 2  = \left( 1 - \frac{\lambda}{2 }  \right)  ^ 2 dr ^ 2 
	\simeq \left(  1 - \lambda  \right)  dr ^ 2  
\] Substituting this in, we get that our metric under this change of coordinates 
is 
\[
	ds ^ 2  = - dt ^  2 + dz ^ 2 + d \tilde{ r } ^ 2 + \left( 1 - 8 \mu G  \right) \tilde{ r } ^ 2 d \phi ^ 2    
\] Scaling the angular coordinate as $ \tilde{ \phi }  = \sqrt{ 1 - 8 \mu G   }  \phi  $, 
this gives our metric as 
\[
 ds ^ 2 =  - dt ^2 + dz ^ 2 + d \tilde{ r } ^ 2 + \tilde{ r } ^2 d \tilde{ \phi } ^ 2    
\] This looks like Minkowski space time after all the first order approximations, 
but our original metric wasn't.
Intuitively, to get 
a double image, our geodesics follow a path as follows.
\hspace*{10cm}
\begin{figure}[htpb]
	\centering
\input{ex2q2.pdf_tex}
\caption{Double image appearing}%
\label{fig:x2q2}
\end{figure}

Here's an efficient way to derive the Green's function. 
\begin{align*}
	\partial  ^ 2 _ x + \partial  ^ 2 _ y G &=  -  1  \\
	\int_{ B _ 1 } d ^ 2 x \partial  ^ 2 G  &=  - 1 \\
	\int _{ S_1 } r d \phi \left( \partial  G  \right)  \cdot  \vec{n} 
						&=  \int _{ S_ 1  } d \phi \partial  _ r G \left( \frac{\partial r }{\partial  \vec{v} }    \right) \cdot  \vec{n}  \\
						&=  2 \pi r \partial _ r G  
\end{align*}
If we integrate through, 
we get that 
\[
  - \overline{ h } _{ 00 }  = \overline{ h } _{ 33 }  = \lambda 
   = 8 G \mu \log \left( \frac{r}{r_0 }  \right) 
\] 

There's a slight caveat in this question. 
if we rescale $ \phi $ as before, 
this means that our periodicity changes, 
we have a slightly smaller period. 
So, two light rays get pinched when we 
take straight lines. 

\pagebreak 

\subsection{Question 3} 
To first approximation, since our rotating sphere is moving 
slowly, our energy-momentum tensor 
is $ T^{ \mu \nu } \sim \rho u ^{ \mu } u ^{ \nu } $, 
where we take all contributions of $ O \left(  \Omega ^ 2  \right)  $ 
as approximately zero. 
\[
T_{ \mu \nu } \simeq \rho 
\begin{pmatrix}  1 & - \Omega y & \Omega x & 0 
\\ - \Omega y & 0 & 0 & 0 
\\ \Omega x & 0 & 0 & 0 \\
0 & 0 & 0 & 0 \end{pmatrix} 
\] 
\subsubsection{First part}
We first solve for our $ \overline{ h } _{ 00 } $ component, 
which is the solution to the equation 
\[
\nabla ^ 2 \overline{ h } _{ 00 }  = - \frac{ 4 M }{ R ^ 2 } \delta \left( r - R  \right) 
\] We switch to spherical polar coordinates and 
assume axisymmetric solutions, and assume that 
$ \overline{ h } _{ 00 }  = \overline{ h } _{ 00 } \left( r  \right)  $. 
In spherical polar coordinates, 
the form of the Laplacian means that 
we have to solve
\[
\frac{1}{ r^ 2  } \frac{\partial }{\partial r }  \left( r ^ 2 
\frac{\partial  \overline{ h } _{ 00  } }{\partial r  }  \right) 
=  - \frac{4 M }{ R ^ 2 } \delta \left( r - R  \right) 
\] 
In the regions $ r < R $ and $ r > R $, we 
want to solve 
\[
\frac{1}{ r ^ 2 } \frac{\partial  }{\partial  r ^ 2 }  \left( 
r ^ 2 \frac{\partial  \overline{ h } _{ 00 } }{\partial   r }  \right) = 0  
\] To ensure we don't have a singular solution at the origin
and that the solution decays at infinity, this means that 
\[
\overline{ h} _{ 00 }  = \begin{cases}
\frac{C}{ R } & r < R  \\ 
\frac{C}{r} & r > R 
\end{cases}
\] where $ C $ is a constant to be determined. 
To determine this constant, we integrate the 
above equation between the bounds $ R + \epsilon $ and $ R  - \epsilon $ 
for small $ \epsilon $. In other words, 
we wish to calculate 
\[
4 \pi \int_{ R - \epsilon } ^{  R + \epsilon } 
dr \, r ^ 2 \left[ \frac{1}{ r ^ 2 } \frac{\partial  }{\partial  r }  
\left( r ^ 2 \frac{\partial  \overline{ h } _{ 00 } }{\partial  r }   \right)  \right]   = - \frac{4 M }{ R ^ 2 } \left( 4 \pi  \right)  
\int_{ R - \epsilon } ^{ R + \epsilon }  dr \, r ^ 2 \delta \left( r - R  \right)  
\] Hence we have that 
\[
\left[  R ^ 2 \frac{\partial  \overline{ h } _{ 00 } }{\partial  r }   \right] ^{ R ^ + } _{ R _  -  }   =  - 4 M \implies C = 4 M  
\] since $ \overline{ h } _{ 00 } $ is constant taking the 
limit from below. 
Hence
\[
\overline{ h } _{ 00 }  = \begin{cases}
\frac{4M}{R } & r < R \\
\frac{4M}{r } & r > R 
\end{cases}
\] We invert this to find $ h_{ 00}  $. Since $ \overline{ h } _{ 00}  $ 
is the only diagonal component of $ \overline{ h } _{ \mu \nu } $, 
we have that $ \overline{ h }  =  - \overline{ h } _{ 00 } $ to first order. 
This means that our solution for $ h_{ 00 } $ is given by 
\[
h _{ 00 }  = \overline{ h } _{ 00 }  - \frac{1}{2 } \eta _{ 00 } \overline{ h } _{ 00 } = \begin{cases}
\frac{2M}{R  }  & r < R \\
\frac{2M}{ r } & r > R 
\end{cases} 
\] Our metric is thus 
\[
ds^ 2 =  - \left( 1  - \frac{2M}{ r }  \right) dt ^ 2  
+ \left( 1 + \frac{2M}{r }  \right)  \left( dr ^ 2 + r ^ 2 d \phi ^ 2  \right)  
\] This is the form of Newtonian gravity!

\subsubsection{Second Part} 
We proceed with the same strategy we used before. 
Making use of the spherical Laplcian and 
the ansatz that $ H = f ( r ) \sin \theta e ^{ i \phi } $, 
we want to solve the following equation 
in the regions $ r  > R $ and $ r < R $: 
Substituting in this ansatz with the Laplacian in spherical 
coordinates, we have that we want to solve the following 
equation in the two regions (with the correct boundary conditions)
\[
\sin ^ 2 \theta \left( \frac{\partial  }{\partial  r }  \left(  r ^ 2 \frac{\partial  f }{\partial  r }   \right) \right) e ^{ i \phi  }  - e ^{ i \theta } \sin \theta f + e ^{ i \phi } f \sin \theta 
\left( \cos ^ 2  - \sin ^ 2 \theta  \right)  = 0 
\] 
We make the appropriate cancellations 
and make use of standard trigonometric 
identities 
\begin{align*} 
\sin ^ 2 \theta \left(  \frac{\partial  }{\partial  r }  \left(  r ^ 2 \frac{\partial  f }{\partial  r }   \right)   \right)   - f + f \left(  \cos ^ 2 \theta  - \sin ^ 2 \theta  \right)   &= 0 \\ 
\frac{\partial  }{\partial  r  }  \left( r ^ 2 \frac{\partial  f }{\partial  r }   \right)   - 2f  & = 0 
\end{align*} 
We then make the ansatz that $ f = A r ^{ \alpha } $ for some power exponent 
$ \alpha $. The solutions substituting in this ansatz, we 
find that the two solutions are $ \alpha = 1, -2 $. Due to 
regularity conditions, these are the exponents for the regions 
$ r < R $ and $ r > R $ respectively. Imposing that 
the function is continuous at  $ r  = R $, 
we find that 
\[
f\left(   r  \right)  = \begin{cases}
A r & r < R \\ 
\frac{ A R ^ 3}{ r ^ 2  } & r > R 
\end{cases}
\] Now all that's left to do is to determine the constant $ A $. 
We do this by integrating the equation over 
the troublesome coordinate over a small 
volume around radius $ R $. We only care about the 
radial part of the Laplacian since 
continuity takes every other term to zero. 
Thus, we integrate over 
\[
\sin \theta e ^{ i \phi } \frac{\partial  }{\partial  r }  \left( r ^ 2 
\frac{\partial  f }{\partial  r }  \right)   =  e^{ i \phi } \sin \theta 
\frac{ 4 \Omega }{ R ^ 2 } M \delta \left( r - R  \right)   
\]   We cancel off the functions of $ \phi $ and $ \theta $
, and remembering to include the measure $ dr \, r ^ 2 $, 
integrating the radial coordinate 
\[
\left[  R ^ 2 \frac{\partial  f }{\partial  R }   \right] ^{ R _ + }_{ R_{ - } }  =  - i 4 \Omega R M 
\] Now, the crucial observation 
here is that we substitute in the limits from 
above and below according to the \textbf{different} forms 
of $ f \left( r  \right)  $. This means that 
\[
A R ^ 2 \left[  1 - \left(  - 2 \right)   \right]  = - i 4 \Omega R M 
\]  Thus, we have that 
$ A =  - \frac{4}{3 }  \Omega R M $.
Substituting in our value of $ A $ and comparing the real and imaginary parts, we recover
our formula for $ \overline{ h }_{ 0i  }  = h _{ 0i } $   
as 
\[
h_{ 0i }  = \begin{cases}
\omega  \left(  y , -x, 0  \right)  & r < R \\
\frac{\omega  R ^ 3 }{ r ^ 3 }  \left( y , -x , 0  \right)  & r > R 
\end{cases}
\] 
This means that our new metric is 
\[
ds ^ 2  = \left(  1 - \frac{2M}{ R }  \right)  dt ^ 2 + \left(  1 + \frac{2M}{ R }  \right)  
\left( dx ^ 2 + dy ^ 2 + dz ^ 2  \right)  + 2 \omega  y dx dt - 2 \omega x dy dt 
\] Our corresponding Lagrangian is 
\[
L =  - \left(  1 - \frac{2M}{r }  \right) \dot{  t } ^ 2 +  \left(  1 + \frac{2M}{r }  \right) 
\left( \dot{ x } ^ 2 + \dot{ y } ^ 2 + \dot{ z } ^ 2     \right)  + 
2 \omega y \dot{ x } \dot{ t } - 2 \omega x \dot{ y } \dot{ t }     
\] Let's look at what our geodesic equation for 
$ x $ is. Making use of the Euler-Lagrange equations, 
we have that for the $ x $ coordinate 
\[
\ddot{x} \left(  1  +\frac{2M}{ r }  \right)  + \dot{ x } \dot{ r } (  - \frac{2M}{r ^ 2 } ) + 
\omega  \dot{ y } \dot{ t } + \omega y \ddot{t}  = -  \omega \dot{ y } \dot{ t }    
\]  Now, to first order we have that $ \dot{ t } \simeq 1  $. The second term 
on the left hand side above is second order in $ | \dot{ x } |   $ and $ \omega y  \ddot{ t  }$ 
is as well. 
So, we're left with the equation 
\[
\ddot{x} (  1 + \frac{2M}{r } ) +  =  - 2 \omega \dot{ y }   
\]  where this time we're differentiating with respect to $ t $ 
since it's the same as differentiating 
with respect to $ \lambda $ to first order. 
Similarly, we have that \[ 
\ddot{y} (  1 + \frac{2M}{r } ) +  =  2 \omega \dot{ x}   
\] Taking the $ \left( \frac{2M}{ r }  \right)  $ perturbation as 
small to the other side, we recover the Coriolis force.

From the above, 
we can just integrate out 
the angular parts straight way to arrive 
at the differential equation 
\[
	\frac{d}{dr } \left(  r^ 2 
	\frac{d }{ d r } \overline{ h } _{00 } \right)  = - 
	4 M  G \delta \left(  r   - R  \right) 
\] 

Another way to do the last part 
would be to observe that since 
the gravitational potential inside the 
shell is zero. 
Hence, we're left with just the kinetic part of 
the Lagrangian which is 
\[
	\mathcal{ L }  = \vec{v} \cdot  \vec{v} + 2 h_{ 0i } \cdot \vec{v}
\] 
We have that $ h _{ 0i  }  = \vec{r} \times \omega $. 
This means that our Lagrangian is 
\[
	\vec{v} \cdot  \vec{v} +  2 \left( \omega \times \vec{v}  \right)  \cdot  \vec{r}
\] Using the Euler-Lagrange equations
\begin{align*}
	\frac{ d }{ dt } \left( \frac{\partial  \mathcal{ L } }{\partial  d \vec{v} }  	
	\right)  & = 2 \frac{ d }{ dt } \vec{v} 	 + 
	 2 \vec{v} \times \omega \\
	 \frac{\partial  \mathcal{ L  } }{\partial \vec{r} 	 }
		 &=  2 \left( \omega \times \vec{r}  \right)   \\ 
		 \frac{d}{ dt  } \vec{v} &=  2 \omega \times \vec{r} 
\end{align*}
\pagebreak
\subsection{Question 4}
Our second order contribution from 
our Christoffel component is 
\[
\Gamma ^ \mu _{ \nu \rho }  = - \frac{1}{2 } h ^{ \mu \sigma } ( h_{ \sigma \nu , \rho } 
+ h _{ \sigma \rho , \nu }  - h _{ \nu \rho , \sigma } 	 )
\] Our Ricci tensor is given by 
\[
R_{ \mu \nu } ^{ \left(  2  \right)  } \left[  h  \right]   = 
\partial  _ \rho \Gamma ^{ \rho } _{ \nu \mu }  - \partial  _ \nu \Gamma ^{ \rho } _{ \rho \mu } 
+ \Gamma ^{ \alpha } _{ \mu \nu } \Gamma ^{ \rho } _{ \alpha \rho }  - 
\Gamma ^{ \alpha } _{ \mu \rho } \Gamma ^{ \rho } _{ \alpha \nu } 
\] To keep things simple, we look at contributions 
by type, and see if they match with what we're given. 
Specifically, terms of the schematic $ h \partial  \partial  h $ 
come from the $ \partial  \Gamma  - \partial  \Gamma $ term. 
After working though the algebra, this term is given by 
\[
R_{ \mu \nu } ^{ \left(  2  \right)  } \left[  h \partial  \partial  h  \right]  
=  - \frac{1}{2 } h ^{ \rho \sigma } h_{ \sigma \nu , \mu \rho } + \frac{1}{2 } 
h ^{ \rho \sigma } h _{ \nu \mu , \rho \sigma  } + 
\frac{1}{2 } h ^{ \sigma \rho } h _{ \rho \sigma, \mu \nu } - \frac{1}{2 } 
h _{ \rho \mu , \sigma \nu } 
\] This agrees with the form shown.
Next, we need to calculate the only other type of 
term there is in the expansion which is of the form $ \partial  h \partial  h $ 
schematically. 
The easiest way it seems to go about doing this 
is by first calculating the term $ \partial  _ \rho \Gamma ^ \rho_{ \nu \mu }  + \Gamma ^{ \alpha } _{ \mu \nu } \Gamma ^{ \rho } _{ \alpha \rho }$ and then proceed to 
anti-symmetrise over the indices $ \rho , \nu $ on the bottom.

The contribution 
comes from 
\begin{align*}
\partial  _ \rho \Gamma ^ \rho_{ \nu \mu }  + \Gamma ^{ \alpha } _{ \mu \nu } \Gamma ^{ \rho } _{ \alpha \rho } &=   - \frac{1}{2 } h ^{ \rho \sigma } _{ , \rho } \left(  h _{ \sigma \nu , \mu } 
+ h _{ \sigma \mu , \nu }  - h _{ \nu \mu , \sigma } \right)  + 
\frac{1}{4 } \eta ^{ \alpha \sigma } \eta ^{ \rho \beta } \left(  
h_{ \mu \sigma , \nu } + h _{ \nu \sigma , \mu }  - h _{ \mu \nu , \sigma } \right)  \left( 
h_{ \alpha \beta, \rho } + h _{ \rho \beta , \alpha }  - h _{ \alpha \rho , \beta }  \right)  
\end{align*}
This term here is quite gnarly to deal with. 
We get three terms from the first term in the sum, 
and a further nine terms from the two first order $ h $ brackets multiplied together. 

\subsubsection{Deriving the Linearised Einstein Hilbert action} 
Recall that in the Lagrangian formulation 
our Einstein-Hilbert action can be written as 
\[
S_{ EH  } = \int d ^ 4 x \, \sqrt{  - g }  R 
\] In this question, 
we have to be slightly careful. 
We need to include terms of both order $ O \left(  h    \right) $ 
and order $ O \left( h ^ 2  \right)  $ in our $ R _{ \mu \nu } $ term, 
since $ g ^{ \mu \nu } = \eta ^{ \mu \nu }  - h ^{ \mu \nu } $. 
We first have to find out what $ \sqrt{  - g}  $ is.
To first order, using the fact that the derivative 
of the determinant of a matrix is the trace, we have that 
\[
g =   - 1  - h \implies \left( -1 - h  \right)^{ \frac{1}{2 } }  \simeq - 1  - \frac{h}{2 } 
\] So the total term 
we have to consider is 
\[
S_{ EH }  = \int d ^ 4 x \left( 1 - \frac{h}{2 } \right) 
\left( \eta^{ \mu \nu }  - h ^{ \mu \nu }  \right) R _{ \mu \nu } \] 
Our first order contribution to the 
Ricci tensor is 
\[
	R_{ \mu \nu } ^{ \left(  1  \right)  } \left[  h  \right]   = 
	\partial  ^ \rho \partial  _{ ( \mu } h _{ \nu  ) \rho 	  }  - 
	\frac{1}{2 } \partial  ^ \rho 	\partial  _ \rho h _{ \mu \nu } 
	 - \frac{1}{2 } \partial  _ \mu \partial  _ \nu h
\] To make things clearer, we go term by term.
The first term that's easiest to check out 
is the $  \frac{1}{4 }\partial  _ \rho h \partial  ^ \rho h  $ 
term in the integrand. We can 
selectively pick out terms from each bracket 
which will get us what we want. This is 
a technique we can try. 

\pagebreak 

\subsection{Question 5}
Let's do some counting here. 
In our gauge transformation, 
we have $ 4 $ degrees of freedom which 
are chosen from $ \lambda $. 
But, we have that it looks like we 
have five conditions from 
the conditions 
\[
 H _{ 0 \mu }  = H  = 0 
\] What happened? 
Well, we need to impose the de Donder gauge
condition so that 
\[
  - \overline{ H } _{ 00 } k _ 0 + \overline{ H } _{ 0i  } k _ i  = 0
\] This removes one constraint, namely, 
if $ \overline{ H  }_{ 0i } = 0 $, then since $ k _ 0  = 0 $  
then we get that $ H _{ 00}  =  0 $ for free! 

This question is about solving the right 
equations to force us in the 'transverse-traceless' 
gauge, or TT gauge. Since we're solving 
a wave equation in two indices, we have a 
polarisation matrix $ H _{ \mu \nu } $. 
All we're doing in this question is showing that 
when we move to a different gauge, 
\[
	H_{ \mu \nu } \to H _{ \mu \nu }  + i \left( 
	k _ \mu X _ \nu + k _ \nu X _ \mu  - \eta _{ \mu \nu  } k ^ \rho X _ \rho \right) 
\] we can solve the right equations to get that 
\[
 H _{ 0 \mu }  = H \indices{^ \mu _ \mu }  =0  
\] It's not 
always obvious a priori that we have enough 
degrees of freedom in our gauge transformations 
to be able to solve things like this, so 
the whole point of the question is to indeed 
check that this is possible.

First observe that due to the 
condition $ k _ \mu k ^ \mu  = 0 $, 
we have that all non-trivial solutions 
to the wave equation must include 
 $ k _ 0 \neq  0 $.

 Suppose that initially 
 \[
  H_{ 0 \mu }  = f _{ 0 \mu } 
 \] Imposing our transverse condition means 
 that we require
 \[
	 f _{ 0 \mu }  =  - i \left( k _ \mu X _ 0 
	 + k _ 0 X _ \mu  - \eta _{ 0 \mu } k _ \rho X ^ \rho \right)  
 \] If we set $ \mu   = 0 $, carrying the $ i$ to the
 other side gives us the equation 
 \[
  i f _{ 00 }   = k _ 0 X _ 0 + k _ i X _ i 
 \] In addition, 
 imposing our traceless condition means that 
 if we set $ H \indices{ ^ \mu  _\mu }  = \rho   $, 
 then we get the condition 
 \[
   \frac{  - i \rho }{ 2} = k _ \rho X ^ \rho  =  - k _ 0 X ^ 0 + k _ i X ^ i 
 \] Combining this with the condition 
 that $ H _{ 0 0 }  = 0 $ above, we find that 
 we can solve for $ X _ 0 $ by setting 
 \[
	 X _ 0  = \frac{ i }{ 2 k _ 0  }\left(  f _{ 00 } + \frac{\rho}{2} \right) 
 \] The other transverse 
 conditions are 
 \[
	 H  _{ 0 i }  =f _{ 0i }  = - i \left( k _ 0 X _ i + k _ i X _ 0 \right) 
 \] However, since we've already solved 
 for $ X _ 0 $, we can just invert to get 
 \[
  \frac{ i f _{ 0 i }  - k _ i X _ 0 }{ k _ 0 }  = X _ i 
 \] These are all the gauge conditions we need. 
 However, we still need to check that these solutions are consistent with one another. 
To do this, the simplest thing to do is 
to take our wave vector to lie in the $z  $ direction 
without loss off generality, so $ k = \omega \left( 1, 0 , 0 , 1 \right) $. 
This yields a system of equations we have to solve 
as well as check for consistency. 
\begin{align*}
	i f_{00 } &=  \omega \left( X _ 0 + X _ 3  \right)   \\
	- \frac{ i \rho }{ 2 } &  = \omega (  - X_0 + X_3) \\
	i f _{ 01  } &=  \omega X _ 1 \\ 
	i f _{ 02 } &= \omega X_2 \\
	i f_{ 0 3 } &=  \omega \left(  X _ 0 + X _ 3 \right)  \\
\end{align*}
We have explicit solutions for $ X _ 1 $ and $ X _ 2 $, 
but we need to check for consistency for the $ X _ 0 $ and 
$ X _ 3 $ components. 
Solving for the first two equations, we have 
that
\[
	X_ 0  = \frac{1}{ 2 \omega } i \left( f _{ 00  } + \frac{\rho}{2 }  \right), 
	X_3 = \frac{1}{2 \omega } i \left( f_{00}  - \frac{\rho}{2 } \right) 
\] The only thing we need to check is that now, 
$ f _{ 01 } = f_{ 0 3 } $. However, this 
is okay since our de Donder gauge condition $ k_ \mu H ^{ \mu \nu }  = 0 $ for 
$ \nu  =  0 $ implies that $ f _{ 00 } = f _{ 0 3 } $. 
Thus, we have a consistent solution. 


Question: is it always valid to 
take the wave vector in the $ z $ direction without loss 
of generality? 


\pagebreak
\section{Question 6}
Assuming we have slow moving bodies, 
the four momenta of an object is given by 
the vector with unity in the zeroth component 
as well as the Newtonian velocity. 

Be careful 
when contracting indices. For example 
if we want to compute $ \dddot{ I }_{ kk }  $, 
make sure you calculate $  I _{ kk} $ first in 
the definition of $ Q _{ ij } $ first. 

For each particle, the associated energy-momentum 
tensor is the mass of each particle along with 
a delta function for the coordinate.
\[
	T^{ 00 }  = m_1 \delta \left( z \right)  \delta \left( x -r_1 \cos \omega t  \right) \delta\left( y  - r_1 \sin \omega t  \right)  + m_2 \delta\left( z  \right)  \delta \left( x + r_2 \cos \omega t  \right)  \delta\left( y + r_2 \sin \omega t  \right)  
\] Using standard double angle formulae we have
that 
\[
	I_{ ij }  = \left( \frac{ m_1 r_1 ^ 2 + m_2 r_2^ 2 }{ 2 }  \right)  
	\begin{pmatrix} 1 + \cos \omega t &  2 \sin \omega t & 0 \\
	\sin 2 \omega t & 1 - \cos 2 \omega t & 0 \\ 
0 & 0 & 0 \end{pmatrix}  
\] Using this, our quadrapole moment is given by 
\[
	Q_{ ij} \left( t  \right)  = I_{ ij }  - \frac{1}{3 } I_{ kk } \delta_{ ij } 
	 =  \left( \frac{ m_1 r_1 ^ 2 + m_2 r_2^ 2 }{ 2 }  \right)  
	 \begin{pmatrix}  \frac{1}{3 } + \cos 2 \omega t  & \sin 2 \omega t & 0 \\ 
	 \sin 2 \omega t & \frac{1}{3 } - \cos 2 \omega t & 0 \\
         0 & 0 &  - \frac{2}{3 } \end{pmatrix}   
\] Differentiating this with respect to time 3 times, 
and then using the expression for reduced mass $ \mu U R  = m_1 r_1 = m_2 r_2 $, $ R  = r_1 + r_2 $, we have that 
\[
	 \dddot{Q}_{ ij } = 
	 4 \omega ^ 3 \mu R ^ 2   \begin{pmatrix} \sin 2 \omega t & \cos 2 \omega t & 0 \\ \cos 2 \omega t &  - \sin 2 \omega t & 0 \\ 
	 0 & 0 & 0 \end{pmatrix} 
\] Our expression for power 
is given by the expression 
\[
 \mathcal{ P }  = \frac{1}{5 } 
 \dddot{ Q}_{ ij  } \dddot{ Q } ^{ ij }
\] Recalling a factor of $ 2 $ that 
comes from summing the triogonometric 
terms in the matrix, we have that 
\[
 \mathcal{ P }  = \frac{32}{ 5 } \omega^  6 \mu ^ 2 R ^ 4 
\] Substituting our values of $ \omega $ and $ \mu $, this gives our value of $ \mathcal{ P } $ as 
\[
 \mathcal{ P }  = \frac{32}{5 } \frac{ G ^ 4 
 m_1 ^ 2 m_2 ^ 2 \left( m_1 + m_2  \right)  }{ R ^ 5 }
\] 

For the second part of the question, 
we want to solve $ \frac{  d E }{  dt }  =  -\mathcal{ P } $. 
Using the chain rule for our expression for $ E $, we have that 
\[
	\dot{ R } \frac{ G m_1 m_2 }{ 2 R ^ 2 }  =   - \frac{32}{5 } \frac{G ^ 4 m_1 ^ 2 m_2 ^ 2 \left( m_1 + m 2  \right)  }{ R ^ 5 } 
\]
Simplifying, we have that 
\[
	\frac{d R }{ d t }  =  - \frac{64 G ^ 3 m_1m_2 \left( m_1 +  m_2  \right)  }{ 5 R ^ 3 }
\] Substituting in our initial condition 
that $ R =R_0 $ at $ t = 0 $, then we 
get that 
\[
	\frac{R^ 4 }{ 4 }  = \frac{R_0^ 4 }{ 4 }  - \frac{64 }{ 5 } G ^ 3 m_1 m 2 \left( m_1 +  m_2 \right) 
\]

To extract the chirp mass from 
experimentally determining $ \omega $ and $ \dot{ \omega }  $, 
we use the following facts (note that I've emitted constants here).
We have from the definition that 
\[
	\omega \propto \frac{\left( m_1 + m_2  \right)  ^{ \frac{1}{2 } }}{ R ^{ \frac{3}{2 } } }
\] Differentiating $ \omega ^ 2 $ with respect to time, 
and substituting our expression for $ \dot{R }  $ above, 
we find that 
\[
	\dot{ \omega }  \propto \left( m_1 + m_2  \right)  ^{ \frac{3}{2 } } m_1 m_2 R ^{  - \frac{11 }{ 2 } } 
\] Now, just sub out $ R $ in favour of $ \omega $ in to get 
\[
	\dot{ \omega }  = m_1 m_2 \left( m_1 + m_ 2  \right)^{  - \frac{1}{3 } } \omega ^{ \frac{11}{3  }}  
	\] This means 
	that we get an experimentally measured value of $ \frac{m_1 m_2 }{  \left( m_1 + m_2  \right)  ^{ \frac{1}{3 } } } $. Just take this to the power of $ \frac{3}{5 } $ to get the chirp mass. 

\pagebreak

\section{Problem Solving Tips} 



\subsection{Being careful about how we expand things} 
We need to be careful about trying to 
apply the covariant derivative to Christoffel symbols. 
For example, writing out 
\[
 \nabla _ \nu \Gamma ^ \nu _{ \rho \mu } 
\]  doesn't make sense since $ \Gamma $ is not a tensor!
$ \Gamma $ doesn't transform as a function. 
Thus, we need to expand out things the right way. 
Let's try to expand out the 
term 
\[
	\nabla _{ [ \mu } \nabla _{ \nu ] } f 
\] We need to work out from in, because 
we know that $ \nabla _{ \nu } f $ is a covector. 

\subsection{Tricks in the Levi-Civita connection} 
In the Levi-Civita connection, we have that 
raising indexes 

\subsection{Integrating over volume forms} 
Always integrate 'as is'  - we don't have to add
the metric if it's not there. 
When trying to calculate the hodge star, 
write things out in terms of the Vielbein basis. 

The minus sign when taking our hodge star comes
from our $ \hat{\theta } ^ 0 $ has the Minkowski metric. 

(Weird thought - 

\subsection{Showing that magnitude is constant} 
It's a lot easier to 
show things like the angle between two 
vectors being constant 

Sometimes we need to avoid coordinate singularities, 
a work around is to rotate the manifold. 

\subsection{Symmetries of the Riemann Tensor} 
We have 4 main symmetries of the Riemann Tensor to worry about. 
Antisymmetry in the 1 and 2 indices, anti-symmetry in the 3 and 4 indices, 
Symmetry in swtiching, and Bianchi identity.

We do the first steps by writing things out in
terms of an anti-symmetric basis. Then, adding 
on the Bianchi identity is a matter of counting the extra independent 
constraints we can get.

In 2D, this means that $G_{ \mu \nu} =0 $, so no
matter exists. 

\subsection{General techniques} 
Looking at the number of components 
can help simplify things greatly. 
