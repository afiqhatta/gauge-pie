\subsection{A note about dual spaces}
In linear algebra, if we have a vector space which we call $ V$, then we can define a natural object which we call it's dual space, denoted by $V^* $. The dual space is the space of linear functions which takes $V \to  \R$: 
\[
V^* = \{ f \mid f : V \to  \R, \quad \text{ f is linear } \} 
\] 
Now it may seem from first glance that the space of all functions is a lot larger than our original vector space, so it's counter intuitiveto call it the 'dual'. However, we can prove that these vector spaces are isomorphic. Suppose that $\{ e_\mu \} $ is a basis of $V$. Then we pick what we call a dual basis of $V^*$, by choosing the 
\[
\mathcal{B}( V ^* ) = \{ f^\mu \mid f^\mu ( e_\nu )  =\delta \indices{^\mu_\nu} \}  
\] 
One can show that this set indeed forms a basis of $V^* $.
\begin{thm}
The above basis forms a basis of $V^* $. 
 \begin{proof}
	First we need to show that the linear maps above, span our space. 
	This means we need to be able to write any linear map, say $ \omega $, as a linear combination 
	\[
		\omega = \sum \omega_\mu  f^\mu , \omega_\mu  \in F  
	\] 
	To do this, we appeal to the fact that if two linear maps agree on the vector space's basis, 
	then they agree. 
	So, let the values that $ \omega $ takes on the basis be $\omega_\mu  = \omega (e_\mu  ) $. Then, taking 
	\[
	 \Omega  = \sum \omega_\mu  f^\mu  
 \] We find that $ \Omega$ also satisifes $ \Omega( e_\mu  )  = \omega_\mu  $. 
 Thus, the maps are the same. 
 Hence, $\omega$ can be written as the span of our dual basis vectors. 
 To show that these basis vectors are linearly independent, 
 we assume that there exists a non trivial sum such that they add to the zero map.  
	\[
	 0  = \sum \lambda_\mu  f^\mu  
	\] If we apply this map to an arbitrary basis vector $ e_i$, then we get  \[
	0 = \sum \lambda_\mu  f^\mu  ( e_i ) = \lambda_i 				
	\] for arbitrary $i $. Hence, we must have that all $\lambda_i$ are zero. 
	Thus the basis vectors are independent. 
\end{proof}
\end{thm}
Now, assuming that our original vector space $ V$ had finite dimension, the way we've defined the basis of $V^* $
means that we had the same number of basis elements. 
This means that $V $ and $V^* $ have the same dimesion. 
One can prove that vector spaces with the same dimension are isomorphic, so we have that 
\[
V \simeq V^* 
\] Think of a dual space as a 'flip' of a vector space. 
We can identify the dual of a dual space as the original space itself, so that 
\[
\left( V^*  \right)^ * = V 
\] This is because given an object in the dual space $\omega$, we can define a natural map from 
$V: V^* \to  \R $ given by
\[
V \left( \omega  \right)  = \omega \left( V  \right) \in \R
\] 
\subsubsection{Vector and Covector spaces} 
Now, since we've identified tangent spaces as vector spaces, we can proceed to construct its dual. 
If we have a tangent space $ T_p ( \mathcal{ M}) $ with a basis $\{ e_\mu  \}$, our natural dual basis is 
given by 
\[
\mathcal{B }( T_p^* ( \mathcal{M})) = \{ f^\mu  \mid f^\mu  ( e_\nu  ) = \delta \indices{^\mu _\nu  } 
\] The corresponding dual space is denoted as $T_p^* ( \mathcal{ M}) $, and is known as 
the cotangent vector space. 
For brevity, elements in this space are called \textbf{ covectors}. 
In this basis, the elements $ \{ f^\mu  \}$ have the effect of 'picking' out components of a vector $ V = V^\mu e_\mu  $. 
\[
f^\nu  ( V )  =f^\nu  ( V^\mu  e_\mu  )  = V^\mu  f^\nu  ( e_\mu )  = V^\nu  
\] There's a different way to pick elements of this dual space in a smooth way. 
They're chosen by picking elements of a set called the set of 'one forms'. 
We denote the set of one forms, with an index $1 $ as $ \Lambda^{1}( \mathcal{M }) $. 
We can construct elements form this set by taking elements from $ C^\infty ( \mathcal{M} )$. 
Suppose that we have an $ f \in C^\infty ( \mathcal{ M } ) $, then the corresponding one-form is a map 
\[
df : T_p ( \mathcal{ M}) \to  \R \quad V \mapsto V( f) 
\] From one the set of one forms, we then have an obvious way to get the dual basis. 
The dual basis is obtained by just taking the coordinate element of the manifold, so that our one form is 
\[
dx^\nu  : T_p ( \mathcal{ M }) \to  \R
\] This satisfies the property of a dual basis, since 
\[
dx^\nu ( e_\mu  )  = \frac{\partial x^\nu  }{\partial x^\mu  } = \delta \indices{^\nu _\mu  } 
\] With this convention, we can check that 
$ V ( f) $ is what its supposed to be by observing that
\[
df( X)  = \frac{\partial f}{\partial x_\mu} dx^\mu ( X^\nu \partial _\nu )  = X^\nu \frac{\partial f}{\partial x^\nu} = X_ ( f)  
\]  So we recover what we expect 
by setting this as a basis. Now, we should
check whether a change in coordinates leaves our properties invariant. 

Suppose we change our basis from $ x^\mu \to \tilde{z}^\mu ( x)  $, 
then, we know that our basis vector transforms like 
\[
\frac{\partial }{\partial \tilde{x }^\mu }  = \frac{\partial \tilde{x }^\mu }{\partial x^\nu} dx^\nu  
\] We guess that our basis of one forms 
should transform as 
\[
d \tilde{x }^\mu = \frac{\partial \tilde{x}^\mu }{\partial x^\nu} dx^\nu  
\] This ensures that, when we contract a transformed basis one form 
with a transformed basis vector, that
\begin{align*}
d \tilde{x }^\mu \frac{\partial }{\partial \tilde{x }^\nu } &=  \frac{\partial \tilde{x }^\mu}{ \partial x^\rho} dx^\rho \frac{\partial x^\sigma}{\partial \tilde{x }^\nu } \frac{\partial }{\partial x^\sigma}    \\
							    &=  \frac{\partial \tilde{x }^\mu }{\partial x^\sigma} \frac{\partial x^\sigma}{\partial \tilde{ x}^\nu } dx^\rho \left( \frac{\partial }{\partial dx^\sigma}  \right)    \\
							    &=  \frac{\partial \tilde{x}^\mu }{\partial x^\rho} \frac{\partial x^\rho}{\partial \tilde{x }^\nu }  = \delta \indices{^\mu_\nu} 
\end{align*} It's no coincidence that this looks like a Jacobian! 

Now, as with basis elements in our vector space, 
we need to determine how these objects transform under a change of basis. 
*Need to finish this section on basis transformations for covectors* 

