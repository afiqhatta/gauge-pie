\subsection{Question 7} 

\subsubsection*{Lie derivative of 1-form} 
Using the Leibniz rule for our Lie derivative, we consider the Lie derivative 
for $ \mathcal{ L }_X ( \omega Y ) $; 
\[
\mathcal{ L }_X ( \omega( Y) ) = \omega ( \mathcal{ L }_X Y ) + ( \mathcal{ L }_ X \omega) ( Y) 	
\] This expression is basis independent. 
Now, observe that $ \omega ( Y ) $ is a function in  $ C^ \infty ( \mathcal{ M }) $. 
Thus, the Lie derivative for this term is just given by $ X ( \omega ( Y) )  $. 
We also know that $ \mathcal{  L }_X ( Y )  = [ X, Y ] $. 
Thus, 
\begin{align*}
X^\mu\partial _\mu ( \omega_\nu Y ^\nu) & = (\mathcal{L }_X \omega)_\nu Y ^\nu + \omega_\nu [ X, Y ] ^\nu  \\
Y_\nu X^\mu \partial _\mu \omega^\nu + \omega_\nu X^\mu \partial _\mu Y^\nu  &= ( \mathcal{ L }_ X \omega)_\nu Y ^\nu + \omega_\nu X^\mu \partial_\mu Y^\mu  - \omega_\nu Y^\mu \partial _\mu X^\nu
\end{align*} Up to index relabelling of the dummy indices, the last term of the LHS 
and the second term of the RHS are the same, so they cancel out. 
Moving the negative term on the LHS to the right hand side and relabelling gives 
\[
Y_\nu ( X^\mu \partial_\nu \omega^\nu + \omega_\mu \partial_\nu X^\mu ) = ( \mathcal{ L }_X \omega) _\nu Y ^\nu
\] However, since $ Y $ was arbitrary we can just read off the 
basis independent components here. 
\[
( X^\mu \partial_\nu \omega^\nu + \omega_\mu \partial_\nu X^\mu ) = ( \mathcal{ L }_X \omega) _\nu 
\]
\subsubsection*{Lie derivative for a 2-tensor} 
We play exactly the same game with 
the rank $ ( 0 , 2) $ tensor as well. 
 \[
	 \mathcal{L }_ X  ( g ( V , W) ) = (\mathcal{L }_{X } g ) ( V, W) + g ( \mathcal{L }_X V, W ) + g ( V, \mathcal{L }_X W )  
\] In components, and multiplying out with the product rule, 
this term is 
\begin{align*}
	X^\nu (\partial  _\mu g_{ \alpha \beta }) V^\alpha W^\beta & + X^\mu g_{ \alpha \beta } W ^ \beta \partial _\mu V ^\alpha + X^\mu g _{ \alpha \beta } V^\alpha \partial _\mu W^ \beta  = \\ 																	    & = ( \mathcal{ L }_X g ) _{ \alpha \beta } V^ \alpha W ^ \beta + g_{ \alpha \beta } [ X , V]^ \alpha \partial _\mu W^ \beta + g_{ \alpha \beta } V ^ \alpha [ X, W ] ^ \beta 	    
\end{align*} The right hand side 
is just equal to, expanding the commutators in terms of components, 
\[
	= ( \mathcal{ L }_X g ) _{ \alpha \beta } V ^ \alpha W ^ \beta + g_{ \alpha \beta } X^\nu \partial  _\nu V ^ \alpha W ^ \beta  -g _{ \alpha \beta } V ^\nu \partial _\nu X ^ \alpha W ^ \beta + g_{ \alpha \beta } V ^ \alpha X ^ \nu \partial  _\nu W ^ \beta - g_{ \alpha \beta } V ^ \alpha W ^ \beta \partial  _\nu X  ^ \beta 
\] Now up to index relabelling $ \mu $ and $ \nu $, the second and fourth 
terms of this equation cancel out with the second and third terms on the LHS of our first equation. 
Thus, we're left with 
\[
X^\mu \partial _\mu g_{ \alpha \beta } V ^ \alpha W ^ \beta + g_{ \alpha \beta } V ^ \nu ( \partial  _\nu X ^ \alpha ) W ^ \beta + g _{ \alpha \beta } V ^ \alpha W^\nu ( \partial  _\nu X ^ \beta ) = ( \mathcal{ L }_ X g )_{ \alpha \beta } V ^ \alpha W ^ \beta 
\] Now, as before, relabelling $ \nu , \alpha$ in the second term and $ \nu , \beta $ in the third term 
recovers the expression in the question (after factorising out  $ V , W $).

\subsubsection*{Last part} 
The last part of the question is just a matter of 
substituting in definitions. 
\begin{align*}
	( \iota_X d \omega ) _ \mu &=  X^\nu ( d \omega) _{ \nu\mu } \\
				   &=  X ^ \nu 2 \partial  _{ [ \nu } \omega _{ \mu ] } \\
				   &=  X^\nu \partial  _\nu \omega_{ \mu } - X ^ \nu \partial  _{ \mu } \omega _{ \nu } 
\end{align*}
Also, we have 
\begin{align*}
	d ( \iota _{ X  } \omega) _{ \mu }  &=  \partial  _\mu ( \iota _ X \omega )  \\
					    & =  \partial _ \mu ( X^\nu \omega_\nu) \\
					    &=  \omega_ \nu \partial  _{ \mu } X ^ \nu + X ^ \nu \partial _ \mu \omega _ \nu 
\end{align*}
Adding these terms together gives 
\[
 ( \iota _ X d \omega ) _ \mu + d ( \iota _ X \omega )_\mu  = X^ \nu \partial  _ \nu \omega _ \mu - X ^ \nu \partial  _ \mu \omega _ \nu + \omega _ \nu \partial  _ \mu X ^ \nu + X ^ \nu \partial  _ \mu \omega _ \nu  = X^ \nu \partial  _ \nu  \omega _ \mu + \omega _ \nu X ^ \nu 
\] since we have cancellation with the second and last term. 



\pagebreak 

