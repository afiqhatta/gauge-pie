\pagebreak
\subsection{Derived tensors from the Riemann Tensor} 
From our Riemann tensor, we 
can easily build new tensors by contracting over some indices. 
In this bit, we'll define some really important 
tensors given by our curvature. 

\begin{defn}{(Ricci Tensor)} 
	Our Ricci tensor is given by contracting our first 
index with our third component of the Riemann tensor. 
\[
 R_{ \mu \nu } = R \indices{ ^ \rho _{ \mu \rho \nu } } 
\] This obeys the symmetry property that $  R_{ \mu \nu  }  = 
R_{ \nu \mu } $. 
\end{defn} 

\begin{defn}{(Ricci Scalar)} 
From this object, we can then 
get our Ricci scalar which is given by 
\[
 R = g^{ \mu \nu } R _{ \mu \nu }
\] 
\end{defn} 
Applying our Bianchi identity to the Riemann tensor, we have 
that 
\[
 \nabla ^ \mu R_{ \mu \nu } = \frac{1}{2 } \nabla _ \nu  R 
\] This can be shown by considering 
the Bianchi idenity 
\[
	\nabla _{ [ \rho } R _{ \alpha \beta ] \gamma \delta }  = 0 , 
	\implies
	\nabla _ \rho R_{ \alpha \beta \gamma \delta } + 
	\nabla _\alpha R_{ \beta \rho \gamma \delta } + 
	\nabla _ \beta R_{ \rho \alpha \gamma \delta } = 0 
\] Now, raising and lowering 
indices, and making use of the symmetries 
of the Riemann tensor should make 
the above result pop out. 
In this spirit, we define the Einstein tensor
\[
 G_{ \mu \nu } = R _{ \mu \nu } - \frac{1}{2 } R g_{ \mu \nu }
\] This obeys the rule that the covariant derivative 
is 
\[
 \nabla ^ \mu G _{ \mu \nu } = 0 
\] 
\subsection{Connection 1-forms} 
We now will introduce a handy technology to compute 
our Riemann tensor, which will make it easier to do instead of 
computing all the Christoffel components. 
If we had a basis $ \left\{  e _ \mu  \right\}  = \left\{  \partial  _ \mu  \right\}  $, 
we can always introduce a different basis which is a linear 
sum of the coordinate induced basis. We call this 
basis 
\[
 \hat{e}_a = e \indices{ _ a ^ \mu } \partial  _ \mu  
\] The upshot of this is that on a
Riemannian manifold, we can pick a basis such that 
\[
	g ( \hat{e }_a , \hat{e}_b ) = g_{ \mu \nu } e \indices{ _ a ^ \nu } e \indices{ _ b ^ \nu } = \delta _{ ab }  
\] We call the components 
$ e \indices{ _ a ^ \mu }  $ are called vielbeins. 
We can raise and lower indices using $ g_{ \mu \nu } $ and a, b using $ \delta _{ a b } $
The basis of one forms $ \left\{  \hat{\theta } ^ a   \right\} $ obey 
our standard dual basis relation 
\[
	\hat{\theta } ^ a ( \hat{e } _ b ) = \delta \indices{ ^ a _ b }  
\]  They are $ \hat{ \theta } ^ a = e \indices{ ^ a _ \mu } dx ^ \mu  $, 
with $ e \indices{ ^ a _ \mu } e \indices{ _ b ^ \mu }  = \delta \indices{ ^ a _ b }  $. 
This property satisfies $ e \indices{ ^ a _ \mu } e \indices{ _ a ^ \nu } = \delta \indices{ _ \mu ^ \nu } $. 
Our metric is 
\[
 g = g_{ \mu \nu } dx ^ \mu \otimes dx ^ \nu = \delta _{ ab } \hat{\theta } ^ a \otimes \hat{ \theta } ^ b 
\] This means that 
\[
 g_{ \mu \nu } = e \indices{ ^ a _ \mu } e ^{ ^ b _ \nu } \delta _{ ab }  
\] An example would be to consider the metric 
(Schwarzchild metric) 
\[
	d s ^ 2 = - f ( r ) ^ 2 dt ^ 2 + f ( r) ^{ - 2} dr ^ 2 
	+ r ^ 2 ( d \theta ^ 2 + \sin ^2 \theta d \phi ^ 2 ) 
\] This is $ ds ^ 2   = \eta ^{ ab  } \hat{ \theta } ^ a \otimes \hat{ \theta } ^ b $, 
with non-coordinate 1 forms 
\[
 \hat{ \theta } ^ 0 = f dt , \quad \hat{ \theta  } ^ 1 = f ^{ -  1 } dr, \quad \hat{ \theta } ^ 2 = r d \theta , 
 \hat{ \theta  } ^ 3 = r \sin \theta d \phi 
\] In the basis $ \left\{  \hat{ e } _ a  \right\}  $, the components 
of our connection are defined as (not to be confused with our usual 
connection) that 
\[
 \nabla _{ \hat{ e } _ c } \hat{ e } _ b : = \Gamma ^ a _{ b c } \hat{ e } _a 
\] We define the connection 
1-form to be $ \omega  \indices{ ^ a _ b } = \Gamma ^ a _{ bc } \hat{ \theta } ^ c  $, 
this is called the spin connection. 

We have the first Cartan structure equation. 
\[
 d \hat{ \theta } ^ a + \omega  \indices{ ^ a _ b } \wedge  \hat{ \theta } ^ b = 0  
\]
In addition, we have for the Levi-Civita connection, that 
our Cartan structure components $ \omega_{ ab } = - \omega _{ b a}$. 
In the Vielbein basis, we set 
\[
	R \indices{ ^ a _{ bcd } } = R ( \hat{ \theta } ^ a ; \hat{ e } _ c , \hat{ e } _ d , \hat{ e } _ b )  
\] with $ R \indices{ ^ a _{ b cd } }   = - R \indices{ ^ a _{ b d c } }  $. We 
define the curvature 2 form to be 
\[
 R \indices{ ^ a _ b } = \frac{1}{2 } R \indices{ ^ a _{ b cd  } } \hat{ \theta } ^ c 
 \wedge  \hat{ \theta } ^  d
\] This gives our second structure equation. 
\[
 R \indices{ ^ a _ b } = d \omega \indices{ ^ a _ b } + \omega \indices{ ^ a _ c } \wedge  \omega \indices{ ^ c _ b}    
\] We should think about what information is compressed and what is not.

Let's go back to our example. We compute 
$ d \hat{ \theta } ^ a $, which gives us 
\begin{align*}
	d \hat{ \theta } ^ 0 &= f' dr \wedge  dt \\
	d \hat{ \theta } ^ 1 &=  f' dr \wedge  dr =  0  \\
	d \hat{ \theta } ^ 2 &=  dr \wedge  d \theta  \\ 
	d \hat{ \theta } ^ 3 &=  \sin \theta dr \wedge  d \phi - r \cos d \theta \wedge  d \phi  \\
\end{align*} 
Now, we use our Cartan structure equation to get 
\[
 d \hat{ \theta } ^ 0 = 0 \omega \indices{ ^ 0 _ b } \wedge  \hat{ \theta } ^ b 
 \implies \omega \indices{ ^ 0 _ 1 } = f'  f dt = f ' \theta ^ 0  
\] But, we use our anti-symmetry and the Minkowski 
metric to get that 
\[
 w \indices{ ^ 0 _ 1 } = - \omega _{ 0 1} = \omega _{ 10 } = \omega \indices{ ^ 1 _ 0 }  
\] To check the consistency of this system, 
we check that our equation for 
\[
 d \hat{ \theta } ^ 1 = 0 
\] Proceeding like this, we find that 
\begin{align*}
	\omega \indices{ ^ 0 _ 1 } &=  \omega \indices{ ^ 1 _ 0 } = f ' \hat{ \theta  } ^ 0   \\
	\omega \indices{ ^ 2 _ 1 } &= - \omega \indices{ ^ 1 _ 2  } = \frac{ f }{ r } \hat{ \theta }^ 2 \\
	\omega \indices{ ^ 3 _  1 }   &= - \omega \indices{ ^ 1 _ 3 } = \frac{f}{r } \hat{ \theta } ^ 3 \\
	\omega \indices{ ^ 3 _ 2 } &= - \omega \indices{ ^ 2 _ 3 }   = \frac{ \cot \theta }{ r  } \hat{ \theta } ^ 3 
\end{align*}
Now the curvature tensor 
\begin{align*}
	\mathcal{ R } \indices{ ^ 0 _ 1 } &=  d \omega \indices{ ^ 0 _ 1 }  + \omega \indices{ ^ 0 _ c 
	} \wedge  \omega \indices{ ^ c _ 1 }     \\
	&=  f ' d \hat{ \theta } ^ 0 + f'' dr \wedge  \hat{ \theta } ^ 0  \\
	&=  ( ( f ' ) ^ 2 + f '' f ) dr \wedge  dt  \\
	&=   - ( ( f ' ) ^ 2 + f '' f ) \hat{ \theta } ^ 0 \wedge  \hat{ \theta } ^ 1  \\
	& \implies R_{ 0101 }  = ( f f '' + f '' f ) 
\end{align*} 
We can convert back by using 
\[
	R_{ \mu \nu \rho \sigma } = e \indices{ ^ a _ \mu } e ^{ ^ \beta _ \nu } e \indices{ ^ c _ \rho } 
	e \indices{ ^ d _ \sigma } R _{ abc d} 
\] This means that $ R _{ trtr }  = f f'' + ( f' ) ^ 2$. 
