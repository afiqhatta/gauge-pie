\subsection{Integral curves} 
We'll now do an interesting diversion to discuss flows on a manifold. Think of a body of water moving smoothly on a surface, where each point on the manifold moves to a different point after some amount of time. This is a what a flow is. More specifically, it is a smooth map $\sigma_t : \mathcal{M} \rightarrow \mathcal{M} $ on the manifold (which makes it a diffeomorphism), where $t$ is our 'time' parameter that we were talking about. As such, these flow maps actually form an abelian group, where 
\begin{itemize} 
\item $\sigma_0 = I_\mathcal{M} $. So, after time $t = 0$ has passed, nothing has moved so we have that this is the identity map. 
\item If we compose the same flow after two intervals in time, we should get the same flow when we've let the sum of those times pass over. So, 
\[ \sigma_s \circ \sigma_t  = \sigma_{ s + t } \] 
\end{itemize} 
If we take a flow at a given point $p \in \mathcal{M}$, we can define a curve on the manifold by setting: 
\[ 
\gamma(t): \mathbb{R} \rightarrow \mathcal(M), \quad \gamma(t) = \sigma_t ( p ) 
\] where without loss of generality we have $\gamma(0) = p$. Since this is a curve, we can define it's associated curve in $R^n$ space with a given coordinate chart, and hence associate with it a tangent vector $X_p$. We can also work backwards. From a given vector field, we can solve the differential equation 
\[ 	
X^\mu ( x(t) )  = \frac{ d x^\mu ( t) }{ dt} 
\] with the initial condition that $x^\mu ( 0 )  = \psi ( p ) $ at some point in the manifold, and have that this defines a unique curve. The set of curves then together form a flow. 
Thus, we've seen a one to one correspondence between vector fields and flows. 	
\pagebreak 

