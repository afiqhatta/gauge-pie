\documentclass[11pt, a4paper]{article}   	% use "amsart" instead of "article" for AMSLaTeX format
 
\usepackage[margin = 1.0in]{geometry}            		% See geometry.pdf to learn the layout options. There are lots.
\geometry{letterpaper}                   		% ... or a4paper or a5paper or ... 
\usepackage[parfill]{parskip}    		% Activate to begin paragraphs with an empty line rather than an indent
\usepackage{graphicx}				% Use pdf, png, jpg, or eps§ with pdflatex; use eps in DVI mode
							% TeX will automatically convert eps --> pdf in pdflatex	
\usepackage{adjustbox}	
\usepackage[section]{placeins}


%% LaTeX Preamble - Common packages
\usepackage[utf8]{inputenc}
\usepackage[english]{babel}
\usepackage{textcomp} % provide lots of new symbols
\usepackage{graphicx}  % Add graphics capabilities
\usepackage{flafter}  % Don't place floats before their definition
\usepackage{amsmath,amssymb}  % Better maths support & more symbols
\usepackage[backend=biber]{biblatex}
\usepackage{amsthm}
\usepackage{bm}  % Define \bm{} to use bold math fontsx
\usepackage[pdftex,bookmarks,colorlinks,breaklinks]{hyperref}  % PDF hyperlinks, with coloured links
\usepackage{memhfixc}  % remove conflict between the memoir class & hyperref
\usepackage{mathtools}
\usepackage[T1]{fontenc}
\usepackage[scaled]{beramono}
\usepackage{listings}
\usepackage{physics}
\usepackage{tensor}
\usepackage{tikz}
\usepackage{multicol} 

%% Commands for typesetting theorems, claims and other things.
\newtheoremstyle{slplain}% name
  {.5em}% Space above
  {.5em}% Space below
  {}% Body font
  {}%Indent amount (empty = no indent, \parindent = para indent)
  {\bfseries}%  Thm head font
  {. }%       Punctuation after thm head
  {}%      Space after thm head: " " = normal interword space;
        %       \newline = linebreak
  {}%       Thm head spec

\theoremstyle{slplain}
\newtheorem{theorem}{Theorem}
\newtheorem*{thm}{Theorem}
\newtheorem*{claim}{Claim}
\newtheorem*{example}{Example}
\newtheorem*{defn}{Definition}
\newtheorem*{tech}{Technique}
\newtheorem*{question}{Questions} 

\newcommand{\Lagr}{\mathcal{L}}
\newcommand{\vc}[1]{\mathbf{#1}}
\newcommand{\pdrv}[2]{\frac{\partial{#1}}{\partial{#2}}}
\newcommand{\thrint}[1]{\int d^3 \vc{x} \left( {#1} \right)}
\newcommand{\R}{ \mathbb{R}}

\author{Afiq Hatta} 
\title{Cool stuff in General Relativity}
\begin{document}
\maketitle

\tableofcontents

\section{Starting off with non-relativistic particles} 
Deriving the Euler Lagrange equations. We perturb our action \[ S = \int_{t_1}^{t_2} dt \, L \] 
If we perturb our action slightly, 
\begin{align*} 
 S[x^i + \delta x^i ]& = \int_{t_1}^{t_2} dt \, L( x^i + \delta x^i, \dot{x}^i + \delta  \dot{x}^i ) \\ 
& = S[x^i ] + \int_{t_1}^{t_2} dt \, \left( \frac{ \partial L}{\partial x^i } \delta x^i + \frac{ \partial L }{ \partial \dot{x}^i } \delta \dot{x} ^i \right) \\
& = S[x^i] + \int_{ t_1 }^{t_2} dt \, \delta x^i \left( \frac{ \partial L }{ \partial x^i }  - \frac{d}{dt} \left( \frac{ \partial L }{ \partial \dot{x}^i } \right) \right ) 
\end{align*} 
This implies that the integrand goes to zero. 

\subsection{Exploring different Lagrangians} 
Consider the Lagrangian in Euclidean coordinates 
\[ L = \frac{1}{2} m ( \dot{x}^2 + \dot{y}^2 + \dot{z}^2 ) 
\] 
The E-L equations imply that $\ddot {x} = 0 $. So we have a constant velocity. Now in different coordinates, 
\[ L = \frac{ 1}{2} g_{ ij } (x) \dot{x}^i \dot{x}^j \] 
This is a metric. Our distance in these general coordinates between $x^i \rightarrow x^i + \delta x^i $ is now 
\[ 
ds^2 = g_{ij} dx^i dx^j \] 
Some $g_{ij} $ do not come from $\mathbb{R}^3 $, and these spaces are \textbf{curved}. This means there is no smooth map back into $\mathbb{R}^3$. Our equations of motion that comes from the Euler Lagrange equations are the geodesic equations.  

Observe that 
\begin{align*} 
\frac{d}{dt} \left( \frac{ \partial L}{\partial \dot{x}^i } \right) &= \frac{d}{dt} g_{ ik}\dot{x}^k \\
&= g_{ik} \ddot{x}^k + \dot{x}^k \dot{x}^l \partial_l g_{ ik}\end{align*} 

And, differentiating the lagrangian with $\partial_i$, we get
\[ 
\frac{ \partial L}{ \partial x^i} = \frac{ 1}{2} \partial_i g_{kl} \dot{x}^k \dot{x}^l \]
Substiting this into the EL equations, this reads 
\[ 
g_{ik} \ddot{x}^k + \dot{x}^k \dot{x}^l \partial_l g_{ ik}  - \frac{1}{2} \partial_i g_{kl} \dot{x}^k \dot{x}^j  =0 
\]  
Now, the second term is symmetric in $k, l$, so we can split this term in two. We also multiply by the inverse metric to cancel out this annoying factor of $g$ that we have in front of everything. This gives us the final expression that 
\[ 
\ddot{x}^i + \frac{1}{2} g^{il} \left( \partial_k g_{lj} + \partial_j g_{lk}  - \partial_l g_{jk} \right) \dot{x}^j \dot{x}^k = 0
\] 

\section{Special relativity} 
We'll now put time and space on the same 'footing' per se, and talk about special relativity. In special relativity, instead of time being its own separate variable, we have that dynamic events take place in 4 spacetime coordinates denoted $x^\mu  = (t, x, y, z)$, where we now use greek indices to denote four components $\mu = 0, 1, 2 , 3$. Now, we wish to construct an action and extremize this path, but since our $t$ variable is already taken, we need to parametrise paths in spacetime by a different parameter. We'll call this parameter $\sigma$, and show that there's a natural choice for this, something called 'proper time', later. 

We define our metric, the Minkowski metric, on this spacetime to be $\eta^{ \mu\nu} = diag( -1, +1, +1, +1) $. Thus, distances in Minkwoski spacetime are denoted as \[ 
ds^2  = \eta_{ \mu \nu} dx^\mu dx^\nu  =  -dt^2 + dx^2 + dy^2 + dz^2 \] 

We have names for different events based on their infinitesimal distance. Since our metric is no longer positive definite, we have that events can have a distance of any sign. 

\begin{itemize} 
\item If $ds^2 < 0$, events are called timelike. 
\item If $ds^2 = 0$, events are called null. 
\item If $ds^2 > 0$, events are called spacelike. 
\end{itemize} 


Our action, then, should look like (now with the use of an alternate parameter $\sigma$ to parametrize our paths) 
\[ 
S[ x^\mu ( \sigma) ] = \int_{ \sigma_1}^{\sigma_2} \sqrt{ - ds^2 } \]
Now, we can parametrise the integrand with sigma to get 
\[ 
S[x^\mu (\sigma) ] = m \int_{\sigma_1}^{ \sigma_2 } d\sigma \, \,  \sqrt{ - \eta_{ \mu \nu} \frac{ d x^\mu}{ d \sigma} \frac{ dx^\nu}{ d \sigma} } 
\] 
In this case, our Lagrangian $L = 
m \sqrt{ - \eta_{ \mu \nu} \frac{ d x^\mu}{ d \sigma} \frac{ dx^\nu}{ d \sigma} }$. Now, before we begin analysing what this equation gives us, there are two symmetries we'd like to take note of. One of our symmetries is invariance under Lorentz transformations. This means, if we boost our frame with a Lorentz transformation $x^\mu \rightarrow \Lambda\indices{^\mu_\nu} x^\nu $, one can easily verify, using the condition that \[ 
\Lambda\indices{^\alpha_\mu} \Lambda\indices{^\beta_\nu} \eta_{\alpha\beta} = \eta_{ \mu \nu} 
\] 
that the Lagrangian remains invariant under this. One can also verify that this action is invariant under reparametrisation of the curve via a new function $\sigma' = \sigma' ( \sigma)$.  

Using the chain rule, we reparametrise by rewriting the action as 
\[ 
S = m \int \frac{d \sigma}{ d \sigma' } d\sigma' \, \quad \sqrt{  - \eta_{ \mu\nu} \left( \frac{ d \sigma' }{ d\sigma} \right)^2 \frac{ dx^\mu}{ d \sigma' } \frac{ dx^\nu}{ d \sigma' } } = \int d \sigma'\,  \sqrt{  - \eta_{ \mu\nu} \frac{ d x^\mu}{ d \sigma'} \frac{ dx^\nu}{ d\sigma'} } \] 

In this case, we're just applying the chain rule but factoring out the the $\frac{ d \sigma ' }{ d \sigma} $ term. But this is exactly the same as what we had before. Thus, we have reparametrisation invariance.In analogy with classical mechanics, we compute the conjugate momentum \[ 
p_\mu = \frac{ \partial L }{ \partial \dot{x}^\mu }  
\] 


\pagebreak 
\section{Introducing Differential Geometry for General Relativity} 
Our main mathematical objects of interest in general relativity are manifolds. Manifolds are topological spaces which, at every point, has a neighbourhood which is homeomorphic to a subset of $\mathbb{R}^n$, where we call $n$ the dimension of the manifold. In plain English, manifolds are spaces in which, locally at a point, look like a flat plane. This can be made more rigourous by the creation of maps, which we call 'charts', that take an open set around a point (a neighbourhood), and mapping this to a subset of $\mathbb{R}^n$. 

Precisely, for each $p \in \mathcal{M}$, there exists a map $\phi : \mathcal{O} \rightarrow \mathcal{U} \subset \mathbb{R}^n$, where $p \in \mathcal{ O} \subset \mathcal{M}$, and $\mathcal{ O} $ is an open set of $M$ defined by the topology. Think of $\phi$ as a set of local coordinates, assigning a coordinate system to $ p$. We will write $\phi(p) = (x_1, \dots x_n) $ in this regard.   

This map must be a 'homeomorphism', which is a continuous, invertible map with a continuous inverse. In this sense, our idea of assigning local coordinates to a point in $\mathcal{M}$ becomes even more clear. 

We can define different charts to different regions, but we need to ensure that they're we'll behaved on their intersections. Suppose we had two charts and two open sets defined on our manifold, and looked at how we transfer from one chart to another. For charts to be compatible, we require that the map 
\[ 
\phi_\alpha \circ \phi_\beta^{ -1}: \phi_\beta ( \mathcal{ O}_\alpha \cap \mathcal{O}_\beta ) \rightarrow \phi_\alpha (\mathcal{ O}_\alpha \cap \mathcal{O}_\beta )
\] is also smooth (infinitely differentiable). 

A collection of these maps (charts) which cover the manifold is called an atlas, and the maps taking one coordinate system to another ($\phi_\alpha \circ \phi_\beta^{ -1} $), are called transition functions. 

Some examples of manifolds include 
\begin{itemize} 
\item $\mathbb{R}^n$ and all subsets of $\mathbb{R}^n$ are n-dimensional manifolds, where the identity map serves as a sufficent chart. 
\item $S^1, S^2$ are manifolds, with modified versions of polar coordinates patched together forming a chart (as we'll see in the case of $S^1$. 
\end{itemize}
Let's start simple and try to construct a chart for $S^1$. Our normal intuition would be to use a single chart $S^1 \rightarrow [0, 2 \pi) $, which indeed covers $S^1 $ but doesn't satisfy the condition that the target set is an open subset of $\mathbb{R}$. This yields problems in terms of differentiation functions at the point $0 \in \mathbb{R}$, because the interval is closed there, not open. One way to remedy this is to define two coordinate charts then path them together to form an atlas. Our first open set will be the set of points on the circle which exclude the rightmost point on the diameter, a set denoted by $\mathcal{O}_1$, and our second open set is the whole sphere excluding the leftmost point. We'll denote this $\mathcal{ O}_2 $. 

We assign the following charts which are inline with this geometry
\begin{align*} 
\phi_1 : \mathcal{O}_1 &\rightarrow \theta_1 \in (0, 2 \pi ) \\
\phi_2 : \mathcal{O}_2 &\rightarrow \theta_2 \in ( - \pi , \pi ) 
\end{align*} 

It's easy to verify that if we take a point on the manifold, our transition matrix reads that 
\[ 
\theta_2 = \phi_2 ( \phi_1^{ -1} ( \theta_1) )  = \begin{cases} 
\theta_1, \, \theta_1 \in (0 , \pi ) \\
\theta_1  - 2 \pi,  \, \theta_1 \in ( \pi, 2 \pi ) 
\end{cases} 
\] 
Now that we have coordinate charts, we can do things that we usually do on functions described in $\mathbb{ R}^n $, like differentiate. 
Furthermore, we can define maps between manifolds (which don't necessarily have the same dimension), where smoothness is defined via smoothness on coordinate charts. These are called diffeomorphisms. A function 
\[
f: \mathcal{ M} \rightarrow \mathcal{ N } 
\]
is a diffeomorphism if the corresponding map between $\mathbb{R}^{ dim \mathcal{M} } $ and $\mathbb{R}^{ dim \mathcal{N } } $ is smooth: 
\[ 
\psi \circ f \circ \phi^{ -1} : U_1 \rightarrow U_2 
\] for all coordinate charts $\phi : \mathcal{O}_1 \rightarrow U_1$ and $\psi: \mathcal{O}_2 \rightarrow U_2$ defined on the manifolds $\mathcal{M}$ and $\mathcal{N}$ respectively. 

\subsection{Tangent vectors}
Throughout our whole lives, we've been thinking of a 'vector' as a way to denote some position in space. However, this idea of a vector is only really unique to the manifold $\mathbb{R} ^n $. A much more universal concept of a vector is the idea of 'velocity', the idea of movement and direction at a given point.  A tangent vector is a 'derivative' form at a given point in the manifold. This means that we define it to obey properties that one might expect in our usual notion of a derivative for functions in $\mathbb{R}$. We denote a vector at a point $ p \in \mathcal{M}$ as $X_p$. 
This means that a vector is simply a map $X_p : C^\infty \rightarrow \mathbb{R} $, which satisfies  
\begin{itemize} 
\item Linearity: 
\[ X_p( \alpha f + \beta g ) = \alpha X_p ( f) + \beta X_p (g), \quad \forall f, g \in C^\infty (\mathcal M ), \, \, \alpha, \beta \in \mathbb{R} \]
\item  $X_p( f)  = 0$ for constant functions on the manifold.
\item Much like the product rule in differentiation, tangent vectors should also obey the Leibniz rule where 
\[ 
X_p(fg) = f(p ) X_p (g) + g(p) X_p (f) 
\] 
Remember that with the Leibniz rule, the functions which are not differentiated are evaluated at $p$! This is useful for our theorem afterwards.  
\end{itemize} 
This next proof is about showing that tangent vectors can be built from differential operators in the $n$ dimensions of the manifold. 
We will now show that all tangent vectors $X_p$ have the property that they can be written out as 
\[ 
X_p = X^\mu  \left. \frac{ \partial }{ \partial x^\mu } \right\vert_{p} 
\] What we're saying here is that $ \partial_\mu $ at the point $p \in \mathcal{M} $ forms a basis for the space of tangent vectors at a point. 
To do this, take your favourite arbitrary function $f: \mathcal{ M} \rightarrow \mathbb{R} $. Since this is defined on the manifold, to make our lives easier we'll define $F = f \circ \phi^{ -1}: \mathbb{R}^n \rightarrow \mathbb{R} $, which we know how to differentiate. The first thing we'll show is that we can locally move from $F(x(p)) \rightarrow F(x(q)) $ by doing something like a Taylor expansion: 
\[ 
F( x(q)) = F(x(p))    + ( x^\mu (q)  - x^\mu (p) ) F_\mu ( x( p)) 
\] 
Here, we're fixing $p \in \mathcal{ M} $ and $F_\mu$ is some collection of $n$ functions. One can easily verify that $F$ can be written in this way by precisely doing a Taylor expansion then factorising out the factors of $(x^\mu (q)  - x^\mu (p) )$. We can find an explicit expression for $F_\mu ( x(p ))$ by differentiating both sides and then evaluating at $x( p)$. We have that 
\[ 
\left. \frac{ \partial F}{ \partial x^\nu } \right\vert_{ x(p)}  = \delta\indices{^\mu_\nu} F_\mu + (x^\mu (p)  - x^\mu( p) ) \left. \frac{ \partial F_\mu}{ \partial x^\nu } \right\vert_{ x( p)}  = F_\nu \] 
The second term goes to zero since we're evaluating at $x(p)$, and our delta function comes from differentiating a coordinate element. Our initial $F(x(p))$ term goes to zero since it was just a constant. Recalling that $\phi^{ -1} \circ x^\mu ( p )  = p$, we can just rewrite this whole thing as 
\[ 
f(q) = f(p) + (x^\mu (q) - x^\mu ( p)) f_\mu (p ) 
\] 
where in this case we've defined that $f_\mu (p ) = F_\mu \circ \phi^{ - 1} $. However, we can figure out what this is explicitly
\[ 
f_\mu ( p ) = F_\mu \circ \phi ( p )  = F_\mu ( x(p)) =  \frac{\partial F (x(p)) }{ \partial x^\mu } = \frac{ f \circ \phi^{ -1 } ( x (p))}{ \partial x^\mu } : = \left. \frac{ \partial f}{ \partial x^\mu } \right\vert_{ p } 
\] 
Now, its a matter of applying our tangent vector to our previous equation, recalling that $X_p (k ) = 0 $ for constant $k$, and that all functions are evaluated at the point $ p$. We have that, upon application of the Leibniz rule 
\begin{align*} 
X_p ( f(q)) & = X_p ( f(p)) + X_p ( x^\mu (q) - x^\mu (p) ) f_\mu (p) + ( x^\mu (p) - x^\mu(p) )X_p ( f_\mu ( p)) \\
& = X_p ( x^\mu (p)) f_\mu ( p) \\
&= X^\mu f_\mu ( p) \\
&= X^\mu \left. \frac{ \partial f }{ \partial x^\mu } \right\vert_p 
\end{align*} 
In the first line we've replaced $q$ with $p$ in the last term since Leibniz rule forces evaluation at $p$. We've declared $X^\mu = X_p (x^\mu) $ as our components. Since $f$ was arbitrary, we have now written that 
\[ 
X_p = X^\mu \frac{ \partial }{ \partial x^\mu } 
\] 
To show that $\{ \partial_\mu \} $ forms a basis for all tangent vectors, since we've already shown that they span the space we need to show they're linearly independent. Suppose that 
\[ 
0 = X^\mu \partial_\mu
\] 
Then, this implies that if we take $f = x^\nu$, then $0 = X^\nu$ for any value of the index $\nu$ we take. So, we have linear independence.   

\subsubsection*{Tangent vectors should be basis invariant objects} 
A tangent vector is a physical thing. However, so far we've expressed it in terms of the basis objects $ \{ \partial_\mu \} $ which are chart dependent.So, suppose we use a different chart which is denoted by coordinates $ \tilde{x}^\mu$. This means that our new tangent vector needs to satisfy the condition that 
\[ 
X_p = X^\mu \left.  \frac{ \partial}{ \partial x^\mu } \right\vert_{ p}   = \tilde{X}^\mu \left. \frac{ \partial }{ \partial \tilde{x}^\mu} \right\vert_p
\] 
This relation allows us to appropriately relate the components $X^\mu$ to that of $\tilde{X}^\mu$, in what is called a contravariant transformation. Using the chain rule, we have that 
\[ 
X^\mu \left. \frac{\partial}{ \partial x^\mu} \right\vert_p  = X^\mu \left. \frac{ \partial \tilde{ x}^\nu}{ \partial x^\mu} \right\vert_{ \phi(p )} \left. \frac{ \partial}{ \partial \tilde{x}^\nu } \right\vert_p 
\] Notice that when differentiating a coordinate chart with respect to another, we're evaluating at the coordinate chart of the point. This is why we subscript with $\phi( p)$ in the terms. Comparing coefficients, we have that 
\[ \tilde{X}^\nu = X^\mu \left. \frac{ \partial \tilde{x}^\nu }{ \partial x^\mu } \right\vert_{ \phi(p) } \] 


\subsection{Treating tangent vectors as derivatives of curves on the manifold}
We present a different way to think of tangent vectors, which is viewing them as 'differential operators along curves'. Consider a smooth curve along our manifold, which we can parametrise from on an open onterval $I = (0 , 1)$, and define the starting point of this curve at $p \in \mathcal { M} $; 
\[ 
\lambda: ( 0, 1) \rightarrow \mathcal{ M}, \quad \lambda( 0 ) = p 
\]  
We now ask the question, how do we differentiate along this thing? To do this, we'll have to apply coordinate charts so that we can make sense of differentiation. So, suppose we would like to differentiate a function $f$ along this manifold. We apply our chart $\phi$ to $\lambda$ to get a new function $ \phi \circ \lambda : \mathbb{R} \rightarrow \mathbb{R}^n$, which we'll suggestively write as $x^\mu ( t) $. In addition, to be able to differentiate $f$ in a sensible way we also construct the function $F = f \circ \phi^ { -1} $ . Thus, differentiating a function along a curve $x^\mu ( t) = \phi \circ \lambda ( t) $ should look like 
\begin{align*} 
\frac{ d}{ dt} f(t) &= \frac{ d}{ dt} \left( F \circ \phi^{ -1} \circ \phi \circ \lambda (t) \right) \\
& = \left. \frac{d}{ dt} \left( F \circ \phi^{ -1} \circ x^\mu ( t) \right) \right\vert_{ t = 0} \\
&= \frac{ dx^\mu}{ dt} \left. \frac{ \partial F \circ \phi^{ -1} }{ \partial x^\mu } \right\vert_{ \phi( p ) } \\
&= \left. \frac{dx^\mu}{dt} \right\vert_{t =0} \left. \frac{ \partial f}{ \partial x^\mu } \right\vert_{ p } \\
&= X^\mu \partial_\mu (f) \\	
&= X_p (f)
\end{align*} 
Thus, differentiating along a curve gives rise to a tangent vector acting on $f$. 

\subsection{Vector fields} 
Thus far we've defined tangent spaces at only a specific point in the manifold, but we'd like to know how we can extend this notion more generally. A vector field $X$ is an object which takes a function, and then assigns it avector at any given point $p \in \mathcal{M}$. So, we're taking 
\[ 
X: C^\infty ( \mathcal{ M} ) \rightarrow C^\infty( \mathcal{M} ), \quad f \mapsto X(f) 
\]
$X(f)$ is a function on the manifold which takes a point, and then differentiates $f$ according to the tangent vector at that point. 
\[ 
X(f)( p) = X_p ( f), X = X^\mu \partial_\mu 
\] 
In this case, $X^\mu$ is a smooth function which takes points on the manifold to the components of the tangent vector $X_p^\mu$ at $p$. We call the space of vector fields $X$ as $\mathcal{ X} (\mathcal{M}) $.  
So, since $X(f)$ is now also a smooth function on the manifold, we can apply another vector field $Y$ to it, for example. However, is the object $XY$ a vector field on it's own? The answer is no, because vector fields also have to obey the Liebniz identity at any given point, ie the condition that 
\[ 
X(fg) = f X(g) + g X( f) 
\] 
However, the object $ XY$ does not obey this condition since 
\begin{align*} 
XY( fg )& = X ( f Y (g) + g Y ( f) ) \\
	&= X(f) Y(g) + f XY ( g) + X( g) Y(f) + g XY( f) \\ 
	& \neq g XY ( f) + f XY ( g) 
\end{align*} 
We do get from this however, that 
\[ 
XY  - YX := [ X, Y ] 
\]  
does obey the Leibniz condition, because it removes the non-Leibniz cross terms from our differentiation. The commutator acts on a function $f$ by 
\[ 
X( Y (  f))  - Y( X( f)) = [ X, Y] f 
\] One can check that the components of the new vector field $[ X, Y]^\mu $ are given by 
\[ 
[ X, Y]^\mu = X^\nu \frac{\partial Y^\mu}{ \partial x^\mu }  - Y^\nu \frac{ \partial X^\mu }{ \partial x^\nu } 
\]
The commutator obeys the Liebniz rule, where 
\[ 
[X , [ Y, Z]] + [ Y, [ Z, Z]] + [Z, [ X, Y]] = 0
\] 

\subsection{Integral curves} 
We'll now do an interesting diversion to discuss flows on a manifold. Think of a body of water moving smoothly on a surface, where each point on the manifold moves to a different point after some amount of time. This is a what a flow is. More specifically, it is a smooth map $\sigma_t : \mathcal{M} \rightarrow \mathcal{M} $ on the manifold (which makes it a diffeomorphism), where $t$ is our 'time' parameter that we were talking about. As such, these flow maps actually form an abelian group, where 
\begin{itemize} 
\item $\sigma_0 = I_\mathcal{M} $. So, after time $t = 0$ has passed, nothing has moved so we have that this is the identity map. 
\item If we compose the same flow after two intervals in time, we should get the same flow when we've let the sum of those times pass over. So, 
\[ \sigma_s \circ \sigma_t  = \sigma_{ s + t } \] 
\end{itemize} 
If we take a flow at a given point $p \in \mathcal{M}$, we can define a curve on the manifold by setting: 
\[ 
\gamma(t): \mathbb{R} \rightarrow \mathcal(M), \quad \gamma(t) = \sigma_t ( p ) 
\] where without loss of generality we have $\gamma(0) = p$. Since this is a curve, we can define it's associated curve in $R^n$ space with a given coordinate chart, and hence associate with it a tangent vector $X_p$. We can also work backwards. From a given vector field, we can solve the differential equation 
\[ 	
X^\mu ( x(t) )  = \frac{ d x^\mu ( t) }{ dt} 
\] with the initial condition that $x^\mu ( 0 )  = \psi ( p ) $ at some point in the manifold, and have that this defines a unique curve. The set of curves then together form a flow. 
Thus, we've seen a one to one correspondence between vector fields and flows. 	
\pagebreak 

\subsection{Differentiating vector fields with respect to other vector fields}

\subsubsection{Push-forwards and Pull-backs} 
A sensible question to now as is that, since we have these smooth vector fields, how do we differentiate a vector field with respect to another one? For example, if we have $X, Y \in \mathcal{M}$, what constitutes the notion of a change in $X$ with respect ot $Y$. The notion of derivatives on manifolds is difficult because we can't compare tangent spaces at different points in the manifold, for example $T_p(M )$ and $T_q(M) $ are tangent spaces at different points, and we could define different charts for each space, hence we have some degrees of freedom (and our derivative  wouldn't make sense).  To make sense of comparing different tangent spaces, we need to create way to compare the same functions, but on different manifolds. These are called push forwards and pull backs. 

Let's start by defining a smooth map between manifolds $\phi : M \rightarrow N$ ($\phi$ is not a chart here). We're not assuming that $M$ and $N$ are even the same dimension here, and so we can't assume $\phi^{ -1} $ doesn't even exists. 

Suppose we have a function $f: N \rightarrow \mathbb{R} $. How can we define a new function based on $f$ that makes sense, which goes from $M \rightarrow \mathbb{ R} $? We define the pull back of a function $f$, denoted $(\psi^* f) : M \rightarrow \mathbb{R} $ as 
\[
( \psi^* f ) ( p )  = f ( \psi ( p ) ), \quad p \in \mathbb{ M } 
\] So, we've converted this thing nicely. 

Our next question then is how, from a vector field $Y \in \mathcal{X} ( M )$, can we make a new vector field in $X \in \mathcal{X} ( M ) $? We can, and this is called the push-forward of a vector field, denoted $ \phi_* Y \in \mathcal{ X} ( N )$. We define that object as the vector field which takes 
\[ 
(\phi_*  Y ) ( f)  = Y ( \phi^* f ) 
\] 
This makes sense because $\phi^* f \in C^{\infty} ( M )$, so applying $Y$ makes sense. Now, to show $\phi_* Y \in \mathcal{X} ( N )$, we should verify that 
\[ 
\phi_* Y : C^\infty ( N ) \rightarrow C^\infty( N ), \quad f \mapsto C^{\infty} 
\] 
Well, this object philosophically maps 
\[ 
 f \mapsto \phi_* Y ( f)  = Y ( \phi^* f ) 
\]  But the object on the left hand side is a vector field ready to be turned into a tangent vector when we assign it to a point on the manifold: 
\[ 
p \mapsto Y_p ( \phi^* f ) 
\] Hence this object agrees with our definition. The fact that we have a map $\phi: M \rightarrow N$ and are pushing the vector field from $\mathcal{ X} ( M ) $ to $\mathcal{ X} ( N )$ is the reason why we call this new mapping a push forward. 

\subsubsection{Components for Push-forwards and Pull-backs} 
Now, since $\psi_* Y $ is a vector field, it's now in our interest to find out about what the components are for this thing. We want to find that the components $( \psi_*Y )^\nu$ such that 
\[ 
\psi_* Y  = ( \psi_* Y )^\nu \partial_\nu 
\]
We can work first by assigning coordinates to $\phi( x )$, which we denote by $y^\alpha  ( x) = \phi( x), \quad x\in M,  \alpha = 1, \dots dim(N)$. 
If we write out our vector field $Y$ as $Y  = Y^\mu\partial_\mu$, then our push-forward map in summation convention looks like 
\[ 
(\phi_* Y ) f = Y^\mu \frac{\partial f ( y ( x))}{ \partial x^\mu}  = Y^\mu \frac{ \partial y^\alpha }{ \partial x^\mu } \frac{ \partial f}{ \partial y^\alpha } 
\] 
In the second equality, we've applied the chain rule. Remember, $y$ pertains to coordinates in the manifold $N$, so on our push-forward, we have that our new components on the manifold $N$, we have that 
\[ 
(\phi_* Y )^\alpha  = Y^\mu \frac{ \partial y^\alpha }{ \partial x^\mu } 
\] 

\subsubsection{Introducing the Lie derivative} 
In what we've just presented, some objects are naturally pulled back and some are naturally pushed forward. However, things become when our map between manifolds is a diffeomorphism and hence invertible; which means we can pull back and push forward with whatever objects we want. 
We can use this idea to differentiate vector fields now. Recall that if we've given a vector field, $X \in \mathcal{X} (M )$, we can define a flow map $\sigma_t : M \rightarrow M$ This flow map diffeomorphism allows us to push vectors along flow lines in the manifolds, from the tangent spaces 
\[ 
T_p ( M ) \rightarrow T_{\sigma_t ( p )} ( M ) 
\] 

This is called the Lie derivative $\mathcal{L}_X$, a derivative which is induced by our flow map generated by $X$. For functions. we have that 
\[ 
\mathcal{L}_X f = \lim_{ t \rightarrow 0 } \frac{ f( \sigma_t ( x ))  - f( x) }{ t }  = \left. \frac{ df ( \sigma_t( x) ) }{ dt } \right\vert_{ t =0 } 
\] However, the effect of doing this is exactly the same as if we were to apply the vector field $X$ to the function: 
\[ 
\frac{ df}{ dx^\mu } \left. \frac{ dx^\mu } { dt } \right\vert_{ t = 0 }  = X^\mu \frac{ \partial f }{ \partial x^\mu }  = X( f)
\] Thus, a Lie derivative specialised to the case of functions just gives us $\mathcal{L}_X ( f)  = X( f) $. 
Now, the question is about how we can do this differentiation on vector fields. What we need to do is to 'flow' vectors at a point back in time to where they originally started, and look at this difference. 
\[ 
( \mathcal{L}_X Y )_p  = \lim_{ t \rightarrow 0 } \frac{ ( \sigma_{-t }^* (Y)_p - Y_p )  }{ t } 
\] 

Let's try and compute the most basic thing first, the Lie derivative of a basis element of the tangent space  $\partial_\mu$: 
\[
\sigma_{ -t}^* \partial_\mu = (\sigma_{ -t}^* \partial_\mu)^\nu \partial_\nu
\] Let's try and figure out what $ (\sigma_{ -t}^* \partial_\mu)^\nu$ is. Because of the fact that the diffeomorphism is induced by the vector field $X$, we have that 
\[
( \sigma_{ -t}^* (x))^\nu = x^\nu  - t X^\nu + \dots
\] Thus our components of a push-forward of an arbitrary vector field are given by 
\[ 
( \sigma_{ -t}^*  Y )^\nu  = Y^\sigma \frac{ \partial ( \sigma_{ -t } ( x) )^\nu}{ \partial x^\sigma } , \quad \text{ in our case }  Y^\sigma = \delta\indices{^\sigma_\mu} 
\] 
Substituting the expressions above with one another gives us that 
\[ 
( \sigma_{ -t} ( x) \partial_\mu )^\nu  = \delta\indices{^\nu_\mu}  - t \frac{ \partial X^\nu }{ \partial x^\mu } + 
\] Contracting this with $\partial_\nu$, and then subtracting off $\partial_\mu$, we have that
\[ 
\mathcal{L}_X \partial_\mu =  - \frac{ \partial X^\nu}{ \partial x^\nu} \partial_\nu 
\] We require that a Lie derivative obeys the Leibniz rule, so we have that applying on a general vector field $Y$, 
\begin{align*}
\mathcal{L}_X (Y)  & = \mathcal{L}_X( Y^\mu \partial_\mu ) \\
				&= \mathcal{L}_X ( Y^\mu ) \partial_\mu + Y^\mu \mathcal{L}_X ( \partial_\nu ) \\
				&= X^\nu \frac{ \partial Y^\mu }{ \partial x^\nu } \partial_\mu - Y^\mu \frac{ \partial X^\nu } { \partial x^\mu } \partial_\nu 
\end{align*} 
We realise that this however are just components of the commutator! So
\[
L_X ( Y ) = [ X, Y ] 
\] 

\section{Tensors} 

\subsection{A note about dual spaces}
In linear algebra, if we have a vector space which we call $ V$, then we can define a natural object which we call it's dual space, denoted by $V^* $. The dual space is the space of linear functions which takes $V \to  \R$: 
\[
V^* = \{ f \mid f : V \to  \R, \quad \text{ f is linear } \} 
\] 
Now it may seem from first glance that the space of all functions is a lot larger than our original vector space, so it's counter intuitiveto call it the 'dual'. However, we can prove that these vector spaces are isomorphic. Suppose that $\{ e_\mu \} $ is a basis of $V$. Then we pick what we call a dual basis of $V^*$, by choosing the 
\[
\mathcal{B}( V ^* ) = \{ f^\mu \mid f^\mu ( e_\nu )  =\delta \indices{^\mu_\nu} \}  
\] 
One can show that this set indeed forms a basis of $V^* $.
\begin{thm}
The above basis forms a basis of $V^* $. 
 \begin{proof}
	First we need to show that the linear maps above, span our space. 
	This means we need to be able to write any linear map, say $ \omega $, as a linear combination 
	\[
		\omega = \sum \omega_\mu  f^\mu , \omega_\mu  \in F  
	\] 
	To do this, we appeal to the fact that if two linear maps agree on the vector space's basis, 
	then they agree. 
	So, let the values that $ \omega $ takes on the basis be $\omega_\mu  = \omega (e_\mu  ) $. Then, taking 
	\[
	 \Omega  = \sum \omega_\mu  f^\mu  
 \] We find that $ \Omega$ also satisifes $ \Omega( e_\mu  )  = \omega_\mu  $. 
 Thus, the maps are the same. 
 Hence, $\omega$ can be written as the span of our dual basis vectors. 
 To show that these basis vectors are linearly independent, 
 we assume that there exists a non trivial sum such that they add to the zero map.  
	\[
	 0  = \sum \lambda_\mu  f^\mu  
	\] If we apply this map to an arbitrary basis vector $ e_i$, then we get  \[
	0 = \sum \lambda_\mu  f^\mu  ( e_i ) = \lambda_i 				
	\] for arbitrary $i $. Hence, we must have that all $\lambda_i$ are zero. 
	Thus the basis vectors are independent. 
\end{proof}
\end{thm}
Now, assuming that our original vector space $ V$ had finite dimension, the way we've defined the basis of $V^* $
means that we had the same number of basis elements. 
This means that $V $ and $V^* $ have the same dimesion. 
One can prove that vector spaces with the same dimension are isomorphic, so we have that 
\[
V \simeq V^* 
\] Think of a dual space as a 'flip' of a vector space. 
We can identify the dual of a dual space as the original space itself, so that 
\[
\left( V^*  \right)^ * = V 
\] This is because given an object in the dual space $\omega$, we can define a natural map from 
$V: V^* \to  \R $ given by
\[
V \left( \omega  \right)  = \omega \left( V  \right) \in \R
\] 
\subsubsection{Vector and Covector spaces} 
Now, since we've identified tangent spaces as vector spaces, we can proceed to construct its dual. 
If we have a tangent space $ T_p ( \mathcal{ M}) $ with a basis $\{ e_\mu  \}$, our natural dual basis is 
given by 
\[
\mathcal{B }( T_p^* ( \mathcal{M})) = \{ f^\mu  \mid f^\mu  ( e_\nu  ) = \delta \indices{^\mu _\nu  } 
\] The corresponding dual space is denoted as $T_p^* ( \mathcal{ M}) $, and is known as 
the cotangent vector space. 
For brevity, elements in this space are called \textbf{ covectors}. 
In this basis, the elements $ \{ f^\mu  \}$ have the effect of 'picking' out components of a vector $ V = V^\mu e_\mu  $. 
\[
f^\nu  ( V )  =f^\nu  ( V^\mu  e_\mu  )  = V^\mu  f^\nu  ( e_\mu )  = V^\nu  
\] There's a different way to pick elements of this dual space in a smooth way. 
They're chosen by picking elements of a set called the set of 'one forms'. 
We denote the set of one forms, with an index $1 $ as $ \Lambda^{1}( \mathcal{M }) $. 
We can construct elements form this set by taking elements from $ C^\infty ( \mathcal{M} )$. 
Suppose that we have an $ f \in C^\infty ( \mathcal{ M } ) $, then the corresponding one-form is a map 
\[
df : T_p ( \mathcal{ M}) \to  \R \quad V \mapsto V( f) 
\] From one the set of one forms, we then have an obvious way to get the dual basis. 
The dual basis is obtained by just taking the coordinate element of the manifold, so that our one form is 
\[
dx^\nu  : T_p ( \mathcal{ M }) \to  \R
\] This satisfies the property of a dual basis, since 
\[
dx^\nu ( e_\mu  )  = \frac{\partial x^\nu  }{\partial x^\mu  } = \delta \indices{^\nu _\mu  } 
\] With this convention, we can check that 
$ V ( f) $ is what its supposed to be by observing that
\[
df( X)  = \frac{\partial f}{\partial x_\mu} dx^\mu ( X^\nu \partial _\nu )  = X^\nu \frac{\partial f}{\partial x^\nu} = X_ ( f)  
\]  So we recover what we expect 
by setting this as a basis. Now, we should
check whether a change in coordinates leaves our properties invariant. 

Suppose we change our basis from $ x^\mu \to \tilde{z}^\mu ( x)  $, 
then, we know that our basis vector transforms like 
\[
\frac{\partial }{\partial \tilde{x }^\mu }  = \frac{\partial \tilde{x }^\mu }{\partial x^\nu} dx^\nu  
\] We guess that our basis of one forms 
should transform as 
\[
d \tilde{x }^\mu = \frac{\partial \tilde{x}^\mu }{\partial x^\nu} dx^\nu  
\] This ensures that, when we contract a transformed basis one form 
with a transformed basis vector, that
\begin{align*}
d \tilde{x }^\mu \frac{\partial }{\partial \tilde{x }^\nu } &=  \frac{\partial \tilde{x }^\mu}{ \partial x^\rho} dx^\rho \frac{\partial x^\sigma}{\partial \tilde{x }^\nu } \frac{\partial }{\partial x^\sigma}    \\
							    &=  \frac{\partial \tilde{x }^\mu }{\partial x^\sigma} \frac{\partial x^\sigma}{\partial \tilde{ x}^\nu } dx^\rho \left( \frac{\partial }{\partial dx^\sigma}  \right)    \\
							    &=  \frac{\partial \tilde{x}^\mu }{\partial x^\rho} \frac{\partial x^\rho}{\partial \tilde{x }^\nu }  = \delta \indices{^\mu_\nu} 
\end{align*} It's no coincidence that this looks like a Jacobian! 

Now, as with basis elements in our vector space, 
we need to determine how these objects transform under a change of basis. 
*Need to finish this section on basis transformations for covectors* 

\subsection{Taking the Lie Derivative of Covectors} 
We would like to repeat what we did for vectors, and take derivatives of covectors. 
To do this, we need to define pull-backs for covectors.
Suppose we had a covector $\omega$ living in the tangent space 
of some manifold $N$, in $T_p^*( N)$. 
We can then define the pull back of this vector field based on first pushing forward 
the vector field $X$. Thus, if $\phi : M \to  N$, then we define the pullback 
$\phi^ * \omega $ as the covector field in $T_p ( M ) $ as 
\[
( \phi^* \omega ) ( X)  = \omega ( \phi_* X ) 
\] What information can we glean from this? 
Well, we can try to figure out what the components of $\phi^ * \omega $ are. 
If we let $\{ y ^ \alpha \} $ to be coordinates on $N$, 
then we expand this covector as 
\[
\omega = \omega_\mu  dy^\mu  
\] Also recall that the components of a pushed forward vector field are 
\[
( \phi_* X )^ \mu   = X^\nu \frac{\partial y ^ \mu  }{ \partial x^\nu }
\] Now, if we take the equation 
\[
( \phi ^ * \omega )  = \omega ( \phi_* X ) 
\] Then, expanding out in terms of components, we have that 
\[
( \phi^* \omega ) ^\mu  dx_\mu  ( X^\nu  e_\nu  )  = w^\mu  dy_\mu  ( \phi_* X)^\nu  \frac{\partial }{\partial y ^\nu  }
\] Remember, we have to expand in the correct coordinates. The object $ \phi^* \omega $ lives in 
the space $M $ so we expand in the $ dx^ \mu  $ basis. On the other hand
we have that $\omega $ originally lives in $N $ so we expand in 
$ dy^ \mu  $. 
Substituting our expression for our push forward vector field, 
and we get that 
\[
( \phi^ * \omega )_\mu  X^\mu  = \omega_\nu  \frac{\partial y^\nu  }{\partial x^\mu }X^\mu  
\] This step requires a bit of explanation. 
After we substitute in the components for the pushed forward vector field, 
we then use the fact that on both manifolds, our basis vectors and 
our basis covectors contract to give a delta function 
\[
dx^\mu  \left(  \frac{\partial }{\partial x^ \nu } \right)= \delta \indices{^\mu  _\nu }, \quad dy^\mu  \left( \frac{\partial }{\partial y^\nu  }  \right) = \delta \indices{^\mu _\nu }						  
\] This implies that the components of our pulled back vector field are 
\[
( \phi^* \omega )_\mu  = \omega_\nu  \frac{\partial y^\nu }{\partial x^\mu  }
\] 
As in the case of vectors, we can also make rigourous the definition of a Lie derivative 
with respect to a covector field. 
This is denoted $\mathcal{L}_X \omega$, where $X$ is our underlying vector field
we're differentiating with. 
If our vector field $X$ imposes a flow map which we label as 
$\sigma_ t$, then our corresponding Lie derivative is derivative is defined as 
\[
\mathcal{ L }_X \omega  =\lim_{t \to 0} \frac{( \sigma_t^* \omega)   - \omega }{t }	 
\]There's an important point to be made here. 
In our previous definition of a Lie derivative for a vector field, 
we took the inverse diffeomorphism $\sigma_{ - t} $. 
But in this case, we need to take $t $ positive since 
we're doing a \textbf{pull-back} instead of a pushforward, like we did 
with vector fields. 

Let's go slow and try to compute 
the components of this derivative. 
Recall that for a flow map, we have that infinitesimally, 
\[
y^\nu  = x^\nu  + t X^\nu , \implies \delta \indices{^\nu _\mu  } + t \frac{dX^\nu }{dx^\mu  }	  
\] Thus for a general covector field, our components for the pull back 
are 
\[
( \sigma_t \omega )_\mu  = \omega_\mu  + t \omega_\nu \frac{dX^\nu }{dx^\mu  }
\] Hence, the components of a basis element under this flow becomes 
\[
( \sigma_t^* dx^\nu )  = dx^ \nu  + t dx^\mu  \frac{dX^\nu  }{dx^\mu  }
\] So, taking the limit, we have that our components of our 
Lie derivative are given by 
\[
\lim_{t \to 0  } \frac{\sigma^*_t dx^\nu   - dx^\nu  }{ t }  = dx^\mu  \frac{dX^\nu }{dx^\mu  }		 
\] Now, as before, we impose the 
Liebniz property of Lie derivatives, and expand out a general covector. 
Hence, we have that 
\begin{align*}
\mathcal{L}_X( \omega_\mu  dx^\mu  ) &= \omega_\mu  \mathcal{L}dx^\mu  + dx^\mu  \mathcal{L}_X( \omega_\mu  )  \\
 &= \omega_\mu  dx^ \nu  \frac{dX^\nu  }{dx^\mu  } + dx^ \nu  X^\mu   \frac{d \omega_\nu }{dx^ \mu  }
\end{align*} This implies that 
our components of our Lie derivative can be written nicely as 


\[
\mathcal{L }_X( \omega )_\mu   = ( X^\nu \partial_\nu \omega _\nu + \omega _\nu \partial_\mu X^\nu ) 
\]  
\subsection{Tensor fields} 
Now, we can combine both maps from tangent and cotangent spaces to create tensors. 
A tensor of rank $( r, s) $ is a \textbf{multilinear} map from 
\[
T: T_p^* ( M ) \times \dots T_p^*( M )  \times T_p(M) \times \dots T_p( M) \to  \mathbb{ R}  
\] where we have $r  $ copies of our cotangent field and $ s$ copies of our tangent field. 
We define the total rank of this multilinear map as $ r + s$. 
Since a cotangent vector is a map from vectors to the reals, 
this is a rank  $ ( 0, 1 ) $ tensor. 
Also, a tangent vector has rank $ ( 1, 0 ) $ since it's a map from the
cotangent space. 
A tensor field is the smooth assignment of a rank $ ( r, s) $ tensor to a point
on the manifold $ p \in M $. 
We can write the components of a tensor object 
by writing down a basis, then sticking this into the object. 
If $ \left\{  e_\nu \right\} $ was a basis of $ T_{p }( M ) $, and 
$ \left\{  f^\nu \right\} $ was the basis of the dual space, then 
the tensor has components 
\[
T \indices{^{\mu_1 } ^{\mu_2} ^{\dots } ^{\mu_r}_{\nu_1 }_{\nu_2 } _{\dots }_{\nu_s}} = T(f^{\mu_1 }, f^{\mu_2}, \dots, f^{\mu_r}, e_{\nu_1 }, \dots e_{\nu_s} )  
\]
For example, a $ ( 2 , 1) $ tensor acts as 
\[
T ( \omega , \epsilon; X) = T ( \omega_{\mu } f^\mu, \epsilon_{ \nu }f^\nu, X^\rho e_{ \rho }) 
\] We have that the covectors $ \omega  , \epsilon \in \Lambda^ 1 ( M ) $, and $  X\in \mathcal{ X }( M ) $. 
The object above is then equal by multilinearity to 
\[
= \omega _{ \mu } \epsilon_{\nu } X^\rho T \indices{ ^\mu ^ \nu _\rho }  
\] Under a change of coordinates, we have that 
\[
\tilde{ e}_{ \nu } = A \indices{^\mu _ \nu } e_{ \mu }, \quad A \indices{ ^ \mu _ \nu } = \frac{\partial  x^\mu }{\partial \tilde{ x}^\nu}   
\] 
Similarly, we have that for covectors we transform as 
\[
\tilde{ f} ^\rho = B \indices{ ^\rho _\sigma } f^\sigma, \quad B \indices{ ^\rho _\sigma } \frac{\partial  \tilde{ x}^\rho}{\partial x^\sigma}    
\] Thus, a rank $ ( 2, 1 ) $ tensor transforms as 
\[
\tilde{ T } \indices{^\mu^\nu_\rho}  = B \indices{^\mu_\sigma } B \indices{^\nu _\tau} A \indices{^\lambda _\rho} T \indices{^\sigma ^\tau _ \lambda}     
\] There are a number of operations which we can perform on tensors. 
We can add or subtract tensors. We can also take the tensor product. 
If $ S $ has rank $ ( p , q) $, and $  T $ has rank $ ( r, s ) $, then 
we can constrict $ T \otimes S $, which has rank $ ( p + t , q + s ) $. 
\[
S \otimes T ( \omega_1, \dots \omega_p, \nu_1 , \dots \nu_{ r }, X_1, \dots X_q , Y_1, \dots Y_s )  = S( \omega_1 , \dots , \omega_p, X_1, \dots, X_q ) T ( \nu_1 , \dots,  \nu_r, Y_1 , \dots Y_s) 
\] 
Our components of this are 
\[
(  S\otimes T )\indices{ ^{ \mu_1 } ^{ \dots \mu_{ p }} ^{ \nu_{1 } \dots \nu_{ r }} _{ \rho_1 \dots \rho_{ l } \sigma_1 \dots \sigma _{ s}}} = S \indices{^{ \mu_1 \dots \mu_{ p}}_{ \rho_1 \dots \rho_{ q}}} T \indices{ ^{ \nu_1  \dots \nu_{ q }} _{ \sigma_{ 1} \dots \sigma_{ s}}}   
\] We can also define a contraction.
We can turn a $ ( r, s $ tensor into an  $ ( r - 1, s - 1) $ tensor. If we have
$ T $ a $ \left( 2, 1  \right) $ tensor, then we can define a 
\[
S( \omega  ) = T ( \omega  , f^\mu, e_\mu ) 
\] The sum over $ \mu $ is basis independent. 
This has components 
\[
S^\mu = T \indices{ ^\mu ^\nu _\nu }  
\] This is different from $ ( S ') ^\mu = T \indices{ ^\nu ^\mu _\nu} $ 
We can also symmetrise and anti symmetrise. Given a $ ( 0 , 2 ) $ tensor, 
we can define 
\[
S( X, Y ) = \frac{1}{ 2 } ( T ( X, Y ) + T ( Y , X) ), \quad A( X, Y ) = \frac{1}{2 } ( T ( X, Y ) - T ( Y , X) ) 
\] This has components which we write as 
\begin{align*}
T_{ ( \mu \nu ) } : = & \frac{1}{2 } ( T \indices{ _\mu_\nu} + T \indices{ _\nu_\mu} ) \\
T_{ [ \mu \nu ] } : = & \frac{1}{2 } ( T \indices{ _\mu_\nu} - T \indices{ _\mu_\nu}) 
\end{align*}
We can also symmetrise or anti symmetrise over multiple indices. 
So 
\[
T \indices{ ^\mu _{ ( \nu \rho \sigma  )  }}  = \frac{1}{ 3 ! } ( T \indices{^\mu _\nu _\rho _\sigma } + \text{ 5 perms } )  
\] We can also anti symmetrise by multiplying by the sign of permutations. 
\[
T \indices{ ^\mu _{ [  \nu \rho \sigma  ]   }}  = \frac{1}{ 3 ! } ( T \indices{^\mu _\nu _\rho _\sigma } + sgn ( perm)  \text{ for 5 perms } )
\]

\pagebreak
\subsection{Differential forms} 
Differential forms are totally antisymmetric $ ( 0 , p )  $ tensors, and are denoted $ \Lambda^ p ( M ) $. 
0 -forms are functions. If  $ dim ( M )  = n$, then p-forms have n choose p components by anti-symmetry. 
n-forms are called top-forms. 

\subsubsection{Wedge products} 
Given a $ \omega  \in \Lambda^ p ( M ) $ and $ \epsilon \in \Lambda^ q ( M ) $, we can form a  $ ( p + q ) $ form 
by taking the tensor product and antisymmetrising. 
This is the wedge product. 
Our components are given by 
\[
( \omega \wedge  \epsilon )_{ \mu_1 \dots \mu_{ p } \nu_1 \dots \nu_{ q }} = \frac{ ( p + q ) ! }{p ! q ! } \omega _{ [ \mu_1 \dots \mu_p } \epsilon_{ \nu_1 \dots \nu_q ] }
\] An intuitive way to thing about this is that 
we are simply just adding anti-symmetric combinations of 
forms, without dividing (other than to make up for the previous 
anti-symmetry). So, we have that, for example, when 
we wedge product the forms $ dx ^ 1  $ with $dx^ 2 $, that 
\[
dx^ 1 \wedge  dx^ 2 = dx^ 1 \otimes dx^ 2 - dx^ 2 \otimes dx^ 1 
\] In terms of components one can check that, for example, for one forms we have that 
\[
( \omega \wedge  \epsilon )_{ \mu \nu } = \omega_{ \mu } \epsilon_{ \nu } - \omega_{ \nu } \epsilon_{ \mu }
\] We can iteratively wedge the basis of forms $\left\{  dx^ \mu  \right\}  $
together to find that 
\[
dx ^ 1 \wedge  \dots dx^ n = \sum_{ \sigma \in S_n } \epsilon ( \sigma)  dx^{ \sigma ( 1) } \otimes \dots \otimes dx^{ \sigma( n ) }
\] To show this, we use an example. Note that the 
components of $ dx^ 1 \wedge  dx^ 2 $ are 
\[
dx^ 1 \wedge  dx^ 2 = 2 \delta^ 1 _{ [ \mu } \delta^ 2 _{ \nu ] } dx^ \mu dx^ \nu 
\]  Now, this means that wedging this with $ dx^ 3 $ gives 
components  \[
( dx^ 1 \wedge  dx^ 2 ) \wedge  dx^ 3 = \frac{3!}{2 } 2 \delta^1_{ [ [ \mu } \delta^ 2_{ \nu ] } \delta^ 3 _{ \rho ] } dx^ \mu dx^ \nu dx^ \rho 
\] But this is the sum of all permutations multiplied 
by the sign, since a set of antisymmetrised indices nested 
in a bigger set it the original set.

\subsubsection{Properties of wedge products} 
Our antisymmetry property of forms 
gives it properties we might expect. 
One of these is that switching a $ p $ and $ q $ form 
picks up a sign:  we have that 
\[
\omega \wedge \epsilon  = ( - 1) ^{ p q } \epsilon \wedge  \omega    
\] In general, for an odd form we have that 
\[
\omega  \wedge  \omega = 0 
\] For the manifold $ M = \mathbb{ R} ^ 3 $, with $ \omega  , \epsilon \in \Lambda^ 1 ( M) $, 
we have that 
\[
( \omega  \wedge  \epsilon ) = ( \omega_1 dx^1 + \omega_2 dx^ 2 + \omega_3 dx^ 3 ) \wedge  ( \epsilon_1 dx^1 + \epsilon_2 dx^2 + \epsilon_3 dx^3 ) 
\] expanding this thing, we have that 
\begin{align*}
\omega  \wedge  \epsilon & = ( \omega_1 \epsilon_2  - \epsilon_2 \omega_1 ) dx^1 \wedge  dx^ 2 \\
			 & + ( \omega_2\epsilon_3 - \omega_3 \epsilon_2 ) dx^ 2 \wedge  dx^ 3 \\
			 & + ( \omega_3\epsilon_1  -\omega_1 \epsilon_3 ) dx^3 \wedge  dx^1 
\end{align*}
These are the components of the cross product. The cross product 
is really just a wedge product between forms. 
In a coordinate basis, we write that 
\[
\omega  = \frac{1}{p ! } w_{ \mu_1 \dots \mu_{ p }} dx^{ \mu_1 } \wedge  \dots \wedge  dx^{ \mu_{ p }}, \quad \omega =  w_{ \mu_ 1 \dots \mu_{ p  } } dx^{ \mu_ 1 } \otimes \dots \otimes dx^{ \mu_{ p } }  
\] This is useful because writing out forms in terms of wedge 
products as their basis turns out to make calculations a lot easier. 

\subsection{The exterior derivative} 
Notice that given a function $ f$, we can construct a  1-form 
\[
df = \frac{\partial f }{\partial  x^{ \mu} } 
\] In general, there exists a map 
$ d : \Lambda^ p ( M ) \to \lambda^{ p + 1 } (  M ) $. 
this is the exterior derivative. In coordinates, 
we have that 
\[
dw = \frac{1}{ p ! } \frac{\partial \omega_{ \mu_1 \dots \mu_{ p  } }}{\partial x^\nu} dx^\nu \wedge  \dots \wedge  dx^{ \mu_{ p }} 
\] One should view this as a 
generalised curl of some vector field. 
In components, we have that 
\[
( d \omega  ) _{ \mu_1 \dots \mu_{ p + 1  }} = ( p +  1) \partial_{ [ \mu_1 } \omega_{ \mu_2 \dots \mu_{ p + 1 }]  }
\] Let's try to gain an intuition about 
why these two definitions are equivalent.
If we contract our component definition with the tensor 
product $ dx^ 1 \otimes \dots \otimes dx^{p +1}$, 
then we are summing over 
\[
	d \omega = \frac{1}{p ! } \sum_{ \sigma \in S_ n } \epsilon ( \sigma)  \partial_{ \sigma ( \mu_ 1 ) } w_{ \sigma ( \mu _ 2 )  \dots \sigma ( \mu _{ p + 1 ) }} dx^{ \mu _ 1 } \otimes \dots \otimes dx^{ \mu _{ p +1}}
\] However we can transfer our 
permutations to permutations on tensor product, 
but by definition this would just be the 
wedge product on our basis one-forms. 

By antisymmetry,  we have a very significant identity that 
\[
d ( d \omega  ) = 0 
\] We write this as $ d^ 2 = 0 $.
To show this, the easiest way is not to use our 
definition of $ d \omega  $ in components but 
rather to use our definition in terms of wedge product 
basis vectors. Let's think carefully about how the exterior 
derivative acts on some on $ p $ form but with 'components' 
in our wedge product basis $ \left\{  dx^1 \wedge  \dots \wedge  dx^p \right\}  $. 
In our wedge product basis, our components are $ \frac{1}{p !} \omega_{ \mu_ 1 \dots \mu _{ p } } $ since 
\[
	\omega = \frac{1}{p!} \omega_{ \mu_1 \dots \mu_{ p } }
\] We have that under the exterior derivative, we are mapping 
\[
 d : \frac{1}{p!} \omega_{ \mu_ 1 \dots \mu_ p } dx^{ \mu _ 1 } \wedge  \dots \wedge dx ^{ \mu_ p } \mapsto \frac{1}{p!} \partial_\nu \omega_{ \mu _1 \dots \mu _ p } dx^ \nu \wedge dx^{ \mu _ 1 } \wedge  \dots \wedge  dx^{ \mu _ p } 
\] So, in our new fancy wedge product basis, we are mapping 
 \[
 w_{ \mu_1 \dots \mu_{ p }} \mapsto \partial _ \nu \omega_{ \mu _ 1 \dots \mu _{ p } }
\] 
Hence, we have that, in our wedge product basis, our components of 
$ d ( d \omega ) $ are 
\[
d ( d \omega ) = \frac{1}{p ! } \partial _ \rho \partial  _ \nu w_{ \mu _ 1 \dots \mu _ p } dx^ \rho \wedge  dx ^ \nu \wedge  dx ^{ \mu _ 1 } \wedge  \dots \wedge  dx^{\mu _ p } = 0 
\] since we have symmetry of mixed 
partial derivatives in $ \rho , \nu $ contracted 
with the wedge product which is antisymmetric in those 
indices.

It's also simple to show that 
\begin{itemize}
\item $ d ( \omega  \wedge \epsilon ) = d \omega  \wedge  \epsilon + ( - 1)^ p \omega  \wedge  d \epsilon $
\item For pull backs, $ d ( \phi^ * \omega  ) = \phi^* ( d \omega  ) $
\item $ \mathcal{ L }_X ( d \omega  ) = d ( \mathcal{ L }_ X \omega $
\end{itemize}
A p-form is closed if $ d \omega   = 0 $ everywhere. 
A p form is exact if $ \omega   = d \epsilon$ everywhere for some $ \epsilon$. 
We have that  
\[
d^ 2 = 0 \implies \text{exact} \implies \text{closed} 
\] 
\textbf{Poincare's lemma} states that on $ \mathbb{ R}^n$, or locally on  $ \mathcal{  M }$, exact implies closed. 

\pagebreak 
\subsubsection{Examples} 
Consider a one form $ \omega = \omega_\mu( x) dx^\mu $. Using our formula for the exterior derivative: 
\[
( d \omega)_{ \mu \nu } = \partial _\mu \omega_\nu - \partial  _\nu \omega_\mu 
\] Or, in terms of our form basis, 
\[
d \omega = \frac{1}{2 } ( \partial _\mu \omega_\nu - \partial _\nu \omega _\mu ) dx^\mu \wedge  dx ^ \nu 
\] In three dimensions, 
\[
d \omega = ( \partial_1 \omega_2 - \partial_2 \omega_1 ) dx^ 1 dx ^ 2 + ( \partial_2 \omega_3 -\partial_3 \omega_2 ) dx^ 2 \wedge  dx^ 3 + ( \partial_3 \omega_1 - \partial_1 \omega_3  ) dx^ 3 \wedge  dx^ 1 
\] These are the components of $ \nabla \times \omega $. 
The exterior derivative of a 1 form is a 2 form, but 2 forms have just three components in $ \mathbb{ R} ^3$ by anti symmetry ( with the components shown there). So we think of it has another vector field, if we identify the basis vectors $ \left\{  dx^ i \wedge  dx^ j  \right\} $ with components in $ \mathbb{R}^ 3 $!. 
However, getting another 'vector field' by doing an exterior derivative on the 
same type of object we had before is not the case in general.

Consider  $ B \in \Lambda^ 2 ( \mathbb{ R} ^ 3 ) $. So, we have
a 2-from in a 3 dimensional manifold. Let's label the components out 
explicitly here. 
\[
 B = B_1 ( x) dx^ 2 \wedge  dx^ 3 + B_2 ( x) dx^ 3 dx^ 1 + B_3 ( x) dx^ 1 \wedge  dx^ 2 
\] Before we do any explicit calculation, 
we know that the exterior derivative pushes this up to a 3 -form, which only has one component
in a three dimensional manifold. We compute the exterior derivative 
explicitly by differentiating each component
and then adding on the wedge.
\begin{align*}
	d B &=  \partial _ 1 B_ 1 dx ^1 \wedge  dx ^ 2 \wedge  dx^ 3  + \partial _ 2 B_2 dx^ 2 \wedge  dx^ 3 \wedge  dx^ 1 + \partial  _ 3 B _ 3 dx^ 3 \wedge  dx^ 1 \wedge  dx^ 2  \\
	    &=   \left( \partial  _ 1 B _ 1 + \partial  _ 2 B _ 2 + \partial  _ 3 B _ 3 
	     \right) dx^ 1 \wedge  dx^ 2 \wedge  dx^ 3 
\end{align*} 
In the last step, we permuted the indices cyclically 
so that we don't have a sign change. 
Note that we get our components of a grad
operator acting on $ \mathbf { B } $! 

For our final example, we take something from
electromagnetism. The gauge field, or perhaps more commonly 
known as our vector potential $ A \in \Lambda ^ 1 ( \mathbb{ R} ^ 4 ) $, 
can be written out as a one-form in $ \mathbb{ R } ^ 4 $. 

If we expand this as a one form, we can write 
\[
 A = A_ \mu dx^ \mu 
\] 
What happens when we take the exterior 
derivative of this thing? 
We get that 
\begin{align*}
	dA &= \partial  _ \nu A _ \mu dx^ \nu \wedge  dx ^ \mu   \\
	   &=  \frac{1}{2 } \partial _{ [ \mu } A _{ \mu ] } dx^ \nu \wedge  dx ^ \mu \\ 
	   &=  \frac{1}{2 } \left(  \partial  _ \nu A _ \mu - \partial  _ \mu 
	   A _ \nu  \right) dx^ \nu \wedge dx ^ \mu \\
	   &=  \frac{1}{2 } F_{ \mu  \nu  } dx^ \mu  dx ^ \nu  \\
\end{align*} 
We identify here that $ F_{ \mu \nu} = \partial  _ \mu A _ \nu - \partial  _ \nu A _ \mu $ 
is our electromagnetic field strength tensor! 
We also have that since $ d F = d ^2 A = 0 $, 
we get for free what one may recognise as the Bianchi 
identities. 

We can also introduce gauge transformations
which act on our electromagnetic 
vector potential. These act as 
\[
A\to A+ d\alpha  \implies F \to d ( A d\alpha ) = dA \text{ invariant } 
\] where we treat $ \alpha \in \Lambda^ 0 ( \mathcal{ M } ) = C^\infty ( \mathcal{ M }) $
We also get Maxwell's equations for free! 
\[
F = dA \implies dF = d^ 2 A = 0
\] There's are two of Maxwell's equations. 

\subsection{Integration} 
On a manifold, we integrate functions 
\[
	f : \mathcal{ M } \to \mathbb{ R } 
\] with the help of a special kind of 
a special kind of top form. 
The kind of form we need is called a volume form or orientation. 
This is a nowhere vanishing top form. Locally, 
it can be written as 
\[
v = v ( x) dx^ 1 \wedge  \dots \wedge  dx^ n , \quad v ( x) \neq 0 !
\] For some manifolds, globally we may not be able to glue
volume forms together over the whole manifold. 
If such a form exists, the manifold is said to be orientable. 
Not all manifolds are orientable, for example the Mobius strip. 
This says that $ v( x) $ must change direction and hence be zero. 
Or, $ \mathbb { RP } ^ n $. 
Given a volume form, we can integrate 
any function  $f: M \to \mathbb{ R} $ over  $ \mathcal{ M } $. 
In chart $ \mathcal{ O } \subset \mathcal{ M } $, we define
\[
\int_{ \mathcal{ O } } f v =  \ \int_{\mathcal{ U } } dx^ 1 \dots dx^ n f( x) v( x) 
\] Now, this tells us how to integrate a patch. 
Then, summing over patches gives us the whole integral. 
$ v ( x) $ can be thought of as our measure - ' the volume of some part of 
the manifold'. There is freedom in our choice of volume form here, 
we could've chosen lots of different volume forms which 
satisfy our condition above.

\subsubsection{Integrating over submanifolds} 

We haven't defined how to integrate, say a function 
over a $ p $-form on an 
$ n $ dimensional manifold, where $ p < n $ 
(since so far our definition of integration 
has only pertained to top forms. 

So to do things like this, we need to 
find a way to 'shift down' into a lower dimensional 
subspace and do things there. This is why we define the concept of a submanifold.

A new manifold $ \Sigma $ of dimension $ k < N $ is called a submanifold
of $ \mathcal{ M  }$ if there exists an injective map $ \phi : \Sigma \to \mathcal{ M }$ 
such that $ \phi^ * : T_p ( \Sigma ) \to T_p ( \mathcal{ M } ) $, the 
\textbf{pullback} of $ \phi $, is also 
injective. We require the condition of injectivity 
so that our submanifold doesn't intersect itself 
when we embedded it in our larger manifold.
The first condition is so that there are no crossings. 
The second condition is there so that we have no cusps in our tangent space.

We're now fully equipped to integrate over 
some portion of a submanifold of $ \mathcal{ M } $. 
You can think of this as a 'surface' or 'line' of 
some sort embedded in our manifold. 
We can integrate any $ \omega  \in \Lambda^ k ( \mathcal{ M } ) $ over  $\Sigma $ 
by first identifying it with the embedded portion of $ \Sigma $
in  $ \mathcal{ M } $, and then pulling it back into $\Sigma $
itself. Now, we're in a  $ p $ dimensional space, 
and we know how to integrate here since $ \phi ^ * \omega $ is 
now a top-form
\[
\int_{ \phi ( \Sigma ) } \omega = \int_{ \Sigma } \phi^ * \omega
\]  
Let's do an example where we integrate say over a line 
embedded in a bigger manifold. 
We define an injective map $ \sigma $ which 
takes our line $ C $ into our manifold. 
\[
\sigma : C\to \mathcal{ M }
\] defines a non intersecting curve in M. Then, for 
$ A \in \Lambda ^ 1 ( \mathcal{ M } ) $, we have, 
integrating over our embedding of our line in $\mathcal{ M } $, 
\[
\int_{ \sigma ( C) } A = \int_{ C } \sigma^ * A = \int d \tau A_{ \mu } ( x) \frac{ dx^ \mu }{d \tau }
\] 
The last equality comes from the fact that the 
components of a one-form pulled back transforms as
\[
 ( \sigma ^ * A )_\nu = A_\mu  \frac{ d x ^ \mu  }{ dt ^ \nu }
\] where in this case, since 
we're pulling back to a one dimensional 
manifold, we only have $ \nu = 0 $  (which we write
for brevity as $ \tau $). 

\subsubsection{Stokes' theorem}
\begin{questions}
\textit{How does this tie in to the bigger picture? }  
\textit{What use does Stoke's theorem have in general relativity?} 
\textit{How do we prove Stoke's theorem?} 
\end{questions}

\begin{defn}{(Boundaries).} 
So far we've only considered manifolds which are smooth. 
However, we can 'chop off' a portion of 
our manifold to make a slightly different map that what we 
are used to, a new map 
\[
\phi_\alpha : \mathcal{ O }_\alpha \to \mathcal{ U }_\alpha \subset \frac{1}{2 } \mathbb{R}^n  = 
\left\{ (x_1, \dots x_ n ), \mid x_1 \geq 0  \right\} 
\] Our boundary of our manifold is the
set of points on $ \mathcal{ M }  $ which are mapped to 
$ ( 0 , x_2 , \dots x_ n ) $. 
Boundaries are $ n - 1 $ dimensional manifolds 
of our original manifold which has dimension $n$

\end{defn}

\begin{figure}[h]
	\centering
	\includegraphics[width=0.5\linewidth]{figures/submanifold.png}
	\caption{Here we have a manifold with boundary.}%
\end{figure}

\begin{thm}{Stokes' Theorem.}
Let $ \mathcal{ M } $ be a manifold with boundary. 
This is a manifold which just stops and gets cutoff somewhere. 
If we call the manifold $ \mathcal{ M } $, we call the boundary 
$ \partial  \mathcal{ M }$. 
If we take $ \omega  \in \Lambda^{ n - 1} ( \mathcal{ M } ). $
Our claim is that
\[
\int_{ \mathcal{ M } } d \omega = \int_{ \partial  \mathcal{ M }} \omega
\] This Stokes' theorem. It's 
presented in a more general form 
than what we're used to, but we'll see in the 
examples that this way of expressing this 
gives us both Stokes' theorem in three dimensions 
as well as Green's theorem. 

\end{thm}

\begin{example}{Stokes' theorem in one dimension. } 
Take $ \mathcal{ M } $ as the interval $ I $ with $ x \in [ a, b ] $. 
$ \omega ( x)  $ is a function and 
\[
d \omega = \frac{ d \omega }{ dx  } \cdot  dx
\] We have that 
\[
\int_{ \mathcal{ M } } d \omega = \int_a^ b \frac{ d \omega }{ dx } dx \quad \int_{ \partial  \mathcal{ M } } \omega = \omega( b ) - \omega ( a) 
\] In one dimension, we've
recovered integration by 
parts on a line. 

\end{example}

\begin{example}{Stokes' theorem in two dimensions. } 
In the second case, we have that 
\[
M \subset \mathbb{ R} ^ 2, \omega \in \Lambda^ 1 ( \mathcal{ M } ) 
\] This recovers 
\[
\int_{ \mathcal{ M } } d \omega = \int_{ \mathcal{ M } } \left( \frac{\partial \omega_2 }{\partial  x^ 1 }  - \frac{\partial  \omega_1 }{\partial  x^ 2}  \right) dx^ 1 \wedge  dx^2  
\] By Stokes theorem this is 
\[
\int_{ \partial  \mathcal{ M  } } \omega = \int_{ \partial  M } \omega_1 dx^ 1 + \omega_2 dx^ 2  
\] This equality is Green's theorem in a plane. 

\end{example}

\begin{example}{Stokes' theorem in three dimensions.} 
Finally, take $ \mathcal{ M } \subset \mathbb{ R}^ 3 $ and $ \omega \in \Lambda^ 2 ( \mathcal{ M } ) $
\begin{align*}
\int_{ \mathcal{ M } } d \omega & = \int dx^ 1 dx^ 2 dx^ 3 ( \partial_1 \omega_1 + \partial_2 \omega_2 + \partial  _ 3 \omega _ 3 ) \\
\int_{ \partial  M   } \omega  &= \int_{ \partial  M  } \omega_1 dx^ 2 dx^ 3 + \omega_ 2 dx^ 3 dx^ 1 + \omega_3 dx^ 1 dx^ 2 
\end{align*}

Equating the two expressions above 
gives us our usual notion of Stokes' theorem in three dimensions, which 
is disguised as Gauss' divergence theorem. 
\end{example}

\pagebreak 
\section{Introducing Riemannian Geometry} 
\subsection{ The metric}

\begin{defn}{(The metric tensor). } 
We'll now do introduce a tensor object which 
turns our tangent space into an inner product space. 
We do this by introducing an object called a metric, 
which intuitively has been a way in which we define 
the notion of 'distance' in a space. 
A metric $ g $ is a $ ( 0, 2 ) $ tensor that is
\begin{itemize}
\item  symmetric $ g ( X, Y )  = g ( Y , X) $
\item non-degenerate: $ g ( X, Y )_ p = 0, \quad \forall Y_p \in T_p ( \mathcal{ M }) \implies X_p = 0$ 
\end{itemize}
\end{defn}
Notice that our condition for non-degeneracy is \textbf{not}
the same as having a point where  $ g ( X, X )_p = 0 $ as we
shall soon see. 
In a coordinate basis, $ g = g_{ \mu \nu } dx ^ \mu  \otimes dx^ \nu $.
The components are obtained by our standard way of 
subbing in basis vectors into our tensor. 
\[
	g _{\mu \nu  } = g \left( \frac{\partial }{\partial  x^ \mu } , \frac{\partial }{\partial x^ \nu } \right)  
\]  We often write this as 
a line element which we call 
\[
ds^ 2 = g_{ \mu \nu } dx^ \mu dx ^ \nu 
\] This is something that perhaps we're more 
familiar with. 
Since our metric is non-degenerate, it is 
a theorem in linear algebra that we can diagonalise this thing, and furthermore 
we have no zero eigenvalues. 
If we diagonalise $ g _{ \mu \nu }$, 
it has positive and negative elements (non are zero). 
The number of negative elements is called the 
signature of the metric. 
There's a theorem in linear algebra (Sylvester's law of inertia) 
which says that the signature is invariant, which means that 
it makes sense to talk about signatures in a well defined sense. 

\subsubsection{Riemannian Manifolds} 
A Riemannian manifold is a manifold with metric 
with signature all positive. 
For example, Euclidean space in $ \mathbb { R} ^ n $ endowed with 
the usual Pythagorean metric. 
\[
g = dx^ 1 \otimes dx ^ 1 + \dots + dx^ n \otimes dx^ n 
\]  A metric gives us a way to measure the
length of a vector $ X \in \mathcal{ X } ( \mathcal{ M } ) $. 
Since our signature is positive, we have that $ g ( X, X ) $ is a 
positive number, and hence we can take a square root to 
define a norm. 
\[
| X | = \sqrt{ g ( X, X ) } 
\] We can also find the angle between vectors, where
\[
g ( X, Y ) = | X| | Y | \cos \theta 
\] It also gives us a way to measure the distance between two points, 
$ p $, $ q $. Along the curve 
\[
\sigma : [ a, b ] \to \mathcal{ M }, \quad \sigma ( a) = p, \sigma ( b ) = q
\] our distance is given by the integral of the metric at 
that point where $ X $ is the tangent to the curve' 
\[
s = \int_ a ^ b dt \, \sqrt{ g ( X, X) } \mid_{ \sigma( t) }  
\] where at each point $ X $ is our tangent to the curve. 
If our curve has the coordinates $ x^ \mu ( t ) $, 
then our distance is 
\[
 s = \int_ a ^ b dt\, \sqrt{ g_{ \mu \nu } ( x) \frac{dx^ \mu }{ dt } \frac{ dx^ \nu }{ dt }} 
\]
Note that this notion of distance still makes sense since and 
is well defined since it's easy to check that 
$ s $ is invariant under re-parametrisations of our curve. 

\subsubsection{Riemannian Geometry} 
A Lorentzian manifold is a manifold equipped with 
a metric of signature $ ( - + + ... ) $. 
For example, Minkowski space is  $ \mathbb{ R} ^ n $
but our metric is 
\[
\eta = - dx^ 0 \otimes dx^ 0 + dx^ 1 \otimes dx^ 1 + \dots + dx^{ n-1 } \otimes dx^{ n- 1}
\] with components
\[
\eta_{ \mu \nu  } = diag ( - 1, 1 \dots, 1 )  
\]  This is slightly different to our Riemannian manifold 
case since now we can have vectors with negative or zero length. 
We classify vectors $ X_ p \in T_p ( \mathcal{ M } ) $
as 
\[
g ( X_p , X_p ) = \begin{cases}
 < 0 & \text{ timelike} \\
 = 0 & \text{null} \\
 > 0 & \text{spacelike}
\end{cases}
\] At each point $ p \in \mathcal{ M } $, we draw null tangent vectors called lightcones, 
and as we'll soon see, this region outlines our area of possible causality.

A curve is called timelike if it's tangent 
vector at every point is timelike. We can see this in the figure 
where we have two light-cones for future and past time. 
\begin{figure}[h]
	\centering
	\hspace*{3cm}\includegraphics[width=0.5\linewidth]{figures/lightcone.png}
	\caption{Here we show timelike vectors with negative norm!}%
	\label{fig:figures / lightcones}
\end{figure}
In this case, we can measure the distance between two points. 
\[
\tau = \int_{ a} ^ b dt \sqrt{  - g_{ \mu \nu } \frac{dx^ \mu }{ dt } \frac{ dx ^ \nu }{ dt }} 
\] This object $ \tau $ is called the proper time 
between two points. Philosophically, this is a parameter which is 
invariant in all frames. 
If we were to reparametrise this curve, our 
definition of $ \tau $ remains invariant. 

\subsubsection{The Joys of a metric}

\begin{claim}{Metrics induce a natural isomorphism from vectors to 1-forms.} 
The metric gives a natural (basis independent) isomorphism 
\[
g : T_p ( \mathcal{ M }) \to T_p ^ * ( \mathcal{ M } ) 
\] Given $ X \in \mathcal{ X } \left( M  \right) $, we can construct 
$ g ( X , \cdot  ) \in \Lambda^ 1 ( \mathcal{ M }) $. 
If $ X = X^ \mu \partial _ \mu $, our corresponding one form is 
\[
g_{ \mu \nu } X^ \mu dx^ \nu : = X_\nu dx^ \nu 
\] In this formula, we've written the index on $ X $ downstairs!
The metric provides a natural isomorphism between 
our vector space and our one-forms. Hence, 
this metric allows us to raise and lower indices ,
which means that our metric switches the mathematical space 
we are working in. 
Lowering an index is really the statement that there's a natural 
isomorphism. 
Because $ g $ is non-degenerate, there's an inverse
\[
g ^{ \mu \nu } g _{ \nu \rho } = \delta \indices{ ^ \mu _ \rho } 
\] This defines a rank $ ( 2, 0 ) $ tensor $ \hat{g} = g^{ \mu \nu } \partial  _\mu \otimes \partial  _ \nu  $, and 
we can use this to raise indices. We have 
\[
X^\mu = g ^{ \mu \nu } X_ \nu  
\]
\end{claim} 

\begin{claim}{Metrics induce volume forms to integrate with.}
There's something else that the metric gives us. 
We also get a natural volume form. On a Riemannian manifold, our 
volume form is defined to be 
\[
v = \sqrt{ det g_{ \mu \nu } }  dx^ 1 \wedge  \dots \wedge  dx^ n 
\] We write $ g = det g_{ \mu \nu }$
On a Lorentzian manifold, $ v = \sqrt{ - g }  dx^ 0  \wedge  \dots \wedge  dx^{ n- 1}$
This is independent of coordinates. 
In new coordinates, 
\[
dx ^ \mu = A \indices{ ^ \mu _ \nu } \quad, A \indices{ ^ \mu _ \nu } \frac{\partial x^ \mu }{\partial  x^ \nu }    
\] 
We see how they change. 
\begin{align*}
dx^ 1 \wedge  \dots \wedge  dx^ n & = A \indices{ ^ 1 _{ \mu _ 1 }} \dots A \indices{^ n _{ \mu _ n } } d \tilde{x} ^{ \mu _ 1} \wedge  \dots \wedge  d \tilde{x }^{ \mu _ n }   \\
				  &=  \sum_{ \text{ perms } \pi } A \indices{ ^ 1 _{ \pi ( 1)  } } \dots A \indices{ ^ n _{ \pi ( n ) } } d \tilde{x }^ 1 \wedge  \dots \wedge  d \tilde{x  }^ n      \\
		& = det ( A) d \tilde{x  }^ 1 \wedge  \dots \wedge  d \tilde{x } ^ n   
\end{align*} If we have that $ det A > 0 $ , coordinate change preserves 
orientation. 
Meanwhile, 
\begin{align*}
g_{ \mu \nu } &=  \frac{\partial  \tilde{x } ^ \rho  }{\partial  x ^ \mu }  \frac{\partial  \tilde{x } ^ \sigma  }{\partial  x^ \nu }  \tilde{ g }_{ \rho \sigma }  \\ 
       &=  ( A^{ - 1} ) \indices{ ^ \rho _  \mu } ( A ^{ - 1}) \indices{ ^ \sigma _ \nu } \tilde{g }_{ \rho \sigma }    \\
\end{align*} 
Hence, 
\[
det g _{ \mu \nu } = ( det A^{ - 1} ) ^ 2 det \tilde{ g } _{ \rho \sigma } 
\] Thus we have that 
\[
v = \sqrt{ | \tilde{ g } |  } d \tilde{x }^ 1 \wedge  \dots \wedge  d \tilde{ x }^ n    
\] in components, 
we have that 
\[
v = \frac{1}{ n ! } v_{ \mu_1 \dots \mu_ n } dx^{ \mu _ 1 } \wedge  \dots \wedge  dx^{ \mu _ n }
\] where our components are given by 
\[
v _{ \mu _ 1 \dots \mu _ n } \epsilon_{ \mu _ 1 \dots \mu _ n }
\] we can integrate functions as 
\[
\int_{ \mathcal{ M } } fv = \int_{ \mathcal{ M } } d^ n x \sqrt{ | g | } f (x)  
\] 

The metric provides a map from $ \omega  \in \Lambda^p ( \mathcal{ M } ) $
to $ ( * \omega ) \in \Lambda ^{ n - p } ( \mathcal{ M } ) $  defined by
\[
 ( * \omega )_{ \mu_1 \dots \mu_{ n - p }  } = \frac{1}{p! } \sqrt{  | g ! | }  \epsilon_{ \mu_1 \dots \mu_{ n - p } \nu_1 \dots \nu_{ p } } \omega^{ \nu_1 \dots \nu_{ p } }
\] This object is called Hodge dual. We can check that 
\[
	* ( * \omega ) = \pm ( -1) ^{ p ( n - p ) } \omega 
\] with $ + $ used in with a Riemannian metric, and $ - $ used 
with a Lorentzian metric. 
We can then define an inner product on forms. 
Given $ \omega, \eta \in \Lambda ^ p ( \mathcal{ M } ) $, let
\[
 \left< \eta , \omega  \right>  = \int_{ \mathcal{ M } } \eta \wedge  * \omega 
\] The integrand is a top form so this is okay. 
This allows us to introduce a new object. 
If we have a p-form $ \omega \in \Lambda ^ p ( \mathcal{ M } ) $, and a 
p-1 form $ \alpha \in \Lambda ^{ p - 1 }( \mathcal{ M } )  $, then 
\[
 \left< d\alpha, \omega    \right> = \left< \alpha, d^ \dagger \omega \right>
\] when $ d ^ \dagger : \Lambda ^ p ( \mathcal{ M } ) \to \Lambda ^{ p -1 } ( \mathcal{ M } ) $, is 
\[
 d^ \dagger  = \pm ( - 1)^{ np + n - 1 } * d * 
\] where again our $ \pm $ signs depend on whether we have 
a Riemannian or Lorentzian metric. 
To show this, on a closed manifold Stokes' theorem implies that 
\[
	0 = \int_{ \mathcal{ M } } d ( \alpha \wedge  * \omega )  = \int_{ \mathcal{ M } } d \alpha \wedge  * \omega + ( - 1) ^{ p - 1 }\alpha \wedge  d * \omega 
\] But the term on the right is just 
\[
	= \left< d\alpha , \omega  \right> + ( - 1 ) ^{ p - 1 } \text{sign} \left<\alpha, * d * \omega  \right>
\] When we fix our sign, we 
get the result. 
There's actually a close relationship between forms in differential 
Geometry and fermionic antisymmetric fields in quantum field theory.
\end{claim} 

\subsection{Connections and Curvature} 
What's the point of a connection? 
This is going to be our final way of differentiation as opposed to 
the things we have already written down. 
A connection is a map $ \nabla  : \mathcal{ X } ( \mathcal{ M } ) \times \mathcal{ X } ( \mathcal{ M } ) \to \mathcal{ X } ( \mathcal{ M } ) $

We write this as $ \nabla  ( X, Y )  = \nabla _{ X } Y $. 
The purpose of doing this is to make it look more like differentiation. 
Here, we call $ \nabla  _ X $ the covariant derivative, and it satisfies 
\begin{itemize}
	\item Linearity in the second argument $ \nabla _ X ( Y + Z )  = \nabla _ X Y + \nabla _ X Z $
	\item Linearity in the first argument $ \nabla _{ f X + g Y }  Z = f \nabla _ X Z + g \nabla _ Y Z \forall f, g \in C^\infty ( \mathcal{ M } ) $
	\item Leibniz $ \nabla _ X ( f Y )  = f \nabla _ X Y + ( \nabla _ X F ) Y $, with $ f $ a function. 
	\item In the above, we have that $ \nabla _ X f = X ( f ) $, agrees with usual differentiation. 
\end{itemize} 

Suppose we have a basis of vector fields $ \left\{  e_ \mu  \right\} $. 
We write 
\[
 \nabla _{ e _ \rho } e_{ \nu } = \Gamma^ \mu _{ \rho \nu } e _{ \mu } 
\] This expression is a vector field because we know 
the derivative spits out a vector field.
We use the notation $ \nabla _{ e _ \mu }  = \nabla _ \mu $, 
to make the connection look like a partial derivative. Then, applying the 
Leibniz rule we can do a derivative on a general vector to give 
\begin{align*}
	\nabla _ X Y & = \nabla _ X ( Y ^ \mu e_{\mu  } ) + X ( Y ^ \mu ) e_\mu + Y ^\mu \nabla _ X e _\mu \\
		     &=  X ^ \nu e_\nu ( Y ^\mu ) e _\mu + Y ^\mu X^\nu \nabla _ \nu e _ \mu  \\
		     &=  X ^ \nu ( e _ \nu Y ^ \mu  + \Gamma^\mu _{ \nu \rho } Y ^ \rho ) e _ \mu  
\end{align*}
Because the vector sits out front we can write 
\[
 \nabla _ X Y  = X^ \nu \nabla _ \nu  Y 
\]  with 
\[
	\nabla _ \nu Y = ( e _ \nu ( Y ^ \mu ) + \Gamma^ \mu _{ \nu \rho } Y ^ \rho ) e _ \mu 
\] or, we define we equivalently define 
\[
	( \nabla _ \nu Y ) ^ \mu := \nabla _ \nu Y ^ \mu = e _ \nu ( Y ^ \mu ) + \Gamma ^ \mu _{ \nu \rho } Y ^ \rho 
\] Comparing this to the Lie derivative however, 
we have that $ \mathcal{ L } _ X $ depends on $ X $ and $ \partial X $,
so we can't write " $ \mathcal{ L }_ X = X ^ \mu \mathcal{ L } _ \mu $ ". 
If we take a coordinate basis 
for the vector fields $ \left\{  e _  \mu  \right\}  = \left\{  \partial  _ \mu  \right\} $, 
then was have that 
\[
 \nabla _ \nu Y ^ \mu = \partial  _ \nu Y ^ \mu + \Gamma ^ \mu _{ \nu \rho } Y ^ \rho 
\] In terms of notation, we can replace differentiation with  
punctuation. So, we have that 
\[
 \nabla _ \nu Y ^ \mu : = Y ^ \mu _{ ; \nu } : = Y ^ \mu_{ , \nu } + \Gamma^ \mu _{ \nu \rho } Y ^ \rho 
\] 
The connection is not a tensor! Consider a change of basis 
\[
	\tilde{ e } _ \nu = A \indices{ ^\mu _\nu } e_\mu, \text{ with } A \indices{ ^\mu _\nu} = \frac{\partial x^\mu }{\partial \tilde{x} ^\nu}    
\]  
We have that 
\[
\nabla _{ \tilde{e } _ \rho  } \tilde{ e } _ \nu \tilde{ \Gamma } ^ \mu _{ \rho \nu } \tilde{ e } _ \mu  = \nabla _{ A \indices{ ^\sigma _ \rho } e _ \sigma  } ( A \indices{ ^ \lambda _ \nu } e _ \lambda )  = A \indices{ ^ \sigma _ \rho } \nabla _ \sigma ( A \indices{ ^ \lambda _ \nu  } e _ \lambda  )        
\]  This simplifies further to give 
\begin{align*}
	&=  A \indices{ ^ \sigma _ \rho } ( A \indices{ ^ \lambda _ \nu  } \Gamma ^ \tau _{ \sigma \lambda  } e _ \tau + e _ \lambda \partial  _ \sigma A \indices{ ^ \lambda _ \nu }     \\
	&=  A \indices{ ^ \sigma _ \rho } ( A \indices{ ^ \lambda _ \nu } \Gamma ^ \tau _{ \sigma \lambda } + \partial  _ \sigma A \indices{ ^ \tau _ \lambda } ) e _ \tau     , \quad e _ \tau = ( A^{ -1 } ) \indices{ ^ \mu _ \tau  } \tilde{ e } _ \mu  	 
\end{align*}
This implies that 
\[
	\tilde{ \Gamma } ^ \mu _{ \rho \nu }  = ( A ^{ -1 } ) \indices{ ^ \mu _ \tau } A \indices{ ^ \sigma _ \rho } A \indices{ ^ \lambda _ \nu } \Gamma ^ \tau _{ \sigma \lambda } + ( A ^{ -1 } ) \indices{ ^ \mu _ \tau } A \indices{ ^ \sigma _ \rho } \partial  _ \sigma A \indices{ ^ \tau _ \nu }       
\]  So we have that an extra term is added on. 
We can also use the connection to differentiate 
other tensors. 
We simply ask that it obeys the Leibniz rule. 
For example, $ \omega \in \Lambda ^ 1 ( \mathcal{ M } ) , Y \in \mathcal{ X } ( \mathcal{ M } ) $ 
\[
	X ( \omega ( Y ) ) = \nabla _ X ( \omega ( Y ) )  = ( \nabla _ X \omega ) ( Y ) + \omega ( \nabla _ X Y ) 
\] Thus, rearranging the terms we have that 
\[
	\nabla _ X \omega ( Y )  = X ( \omega ( Y ) ) - \omega ( \nabla _ X Y ) 
\] So, in terms of coordinates, 
\[
	X^ \mu ( \nabla _ \mu \omega _ \nu ) Y ^ \nu = X^ \mu \partial  _ \mu ( \omega  _\nu Y ^ \nu ) - \omega _ \nu X^ \mu ( \partial  _ \mu Y ^ \nu + \Gamma ^ \nu _{ \mu \rho } Y ^ \rho )  = X ^ \mu ( \partial  _ \mu \omega _ \rho - \Gamma ^ \nu _{ \mu \rho } \omega _ \nu ) Y ^ \rho 
\] Hence, we have that 
\[
 \nabla _ \mu \omega _ \rho  = \partial  _ \mu \omega _ \rho  - \Gamma ^ \nu _{ \mu \rho } \omega _ \nu 
\]  %In general, 
%\[
 %\nabla _ \gamma T \indices{ ^{ \mu_1 \dots \mu _{ p }  } _{ \nu _ 1 \dots \nu _ q }} = \partial _ \mu T \indices{ ^{ \mu _ 1 \dots \mu _ p } _{ \nu_ 1 \dots \nu _ q }} + \Gamma^{ \mu _ 1 }_{ \rho \sigma }  T \indices{ ^{ \sigma \mu _ 2 \dots \mu _{ p }} _{ \nu _ 1 \dots \nu _ q } + \dots  - \Gamma^{ \sigma }_{ \rho \nu_1 } T \indices{ ^{ \mu_ 1 \dots \mu _{ p }}_{ \sigma \nu_ 1 \dots \nu _ q }}    
%\]

Given a connection, we can construct two tensors. 
\begin{enumerate}
	\item Torsion is a rank $(1, 2) $ tensor
		\[
			T ( \omega; X, Y )  = \omega ( \nabla _ X Y - \nabla _ Y X - [ X, Y ] ) 
		\] where we have that $ \omega \in \Lambda^ 1 ( \mathcal{ M } ) $, 
		and $ X, Y \in \mathcal{ X } ( \mathcal{ M } ) $. 
		We can also think of $ T $ as a map from 
		$ \mathcal{ X } ( M ) \times \mathcal{ X } ( \mathcal{ M } ) \to \mathcal{ X } ( \mathcal{ M } ) $, with 
		\[
			T ( X, Y ) = \nabla _ X Y - \nabla _ Y X - [ X, Y ] 
		\] 
	\item Our second quantity that we can create is called \textbf{curvature}. 
		This is a rank $ ( 1, 3 )  $  tensor
		\[
			R ( \omega, X, Y , Z ) = \omega ( \nabla _ X \nabla  _ Y Z - \nabla _ Y \nabla  _ X Z - \nabla _{ [ X, Y ]  } Z 
		\] This is called the Riemann tensor. We can 
		also think of it as a map from $ \mathcal{ X  } ( \mathcal{ M } ) \times \mathcal{ X } ( \mathcal{ M } ) $ to a differential operator which 
		acts on $\mathcal{ X } ( \mathcal{ M } ) $. 
		\[
			R( X, Y ) = \nabla _ X \nabla  _ Y - \nabla  _ Y \nabla  _ X - \nabla _{ [ X, Y ] }
		\] 
\end{enumerate}

To show that these objects are tensors, 
we just need to check linearity in all the arguments. 
For example, 

\begin{align*}
	T ( \omega , fX, Y ) &=  \omega ( \nabla _{ f X  } Y - \nabla _ Y ( f X) - [ fX , Y ] )  \\
			     &=  \omega ( f \nabla _ X Y - f \nabla _ Y X - Y ( f) X  - ( f [ X, Y ] - Y ( f ) X ) ) \\
			     &=  f \omega ( \nabla _ X Y - \nabla _ Y X   - [ X , Y ] )   \\
			     &=  f T ( \omega , X, Y ) 
\end{align*}

Linearity is inherited from the fact 
that our covariant derivative is linear when you add. 
In a coordinate basis $ \left\{  e _\mu   \right\}  = \left\{  \partial  _ \mu  \right\}  $, 
and our dual basis of one-forms $ \left\{  f^ \mu \right\}  = \left\{  dx ^ \mu \right\} $, 
we have that the torsion in our components 
is 
\begin{align*}
	T \indices{ ^ \rho _{ \mu \nu } } &=  T ( f ^ \rho, e_ \mu , e _ \nu )  \\
					  &=  f^ \rho ( \nabla _ \mu e _ \nu - \nabla _ \nu e_\mu  - [ e_ \mu , e _ \nu ] )   \\
	&=  \Gamma^ \rho_{ \mu \nu }  - \Gamma ^ \rho _{ \nu \mu ` } 
\end{align*}
A connection with $ \Gamma^ \rho _{ \mu \nu }  = \Gamma ^ \rho _{ \nu \mu } $
has $ T ^ \rho _{ \mu \nu }  = 0 $ and is said to 
be torsion free. 
In addition, our curvature tensor has components 
\begin{align*}
	R \indices{ ^ \sigma _{ \rho \mu \nu }} &=  R ( f ^ \sigma; e_ \mu , e_ \nu , e _ \rho )  \\
						&=  f ^ \sigma ( \nabla _ \mu \nabla _ \nu e _ \rho - 
						\nabla _ \nu \nabla _ \mu e _ \rho 
						- \nabla _{ [ e_ \mu , e _ \nu ] } e _ \rho \\
						&=  f ^ \sigma ( \nabla _ \mu ( \Gamma ^ \lambda _{ \nu \rho } e_{ \lambda } )  - \nabla _ \nu ( \Gamma ^ \lambda _{ \mu \rho } e_{ \lambda  } ) )   \\
						&=  \partial _ \mu \Gamma ^ \sigma _{ \nu \rho } - \partial  _ \nu \Gamma ^ \sigma _{ \mu \rho } + \Gamma ^ \lambda _{ \nu \rho } \Gamma^ \sigma _{ \mu \lambda } 
						 - \Gamma^ \lambda_{ \mu \rho } \Gamma ^ \sigma _{ \nu  \lambda }
\end{align*}
Clearly, we have an antisymmetry property here, 
where \[
 R ^ \sigma _{ \rho \mu \nu } =  - R ^ \sigma _{ \rho \nu \mu } 
\]
\subsection{The Levi-Civita Connection} 
The fundamental theorem of Riemannian 
Geometry is that there exists a unique, 
torsion-free connection with the property 
obeying 
\[
	\nabla _ X g = 0 , \forall X \in \mathcal{ X } ( \mathcal{ M } ) 
\]
To prove this, suppose that this object exists. 
Then, 
\begin{align*}
	X ( g ( Y , Z ) ) &=  \nabla _ X [ g ( Y , Z ) ]  \\
			  &=  \nabla _ X g ( Y , Z ) + g ( \nabla _ X Y , Z ) + g ( Y , \nabla _X , Z )  \\
			  &=  g ( \nabla _ X Y , Z ) + g ( Y , \nabla _ X  Z )  
\end{align*}
The fact that our torsion vanishes implies 
that 
\[
	\nabla  _ X Y - \nabla _ Y X = [ X, Y ] 
\] Hence, our equation on the left hand of 
our blackboard reads 
\[
	X ( g ( Y ,Z ) ) = g ( \nabla _ Y X , Z ) + g ( \nabla _ X Z , Y ) + g ( [ X, Y ] , Z ) 
\] Now we cycle $ X, Y , Z $, where we find that 
\begin{align*}
	Y ( g ( X, Z ) ) & = g ( \nabla _ Z , Y , X ) + g ( \nabla _ Y X , Z )  + g ( [ Y , Z] , X ) \\
	Z ( g ( X, Y ) ) &=  g ( \nabla _ X Z , Y ) + g ( \nabla _ Z Y , X ) + g ( [ Z, X] , Y ) 
\end{align*}
If add the first two equations and 
then subtract by the third one we get that 
\begin{align*} 
	g ( \nabla _ Y X , Z ) &= \frac{1}{2 } \big [  X g ( Y , Z ) + Y g ( X, Z ) - Z g ( X, Y )  \\
		 & - g ( [ X, Y ] , Z ) - g ( [ Y , Z ] , X ) + g ( [ Z, X ] , Y ) 
\end{align*} 
In a coordinate basis, we have
that $ \left\{  e_ \mu  \right\}   = \left\{  \partial _ \mu  \right\} $, we 
have that 
\begin{align*}
	g ( \nabla _ \nu e _ \mu , e _ \rho ) &=  \Gamma ^ \lambda _{ \nu \mu } g _{ \lambda \rho } = 
	\frac{1}{2 } ( \partial _ \mu g _{ \nu \rho } + \partial  _ \nu g _{ \mu \rho }  - \partial  _ \rho 
	g _{ \mu \nu } ) \\
\end{align*} 
Where we have that 
\[
	\Gamma ^ \lambda _{ \mu \nu } = \frac{1}{2 } g ^{ \lambda \rho } ( \partial  _ \mu g _{ \nu \rho } + \partial  _ \nu g _{ \mu \rho } - \partial  _ \rho g _{ \mu \nu } ) 
\] 
This is the Levi-Civita connection, 
and the $ \Gamma ^ \lambda _{ \mu \nu } $ are called 
the Christoffel symbols. 
We still need to show that it transforms as a connection, 
which we leave as an exercise.

\subsubsection{The Divergence Theorem} 
Consider a manifold $ \mathcal{ M } $ with 
metric $ g $, with 
boundary $ \partial  \mathcal{ M } $ , and
let $ n ^ \mu $ be an outward pointing 
vector orthogonal to $ \partial  \mathcal{ M } $. 
Then, for any $  X^ \mu $, our claim is that 
\[
 \int_{ \mathcal{ M } } d^ n x \sqrt{ g }  \nabla _ \mu X^ \mu = \int _{ \partial  \mathcal{ M } } d^{ n - 1} x \sqrt{ \gamma }  n_\mu 	 X ^ \mu 
\]   On a Lorentzian manifold, this also holds 
with $ \sqrt{ g  }  \to \sqrt{  - g}   $ and this 
also holds provided $ \partial   \mathcal{ M } $ is 
purely timelike or purely spacelike. 

First, we need a lemma, that $ \Gamma^{ \mu } _{ \mu \nu }  = \frac{1}{ \sqrt{ g }  } \partial_\nu \sqrt{ g } $. To prove this, we have that, writing out 
the definitions, that 
\[
	\Gamma ^ \mu _{ \mu \nu }  = \frac{1}{ 2  } g ^{ \mu \rho } \partial  _ \nu g _{ \mu \rho } = \frac{1}{2 }\tr ( \hat{ g } ^{ - 1 } \partial  _ \nu \hat{ g } ) 
\] But from this we have that 
\begin{align*}
	\dots &= \frac{1}{2 } \tr ( \partial  _\nu \log \hat{ g } )    \\
	      &=  \frac{1}{2 } \partial  _ \nu \log \det \hat{ g }  \\
	      &=  \frac{1}{2 } \frac{1}{ \det \hat{ g } } \partial  _ \nu \det \hat{ g }  \\
	      &=  \frac{1}{ \sqrt{ g } } \partial _ \nu \sqrt{ g }  	
\end{align*}

\pagebreak 
\section{Example Sheet 1} 
To: João Melo. 
From: Afiq Hatta

Questions 5, 7 (and the rest) 
\pagebreak 
\subsection{Question 1} 
If we're given compoonents of a vector field and want to solve for its integral curve, then we need to solve the equation
\[ 
\left. \frac{ d x^\mu(t) }{ dt} \right\vert_{\phi(p)} = \left. X^\mu( x^\nu(t) ) \right\vert_{ \phi( p)}
\] 

So for the first integral curve, we need to solve the system 
\begin{align*} 
\frac{dx}{dt}  &= y \\
\frac{dy}{dt}  &= -x 
\end{align*} 
This is made a lot easier by writing out the system in polar coordinates (which is indeed a different cart for the manifold $\mathbb{R}^2$, and writing $(x , y)  = ( r \cos \theta, r \sin \theta)$ with the chain rule gives us 
\begin{align*} 
\dot{r} \cos \theta  - r \dot{\theta} \sin \theta  &= r \sin \theta \\
\dot{r} \sin \theta  + r \dot{\theta} \cos \theta &=  - r \cos \theta 
\end{align*} 
If we multiply the first equation by $\sin \theta$ and the second equation by $\cos \theta$, and then subtract the first equation from the second equation, we've eliminated the $\dot{ r}$ term. We're left wwith 
\[ 
\dot{\theta} =  -1 \implies \theta  =  - t + C
\] 
for some constant $C$. Substituting in $\dot{\theta} = - 1$ in our first equation gives the condition that $\dot{r} = 0 \implies r = R$ for $R$ constant. Hence our integral curves are merely circles of arbitrary radius about the origin. 

For our second vector field 
\[ 
X^\mu  = ( x - y, x +y ) 
\] 
we proceed exactly as before, with polar coordinates. One finds instead that $\dot{\theta} = 1$, and hence that $\dot{r}  = r$. Thus, we have that 
\[
\theta = t + A, \quad r = Be^t 
\] 
for arbitrary constants $A, B$. These curves are spirals. 

\pagebreak
\subsection{Question 2} 
We're given that the map $\hat{ H }  : T_p (M) \rightarrow T_p^* ( M ) $ is a linear map. So, since $\hat{ H } $ is linear, 
\begin{align*} 
H ( X, \alpha Y + \beta Z ) & = \hat{H} ( \alpha Y + \beta Z) ( X) \\ &= ( \alpha \hat{ H} ( Y ) + \beta \hat{H} ( Z) ) ( X) \\
&= \alpha \hat{H}( Y) ( X) + \beta \hat{H} (Z) ( X) \\
&= \alpha H ( X, Y ) + \beta H ( X, Z) 
\end{align*} 
Thus, $H$ is linear in the second argument. The only fact that we've used here is that $\hat{H}$ is a linear map. For the linearity in the first argument, we use the fact that $\hat{H} (Y) \in T_p^* (M)$, which means that it's a linear map. So 
\[ 
H ( \alpha X + \beta Z, Y ) = \hat{H}( Y) ( \alpha X + \beta Z)  = \alpha \hat{H} (Y) X + \beta \hat{H} (Y) ( Z)  = \alpha H(X, Y ) + \beta H(Z, Y ) 
\] Thus our map is linear in the first argument. Note that
\[ 
H : T_p ( M) \times T_p ( M) \rightarrow \mathbb{R} 
\] and since the map is multilinear, we have a rank (0, 2) tensor.

Similarly, if we had a linear map 
\[ 
\hat{G} : T_p( M ) \rightarrow T_p( M) 
\]
we could then define a new map 
\[ 
G : T_p^* ( M) \times T_p(M) \rightarrow \mathbb{ R}, \quad G( \omega, X)  = \omega ( \hat{G} ( X)) \] 
which is also bilinear in both arguments, and hence is a rank (1, 1) tensor. If $G$ is the identity map, then our induced function 
\[ 
\delta : T_p^* ( M) \times T_p( M ) \rightarrow \mathbb{R}, \quad \delta ( \omega, X ) = \omega (X) 
\] is indeed our standard Kronecker delta function. If we set $\omega = x^\mu, X = e_\nu$, then $\delta\indices{^\mu_\nu} = \partial_\mu(x^\mu) = \delta\indices{^\mu_\nu}$. 

\pagebreak 
\subsection{Question 3} 
In this question, we show that only symmetric (antisymmetric) parts of a tensor are 'conserved' whe contracted with symmetric (antisymmetric) tensors over the same indices. If $S^{ \mu \nu} $ is symmetric, then 
\begin{align*} 
V^{ ( \mu \nu) } S_{\mu \nu}  & = \frac{1}{2} \left( V^{ \mu \nu} + V^{ \nu \mu} \right)S_{ \mu \nu} \\
&= \frac{1}{2} V^{\mu \nu}S_{ \mu \nu} + \frac{1}{2} V^{ \nu \mu} S_{\mu \nu} \\
&= \frac{1}{2} V^{\mu \nu}S_{ \mu \nu} + \frac{1}{2} V^{\nu \mu}S_{ \nu \mu } \\
&= \frac{1}{ 2} V^{\mu_\nu} S_{ \mu \nu} + \frac{1}{2} V^{\mu \nu}S_{ \mu \nu} \\
&= V^{ \mu \nu} S_{ \mu \nu} 
\end{align*} 
Going into the second last line we've just relabelled over summed indices. Going into the third line we've used the fact that $S$ is a symmetric tensor.The case for when we contract $V^{\mu \nu}$ for an antisymmetric tensor is entirely similar. 



\pagebreak 
\subsection{Question 5} 
We show that our components $F_{\mu \nu}$ transform appropriately under a change of coordinates. This is done with the chain rule. 
\begin{align*} 
F_{ \mu \nu} & \rightarrow F_{ \mu \nu}' \\
&= \frac{ \partial^2 f}{ \partial x'^\mu \partial x'^\nu} \\
&= \frac{ \partial}{ \partial x '^\mu } \left( \frac{ \partial x^\rho }{ \partial x'^\nu } \frac{ \partial f }{ \partial x^\rho } \right) \\
&= \frac{ \partial x^\sigma}{ \partial x'^\mu } \frac{ \partial }{ \partial x^\sigma} \left( \frac{ \partial x^\rho}{ \partial x'^\nu} \frac{ \partial f}{ \partial x^\rho } \right) \\
&= \frac{ \partial x^\sigma}{ \partial x'^\mu} \frac{ \partial^2 x^\rho}{ \partial x^\sigma \partial x'^\nu } \frac{ \partial f }{ \partial x^\rho} + \frac{ \partial x^\sigma }{ \partial x'^\mu }\frac{\partial x^\rho}{\partial x'^\nu}\frac{\partial^2 f}{ \partial x^\rho \partial x^\sigma } \\
&= \frac{ \partial x^\sigma}{ \partial x'^\mu } \frac{ \partial x^\rho}{ \partial x'^\nu } \frac{ \partial^2 f }{ \partial x^\sigma x^\rho } \\
& =   \frac{ \partial x^\sigma}{ \partial x'^\mu } \frac{ \partial x^\rho}{ \partial x'^\nu } F_{\rho \sigma} 
\end{align*} 

There's a reason why we've taken the first term to zero going into the fifth line. 
Since $ df = 0 $ at $ p$,  then for an arbitrary vector $ A$ in any basis, we
have that at $ p \in \mathcal{ M } $,  
\[
df ( A) = A( f) = A^{\mu } \partial_\mu ( f) = 0, \implies \partial_\mu ( f) = 0 \text{ at } p, \quad \forall \mu = 1, \dots D
\] 
So this term goes to zero, since we only have a single derivative acting on $ f $. *
Thus, the Hessian obeys the tensor transformation law. Since our components transform in the two lower indices with a change of coordinates, this object is basis invariant and hence is a rank (0, 2) tensor. 

Since this is a rank (0, 2) tensor, our coordinate independent way of expressing this object would be 
\[ 
F: T_p( M ) \times T_p ( M ) \rightarrow \mathbb{R}
\] 
This specific representation is 
\[ 
F(V, W) = VW(f) 
\] 
We can show this by expanding with coordinates. 
\begin{align*} 
VW(f)  &= V^\mu \partial_\mu ( W^\nu \partial_\mu f ) \\
&= (\partial_\mu W^\nu)( \partial^\mu V_\nu )f + V^\mu W^\nu \partial_\mu \partial_\nu f \\
&= (V^\mu \partial_\mu W^\nu) \partial_\nu f + W^\nu V^\mu \partial_\mu \partial_\nu f \\
&= W^\nu V^\mu \partial_\mu \partial_\nu f \\
&= W^\nu V^\mu F_{\mu \nu}
\end{align*} Here we've used the fact that $df=0$, which implies that for an arbitrary set of components $Z^\mu$, we have that $Z^\mu \partial_\mu f = 0 $. In the above, we identify this as $Z^\nu = V^\mu \partial_\mu W^\nu$, and hence the first term in the third line goes to zero.  

This implies that $F_{\mu \nu}$ are indeed the components of $F$. Multi linearity in both arguments is just inherited 
from the linearity of $ V , W $ as vector fields. 

* 
A different argument would be that one can note that the first term is of the form 
\[
( G_{ \mu' \nu' })^\rho \partial _\rho f  
\] Where we can view $ G_{\mu' \nu' } $ as $D ^ 2 $ separate vectors indexed by $ \mu' $ and $ \nu' $. 
Thus, since  $df = 0 $, this term goes to zero. (I like this way since it's manifestly 
a bit more basis invariant!) 

\pagebreak

\subsection{Question 6} 
This question explores how the determinant of a metric transfors under coordinate transformations. For this question, we denote the determinant of a change of basis as the Jacobian: 
\[ 
\mathcal{ J } = det \left( \frac{ \partial x^\rho }{ \partial x'^\nu } \right) 
\]  
Hence, when we do a coordinate transform, since $det( AB ) = det( A ) det( B) $, we have that
\begin{align*} 
g' &= det (g_{ \mu \nu }' ) \\
&= det \left( \frac{ \partial x^\sigma}{ \partial x'^\mu } \frac{ \partial x^\rho }{ \partial x'^\nu } g_{ \rho \sigma } \right) \\
&= det(g) J^{ -2 } 
\end{align*} 
This is because the expression in the determinant is the inverse of what we've defined the Jacobian to be.



\pagebreak 

\subsection{Question 7} 

\subsubsection*{Lie derivative of 1-form} 
Using the Leibniz rule for our Lie derivative, we consider the Lie derivative 
for $ \mathcal{ L }_X ( \omega Y ) $; 
\[
\mathcal{ L }_X ( \omega( Y) ) = \omega ( \mathcal{ L }_X Y ) + ( \mathcal{ L }_ X \omega) ( Y) 	
\] This expression is basis independent. 
Now, observe that $ \omega ( Y ) $ is a function in  $ C^ \infty ( \mathcal{ M }) $. 
Thus, the Lie derivative for this term is just given by $ X ( \omega ( Y) )  $. 
We also know that $ \mathcal{  L }_X ( Y )  = [ X, Y ] $. 
Thus, 
\begin{align*}
X^\mu\partial _\mu ( \omega_\nu Y ^\nu) & = (\mathcal{L }_X \omega)_\nu Y ^\nu + \omega_\nu [ X, Y ] ^\nu  \\
Y_\nu X^\mu \partial _\mu \omega^\nu + \omega_\nu X^\mu \partial _\mu Y^\nu  &= ( \mathcal{ L }_ X \omega)_\nu Y ^\nu + \omega_\nu X^\mu \partial_\mu Y^\mu  - \omega_\nu Y^\mu \partial _\mu X^\nu
\end{align*} Up to index relabelling of the dummy indices, the last term of the LHS 
and the second term of the RHS are the same, so they cancel out. 
Moving the negative term on the LHS to the right hand side and relabelling gives 
\[
Y_\nu ( X^\mu \partial_\nu \omega^\nu + \omega_\mu \partial_\nu X^\mu ) = ( \mathcal{ L }_X \omega) _\nu Y ^\nu
\] However, since $ Y $ was arbitrary we can just read off the 
basis independent components here. 
\[
( X^\mu \partial_\nu \omega^\nu + \omega_\mu \partial_\nu X^\mu ) = ( \mathcal{ L }_X \omega) _\nu 
\]
\subsubsection*{Lie derivative for a 2-tensor} 
We play exactly the same game with 
the rank $ ( 0 , 2) $ tensor as well. 
 \[
	 \mathcal{L }_ X  ( g ( V , W) ) = (\mathcal{L }_{X } g ) ( V, W) + g ( \mathcal{L }_X V, W ) + g ( V, \mathcal{L }_X W )  
\] In components, and multiplying out with the product rule, 
this term is 
\begin{align*}
	X^\nu (\partial  _\mu g_{ \alpha \beta }) V^\alpha W^\beta & + X^\mu g_{ \alpha \beta } W ^ \beta \partial _\mu V ^\alpha + X^\mu g _{ \alpha \beta } V^\alpha \partial _\mu W^ \beta  = \\ 																	    & = ( \mathcal{ L }_X g ) _{ \alpha \beta } V^ \alpha W ^ \beta + g_{ \alpha \beta } [ X , V]^ \alpha \partial _\mu W^ \beta + g_{ \alpha \beta } V ^ \alpha [ X, W ] ^ \beta 	    
\end{align*} The right hand side 
is just equal to, expanding the commutators in terms of components, 
\[
	= ( \mathcal{ L }_X g ) _{ \alpha \beta } V ^ \alpha W ^ \beta + g_{ \alpha \beta } X^\nu \partial  _\nu V ^ \alpha W ^ \beta  -g _{ \alpha \beta } V ^\nu \partial _\nu X ^ \alpha W ^ \beta + g_{ \alpha \beta } V ^ \alpha X ^ \nu \partial  _\nu W ^ \beta - g_{ \alpha \beta } V ^ \alpha W ^ \beta \partial  _\nu X  ^ \beta 
\] Now up to index relabelling $ \mu $ and $ \nu $, the second and fourth 
terms of this equation cancel out with the second and third terms on the LHS of our first equation. 
Thus, we're left with 
\[
X^\mu \partial _\mu g_{ \alpha \beta } V ^ \alpha W ^ \beta + g_{ \alpha \beta } V ^ \nu ( \partial  _\nu X ^ \alpha ) W ^ \beta + g _{ \alpha \beta } V ^ \alpha W^\nu ( \partial  _\nu X ^ \beta ) = ( \mathcal{ L }_ X g )_{ \alpha \beta } V ^ \alpha W ^ \beta 
\] Now, as before, relabelling $ \nu , \alpha$ in the second term and $ \nu , \beta $ in the third term 
recovers the expression in the question (after factorising out  $ V , W $).

\subsubsection*{Last part} 
The last part of the question is just a matter of 
substituting in definitions. 
\begin{align*}
	( \iota_X d \omega ) _ \mu &=  X^\nu ( d \omega) _{ \nu\mu } \\
				   &=  X ^ \nu 2 \partial  _{ [ \nu } \omega _{ \mu ] } \\
				   &=  X^\nu \partial  _\nu \omega_{ \mu } - X ^ \nu \partial  _{ \mu } \omega _{ \nu } 
\end{align*}
Also, we have 
\begin{align*}
	d ( \iota _{ X  } \omega) _{ \mu }  &=  \partial  _\mu ( \iota _ X \omega )  \\
					    & =  \partial _ \mu ( X^\nu \omega_\nu) \\
					    &=  \omega_ \nu \partial  _{ \mu } X ^ \nu + X ^ \nu \partial _ \mu \omega _ \nu 
\end{align*}
Adding these terms together gives 
\[
 ( \iota _ X d \omega ) _ \mu + d ( \iota _ X \omega )_\mu  = X^ \nu \partial  _ \nu \omega _ \mu - X ^ \nu \partial  _ \mu \omega _ \nu + \omega _ \nu \partial  _ \mu X ^ \nu + X ^ \nu \partial  _ \mu \omega _ \nu  = X^ \nu \partial  _ \nu  \omega _ \mu + \omega _ \nu X ^ \nu 
\] since we have cancellation with the second and last term. 



\pagebreak 

\subsection{Question 8} 
The components of the exterior derivative of a $ p-$ form consists of the antisymmetrisation of 
$ p + 1 $ indices. Suppose that m $ \omega $ is a $ p- $ form. Then 
\[
( d\omega  )_{\mu_{1} \mu_2 \dots \mu_{p + 1 }} = ( p + 1 ) \partial_{[  \mu_1} \omega_{ \mu_2 \dots \mu_{ p + 1} ] } 	
\] Thus, we can expand the components of the exterior derivative of this object as 
\[
	( d ( d \omega  ))_{ \mu_1 \dots \mu_{p + 1}} = ( p + 2 )( p + 1) \partial_{ [ \mu_1 } \partial_{ [  \mu_2 } \omega _{\mu_3 \dots \mu_{ p + 2 } ]]  }
\] Now, we have a tricky thing to deal with here. 
We have an antisymmetrisation nested inside of an antisymmetrisation. 
We claim that nesting an antisymmetrisation inside an antisymmetrisation 
is just the larger antisymmetrisation: 
\[
X_{ [ \mu_1 [ \mu_2 , \dots \mu_{ p } ]] } = X_{[ \mu_1 \mu_2 \dots \mu_{ p}]}
\] We can prove this by expanding out an antisymmetrisation based on 
just the $ \mu_1 $ index first. 

\begin{align*}
X_{[ \mu_1 \dots \mu_{p } ] } &=    \frac{1}{ p ! } \big (  \sum_{\sigma \in S_{p - 1}} \epsilon ( \sigma ) X_{\mu_1 \mu_{ \sigma ( 2)  } \dots \mu_{ \sigma ( p ) } } \\
&- \sum_{ \sigma \in S_{ p - 1}} \epsilon( \sigma ) X_{ \mu_{ \sigma ( 2 ) } \mu_1  \mu_{ \sigma ( 3)  } \dots \mu_{ \sigma ( p ) }} \\
      & \vdots \\
      & + ( - 1)^{p + 1 } \sum_{ \sigma \in S_{ p + 1  } } \epsilon ( \sigma ) X_{ \mu_{ \sigma ( 2)  } \mu_{ \sigma( 3)  }  \dots \mu_{ \sigma ( p ) } \mu_1 }  \big )  
\end{align*}
So, when we nest antisymmetrisations, we have terms in the sum 
that look like
\[
X_{ [ \mu_1 [ \mu_2 \dots \mu_{p } ]]} =\sum_{ \text{ similar sum as above but of } } \frac{1}{ p ! } \sum_{ \sigma \in S_{ p - 1}  } \epsilon ( \sigma ) X_{ \mu_1 [ \mu_{ \sigma ( 2)  } \dots \sigma ( p )  ] } 
\] But, expanding out the definition, we have that this term 
is just equal to 
\[
= \frac{1}{ p ! } \frac{1}{ ( p - 1 )! } \sum_{ \sigma' \in S_{ p - 1 }  } \sum_{ \sigma \in S_{ p - 1  } } \epsilon( \sigma' ) \epsilon ( \sigma )   X_{ \mu_1  \mu_{ \sigma' \sigma ( 2)  } \dots \mu_{  \sigma' \sigma ( p )   }  } 
\] But, we can compose each pair of permutations and write $ \sigma '' = \sigma' \sigma $, and since 
the sign operator for permutations is a homomorphism, 
we can write that  $ \epsilon ( \sigma ) \epsilon ( \sigma ' ) = \epsilon ( \sigma'' ) $. But, 
we have to be careful to make sure to count twice. Hence the term above is 
\[
\frac{1}{p ! } \frac{1}{ ( p - 1) ! } \sum_{ \sigma'  } \sum_{ \sigma } \epsilon ( \sigma'') X_{ \mu_1 \mu_{ \sigma '' ( 2) } \mu_{ \sigma'' ( 3 ) } \dots \mu_{ \sigma '' ( p ) }}
\] Now, we can relabel the $ \sigma $ index as $ \sigma''$, and so we're just summing 
over and extra $ \sigma' $. This gives
\[
 \frac{1}{p ! } \frac{1}{ ( p - 1 ) ! } \sum_{ \sigma '  } \sum_{ \sigma'' } \epsilon( \sigma '' )  X_{ \mu_1 \mu_{ \sigma '' ( 2) } \mu_{ \sigma'' ( 3 ) } \dots \mu_{ \sigma '' ( p ) }} = \frac{1}{p ! } 
\sum_{ \sigma '' } X_{ \mu_1 \mu_{ \sigma '' ( 2) } \mu_{ \sigma'' ( 3 ) } \dots \mu_{ \sigma '' ( p ) }} 
\] But this just removes the effect of an antisymmetric tensor! Hence, given a set of indices, we have
\[
[ [ \mu_1 \mu_2 \dots \mu_{ p } ] ] = [ \mu_1 \mu_2 \dots \mu_{p } ] 
\] So, the nested indices have no effect. 
Thus we have that 
\[
d ( d \omega  )_{ \mu_1 \dots \mu_{ p + 2 }  } = ( p + 2) ( p + 1 ) \partial_{ [ \mu_1  } \partial_{ [ \mu_2  } \omega _{ \mu_3 \dots \mu_{ p + 2  } ] ] }  = ( p + 2 )( p + 1) \partial _{ [ \mu_1 } \partial_{ \mu_2  } \omega _{ \mu_3 \dots \mu_{ p + 2 }  ] } = 0
\] By antisymmetry of mixed partial derivatives. 


Now, we'd like to show a 'product rule for the exterior  
derivatives and one forms. 
\[
d (\omega  \wedge  \epsilon )  = d\omega  \wedge  + ( -1) ^{ p } \omega  \wedge  d \epsilon  
\] The right hand side in components, 
by definition is 
\[
	d (  \omega  \wedge  \epsilon)_{\gamma \mu_1 \dots \mu_{p } \nu_1 \dots \nu_{ q } } = \frac{ ( p + q + 1 ) ( p + q ) ! 	 }{ p ! q ! } \left( \partial_{ [ \gamma } \omega _{ \mu_1 \dots \mu_{ p } } \epsilon_{ \nu_1 \dots \nu_{ q } ] }  + \omega _{ [ \gamma \mu_1 \dots \mu_{ p - 1 } } \partial_{ \mu_{ p  }  } \epsilon_{ \nu_1 \dots \nu_{ q   } ]  } \right)   
\] Let's see what we've done here. 
We used the product rule to expand out the derivatives, 
but when doing this we need to preserve our order of 
our indices, which is why we kept it in this form. 

Now, we reorder the indices $ \gamma, \mu_1 , \dots \mu_{ p} $. 
We do the procedure 
\[
( \gamma, \mu_1, \mu_2 , \dots , \mu_{p } ) \to ( -1 ) ( \mu_1 , \gamma , \mu_2 , \dots \mu_{ p } ) \to \dots \to ( -1)^{ p } ( \mu_1 , \mu_2 \dots \mu_{ p }, \gamma ) 
\] 
So, we've picked up a factor of $ ( - 1)^{ p }$. 
Thus, when we stick in an extra set of antisymmetric 
indices (which doesn't change things as we showed earlier), 
we get that our expression above is equal to  
\[
\frac{ ( p + q + 1 )! }{ p ! q !  } \partial_{ [ \gamma  } \omega  _{ \mu_1 , \dots \mu_{ p }   } \epsilon_{ \nu_1 \dots \nu_{ q } ] } + ( -1)^{ p } \frac{ ( p + q + 1 )! }{ p ! q !  } \omega _{ [ \mu_1 , \dots \mu_{ p } } \partial_{  \gamma  } \epsilon_{ \nu_{1 }, \dots \nu_{q } ]} 
\] Now, we substitute our expression for an 
exterior derivative. 
The above expression is equal to 
\begin{align*}
 & = \frac{ ( p + q + 1 )! }{ p ! q ! } \frac{1}{ ( p + 1 ) } d\omega _{  [ \gamma \mu_1 \dots \mu_{ p } } \epsilon_{ \nu_1 \dots \nu_{ q } ] } +  ( - 1) ^{ p }\frac{ ( p + q + 1 ) !}{ p ! q ! } \frac{1}{ ( p + 1 ) ! } \omega _{ [ \mu_1 \dots \mu _{ p } } d \epsilon_{ \nu 1 \dots \nu _{ q } ]    } \\
 &= \frac{ ( p + q + 1 )! }{ ( p  + 1 ) ! q ! } ( d \omega)_{ [ \gamma \mu_1 \dots \mu_{ p } } \epsilon_{ \nu_1 \dots \nu_{ q  } ]  }  +  ( - 1)^ p \frac{ ( p + q + 1 ) ! }{ p ! ( q + 1 ) ! } \mu_{ [ \mu_1 \dots \mu _{ p }  } d \epsilon_{ \gamma \nu_1 \dots \nu_{ q } ] }
\end{align*} 
But these are explicitly the components of what we 
are looking for. Hence, we've shown that 
\[
d ( \omega  \wedge  \epsilon )  = d \omega \wedge  \epsilon + ( - 1)^ p  \omega  \wedge  d \epsilon 
\] 
Finally, we wish to show that a pull back of the 
one form $ \omega  $ , denoted as $\psi^* \omega $ from 
the manifold $ M $ to $ N $ commutes with our exterior derivative. 
In other words, we wish to show that 
\[
d ( \psi^* \omega  ) = \psi^ * ( d \omega)   
\] We do this by expanding the components explicitly  first. We have 
that 
\begin{align*}
d ( \psi ^* \omega  )_{ \nu \mu_1 \dots \mu_{ p } } &= \partial_{ [ \nu }( \psi^* \omega  )_{ \mu_1 \dots \mu_{ p ] }} \\
						    &= \frac{\partial  }{\partial x^{ [ \nu }} \frac{\partial y^{ \alpha 1 } }{\partial x^{ \mu_1 }} \dots \frac{\partial  y^{ \alpha_{ p }}}{\partial  x ^{ \mu_{ p } ] }} \omega _{ \alpha_1 \dots \alpha_{ p }}   
\end{align*}

We were careful here to ensure that, since $ \psi^* \omega  $ lives in 
the manifold $M $, we need to differentiate with respect to the coordinates 
$ x^{ \alpha }$. 
Now, here we used the fact that for a general p-form, our components 
change like
\[
( \psi^* \omega  )_{ \mu_1 \dots \mu_{p }} = \frac{\partial  y^{\alpha_1 } }{\partial x^{ \mu_1 }} \dots \frac{\partial y^{\alpha_{p } } }{\partial  x^{ \mu_{ p }}}  \omega _{ \alpha_1 \dots \alpha_{ p }} 
\] Since we have one differential as $ \frac{ \partial }{ \partial x^{ \nu  }}$,
we can use the chain rule to expand this term out, giving that 
the above expression is equal to 
\[
\frac{\partial  }{\partial  y ^{ \beta   }  } \frac{\partial  y^{ \beta  }  }{\partial x^{ [ \nu  }  }  \frac{\partial  y ^{ \alpha_1   }  }{\partial  x^{ \mu_1   }} \dots  \frac{\partial y^{ \alpha_{ p   } }}{\partial x^{ \mu_{ p } }  } \omega _{ \alpha_1 \dots \alpha_{ p } ]} = \frac{\partial y ^{ \beta   }}{\partial x^{ [ \nu  } } \frac{\partial  y^{ \alpha_1  }}{\partial x^{ \mu_1  } }  \dots \frac{\partial  y ^{ \alpha_{ p  }}}{\partial x^{ \mu_{ p   } ]  }} \frac{\partial }{\partial  y ^{ \beta   }  }  \omega _{ \alpha_1 \dots \alpha_{ p }}    
\] Now, this step deserves some 
explanation. By symmetry of mixed 
partial derivatives, 
we're allowed to commute the  $ \frac{\partial  }{\partial y^{ \beta  } } $ term past everything. Because even though the product rule dictates that this has to differentiate each term in this big product, the first terms are derivatives, so by symmetry of mixed partial derivatives inside an antisymmetric tensor, all these extra terms go to zero. 

Finally, due to our ability to relabel dummy indices, one can show that for a vector contraction of the form 
\[
	X \indices{ ^{ \mu_1 }_{ \nu_1 }} \dots X \indices{ ^{ \mu_{ n  } }_{ \nu_{ n }  }} Y_{ [  \mu_1 \dots \mu_{ n  } ] }  = X \indices{ ^{ \mu_1 }_{ [ \nu _{ 1 }}} \dots X \indices{ ^{ \mu_1 } _{ \mu_{ n } ] } } Y_{ \mu_1 \dots \mu_{ n }}   
\] 
This means that indeed, we can shift the antisymmetric terms to the right most indices, giving 
\[
	d ( \psi ^ * \omega  )_{ \nu \mu_1 \dots \mu_{ p }  } = \frac{\partial  y ^{ \beta } }{\partial x^{ \nu  } } \frac{\partial  y ^{ \alpha_1   }}{\partial x^{ \mu 1 }} \dots \frac{\partial  y ^{ \alpha_p  }}{\partial x^{ \mu_{ p  } }} \frac{\partial  }{\partial y ^{ [\beta   }} \omega _{ \alpha_1 \dots \alpha_{ p } ]}  
\] But indeed, these are the components  
of the pulled back one form 
\[
	\psi^ * d( \omega  ) 
\] So we're done! 


\subsection{Question 9}
This question shows the advantage of coming up with a 
tensorial definition of objects first, to simplify calculations for
components. In the case when $ p = 1 $, we set the basis  $ X_1 = e_\mu$, $ X_2 = e_\nu$. 
Then, our definition in tensorial form gives 
\[
	( d\omega ) _{ \mu \nu } = d\omega( e_\mu, e_\nu ) =  e_\mu ( \omega ( e_\nu ) ) - e_\nu ( \omega ( e_\mu ) ) - \omega ( [ e_\mu, e_\nu ] ) 
\] Now note that $ e_\mu , e_\nu $ aren't indexed components per say, 
they're just our choice of basis vector. 
The upshot of doing this is that by symmetry of mixed partial derivatives, 
we have 
\[
 [ e_\mu , e_ \nu ] = \frac{\partial^ 2 }{\partial x^ \mu \partial x^ \nu } - \frac{\partial ^ 2 }{\partial  x^ \nu x ^ \mu }  = 0  
\] Hence, the last term vanishes and thus 
\[
	( d \omega)_{ \mu \nu } = \partial _ \mu ( \omega_ \nu ) - \partial  _ \nu ( \omega _ \mu ) 
\] In the above line we've used the fact that
$ \omega ( e _ \mu ) = \omega_ \mu $ , which can be shown by expanding $ \omega  $ into 
its components and covector basis. 
Once again, when $ p = 3 $, we can still use this trick of using
 the vector basis  $ \left\{  e_\mu  \right\}$, to forget about the 
 commutator terms in the definition. This is because our definition of $ d \omega $ contains terms like 
  \[
	  \omega ( [ e_\mu , e_ \nu ] , e _ \alpha ) 
  \] but since the commutator vanishes and $ \omega $ is multilinear, $ \omega ( 0 , e_ \alpha ) = 0 $. 
  Thus, the only terms that are preserved from 
  the definition is that 
   \[
   d\omega ( e_\mu , e_ \nu , e_ \rho ) = e _ \mu \omega ( e_ \nu , e _ \rho ) - e _ \nu \omega( e _ \rho , e_ \mu ) + e_ \rho \omega ( e_ \mu , e_ \nu ) 
  \] Using the fact that the basis vectors are derivative terms this becomes
  \[
   ( d \omega)_{ \mu \nu \rho } = \partial _ \mu \omega_{ \nu \rho } - \partial _ \nu \omega _{ \rho \mu } + \partial  _ \rho \omega_{ \mu \nu } 
  \] This is consistent with our definition that 
  \[
	  ( d\omega ) _{ \mu \nu \rho } = 3 \partial  _{ [ \mu } \omega_{ \nu \rho ] } = \frac{1}{2 } \sum_{ \text{anti symmetric perms }} \partial _ \mu \omega_{ \nu \rho }
  \] Note the seemingly extraneous factor of two here, but this cancels our since 
  $ \omega $ is a two form and therefore we count twice the number of permutations. 

\subsection{Question 10} 
For now we'll just show that $ d\sigma _ 1  = \sigma_2 \wedge  \sigma_3$. 
Let's calculate the right hand side explicitly, we have that
\begin{align*}
	\sigma_2 \wedge  \sigma_3 & = ( \cos \psi d \theta + \sin \psi \sin \theta d \phi ) \wedge ( d \psi + \cos \theta d \phi )  \\
	&=  \cos \psi d \theta \wedge  d \psi + \sin \psi \sin \theta d \phi \wedge  d \psi + \cos \psi \cos \theta d \theta d \phi + \cos \psi \cos \theta d \theta \wedge  d \phi \\ 
	& + \sin \psi \sin \theta \cos \theta d \phi \wedge  d \phi \\
	&=  \cos \psi d \theta \wedge  d \psi + \sin \psi \sin \theta d \phi \wedge  d \psi + \cos \psi \cos \theta d \theta d \phi + \cos \psi \cos \theta d \theta \wedge  d \phi 
\end{align*}

When we do an exterior derivative in 3 dimensions on a one form, 
we get that in our wedge product basis
\begin{align*}
	d \omega & = ( \partial_ 1 \omega _ 2 - \partial  _ 2 \omega_1 ) d x^ 1 \wedge  dx^ 2 \\
		 & + ( \partial _ 2 \omega_3 - \partial  _ 3 \omega _ 2 ) dx^ 2 \wedge  dx^ 3  \\
		 & + ( \partial  _ 3 \omega_1 - \partial  _ 1 \omega_3 ) dx ^ 3 \wedge  dx ^ 1
\end{align*}
Now, if we identify $d x^ 1 = d \theta, dx^ 2 = d \psi , dx^ 3 = d \phi $, 
then our first component to calculate is 
 \[
	 (  \partial  _{ \theta }  ( \sigma_1)   _{ \psi } - \partial  _{ \psi } ( \sigma_1 )_{ \theta } d \theta \wedge  d \psi )  = \cos \psi d \theta \wedge  d \psi 
\] Similarly, we find that 
\begin{align*}
	( \partial_2 ( \sigma_{1} ) _ 3 - \partial _ 3 ( \sigma_{ 1 }) _ 2 ) dx^ 2 \wedge  dx^ 3 & = \sin \psi \sin \theta d \phi \wedge  d\psi  \\
	( \partial _ 3 ( \sigma_1 ) _{ 1 } - \partial  _ 1 ( \sigma_ 1 ) _ 3 ) dx^ 3 \wedge  dx ^ 1 &=  \cos \psi \cos \theta d \theta \wedge  d \phi \\
\end{align*}

\subsection{Question 11} 
The point of this question is to show that a basis which 
is coordinate induced is equivalent to it's commutator vanishing. 
Showing one way is straightforward, we have that 
\[
	[ e_\mu, e_\nu ] = \frac{\partial ^ 2 }{\partial x^\nu x^\mu  } - \frac{\partial ^ 2 }{\partial  x^\mu x^\nu} = 0   
\] This is by the symmetry of mixed partial derivatives. 

Now we go the other way. From the condition that 
\[
	[ e_\mu, e_\nu ] = \gamma \indices{^\rho _{\mu \nu }} e_{ \rho } 
\] We expand this out 
\[
	[ e \indices{ _\mu^\rho} \frac{\partial }{\partial x^\rho}, e \indices{_\nu^\lambda} \frac{\partial }{\partial  x^\lambda} ] = \gamma \indices{ ^\rho_{ \mu \nu } } e \indices{ _ \rho ^ \lambda } \frac{\partial }{\partial x ^\lambda}        
 \] Writing this out explicitly and cancelling cross terms
 give 
 \[
  e \indices{ _ \mu ^ \rho } \frac{\partial e \indices{ _ \nu ^ \lambda }  }{\partial x ^ \sigma }  \frac{\partial }{\partial x ^ \lambda }  - e \indices{ _ \nu ^ \lambda } \frac{\partial e \indices{ _ \mu ^ \sigma }  }{\partial x ^ \sigma }  \frac{\partial }{\partial x ^ \sigma } = \gamma \indices{ ^ \sigma _{ \mu \nu } } e \indices{ _\sigma ^ \lambda } \frac{\partial }{\partial x ^ \lambda }      
 \] 
 Upon relabelling dummy indices (for example replacing $ \sigma \to \lambda $ in the second term), we end up 
 with the expression 
   \[
  e \indices{ _ \mu ^ \sigma } \frac{\partial  e \indices{ _ \nu ^ \lambda }  }{\partial  x^ \sigma  }\frac{\partial }{\partial x^ \lambda }  - e \indices{ _ \nu ^ \sigma } \frac{\partial  e \indices{ _ \mu ^ \lambda }  }{\partial x ^ \sigma }  \frac{\partial }{\partial  x^ \lambda } = \gamma \indices{ ^ \sigma _{ \mu \nu } } e \indices{ _ \sigma ^ \lambda } \frac{\partial }{\partial  x^ \lambda }     
\]  But, we can just factor out our partials $ \frac{\partial }{\partial  x^ \lambda } $ to get our required expression.  Now, we appeal to the fact that 
\[
e \indices{ _ \mu ^ \rho } f \indices{ ^ \nu _ \rho } = \delta \indices{ _ \mu ^ \nu }   
\] Differentiating both sides, we have that 
\[
f \indices{ ^ \nu _ \rho } \frac{\partial  e \indices{_ \mu ^ \rho }  }{\partial x^ \gamma }  + e \indices{ _ \mu ^ \rho } \frac{\partial f \indices{ ^ \nu _ \rho }   }{\partial x^ \sigma }  = 0   
\] Hence, 
\begin{align*}
  f \indices{ ^ \nu _ \rho } \frac{\partial e \indices{_ \mu ^ \rho }  }{\partial  x ^ \sigma 	 } &=  e \indices{ _ \mu ^ \rho } \frac{\partial f \indices{ ^ \nu _ \rho }  }{\partial x^ \sigma }   \\
  e \indices{ _ \nu ^ \tau } f \indices{ ^ \nu _ \rho } \frac{\partial  e \indices{_ \mu ^ \rho }  }{\partial x^ \sigma } &=  - e \indices{_ \nu ^ \tau } e \indices{ _ \mu ^ \rho } \frac{\partial f \indices{ ^ \nu _ \rho }  }{\partial x^ \sigma }     \\
  \frac{\partial  e \indices{ _ \mu ^ \tau }  }{\partial x^ \sigma  }  &=  - e \indices{_ \nu ^ \tau  } e \indices{ _ \mu ^ \rho } \frac{\partial  f \indices{ ^ \nu _ \rho }  }{\partial x^ \sigma }    \\ 
\end{align*}
Substituting this into the above, 
\[
 - e \indices{_ \mu ^ \sigma  } e \indices{_ \alpha ^ \lambda  } e \indices{ _ \nu ^ \beta } \frac{\partial f \indices{ ^ \alpha _ \beta  }  }{\partial x^ \sigma }  + e \indices{ _ \nu ^ \sigma } e \indices{ _ \alpha ^ \lambda } e \indices{ _ \mu ^ \beta } \frac{\partial f \indices{ ^ \alpha _ \beta }  }{\partial x^ \sigma }  = \gamma \indices{ ^ \alpha _ \mu _  \nu } e \indices{ _ \alpha ^ \lambda }       
\] We can cancel out the $ e \indices{ _ \alpha ^ \lambda } $. 
We get the 
\[
 - e \indices{ _ \mu ^ \sigma } e \indices{ _ \nu ^ \beta } \frac{\partial f \indices{ ^ \alpha _ \beta  }  }{\partial x^ \sigma }  + e \indices{ _ \nu ^ \alpha } e \indices{ _ \mu ^ \beta } \frac{\partial  f \indices{ ^ \alpha _ \beta }  }{\partial x^ \sigma } = \gamma \indices{ ^ \alpha _ \mu _\nu }      
\] Contraction with $ f \indices{ ^ \mu _ \lambda } f \indices{ ^ \nu  _ \sigma}   $, 
gives the result, 
\[
	\frac{\partial  f \indices{ ^ \rho _ \sigma  }  }{\partial x^ \lambda } - \frac{\partial f \indices{ ^ \rho_ \lambda  }  }{\partial  x^ \sigma }  = - \gamma \indices{ ^ \rho _{ \mu \nu } } f \indices{ ^ \mu _ \lambda } f \indices{ ^ \mu _ \sigma }    
\] However, this implies that 
each of $ f ^ \mu $ is closed since if we have $ [ e_\mu, e_\nu] = 0 $, 
then $ \gamma  =0 $ for all indices. 
So, we get that  $ d f^\mu = 0 $ for all  $ \mu $ by 
the above formula. 
Hence, the Poincare lemma states that we can write 
 \[
 f \indices{ ^ \mu _ \nu }  = \partial  _ \nu \eta ^ \mu  
\] Hence, $ \eta ^ \mu $ are a set of functions.  Our condition that 
\[
	\delta \indices{ _ \mu ^ \nu } =  e \indices{_ \mu ^ \alpha } f \indices{^ \nu _ \alpha } = e \indices{_ \mu ^ \alpha } \partial  _ \alpha \eta ^ \nu    = e_{ \mu } ( \eta ^ \nu ) 
\]  Hence, relabelling $ \eta ^ \nu = x^ \nu $ gives us a set of coordinates 
given that  $ \eta  ^ \nu $ are independent. 
However, we know this is the case since if we have a linear sum 
\[
 \sum_{ \mu }   \lambda _ \mu \eta ^ \mu = 0 
\] contracting with $ e $ gives each coefficient $ 0$. 
Hence, we have that  $ n  $ of these are linearly independent. Thus, 
the collection $ \eta ^ i $ is a map from  $ \mathcal{ M } \to \mathbb{ R} ^ n$ is
injective, and since we have $ n $ of these maps, they span (by the Steinitz exchange lemma) they also span. Hence, we have a homeomorphism, and thus $ \left\{  \eta ^ i  \right\} $ is a set of coordinates. 
Thus, the corresponding $ e_ \nu $ are a set of coordinate induced basis vectors. 

\pagebreak 

\section{Example Sheet 4}
\subsection{Question 2} 

The linearised Einstein equations are 
\[
 \Box \overline{h _{ \mu \nu } }  = - 16 \pi G T_{ \mu \nu } , \quad 
 \overline{h } _{ \mu \nu }  = h _{ \mu \nu }   - \frac{1}{2 } \eta _{ \mu \nu } h 
\] We choose to solve for $ \overline{ h } _{\mu \nu } $ 
first because it's easier. We then obtain $ h _{ \mu \nu } $ from 
$ \overline{ h } _{ \mu \nu }  $  by just inverting the above relation. 
This inverse is 
\[
 h _{ \mu \nu }  = \overline{ h } _{ \mu \nu } - \frac{1}{2 } \eta _{ \mu \nu } \overline{h } 
\] We are given the following time independent 
stress-energy tensor to solve for, which is 
\[
	T _{ \mu \nu }  = \mu \delta \left( x  \right)  \delta \left( y  \right)  
	\text{diag}\left( 1, 0 , 0 , - 1  \right)  
\] We proceed by solving component by component. 
Since our stress energy tensor is time-independent we 
can reduce the wave operator $ \Box $ to just the Laplacian 
$ \nabla ^ 2 $. For $ \overline{ h } _{ 00 } $, we wish to solve 
\[
	\nabla ^ 2 \overline{ h } _{ 00 }   = - 16 \pi G \mu \delta \left( x  \right)\delta\left( y \right) 
\] For now, let's work in cylindrical polar 
coordinates and isolate to the case where 
our solution for $ \overline{ h } $ and therefore 
$ \overline{ h }  $ relies solely on $ r $. 
We hence rewrite our equation as 
\[
	\nabla ^ 2 \overline{ h } _{ 00 }\left( r  \right)  = - 16 \pi G \mu \delta\left(r = 0)  
\] Now, if you already know the 
Green's function for this kind of problem then 
you can just solve this thing right here. 
In case you don't, we have that the radial part of the Laplacian 
gives us the equation 
\[
	\frac{1}{r } \frac{d }{ dr } \left( r \frac{ d \overline{ h } _{ 0 0 } }{ d r } 		   \right) = - 16 \pi G \mu \delta \left(  r  =  0  \right)  
\]  For the case $ r > 0 $, when the 
right hand side is zero, our solution is 
\[
	\overline{ h }_{ 0 0 }  = A \log \left( \frac{r}{ r_0} \right) 
\] We need 
to fix the constant  $ A $ consistently with the 
delta function contribution on the 
right hand side of the equation. 
Integrating both sides in cylindrical polars, to 
some radius  $ R $, and then taking the limit as  $ R  \to 0 $, 
we have that 
\begin{align*}
	2 \pi A \int _{ 0 } ^ R dr \, r \frac{1}{r } \frac{ d  }{ dr } 
	\left( r \frac{ df }{ dr}  \right) &  =  - 16 \pi G \mu \\
	2 \pi R \frac{ d \log \frac{ R }{ r_ 0} }{ d R }  & =  - 16 \pi G \mu \\
	A & = - 8 G \mu 
\end{align*}
Thus $ \overline{h } _{ 00 }  = - 8 G \mu \log \left( \frac{ r }{ r _ 0 }  \right) $ 
Additionally, we have that $ \overline{ h } _{ 33 }  = 8 G \mu \log \left( \frac{ r }{ r^0 } \right) $. All of the other components of $ \overline{ h} _{ ij } $  
can be set to zero, since this is a valid solution 
in the homogeneous case. We thus have $ \overline{ h }  = 16 G \mu 
\log \left( \frac{r}{ r_0 }  \right) $. Inverting with the formula above, 
to get $ h $, the only non-zero contributions are
$ h_{ 11 }  = h _{ 22 }  =  -  8 \mu G \log \left( \frac{r}{ r_0 } \right) $

Adding on the perturbation, our metric is thus 
\[
	ds ^ 2  = - dt ^ 2 + \left( 1 - \lambda  \right) \left( dx ^ 2 + dy ^ 2  \right)  + dz ^ 2 
\] In polar coordinates, the $ dx ^ 2 + dy ^ 2 $ is just $ dr ^ 2  + r ^ 2 d \phi ^ 2 $. 
This means that our resulting metric is 
\[
	ds ^ 2 =  - dt ^ 2 + dz ^  2 + \left( 1 - \lambda   \right)  \left( dr ^ 2 + r ^ 2 d \phi ^ 2  \right) 
\] Using the substitution $ \left( 1 - \lambda ^ 2   \right) r ^ 2  = \left( 1 - 8 \mu G  \right)  \tilde{r } ^ 2   $, our angular component of the 
metric reads $ \left( 1 - 8 \mu G  \right)  d \tilde{ r } ^ 2    $. 
To first order, we have that 
\begin{align*}
	\tilde{ r } &  = \frac{\left( 1 - \lambda  \right)  ^{ \frac{1}{2 } } }{ \left( 1 - 8 \mu G  \right)  ^{ \frac{1}{2 } }  } r  \\
		    &\simeq   r \left( 1 + 4 \mu G \log r   \right) \left(  1 + 4 \mu G   \right)    \quad \text{to first order in } 8 \mu G  \\ 
		    &=  \left( 1 + 4 \mu G  - 4 \mu G \log r  \right)  r  
\end{align*}  
Differentiating, this 
gives 
\[
	d \tilde{ r }  = \left( 1 - 4 \mu G \log r   \right)   dr 
\]  Squaring, we get that this is 
\[
	d \tilde{ r } ^ 2  = \left( 1 - \frac{\lambda}{2 }  \right)  ^ 2 dr ^ 2 
	\simeq \left(  1 - \lambda  \right)  dr ^ 2  
\] Substituting this in, we get that our metric under this change of coordinates 
is 
\[
	ds ^ 2  = - dt ^  2 + dz ^ 2 + d \tilde{ r } ^ 2 + \left( 1 - 8 \mu G  \right) \tilde{ r } ^ 2 d \phi ^ 2    
\] Scaling the angular coordinate as $ \tilde{ \phi }  = \sqrt{ 1 - 8 \mu G   }  \phi  $, 
this gives our metric as 
\[
 ds ^ 2 =  - dt ^2 + dz ^ 2 + d \tilde{ r } ^ 2 + \tilde{ r } ^2 d \tilde{ \phi } ^ 2    
\] This looks like Minkowski space time after all the first order approximations, 
but our original metric wasn't.
Intuitively, to get 
a double image, our geodesics follow a path as follows.
\hspace*{10cm}
\begin{figure}[htpb]
	\centering
\input{ex2q2.pdf_tex}
\caption{Double image appearing}%
\label{fig:x2q2}
\end{figure}

\pagebreak 

\subsection{Question 3} 
To first approximation, since our rotating sphere is moving 
slowly, our energy-momentum tensor 
is $ T^{ \mu \nu } \sim \rho u ^{ \mu } u ^{ \nu } $, 
where we take all contributions of $ O \left(  \Omega ^ 2  \right)  $ 
as approximately zero. 
\[
T_{ \mu \nu } \simeq \rho 
\begin{pmatrix}  1 & - \Omega y & \Omega x & 0 
\\ - \Omega y & 0 & 0 & 0 
\\ \Omega x & 0 & 0 & 0 \\
0 & 0 & 0 & 0 \end{pmatrix} 
\] 
\subsubsection{First part}
We first solve for our $ \overline{ h } _{ 00 } $ component, 
which is the solution to the equation 
\[
\nabla ^ 2 \overline{ h } _{ 00 }  = - \frac{ 4 M }{ R ^ 2 } \delta \left( r - R  \right) 
\] We switch to spherical polar coordinates and 
assume axisymmetric solutions, and assume that 
$ \overline{ h } _{ 00 }  = \overline{ h } _{ 00 } \left( r  \right)  $. 
In spherical polar coordinates, 
the form of the Laplacian means that 
we have to solve
\[
\frac{1}{ r^ 2  } \frac{\partial }{\partial r }  \left( r ^ 2 
\frac{\partial  \overline{ h } _{ 00  } }{\partial r  }  \right) 
=  - \frac{4 M }{ R ^ 2 } \delta \left( r - R  \right) 
\] 
In the regions $ r < R $ and $ r > R $, we 
want to solve 
\[
\frac{1}{ r ^ 2 } \frac{\partial  }{\partial  r ^ 2 }  \left( 
r ^ 2 \frac{\partial  \overline{ h } _{ 00 } }{\partial   r }  \right) = 0  
\] To ensure we don't have a singular solution at the origin
and that the solution decays at infinity, this means that 
\[
\overline{ h} _{ 00 }  = \begin{cases}
\frac{C}{ R } & r < R  \\ 
\frac{C}{r} & r > R 
\end{cases}
\] where $ C $ is a constant to be determined. 
To determine this constant, we integrate the 
above equation between the bounds $ R + \epsilon $ and $ R  - \epsilon $ 
for small $ \epsilon $. In other words, 
we wish to calculate 
\[
4 \pi \int_{ R - \epsilon } ^{  R + \epsilon } 
dr \, r ^ 2 \left[ \frac{1}{ r ^ 2 } \frac{\partial  }{\partial  r }  
\left( r ^ 2 \frac{\partial  \overline{ h } _{ 00 } }{\partial  r }   \right)  \right]   = - \frac{4 M }{ R ^ 2 } \left( 4 \pi  \right)  
\int_{ R - \epsilon } ^{ R + \epsilon }  dr \, r ^ 2 \delta \left( r - R  \right)  
\] Hence we have that 
\[
\left[  R ^ 2 \frac{\partial  \overline{ h } _{ 00 } }{\partial  r }   \right] ^{ R ^ + } _{ R _  -  }   =  - 4 M \implies C = 4 M  
\] since $ \overline{ h } _{ 00 } $ is constant taking the 
limit from below. 
Hence
\[
\overline{ h } _{ 00 }  = \begin{cases}
\frac{4M}{R } & r < R \\
\frac{4M}{r } & r > R 
\end{cases}
\] We invert this to find $ h_{ 00}  $. Since $ \overline{ h } _{ 00}  $ 
is the only diagonal component of $ \overline{ h } _{ \mu \nu } $, 
we have that $ \overline{ h }  =  - \overline{ h } _{ 00 } $ to first order. 
This means that our solution for $ h_{ 00 } $ is given by 
\[
h _{ 00 }  = \overline{ h } _{ 00 }  - \frac{1}{2 } \eta _{ 00 } \overline{ h } _{ 00 } = \begin{cases}
\frac{2M}{R  }  & r < R \\
\frac{2M}{ r } & r > R 
\end{cases} 
\] Our metric is thus 
\[
ds^ 2 =  - \left( 1  - \frac{2M}{ r }  \right) dt ^ 2  
+ \left( 1 + \frac{2M}{r }  \right)  \left( dr ^ 2 + r ^ 2 d \phi ^ 2  \right)  
\] This is the form of Newtonian gravity!

\subsubsection{Second Part} 
We proceed with the same strategy we used before. 
Making use of the spherical Laplcian and 
the ansatz that $ H = f ( r ) \sin \theta e ^{ i \phi } $, 
we want to solve the following equation 
in the regions $ r  > R $ and $ r < R $: 
Substituting in this ansatz with the Laplacian in spherical 
coordinates, we have that we want to solve the following 
equation in the two regions (with the correct boundary conditions)
\[
\sin ^ 2 \theta \left( \frac{\partial  }{\partial  r }  \left(  r ^ 2 \frac{\partial  f }{\partial  r }   \right) \right) e ^{ i \phi  }  - e ^{ i \theta } \sin \theta f + e ^{ i \phi } f \sin \theta 
\left( \cos ^ 2  - \sin ^ 2 \theta  \right)  = 0 
\] 
We make the appropriate cancellations 
and make use of standard trigonometric 
identities 
\begin{align*} 
\sin ^ 2 \theta \left(  \frac{\partial  }{\partial  r }  \left(  r ^ 2 \frac{\partial  f }{\partial  r }   \right)   \right)   - f + f \left(  \cos ^ 2 \theta  - \sin ^ 2 \theta  \right)   &= 0 \\ 
\frac{\partial  }{\partial  r  }  \left( r ^ 2 \frac{\partial  f }{\partial  r }   \right)   - 2f  & = 0 
\end{align*} 
We then make the ansatz that $ f = A r ^{ \alpha } $ for some power exponent 
$ \alpha $. The solutions substituting in this ansatz, we 
find that the two solutions are $ \alpha = 1, -2 $. Due to 
regularity conditions, these are the exponents for the regions 
$ r < R $ and $ r > R $ respectively. Imposing that 
the function is continuous at  $ r  = R $, 
we find that 
\[
f\left(   r  \right)  = \begin{cases}
A r & r < R \\ 
\frac{ A R ^ 3}{ r ^ 2  } & r > R 
\end{cases}
\] Now all that's left to do is to determine the constant $ A $. 
We do this by integrating the equation over 
the troublesome coordinate over a small 
volume around radius $ R $. We only care about the 
radial part of the Laplacian since 
continuity takes every other term to zero. 
Thus, we integrate over 
\[
\sin \theta e ^{ i \phi } \frac{\partial  }{\partial  r }  \left( r ^ 2 
\frac{\partial  f }{\partial  r }  \right)   =  e^{ i \phi } \sin \theta 
\frac{ 4 \Omega }{ R ^ 2 } M \delta \left( r - R  \right)   
\]   We cancel off the functions of $ \phi $ and $ \theta $
, and remembering to include the measure $ dr \, r ^ 2 $, 
integrating the radial coordinate 
\[
\left[  R ^ 2 \frac{\partial  f }{\partial  R }   \right] ^{ R _ + }_{ R_{ - } }  =  - i 4 \Omega R M 
\] Now, the crucial observation 
here is that we substitute in the limits from 
above and below according to the \textbf{different} forms 
of $ f \left( r  \right)  $. This means that 
\[
A R ^ 2 \left[  1 - \left(  - 2 \right)   \right]  = - i 4 \Omega R M 
\]  Thus, we have that 
$ A =  - \frac{4}{3 }  \Omega R M $.
Substituting in our value of $ A $ and comparing the real and imaginary parts, we recover
our formula for $ \overline{ h }_{ 0i  }  = h _{ 0i } $   
as 
\[
h_{ 0i }  = \begin{cases}
\omega  \left(  y , -x, 0  \right)  & r < R \\
\frac{\omega  R ^ 3 }{ r ^ 3 }  \left( y , -x , 0  \right)  & r > R 
\end{cases}
\] 
This means that our new metric is 
\[
ds ^ 2  = \left(  1 - \frac{2M}{ R }  \right)  dt ^ 2 + \left(  1 + \frac{2M}{ R }  \right)  
\left( dx ^ 2 + dy ^ 2 + dz ^ 2  \right)  + 2 \omega  y dx dt - 2 \omega x dy dt 
\] Our corresponding Lagrangian is 
\[
L =  - \left(  1 - \frac{2M}{r }  \right) \dot{  t } ^ 2 +  \left(  1 + \frac{2M}{r }  \right) 
\left( \dot{ x } ^ 2 + \dot{ y } ^ 2 + \dot{ z } ^ 2     \right)  + 
2 \omega y \dot{ x } \dot{ t } - 2 \omega x \dot{ y } \dot{ t }     
\] Let's look at what our geodesic equation for 
$ x $ is. Making use of the Euler-Lagrange equations, 
we have that for the $ x $ coordinate 
\[
\ddot{x} \left(  1  +\frac{2M}{ r }  \right)  + \dot{ x } \dot{ r } (  - \frac{2M}{r ^ 2 } ) + 
\omega  \dot{ y } \dot{ t } + \omega y \ddot{t}  = -  \omega \dot{ y } \dot{ t }    
\]  Now, to first order we have that $ \dot{ t } \simeq 1  $. The second term 
on the left hand side above is second order in $ | \dot{ x } |   $ and $ \omega y  \ddot{ t  }$ 
is as well. 
So, we're left with the equation 
\[
\ddot{x} (  1 + \frac{2M}{r } ) +  =  - 2 \omega \dot{ y }   
\]  where this time we're differentiating with respect to $ t $ 
since it's the same as differentiating 
with respect to $ \lambda $ to first order. 
Similarly, we have that \[ 
\ddot{y} (  1 + \frac{2M}{r } ) +  =  2 \omega \dot{ x}   
\] Taking the $ \left( \frac{2M}{ r }  \right)  $ perturbation as 
small to the other side, we recover the Coriolis force.

\pagebreak
\subsection{Question 4}
Our second order contribution from 
our Christoffel component is 
\[
\Gamma ^ \mu _{ \nu \rho }  = - \frac{1}{2 } h ^{ \mu \sigma } ( h_{ \sigma \nu , \rho } 
+ h _{ \sigma \rho , \nu }  - h _{ \nu \rho , \sigma } 	 )
\] Our Ricci tensor is given by 
\[
R_{ \mu \nu } ^{ \left(  2  \right)  } \left[  h  \right]   = 
\partial  _ \rho \Gamma ^{ \rho } _{ \nu \mu }  - \partial  _ \nu \Gamma ^{ \rho } _{ \rho \mu } 
+ \Gamma ^{ \alpha } _{ \mu \nu } \Gamma ^{ \rho } _{ \alpha \rho }  - 
\Gamma ^{ \alpha } _{ \mu \rho } \Gamma ^{ \rho } _{ \alpha \nu } 
\] To keep things simple, we look at contributions 
by type, and see if they match with what we're given. 
Specifically, terms of the schematic $ h \partial  \partial  h $ 
come from the $ \partial  \Gamma  - \partial  \Gamma $ term. 
After working though the algebra, this term is given by 
\[
R_{ \mu \nu } ^{ \left(  2  \right)  } \left[  h \partial  \partial  h  \right]  
=  - \frac{1}{2 } h ^{ \rho \sigma } h_{ \sigma \nu , \mu \rho } + \frac{1}{2 } 
h ^{ \rho \sigma } h _{ \nu \mu , \rho \sigma  } + 
\frac{1}{2 } h ^{ \sigma \rho } h _{ \rho \sigma, \mu \nu } - \frac{1}{2 } 
h _{ \rho \mu , \sigma \nu } 
\] This agrees with the form shown.
Next, we need to calculate the only other type of 
term there is in the expansion which is of the form $ \partial  h \partial  h $ 
schematically. 
The easiest way it seems to go about doing this 
is by first calculating the term $ \partial  _ \rho \Gamma ^ \rho_{ \nu \mu }  + \Gamma ^{ \alpha } _{ \mu \nu } \Gamma ^{ \rho } _{ \alpha \rho }$ and then proceed to 
anti-symmetrise over the indices $ \rho , \nu $ on the bottom.

The contribution 
comes from 
\begin{align*}
\partial  _ \rho \Gamma ^ \rho_{ \nu \mu }  + \Gamma ^{ \alpha } _{ \mu \nu } \Gamma ^{ \rho } _{ \alpha \rho } &=   - \frac{1}{2 } h ^{ \rho \sigma } _{ , \rho } \left(  h _{ \sigma \nu , \mu } 
+ h _{ \sigma \mu , \nu }  - h _{ \nu \mu , \sigma } \right)  + 
\frac{1}{4 } \eta ^{ \alpha \sigma } \eta ^{ \rho \beta } \left(  
h_{ \mu \sigma , \nu } + h _{ \nu \sigma , \mu }  - h _{ \mu \nu , \sigma } \right)  \left( 
h_{ \alpha \beta, \rho } + h _{ \rho \beta , \alpha }  - h _{ \alpha \rho , \beta }  \right)  
\end{align*}
This term here is quite gnarly to deal with. 
We get three terms from the first term in the sum, 
and a further nine terms from the two first order $ h $ brackets multiplied together. 

\subsubsection{Deriving the Linearised Einstein Hilbert action} 
Recall that in the Lagrangian formulation 
our Einstein-Hilbert action can be written as 
\[
S_{ EH  } = \int d ^ 4 x \, \sqrt{  - g }  R 
\] In this question, 
we have to be slightly careful. 
We need to include terms of both order $ O \left(  h    \right) $ 
and order $ O \left( h ^ 2  \right)  $ in our $ R _{ \mu \nu } $ term, 
since $ g ^{ \mu \nu } = \eta ^{ \mu \nu }  - h ^{ \mu \nu } $. 
We first have to find out what $ \sqrt{  - g}  $ is.
To first order, using the fact that the derivative 
of the determinant of a matrix is the trace, we have that 
\[
g =   - 1  - h \implies \left( -1 - h  \right)^{ \frac{1}{2 } }  \simeq - 1  - \frac{h}{2 } 
\] So the total term 
we have to consider is 
\[
S_{ EH }  = \int d ^ 4 x \left( 1 - \frac{h}{2 } \right) 
\left( \eta^{ \mu \nu }  - h ^{ \mu \nu }  \right) R _{ \mu \nu } \] 
Our first order contribution to the 
Ricci tensor is 
\[
	R_{ \mu \nu } ^{ \left(  1  \right)  } \left[  h  \right]   = 
	\partial  ^ \rho \partial  _{ ( \mu } h _{ \nu  ) \rho 	  }  - 
	\frac{1}{2 } \partial  ^ \rho 	\partial  _ \rho h _{ \mu \nu } 
	 - \frac{1}{2 } \partial  _ \mu \partial  _ \nu h
\] To make things clearer, we go term by term.
The first term that's easiest to check out 
is the $  \frac{1}{4 }\partial  _ \rho h \partial  ^ \rho h  $ 
term in the integrand. We can 
selectively pick out terms from each bracket 
which will get us what we want. This is 
a technique we can try. 

\end{document}
