\pagebreak
\section{The Einstein Equations} 
Space time is a manifold $\mathcal{ M } $ equipped 
with a Lorentzian metric $ g$. We view 
our metric as some matrix which varies from 
point to point on the manifold. The right way 
to write things down here is to write down an 
action principle, then vary the actions to get our 
equations of motion. 

\subsubsection{The Einstein-Hilbert Action} 
The dynamics is governed by the Einstein-Hilbert 
action. 
There's not very much we can write here. 
The only object we have to play with is our metric  $ g $, 
and from this we have our natural volume form we can use. 
The only one we have is the square root of our determinant 
of our metric. Our action must precede 
\[
 S = \int d^ 4 \sqrt{ - g }  
\] Now, what scalar function can we put here? 
The only thing we can do is to pull out our Ricci scalar, 
so that 
\[
 S = \int d^ 4 x \, \sqrt{  - g }  R 
\] Note that schematically, $ R \sim \partial  \Gamma + \Gamma  \Gamma $, 
and $ \Gamma \sim g ^{ - 1} \partial  g $. 
Thus, $ R $ is a function of two derivatives 
of our metric. Note, this 
means that the only connection we 
could cook up is the Levi-Civita connection. This is the simplest thing we can do!
Now, to derive the equations of motion, we need to vary this field. 
We take the metric and push it wish a small change 
\[
 g _{\mu \nu  } \to g _{ \mu \nu } + \delta g _{ \mu \nu }
\] Now, from this we can also deduce how 
the inverse metric changes with this infinitesimal change. 
\[
 g_{ \rho \mu } g ^{ \mu \nu } = \delta \indices{ ^ \nu _ \rho } \implies 
 \delta g_{ \rho \mu } g ^{ \mu \nu } + g _{ \rho \mu  } \delta g ^{ \mu \nu } = 0 
\]  This means 
we can read off our change $ \delta g ^{ \mu \nu }  $ as 
\[
 \delta g ^{ \mu \nu }  =  - g ^{ \mu \rho } g ^{ \nu \sigma } \delta g_{ \rho \sigma}
\] When we vary the whole thing, 
the trick is to write out the Ricci 
scalar in terms of the Ricci tensor 
contracted with the metric. Then, we 
apply the product rule when looking at a small variation.
When we do the small variation, we have to remember 
to hit the colume form as well. 
\[
 \delta S  = \int d ^4  x \, 
 \left[  \delta ( \sqrt{  - g } ) g ^{ \mu \nu } R _{ \mu \nu } + 
 \sqrt{ - g }  \left(  \delta g ^{ \mu \nu } R_{ \mu \nu } + g^{ \mu \nu } \delta R _{ \mu \nu } \right) \right] 
\] We need to figure out 
what $ \delta \sqrt{ - g }   $  is. To do this, 
we prove a claim that 
\begin{claim}
	We claim that 
	\[
	 \delta \sqrt{ - g }   = - \frac{1}{2 } \sqrt{ - g }  g _{ \mu \nu } \delta g ^{ \mu \nu }
	\] 
\begin{proof}
	We use the fact that $ \log \det A = \tr \log A $, which implies 
	that 
	 \[
		 \frac{1}{\det  A } \delta ( \det  A)  = \tr  = \tr ( \delta \log A )  = \tr ( 
		 A^{ - 1} \delta A ) = \tr ( A ^{ - 1 } \delta A ) 
	 \] Now, susbtituting in our metric $ g $ as 
	 the matrix $ A $ in our identity above, this finally implies that 
	 \begin{align*}
		 \delta \sqrt{ - g} &  = \frac{1}{2 } \frac{1}{\sqrt{ - g}  } ( - g ) 
		 g ^{\mu \nu } \delta g_{ \mu \nu } \\
		 &=  \frac{1}{2 } \sqrt{ -g }  g ^{ \mu \nu } \delta g _{ \mu \nu } 
	 \end{align*} 
\end{proof}
\end{claim}

Now, we compute the variation of our Ricci tensor. 
\begin{claim}
	We have that our variation is 
	\[
	 \delta R_{ \mu \nu } = \nabla _ \rho \delta \Gamma ^ \rho _{ \mu \nu } - \nabla_{ \nu  }
	 \delta \Gamma ^ \rho _{ \mu \rho }
	\]  with 
	\[
	 \delta \Gamma ^ \rho _{ \mu \nu } = \frac{1}{2 } 
	 g ^{ \rho \sigma } ( \nabla _ \mu \delta g _{ \sigma \nu } + 
	 \nabla _ \nu \delta g _{ \sigma \mu } - \nabla _ \sigma \delta g_{ \mu \nu } ) 
	\] 
\begin{proof}
	First note that importantly, $ \Gamma ^ \mu _{ \rho \nu } $ is not a tensor. 
	However, the difference $ \delta \Gamma ^ \mu _{ \rho \nu } $ 
	is a tensor since its the difference between two connections. This means that it is a well defined action 
	to take the covariant derivative of the objects
	In normal coordinates, at some point 
	\begin{align*}
		\delta \Gamma^ \rho _{ \mu \nu } &= \frac{1}{2 }  g ^{ \rho  \sigma }\left( 
		\partial  _ \mu \delta g _{ \sigma \nu } + \partial  _ \nu 
	g _{ \sigma \mu } - \partial  _ \sigma \delta g _{ \mu \nu } \right)   \\
						 &=  \frac{1}{2 } g ^{ \rho\sigma } \left(  
						 \nabla _ \mu \delta g _{ \sigma \nu } 
					 + \nabla _ \nu \delta g _{ \sigma \mu } 
				  - \nabla _{ \sigma } \delta g _{ \mu \nu }\right)  
	\end{align*}
	Since we're in normal coordinates, we can now replace our partial 
	derivatives $ \partial $ with covariant derivatives $ \nabla $. 
	This gives us a tensorial relation, and hence this 
	holds true in all coordinate frames. We can play the same 
	game with our Riemann tensor, 
	\[
	 R \indices{ ^ \sigma _{ \rho \mu \nu } }  = \partial  _ \mu \Gamma ^ \sigma _{ \nu \rho } 
	  - \partial  _ \nu \Gamma ^ \sigma _{ \mu \rho }
	\]  This implies that our small change 
	is given by 
	\begin{align*}
		\delta R \indices{ ^ \sigma _{ \rho \mu \nu } } &=  \partial  _ \mu  
		\delta \Gamma ^ \sigma _{ \mu \rho }  - \partial  _ \nu \delta \Gamma 
		^ \sigma _{ \mu \rho }\\
		&=  \nabla _ \mu \delta \Gamma ^ \sigma _{ \nu \rho } - 
		\nabla _ \nu \delta \Gamma ^ \sigma _{ \mu \rho } \\
	\end{align*} Since $ \Gamma  =0 $ in normal coordinates.  
\end{proof}
\end{claim}
\begin{thm}{(Vacuum Einstein Equations)}

Now we have that, after all the dust has settled, 
that 
\[
 \delta S = \int d ^ 4 x \, \sqrt{ - g }  \left[  
 R _{ \mu \nu }  - \frac{1}{2 } R g _{ \mu \nu } \right] \delta g ^{ \mu \nu } + \nabla _ \mu X ^ \mu  
\] Where we've written 
\[
 X ^ \mu = g ^{ \rho \nu } \delta \Gamma ^ \mu _{ \rho \nu }  - g ^{ \mu \nu } \delta \Gamma ^ \rho _{ \nu \rho }
\] 
When we impose the condition of stationarity for some 
arbitrary change in the metric, we then have a condition for our integrand. 
This yields the vacuum Einstein equations. 
\[
 \delta S = 0 , \forall g ^{ \mu \nu } \implies 
 G _{ \mu \nu } = R _{ \mu \nu } - \frac{1}{2 } R g _{ \mu \nu } =0 
\] Multiplying this quantity by $ g ^{ \mu \nu } $, we have that 
$ R = 0 $, which implies the vacuum Einstein equations 
\[
 R_{ \mu \nu } = 0 
\]
	
\end{thm}


\subsubsection{Dimensional analysis} 
We can use dimensional analysis to fill out the 
constants here. Our action $ S $ has dimension 
$ M L ^{ 2 } T ^{ - 1}  $, Our metric is dimensionless, and 
$ R $ has units of $ L ^{ - 2} $. This implies that 
our full action is 
\[
 S = \frac{ c ^ 3  }{  16 \pi G} \int d ^ 4 x \, \sqrt{ -g }  R
\]  Our Planck mass
is $ \mathcal{ M } _{ pl } ^ 2  = \frac{ \planck c }{8 \pi G }$ and 
this is approximately $ \mathcal{ M } _{ pl } \propto 10 ^{ 18 } G eV $. 
Work with units with $ c = 1 $, and $ \planck = 1 $. Throughout 
the next of the section, we shall 
work in terms of this Planck mass. 

\subsubsection{The Cosmological Constant} 
We could add a further term to the action 
 \[
	 S = \frac{1}{2 } \mathcal{ M  }_{ pl } \int d ^ 4 x \, \sqrt{ - g}  ( R - 2 \Lambda ) 
\] We will study these. 
We have the equation 
\[
 R_{ \mu \nu } - \frac{1}{2 } R g _{ \mu \nu } = - \lambda g _{ \mu \nu } \implies 
 R = 4 \Lambda \implies R_{ \mu \nu }  = \Lambda g _{ \mu \nu }
\]

\subsection{Diffeomorphisms Revisited} 
Our metric has $ \frac{1}{2 } 4 \cdot  5  = 10 $ components. 
But two metrics related by $ x ^ \mu \to \tilde{x } ^ \mu ( x)  $ 
are physically equivalent. 
This means that actually we only have $ 10 - 4  = 6 $ degrees of freedom.
The change of coordinates can be viewed as a diffeomorphism 
\[
 \phi : \mathcal{ M } \to \mathcal{ M }
\] such 'diffeos' are the 'gauge symmetry' of general relativity.
Consider diffeomorphisms which take the form 
\[
	x ^ \mu \to \tilde{ x } ^ \mu ( x) = x ^ \mu + \delta x ^ \mu  
\] Moreover, we can 
view this as a diffeomorphism generated by a vector field 
$ X ^ \mu  = \delta x ^ \mu $. The metric 
transforms as $ g_{ \mu \nu } ( x) \to \tilde{ g } _{ \mu \nu } ( \tilde{ x  })   $
with $ \tilde{ g } _{ \mu \nu } ( \tilde{ x  } )  = \frac{\partial \tilde{ x } ^ \rho  }{\partial x ^ \mu }
\frac{\partial \tilde{ x } ^ \sigma  }{\partial  x ^ \nu }  g _{ \rho \sigma }  ( \tilde{ x  } )  $. 
From our definition that these coordinates are generated by a vector field, 
this is equal to 
\begin{align*}
	\dots &  = ( \delta \indices{ _ \mu ^ \rho } + \partial  _ \mu X ^ \rho ) 
	( \delta \indices{ _ \nu ^ \sigma  } + \partial  _ \nu X ^ \sigma ) 
	( g_{ \rho \sigma } ( x ) + X ^ \lambda \partial  _ \lambda  g_{ \rho \sigma } ( x) ) \\
	      &=  g_{ \mu \nu } ( x) + X ^ \lambda \partial  _ \lambda g _{ \mu \nu } ( x) 
	      + g _{ \mu \rho } ( x ) \partial  _ \nu X ^ \rho + 
	      g _{ \nu \rho } \partial  _ \mu X ^ \rho \\
\end{align*}  
This means that our infinitesimal change is 
\begin{align*}
	\delta g _{ \mu \nu }  & = X ^ \lambda \partial  _ \lambda g _{ \mu \nu } + 
	g _{ \mu \rho } \partial  _ \nu X ^ \rho + g _{ \nu \rho } \partial  _ \mu X ^ \rho \\
			       &= (  \mathcal{ L  } _ X g)_{ \mu \nu }  \\
\end{align*}
Alternatively, we can write this as 
\begin{align*}
	\delta g_{ \mu \nu } &=  \partial  _ \mu X _ \nu + \partial  _ \nu X _ \mu 
	+ X ^ \rho \left(  \partial  _ \rho g _{ \mu \nu } - 
	\partial  _ \mu g _{ \rho \nu } - \partial  _ \nu g _{ \mu \rho }\right) \\
	&=  \partial  _ \mu X _ \nu + \partial  _ \nu X _ \mu + 2 X ^ \rho g _{ \rho \sigma } T ^ \sigma _{ 
	\mu \nu } \\
	&=  \nabla _ \mu X _ \nu + \nabla _ \nu X _ \mu  \\
\end{align*}
Now, let's look back at the action
\[
 \sigma S = \int d ^ 4 \sqrt{ -g  }  G ^{ \mu \nu } \delta g _{ \mu \nu }
\] If we restrict to these changes of coordinates, 
the above 
\[
 \dots = 2 \int d ^ 4 x \, \sqrt{ - g }  G ^{ \mu \nu } \nabla _ \mu X _ \nu 
 =0 \forall X  =  -2 \int d ^ 4 x \, \sqrt{ - g }  ( \nabla _ \mu G ^{ \mu \nu } )  X^ \nu 
\]  this is because changing coordinates is a gauge symmetry. 
This means that our Einstein tensor obeys 
\[
 \nabla _ \mu G ^{ \mu \nu }  = 0
\] This means that diffeomorphisms are 
a symmetry implies the Bianchi identity. 
So, not all of the $ G ^{ \mu \nu } $ are independent. 
We have found four conditions, which means that the 
Einstein equations $ G _{ \mu \nu } $ is really only 
6 equations, which is the right number of components 
to determine our metric $ g _{ \mu \nu } $. 

\subsection{Some simple solutions} 
Let's look at what happens when we have a vanishing 
cosmological constant with $ \Lambda  =0 $. 
We need to solve  $ R_{ \mu \nu }  = 0$. 
It's tempting to write $ g _{ \mu \nu } = 0 $, 
but this isn't allowed because the metric requires an inverse! 
This makes gravity different from other field theories, 
we need to have the constraint that non of the eigenvalues are zero. 
This may suggest that the metric is not a fundamental field. There 
are lots of similarities to this an fluid mechanics. 
The simplest solution is Minkowski space time. 
\[
 ds ^ 2 = - d t ^ 2 + d \vec{x} ^ 2 
\] 

\subsubsection{De Sitter Spacetime} 
In the case, $ \Lambda > 0 $, we can look for solutions 
to $ R _{ \mu \nu }  = \Lambda g_{ \mu \nu }$ which have some 
spherical symmetry. We look for solutions of the form 
\[
	d s^ 2 = - f (r ) ^ 2 d t ^ 2 + f ( r ) ^{ - 2} dr ^ 2 + r ^ 2 ( d \theta ^ 2 
	+ \sin^ 2 \theta d \psi ^ 2 ) 
\] 
We can compute our Ricci tensor 
to find 
\[
	R _{ t t }  =  - f^4 R_{ rr }  = f ^  3 \left( f '' + \frac{2 f ' }{ r } + ( f ' ) ^2 f 
		\right), \quad 
	R_{ \psi \psi }  = \sin ^ 2 \theta  R _{ \theta \theta  } = ( 1 - f ^ 2 - 2 f f ' r ) \sin ^ 2 \theta 
\] 
Now, with this ansatz we can solve 
for $ f ( r) $ by setting $ R_{ \mu \nu } = \Lambda g _{ \mu \nu } $, 
and then solving the equation by substituting in 
specific indices. 
By setting $ \mu , \nu = t t , r r $ we get that 
 \[
	 f '' + \frac{2 f ' }{ r } + \frac{ ( f ' ) ^ 2 }{ f }  = - \frac{\Lambda }{ f }
\] 
Similarly, by setting $ \mu,  \nu = \theta \theta , \psi \psi  $, we have 
\[
 1 - 2 f f ' r - f ^ 2 = \Lambda r ^ 2 
\] This equation is solved by 
\[
 f ( r)  = \sqrt{ 1 - \frac{ r ^ 2 }{ R ^ 2 }} , \quad R ^ 2 = \frac{ 3 }{ \Lambda}
\] This means that our resulting, final metric 
that we get takes the form
\[
	ds ^ 2 = -  \left( 1 - \frac{r^ 2 }{R^ 2 } \right)^2 dt ^ 2 + \left( 1 - \frac{r^2}{R^ 2 } 
	\right) ^{ -2 } dr^ 2 
	+  r ^ 2 d \Omega_2 ^ 2
\] To save ink, we've written $ r ^ 2 ( d \theta ^ 2 + \sin ^ 2 \theta d \phi ^ 2 ) $
as $ r ^ 2 d \Omega_2 ^ 2 $, which is the radius squared, multiplied by 
the familiar metric on a unit 2-sphere. In particular, if we have 
a set of three Cartesian coordinates such that 
\[
 x ^ 2 + y ^ 2 + z ^  2 = r
\] then since this parametrises a $ 2 $-sphere of radius $r$, 
our resulting metric in spherical coordinates is 
 \[
 d^ 2 x + d y ^ 2 + d z^ 2 = dr^2 + r ^ 2 d \Omega ^ 2 _ 2 
\] 
This will save a lot of time later in our calculations 
which involve different ways 
to parametrise this metric. 

This is de Sitter space time (or alternatively called the static patch of 
d S ). Note that we have a valid range from $ r \in [ 0 , R ] $, but the 
metric appears to be singular at $ r = R $. Throughout this section, 
we will check whether this singularity comes 
from merely a poor choice of coordinates, or is a genuine
physical space-time singularity. 
Recall, it's totally okay for this to be a coordinate singularity 
(in fact, as we shall see, it is) because 
when we define coordinates,  we only define them 
in terms of coordinate patches. 
We can look at geodesics. We have the action, where $ \sigma $ represents proper time 
\[
 S  = \int d \sigma \left[  
 - f ( r) ^ 2 \dot{t } ^ 2 + f ( r) ^{ - 2 } \dot{ r } ^ 2 
 + r ^ 2 \left(  \dot{ \theta } ^ 2 + \sin ^ 2 \theta \dot{ \phi  }  ^ 2  \right)  \right] 
\] Here, 
there are two conserved quantities. Nothing depends on $ \phi $, 
so we have that 
$ l = \frac{1}{2 } \frac{\partial  L }{ d \dot{\phi }  } = r ^ 2 \sin ^ 2 \theta \dot{\phi  } $. 
The other conserved quantity we get is the energy of the particle with 
\[
	E =  - \frac{1}{2 } \frac{\partial  L }{\partial  \dot{ t } } = f ( r) ^ 2 
	\dot{t }  
\] For a massive particle, 
we also require that the trajectory is timelike.
Since $ \sigma  $ represents proper time, this means 
that the Lagrangian itself is equal to 
\[
 - f ^ 2 \dot{ t } ^ 2 + f ^{ - 2 } \dot{ r } ^ 2 + r ^ 2 
 ( \dot{ \theta } ^ 2 + \sin ^ 2 \theta  \dot{ \phi } ^ 2  ) = - 1  
\]
Look for geodesics with $ \theta = \frac{\pi }{ 2 } $ and $ \dot{ \theta } ^ 2  $. 
This gives 
\[
	\dot{ r } ^ 2 + V_{ eff} ( r)  = E ^ 2  
\] with 
\[
	V_{ \text{eff} }( r) = ( 1 + \frac{l ^ 2 }{ r ^ 2 } ) ( 1 - \frac{ r ^  2 }{R ^ 2  } ) 
\] For $ l  = 0 $, we have that 
\[
	r ( \sigma)  = R \sqrt{ E ^ 2 - 1}  \sinh ( \frac{ \sigma }{ R } ) 
\]  This hits $ r = R $ in finite $ \sigma $. Meanwhile 
\[
	\frac{ d t }{ d \sigma } = E \left(  1 - \frac{ r ^ 2 }{ R ^ 2 }  \right)  ^{ - 1}
\] Solutions to this have $ t \to \infty $ as $ r \to R $. 
To see this, suppose $ r ( \sigma _ * )  = R $ and expand $ \sigma = \sigma_{ * }  - \epsilon $. 
We find that 
\[
 \frac{ d t }{ d \epsilon } \simeq  - \frac{ \alpha }{ \epsilon } 
\] this of course means that $ t \sim  - \alpha \log ( \frac{ \epsilon }{ R } ) $, 
and $ t \to \infty $ as $ `\epsilon \to 0 $. 
so, it takes a particle finite proper time  $ \sigma $ but infinite 
coordinate time $ t $.

In fact, the de Sitter space time can 
be embedded in 5 dimensional Minkwoski space, $ \mathbb{ R} ^{ 1, 4 } $. 
which is 
\[
	d s ^ 2 =  -  ( d X^ 0 ) ^ 2 + \sum_{ i = 1 } ^ 4 ( dX ^ i ) ^ 2 
\]  Our surface which we are embedding is 
\[
	- ( X^ 0 )^ 2 + \sum_{ i = 1 } ^ 4 ( X ^ i ) ^2  = R ^ 2 
\] This can be viewed as a hyperbola in Minkowski space. 
The flat metric is inherited onto the surface. 
We will show that this is true. 
We let $ r ^ 2  = \sum_{ i = 1 } ^ 3 ( X ^ i ) ^ 2 $ and $ X ^ 0 = \sqrt{ R ^ 2 - r ^ 2  } 
\sinh ( \frac{t}{ R } ) $, and also that $ X ^ 4  = \sqrt{ R ^ 2 - r ^ 2 }  \cosh ( \frac{t}{ R } ) $. 
This means that, we have \[
  - ( X ^ 0 ) ^ 2 + ( X ^ 4 ) ^ 2  = R ^ 2 - r ^ 2
\] Now just substitute and plug those in. We can check that 
of you compute $ d X ^ 0  $ and $ d X ^ 4 $, and plug into 
the metric in 5 dimensional Minkowski space, we'll recover 
the de Sitter metric. Note that we above is a really weird 
parametrisation! We only singled out $ X ^ 0$ and $ X ^ 4 $. 
Moreover, $ X ^ 4 $ only runs from $ - \infty $ to $ \infty $, 
but in our parametrisation we only have that 
$ X ^ 4 > 0 $. These coordinates are not particularly symmetric, 
and moreover cover only $ X ^ 4 \geq 0 $ . 
A better choice of coordinates is 
\[
	X ^ 0 = R \sinh \left(  \tau / R  \right) , \text{ and } X ^ i = \cosh \left(  \tau / R  \right)  \cdot   y
	^ i 
\] where we have a normalisation constraint on $ y $, such that $ \sum_{ i   = 1 } ^ 4 ( y ^ i ) ^ 2  = 1$. 
Another small calculation, we substitute this 
into our 5d Minkowski metric. We have that 
\[
 d s^ 2  = - d \tau ^ 2 + R ^ 2 \cosh ^ 2 \left(  \tau / R  \right)  d \Omega_ 3 ^ 2 
\] where $ d \Omega _ 3 ^ 2 $ is a metric on  $  S^ 3 $. 
This is a hard calculation, but we've swept this under the rug 
and put these into the $ 3 $ sphere metric. 
We can see that these are a better 
description of de Sitter space time because they describe the whole space, 
and has no more coordinate singularities. 
These are global coordinates. This metric also solves 
the Einstein equations with the same cosmological constant
since it's just a change of coordinates 
from our original solution. 
So now we have two metric which describes the same space. 
There's a natural cosmological interpretation here. 
We have a 3-sphere, which initially shrinks, then expands. This 
corresponds to a contracting / expanding universe. 

\subsubsection{Anti de-Sitter Space-time} 
The next solution we will look at is for 
$ \Lambda < 0 $. Again, looking for solutions 
\[
 ds^2 = - f ( r) ^ 2 dt ^ 2 + f ( r) ^{ - 2} dr ^ 2 + r ^ 2 d \Omega _ 2 ^ 2 
\] We find that in this case, 
\[
 f ( r) = \sqrt{ 1 + \frac{ r^ 2 }{ R ^ 2 } }  , \quad R ^ 2 = - \frac{ 3 }{ \Lambda}
\] This is called anti-de Sitter space-time (AdS). 
There is no coordinate singularity here! 
This time, massive geodesics obey the rule $ \dot{ r } ^ 2 + V_{ \text{eff} } = E ^ 2  $. 
This potential energy is what it was before but there's 
now a plus sign instead of a minus sign 
\[
	V_{ \text{eff} } ( r) = \left(  1 + \frac{l ^ 2 }{ r ^ 2 }   \right) \left(  
	1 + \frac{ r ^ 2 }{ R ^ 2 }\right) 
\] What's happening is that 
this kind of looks like a gravitational potential well in which we're stuck in. 
In particular, massive particles seemed to be confined to 
the centre of Anti de-Sitter space. The origin 
is special here.

Now let's look at massless particles. 
Massless particles follow null geodesics. This means 
that in the derivation of our geodesics we had a constrant 
which was $- 1 $, but we can change this to zero. 

\[
	- f ^ 2 \dot{ t} ^ 2 + f ^{ - 2 } \dot{ r } ^ 2 + 
	r^ 2 \left(  \dot{ \theta } ^ 2 + \sin ^ 2 \theta \dot{ \phi } ^ 2    \right)  = 0 
\] This means that 
at the point $ \theta = \frac{ \pi }{ 2 } , \dot{ \theta } = 0  $, 
we have that our potential obeys $ \dot{ r } ^ 2 + V_{ eff } ( r) 
 = E^ 2 $, and our null geodesic obeys 
 \[
	 V_{ \text{ null} } = \frac{ l ^ 2 }{ 2 r ^ 2 } \left(  
	 1 + \frac{ r ^ 2 }{ R ^ 2}\right) 
 \] We introduce new coordinates, $ r = R \sinh \rho $. 
 Plugging this into our de-Sitter metric, the sinh is 
 chosen so the radial coordinate factors out. 
 Thus, we get 
  \[
  d s^ 2 = - \cosh ^ 2 \rho d t ^ 2 +  R^ 2  d \rho ^ 2 
  + R ^ 2 \sinh ^ 2 \rho ( d \theta ^ 2 + \sin ^ 2 \theta d \phi ^ 2 ) 
 \] The null geodesic equation is 
 \[
  R \dot{ r }  = \pm \frac{E }{ \cosh \rho } \implies 
  R \sinh \rho  = E ( \sigma - \sigma _{ 0 } ) 
 \]  
 Massless particles hit $ \rho \to \infty$ as $ \delta \to \infty$.
 However, $ E = \cosh^ 2 \rho \dot{ t } \implies R \sinh \rho  = 
 R \tan \left(  t / R  \right)   = E ( \sigma - \sigma _ 0 ) $. 
 SO, $ t \to \frac{ \pi R }{ 2 } $ as $ \sigma \to \infty $. 
 Massless particles reach infinity of AdS in finite coordinate time. 
 AdS can be viewed as a hyperboloid in $ \mathbb{ R} ^{ 2, 3, } $ 
 as \[
	 -( X ^ 0 ) ^ 2 - ( X ^ 4 ) ^ 2 + \sum_{ i = 1 } ^ 3 ( X ^ i ) ^  2 = R^ 2
 \] Now, let
 \begin{align*}
 	X ^ 0 &=  R \cosh \rho \sin \frac{ t }{ R  }  \\ 
	X ^ 4 &=  R \cosh \rho \cos \frac{ t }{R }  \\
	X ^ i &=  R y ^ i \sinh \rho , \quad \sum ( y ^ i ) ^ 2 = 1 \\
 \end{align*} 
 and we hence recover the AdS metric. 
 There is one last set of coordinates which we can 
 examine. 
 \begin{align*}
 	X^ i &=  \frac{ \tilde{ r }  }{ R } x ^ i , \quad i = 0, 1, 2   \\
	X^ 4 - X ^ 3 = \tilde{ r } 
	X^ 4 + X ^ 3  = \frac{ R ^ 2 }{ \tilde{ r } } + \frac{ \tilde{ r }  }{ 
	R ^ 2 } \eta_{ ij }  x^ i x ^ j 
 \end{align*}
 The metric we get out of this is 
 \[
  d s^ 2 = R ^ 2 \frac{d \tilde{ r } ^ 2  }{ \tilde{ r } ^ 2  } 
  + \frac{ \tilde{r } ^ 2  }{R ^ 2  } \eta_{ ij }dx ^ i dx ^ j 
 \]  These don't cover all of AdS. This is called the Poincare patch. 

\subsection{Symmetries}
Let's look at some symmetries of the metric. 
The first thing we'd like to do is explain what a symmetry actually is. 
Think about a 2 dimensional sphere. 
This has the symmetry group $ SO ( 3) $, 
since we can rotate it in any axis. Now think 
of a rugby ball. This has $ SO ( 2) $ symmetry since 
we can only rotate it about one axis. 
The correct way to think about these things is to consider flows. 
Consider a 1-parameter family of diffeomorphisms 
$ \sigma _ t : \mathcal{ M } \to \mathcal{ M } $. 
Recall that this is associated to a vector field 
\[
 K^ \mu =  \frac{dx ^ \mu }{ dt }
\] where $ \frac{ d x ^ \mu }{ dt } $ are tangent to the flow lines. 
We're going to call this a symmetry if we start at any point, 
flow along, then our destination point looks the same.

This flow is an isometry if the metric 
looks the same at each point along the flow, in other words 
\[
 \mathcal{ L }_K g = 0 \iff \nabla _ \mu K _ \nu + \nabla _ \nu K _ \mu  =0 
\] We can show that the second expression 
is equivalent to the first by recalling the expression 
for a Lie derivative, and then working in normal coordinates to 
get the covariant expression out. 
In components, our Lie derivative is 
\[
	( \mathcal{ L } _ X g )_{ \mu \nu }  =  K ^ \alpha \partial  _ \alpha g _{ \mu \nu } 
	 + g_{ \alpha \nu } \partial  _ \mu K ^ \alpha + 
	 g _{ \alpha \mu } \partial  _ \nu K ^ \alpha 
 \] In normal coordinates, our first term 
 disappears since the metric is flat, and we 
 can convert the partial derivatives to covariant derivatives, 
 which gives us the second expression. This equation is 
called the 'Killing equation' and any vector $ K  $ which 
obeys this is 
called the Killing vector. 
This is the equation which we need to solve. 
These objects describe the symmetries of the metric. 

Note, commuting the Lie derivatives is handy because 
\[
	\mathcal{ L } _ X \mathcal{ L } _ Y - \mathcal{ L } _ Y \mathcal{ L } _ X  = \mathcal{ L } _{ [ X, Y ] }
\] You can 
show this easily in the case of the Lie derivative 
acting on either a function or a vector field (the first is trivial, 
for a vector field just apply Jacobi). From this, we start to get a Lie 
algebra structure that emerges for the group 
of continuous symmetries of the metric, (which 
is exactly what we expect from continuous symmetries).

\begin{example}{Minkowski space} 
With Minkwoski space in our metric, 
the Killing equation implies 
\[
 \partial _ \mu k _ \nu + \partial  _ \nu k _ \mu  = 0
\] In full generality, we have that 
\[
	k _{ \mu }  = c_\mu + \omega _{ \mu \nu } x ^ \nu , \omega_{ \mu \nu }  =- \omega  _{  \nu \mu }
\] We have that $ c _ \mu $ represent our translations, 
and $ \omega_{ \mu \nu } $ represent boosts or rotations 
depending on what indices we're choosing. 
We can define Killing vectors 
\[
 P _ \mu  = \frac{\partial   }{ \partial  x ^ \mu } , \text{ and } M_{ \mu \nu } 
  = \eta _{ \mu \rho } x ^ \rho \frac{\partial  }{\partial  x ^ \nu }   - \eta _{ \nu \rho } x ^ \rho \frac{\partial  }{\partial  x ^ \mu } 
\] Now, we find that $ [ P_\mu, P _ \nu  ]  $ , and that 
\[
	[ M_{ \mu \nu }, P _ \sigma ] =  - \eta _{ \mu \sigma } P _ \nu + \eta_{ \sigma \nu } P_{ \mu } 
\] in addition, 
\[
	[ M_{ \mu \nu } , M_{ \rho \sigma  } ] = \eta_{ \mu \sigma } M_{ \nu \rho } + \eta_{ \nu \rho } M_{ \mu \sigma } 
	 - \eta_{ \mu \rho } M_{ \nu \sigma } - \eta_{ \nu \sigma } M_{ \mu \rho }
\] These are the commutation relations of the Poincare group. 	
\end{example}

\subsubsection{More examples} 
The isometries of dS and AdS are inherited from 
the 5 dimensional embedding. 
de-Sitter space time has isometry group $ SO ( 1, 4 ) $, 
and Anti de-Sitter space has isometry group $ SO ( 2, 3 ) $. 
Both groups have dimension  $ 10 $, same as 
$ SO ( 5 ) $, and the same is the Poincare group as well.
Minkowski and de-Sitter space are equally as symmetric.
In 5d, the Killing vectors are 
\[
 M_{ AB }  = \eta_{ AC } X ^ C \frac{\partial  }{\partial  X ^ B } - 
 \eta _{ B C} X ^ C \frac{\partial   }{\partial  X ^ A } 
\] In this case, $ A $ runs from $ A  = 0 , 1, 2, 3, 4$, 
and we have that the metric are given by 
 \begin{align*}
	 \eta &=  ( - , + , + , + , + )  \\
	 \eta &=  (- , -, + , + , + )  \\
 \end{align*} 
 The flows induced by $ M_{AB } $ map the embedding 
 hyperboloid to itself. These are isometries of 
 ( A ) dS. 
 Let's look at an example in de Sitter space in static patch coordinates. 
 If the metric $ g_{ \mu \nu } ( x) $, does not 
 depend on some coordinate $ y $, then $ K = \frac{\partial }{\partial y } $ 
 is a Killing vector since $ \mathcal{ L }_{ \partial  _ y } g   = 
 \frac{ \partial  g_{ \mu \nu } }{ \partial  y }  = 0 $. 
 So, for the static path, we expect $ \frac{\partial  }{\partial  t }  $ 
 to be a Killing vector.

 We had 
 \begin{align*}
	 X ^ 0 &=  \sqrt{ R ^ 2 - r ^ 2 }  \sinh \left(  \frac{t}{R} \right)  \\
	 X ^ 4 &=  \sqrt{ R ^ 2 - r ^ 2 }  \cosh \left(  \frac{t}{R} \right)  \\
 \end{align*}
 Look at 
 \[
  \frac{\partial  }{\partial  t }  =  \frac{\partial X ^ A  }{\partial t }  
  \frac{\partial  }{\partial  X ^ A }   = \frac{1}{R  } \left(  X ^ 4 \frac{\partial  }{\partial  X ^ 0 }
   + X ^ 0 \frac{\partial   }{\partial X ^ 4 } \right) 
 \]  It's interesting to note that 
 timelike Killing vectors such that $ g_{ \mu \nu  }  k ^ \mu k ^ \nu  < 0 $ 
 are used to define energy. Minkowski and AdS have such objects. 
 de-Sitter space has such an object in the static patch, 
 but not globally. 
For example, 
\[
 K = X ^ 4 \frac{\partial  }{\partial  X ^ 0 }  + X ^ 0 \frac{\partial  }{\partial  X ^ 4 } 
\] Now, the first term increases $ X ^ 0 $ then $ X ^ 4 > 0 $, 
and decreases $ X ^ 0 $ when $ X ^ 4 < 0 $. 
The Killing vector is positive and timelike only in the static patch. 
Elsewhere, it is spacelike. Energy is a subtle concept in dS.

\subsection{Conserved quantities} 
Consider a particle moving on a geodesic  $ x ^ \mu ( \tau ) $ 
in a spacetime with Killing vector $ K ^ \mu $. 
Then, we have that 
\[
 Q  = K_ \mu \frac{d x ^ \mu }{ d \tau } \text{is conserved }
\] To see this, differentiate to find that 
\begin{align*}
	\frac{d Q }{ d \tau }   & = \partial  _ \nu k _ \mu \frac{d x ^ \nu }{ d \tau } \frac{ d x ^ \mu }{ d \tau } 
	+ k _ \mu \frac{ d ^ 2 x  }{ d \tau ^ 2 } \\
	&=  \partial  _ \nu k _ \mu \frac{d k ^ \nu }{ d \tau } \frac{ d x ^ \mu }{ d \tau }  - 
	k _ \mu \Gamma ^ \mu _{ \rho \sigma } \frac{d x ^ \rho }{ d \tau } \frac{ d x ^ \sigma }{ d \tau } \\
	&=  \nabla _ \nu k _ \mu \frac{ d x ^ \nu }{ d \tau } \frac{ d x ^ \mu }{ d \tau } \\ 
\end{align*} 
We can also see this from the action 
\[
 S  = \int d \tau g_{ \mu \nu } \dot{ x } ^ \mu \dot{ x } ^ \nu   
\] Consider $ \delta x ^ \mu ( \tau )  = k ^ \mu ( x) $. We have 
 \[
 \delta S  = \int d \tau \, \partial  _ \rho g_{ \mu \nu } \frac{ d x^ \mu }{ d \tau } \frac{ d x ^ \nu }{ d \tau } 
 + 2 g_{ \mu \nu } \frac{d x ^ \mu }{ d \tau } \frac{ dk  ^ \nu }{ d \tau }
\] 
Now, we use the fact that 
\begin{align*}
	g_{ \mu \nu } \frac{d  k ^ \nu }{ d \tau } &=  \frac{d k _\mu }{ d \tau } - \frac{d g _{ \mu \rho } }{ d \tau } k ^ \rho  \\
						   &=  \left(  \partial  _ \nu k _ \mu 
						    - \partial  _ \nu g _{ \mu \rho } k ^ \rho \right)  
						    \frac{ d x ^ \nu }{ d \tau }\\
\end{align*}
Substituting this into the action 
\[
 \delta S = \int d \tau 2 \nabla _ \mu K _ \nu \frac{ d x ^ \mu }{ d \tau } \frac{ d x ^ \nu }{ d \tau }
\] This implies that $ \sigma S = 0 $ iff $ \nabla _{ ( \mu } k_{ \nu ) }  = 0 $, the Killing equation.

\subsection{Asymptotics of Spacetime} 
Given a spacetime $ \mathcal{ M } $, with metric 
$ g _{ \mu \nu } ( x) $, we consider a 
conformal transformation 
\[
	\tilde{ g } _{ \mu \nu } ( x) = \Omega ^ 2 (  x) g_{ \mu \nu } ( x)  
\] where $ \Omega( x) $ is smooth, and non zero. 
These metrics don't necessarily 
have the same symmetries - they're different metrics 
that in general describe  \textbf{different } space times. 
They do have one thing on common, however, 
which is important. A vector which is 
null in the $ g _{ \mu \nu } $ spacetime 
is also null in the $ \tilde{ g } _{ \mu \nu  }  $ space 
time. 
This means that they have the same causal structure. 
\[
 g_{ \mu \nu }  X^ \mu    X^\nu =0   \iff \tilde{ g } _{ \mu \nu } X^\mu X^\nu  = 0 
\] Since 0 is the dividing line between positve and negative, 
null / spacelike / timelike vectors in $ g _{ \mu \nu } $ 
map to null / spacelike / timelike vectors in $ \tilde{ g } _{ \mu \nu }  $. 
Conformal transformations are something that crop up 
everywhere in physics. 

\subsubsection{Penrose Diagrams}
The idea is to use conformal transformations 
to bring the infinity of space time 
a 'little bit closer', in such a way 
that is becomes simple to visualise what infinity looks like. 
There is some fancy technical way to do these diagrams, 
but we'll just do examples to get the feel.

In Minkowski space in 2 dimensions, $ \mathbb{ R} ^{ 1, 1 } $, 
the metric $ ds ^ 2 = - d t ^ 2 + dx ^ 2 $ (the standard metric). 
We'll do two successive coordinate transformations. 
We introduce lightcone coordinates  $ u  = t  -x $, 
and $ v = t + x $. It's simple to see that 
in these coordinates, 
\[
 ds ^ 2 =  - du dv
\] We have to be very careful about the range in which coordinates move. 
We have that $ u , v \in ( - \infty  , \infty ) $, 
since $ t , x  $  are in the same range. 
Now, the idea is to come up with a second coordinate transform which 
takes infinity to a finite number. We can choose any function we 
like which has an infinite range in an infinite domain. 
We can now map this to a finite range, 
\[
 u = \tan \tilde{ u } , \quad v = \tan \tilde{ v }   
\] with $ u , v  \in ( - \frac{ \pi }{ 2 } , \frac{\pi}{ 2 } ) $. 
In these coordinates, the metric is 
\[
 ds ^  2 =  -\frac{1}{\cos ^ 2 \tilde{ u } \cos ^ 2 \tilde{ v}   } d \tilde{ u } d \tilde{ v }   
\] Now we do a conformal map. 
Consider the metric 
\[
 d \tilde{ s } ^ 2 = \cos ^ 2 \tilde{ u } \cos ^ 2 \tilde{v } ds ^ 2  = - d \tilde{ u } d \tilde{ v }      
\] Thus, 
we get something that looks like 
the Minkowski metric but now has a limited range. 
We write this range to include the closure of the set (which acts as infinity) 
so that $ \tilde{ u } , \tilde{ v } \in [ - \frac{\pi}{2 } , + \frac{\pi}{2 } ]   $. 
Adding the points $ \pm \frac{\pi}{ 2 } $ that used to be 
$ \pm \infty$ is called conformal compactification.
Now we draw the spacetime with 
lightrays at $ 45 $ degrees, and time vertical.

Insert penrose diagram here. 
This is the Penrose diagram. Don't
trust distances on these diagrams, but trust the causal structure. 

We can draw various geodesics on this diagram, 
in particular timelike geodesics (constant $ x $), 
and spacelike geodesics (constant $  t$ ). 
These are geodesics in our \textbf{original } choice 
of metric.

All timelike geodesics start at the point $ \peninf ^ -  , [ - \frac{\pi}{2 } , - \frac{\pi}{2  }] $, 
and end at $ \peninf ^ + , [ \frac{\pi}{2 } , \frac{\pi}{ 2   }]$. These are
called past / future timelike infinity. 
Meanwhile, all spacelike geodesics 
start and end at two points $ \peninf ^ 0  , [ - \frac{\pi}{2 }, + \frac{\pi}{2 } ] $ or 
$ [ + \frac{\pi}{2 } , - \frac{\pi}{ 2 } ] $. These are called spacelike 
infinity. 

All null curves start at $ \scri ^ - $ "scri-minus" and at $ \scri ^ + $ "scri-plus". 
These all called past and future null infinity.

There are some things we can read off
from a Penrose diagram. They tell us 
basic things about the spacetime.
For example, any two points on the spacetime 
have a common future and a common past. 

Let's now do the same thing 
for $ \mathbb{ R} ^{ 1, 3 } $. 
We do something similar. 
The metric is best written in 
polar coordinates 
\[
 ds ^ 2 = - dt ^ 2 + d r ^ 2 + r ^ 2 d \Omega^ 2 _ 2 
\] We, as before, 
do a change of coordinates so that 
\begin{align*}
	u &=  t - r = \tan \tilde{ u }   \\
	v &=  t + r  =  \tan \tilde{ v }   \\
\end{align*}

In these coordinates, we get that 
\begin{align*}
	ds ^ 2 &=  - d u dv + \frac{1}{4 } ( u - v ) ^ 2 d \Omega_ 2 ^ 2  \\
	       &=  \frac{1}{4 \cos^ 2 \tilde{ u } \cos ^ 2 \tilde{ v }   } \left(  
	       - 4 d \tilde{ u } d \tilde{ v } + \sin ^ 2 ( \tilde{ u } - \tilde{ v } ) d \Omega _ 2 ^ 2     \right)  \\
\end{align*}

This is still 4 dimensional, so we ignore the spherical part. 
Unlike in Minkwoski space, we require that $ r \geq 0 $ which implies that 
$ v \geq u $, which means that 
\[
  - \frac{\pi}{2 } \leq \tilde{ u } \leq \tilde{ v } \leq \frac{\pi}{2 }   
\] We hence drop the $ S ^ 2 $ and draw the Penrose diagram, 
The left hand line on the diagram 
is not a boundary of space time - it is 
merely where $ \tilde{ u }  = \tilde{ v } \implies r = 0   $, 
and $ S ^ 2$ shrinks to zero here.

\subsubsection{de Sitter} 
In global coordinates, de Sitter space 
is represented by 
\[
 ds ^ 2 = - d \tau ^ 2 + R ^ 2 \cosh ^ 2 \left(  \frac{ \tau }{ R }  \right)  d \Omega_ 3 ^ 2 
\] We introduce conformal time, 
which makes the metric have a factor which sits out front.
Conformal time is given by 
\[
	\frac{ d \eta }{ d \tau }  = \frac{1}{ R \cosh \left(  \tau / R  \right)  } 
	\implies \cos \eta = \frac{1}{\cosh \left(  \tau /  R  \right) }
\] where $ \eta \in ( - \frac{\pi}{2 } , \frac{\pi}{2 } ) $. 
Plugging this in, 
we get the following metric 
\[
	ds ^ 2= \frac{R ^ 2 }{ \cos^ 2 \eta } \left(  
	 - d \eta ^ 2 + d \Omega_ 3 ^ 2 \right)  
\] We write out the 3 sphere metric as $ d \chi ^ 2  + \sin ^ 2 \chi d \Omega _ 2 ^2 $. 
de Sitter is conformal to 
\[
 ds ^ 2 = - d \eta ^ 2 + d \chi ^ 2 + \sin ^ 2 \chi  d \Omega_2 ^ 2 
\] Now we can draw our Penrose diagram. 

We see that the boundary of de Sitter 
is spacelike. 
No matter how long you wait, you cannot 
see the whole space, nor can you influence 
the whole space. This is given by a diagonal line. 
We can check that the static patch coordinates map to 
the intersection of the event horizon and the particle horizon. 
