\pagebreak
\section{The Einstein Equations} 
Space time is a manifold $\mathcal{ M } $ equipped 
with a Lorentzian metric $ g$. We view 
our metric as some matrix which varies from 
point to point on the manifold. The right way 
to write things down here is to write down an 
action principle, then vary the actions to get our 
equations of motion. 

\subsubsection{The Einstein-Hilbert Action} 
The dynamics is governed by the Einstein-Hilbert 
action. 
There's not very much we can write here. 
The only object we have to play with is our metric  $ g $, 
and from this we have our natural volume form we can use. 
The only one we have is the square root of our determinant 
of our metric. Our action must precede 
\[
 S = \int d^ 4 \sqrt{ - g }  
\] Now, what scalar function can we put here? 
The only thing we can do is to pull out our Ricci scalar, 
so that 
\[
 S = \int d^ 4 x \, \sqrt{  - g }  R 
\] Note that schematically, $ R \sim \partial  \Gamma + \Gamma  \Gamma $, 
and $ \Gamma \sim g ^{ - 1} \partial  g $. 
Thus, $ R $ is a function of two derivatives 
of our metric. Note, this 
means that the only connection we 
could cook up is the Levi-Civita connection. This is the simplest thing we can do!
Now, to derive the equations of motion, we need to vary this field. 
We take the metric and push it wish a small change 
\[
 g _{\mu \nu  } \to g _{ \mu \nu } + \delta g _{ \mu \nu }
\] Now, from this we can also deduce how 
the inverse metric changes with this infinitesimal change. 
\[
 g_{ \rho \mu } g ^{ \mu \nu } = \delta \indices{ ^ \nu _ \rho } \implies 
 \delta g_{ \rho \mu } g ^{ \mu \nu } + g _{ \rho \mu  } \delta g ^{ \mu \nu } = 0 
\]  This means 
we can read off our change $ \delta g ^{ \mu \nu }  $ as 
\[
 \delta g ^{ \mu \nu }  =  - g ^{ \mu \rho } g ^{ \nu \sigma } \delta g_{ \rho \sigma}
\] When we vary the whole thing, 
the trick is to write out the Ricci 
scalar in terms of the Ricci tensor 
contracted with the metric. Then, we 
apply the product rule when looking at a small variation.
When we do the small variation, we have to remember 
to hit the colume form as well. 
\[
 \delta S  = \int d ^4  x \, 
 \left[  \delta ( \sqrt{  - g } ) g ^{ \mu \nu } R _{ \mu \nu } + 
 \sqrt{ - g }  \left(  \delta g ^{ \mu \nu } R_{ \mu \nu } + g^{ \mu \nu } \delta R _{ \mu \nu } \right) \right] 
\] We need to figure out 
what $ \delta \sqrt{ - g }   $  is. To do this, 
we prove a claim that 
\begin{claim}
	We claim that 
	\[
	 \delta \sqrt{ - g }   = - \frac{1}{2 } \sqrt{ - g }  g _{ \mu \nu } \delta g ^{ \mu \nu }
	\] 
\begin{proof}
	We use the fact that $ \log \det A = \tr \log A $, which implies 
	that 
	 \[
		 \frac{1}{\det  A } \delta ( \det  A)  = \tr  = \tr ( \delta \log A )  = \tr ( 
		 A^{ - 1} \delta A ) = \tr ( A ^{ - 1 } \delta A ) 
	 \] Now, susbtituting in our metric $ g $ as 
	 the matrix $ A $ in our identity above, this finally implies that 
	 \begin{align*}
		 \delta \sqrt{ - g} &  = \frac{1}{2 } \frac{1}{\sqrt{ - g}  } ( - g ) 
		 g ^{\mu \nu } \delta g_{ \mu \nu } \\
		 &=  \frac{1}{2 } \sqrt{ -g }  g ^{ \mu \nu } \delta g _{ \mu \nu } 
	 \end{align*} 
\end{proof}
\end{claim}

Now, we compute the variation of our Ricci tensor. 
\begin{claim}
	We have that our variation is 
	\[
	 \delta R_{ \mu \nu } = \nabla _ \rho \delta \Gamma ^ \rho _{ \mu \nu } - \nabla_{ \nu  }
	 \delta \Gamma ^ \rho _{ \mu \rho }
	\]  with 
	\[
	 \delta \Gamma ^ \rho _{ \mu \nu } = \frac{1}{2 } 
	 g ^{ \rho \sigma } ( \nabla _ \mu \delta g _{ \sigma \nu } + 
	 \nabla _ \nu \delta g _{ \sigma \mu } - \nabla _ \sigma \delta g_{ \mu \nu } ) 
	\] 
\begin{proof}
	First note that importantly, $ \Gamma ^ \mu _{ \rho \nu } $ is not a tensor. 
	However, the difference $ \delta \Gamma ^ \mu _{ \rho \nu } $ 
	is a tensor since its the difference between two connections. This means that it is a well defined action 
	to take the covariant derivative of the objects
	In normal coordinates, at some point 
	\begin{align*}
		\delta \Gamma^ \rho _{ \mu \nu } &= \frac{1}{2 }  g ^{ \rho  \sigma }\left( 
		\partial  _ \mu \delta g _{ \sigma \nu } + \partial  _ \nu 
	g _{ \sigma \mu } - \partial  _ \sigma \delta g _{ \mu \nu } \right)   \\
						 &=  \frac{1}{2 } g ^{ \rho\sigma } \left(  
						 \nabla _ \mu \delta g _{ \sigma \nu } 
					 + \nabla _ \nu \delta g _{ \sigma \mu } 
				  - \nabla _{ \sigma } \delta g _{ \mu \nu }\right)  
	\end{align*}
	Since we're in normal coordinates, we can now replace our partial 
	derivatives $ \partial $ with covariant derivatives $ \nabla $. 
	This gives us a tensorial relation, and hence this 
	holds true in all coordinate frames. We can play the same 
	game with our Riemann tensor, 
	\[
	 R \indices{ ^ \sigma _{ \rho \mu \nu } }  = \partial  _ \mu \Gamma ^ \sigma _{ \nu \rho } 
	  - \partial  _ \nu \Gamma ^ \sigma _{ \mu \rho }
	\]  This implies that our small change 
	is given by 
	\begin{align*}
		\delta R \indices{ ^ \sigma _{ \rho \mu \nu } } &=  \partial  _ \mu  
		\delta \Gamma ^ \sigma _{ \mu \rho }  - \partial  _ \nu \delta \Gamma 
		^ \sigma _{ \mu \rho }\\
		&=  \nabla _ \mu \delta \Gamma ^ \sigma _{ \nu \rho } - 
		\nabla _ \nu \delta \Gamma ^ \sigma _{ \mu \rho } \\
	\end{align*} Since $ \Gamma  =0 $ in normal coordinates.  
\end{proof}
\end{claim}

Now we have that, after all the dust has settled, 
that 
\[
 \delta S = \int d ^ 4 x \, \sqrt{ - g }  \left[  
 R _{ \mu \nu }  - \frac{1}{2 } R g _{ \mu \nu } \right] \delta g ^{ \mu \nu } + \nabla _ \mu X ^ \mu  
\] Where we've written 
\[
 X ^ \mu = g ^{ \rho \nu } \delta \Gamma ^ \mu _{ \rho \nu }  - g ^{ \mu \nu } \delta \Gamma ^ \rho _{ \nu \rho }
\] When we impose that 
\[
 \delta S = 0 , \forall g ^{ \mu \nu } \implies 
 G _{ \mu \nu } = R _{ \mu \nu } - \frac{1}{2 } R g _{ \mu \nu } =0 
\] Multiplying this quantity by $ g ^{ \mu \nu } $, we have that 
$ R = 0 $, which implies the vacuum Einstein equations 
\[
 R_{ \mu \nu } = 0 
\]
\subsubsection{Dimensional analysis} 
We can use dimensional analysis to fill out the 
constants here. Our action $ S $ has dimension 
$ M L ^{ 2 } T ^{ - 1}  $, Our metric is dimensionless, and 
$ R $ has units of $ L ^{ - 2} $. This implies that 
our full action is 
\[
 S = \frac{ c ^ 3  }{  16 \pi G} \int d ^ 4 x \, \sqrt{ -g }  R
\]  Our Planck mass
is $ \mathcal{ M } _{ pl } ^ 2  = \frac{ \planck c }{8 \pi G }$ and 
this is approximately $ \mathcal{ M } _{ pl } \propto 10 ^{ 18 } G eV $. 
Work with units with $ c = 1 $, and $ \planck = 1 $. 

\subsubsection{The Cosmological Constant} 
We could add a further term to the action 
 \[
	 S = \frac{1}{2 } \mathcal{ M  }_{ pl } \int d ^ 4 x \, \sqrt{ - g}  ( R - 2 \Lambda ) 
\] We will study these. 
We have the equation 
\[
 R_{ \mu \nu } - \frac{1}{2 } R g _{ \mu \nu } = - \lambda g _{ \mu \nu } \implies 
 R = 4 \Lambda \implies R_{ \mu \nu }  = \Lambda g _{ \mu \nu }
\]

\subsection{Diffeomorphisms Revisited} 
Our metric has $ \frac{1}{2 } 4 \cdot  5  = 10 $ components. 
But two metrics related by $ x ^ \mu \to \tilde{x } ^ \mu ( x)  $ 
are physically equivalent. 
This means that actually we only have $ 10 - 4  = 6 $ degrees of freedom.
The change of coordinates can be viewed as a diffeomorphism 
\[
 \phi : \mathcal{ M } \to \mathcal{ M }
\] such 'diffeos' are the 'gauge symmetry' of general relativity. 
