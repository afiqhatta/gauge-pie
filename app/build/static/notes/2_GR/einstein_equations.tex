\pagebreak
\section{The Einstein Equations} 
Space time is a manifold $\mathcal{ M } $ equipped 
with a Lorentzian metric $ g$. We view 
our metric as some matrix which varies from 
point to point on the manifold. The right way 
to write things down here is to write down an 
action principle, then vary the actions to get our 
equations of motion. 

\subsubsection{The Einstein-Hilbert Action} 
The dynamics is governed by the Einstein-Hilbert 
action. 
There's not very much we can write here. 
The only object we have to play with is our metric  $ g $, 
and from this we have our natural volume form we can use. 
The only one we have is the square root of our determinant 
of our metric. Our action must precede 
\[
 S = \int d^ 4 \sqrt{ - g }  
\] Now, what scalar function can we put here? 
The only thing we can do is to pull out our Ricci scalar, 
so that 
\[
 S = \int d^ 4 x \, \sqrt{  - g }  R 
\] Note that schematically, $ R \sim \partial  \Gamma + \Gamma  \Gamma $, 
and $ \Gamma \sim g ^{ - 1} \partial  g $. 
Thus, $ R $ is a function of two derivatives 
of our metric. Note, this 
means that the only connection we 
could cook up is the Levi-Civita connection. This is the simplest thing we can do!
Now, to derive the equations of motion, we need to vary this field. 
We take the metric and push it wish a small change 
\[
 g _{\mu \nu  } \to g _{ \mu \nu } + \delta g _{ \mu \nu }
\] Now, from this we can also deduce how 
the inverse metric changes with this infinitesimal change. 
\[
 g_{ \rho \mu } g ^{ \mu \nu } = \delta \indices{ ^ \nu _ \rho } \implies 
 \delta g_{ \rho \mu } g ^{ \mu \nu } + g _{ \rho \mu  } \delta g ^{ \mu \nu } = 0 
\]  This means 
we can read off our change $ \delta g ^{ \mu \nu }  $ as 
\[
 \delta g ^{ \mu \nu }  =  - g ^{ \mu \rho } g ^{ \nu \sigma } \delta g_{ \rho \sigma}
\] When we vary the whole thing, 
the trick is to write out the Ricci 
scalar in terms of the Ricci tensor 
contracted with the metric. Then, we 
apply the product rule when looking at a small variation.
When we do the small variation, we have to remember 
to hit the colume form as well. 
\[
 \delta S  = \int d ^4  x \, 
 \left[  \delta ( \sqrt{  - g } ) g ^{ \mu \nu } R _{ \mu \nu } + 
 \sqrt{ - g }  \left(  \delta g ^{ \mu \nu } R_{ \mu \nu } + g^{ \mu \nu } \delta R _{ \mu \nu } \right) \right] 
\] We need to figure out 
what $ \delta \sqrt{ - g }   $  is. To do this, 
we prove a claim that 
\begin{claim}
	We claim that 
	\[
	 \delta \sqrt{ - g }   = - \frac{1}{2 } \sqrt{ - g }  g _{ \mu \nu } \delta g ^{ \mu \nu }
	\] 
\begin{proof}
	We use the fact that $ \log \det A = \tr \log A $, which implies 
	that 
	 \[
		 \frac{1}{\det  A } \delta ( \det  A)  = \tr  = \tr ( \delta \log A )  = \tr ( 
		 A^{ - 1} \delta A ) = \tr ( A ^{ - 1 } \delta A ) 
	 \] Now, susbtituting in our metric $ g $ as 
	 the matrix $ A $ in our identity above, this finally implies that 
	 \begin{align*}
		 \delta \sqrt{ - g} &  = \frac{1}{2 } \frac{1}{\sqrt{ - g}  } ( - g ) 
		 g ^{\mu \nu } \delta g_{ \mu \nu } \\
		 &=  \frac{1}{2 } \sqrt{ -g }  g ^{ \mu \nu } \delta g _{ \mu \nu } 
	 \end{align*} 
\end{proof}
\end{claim}

Now, we compute the variation of our Ricci tensor. 
\begin{claim}
	We have that our variation is 
	\[
	 \delta R_{ \mu \nu } = \nabla _ \rho \delta \Gamma ^ \rho _{ \mu \nu } - \nabla_{ \nu  }
	 \delta \Gamma ^ \rho _{ \mu \rho }
	\]  with 
	\[
	 \delta \Gamma ^ \rho _{ \mu \nu } = \frac{1}{2 } 
	 g ^{ \rho \sigma } ( \nabla _ \mu \delta g _{ \sigma \nu } + 
	 \nabla _ \nu \delta g _{ \sigma \mu } - \nabla _ \sigma \delta g_{ \mu \nu } ) 
	\] 
\begin{proof}
	First note that importantly, $ \Gamma ^ \mu _{ \rho \nu } $ is not a tensor. 
	However, the difference $ \delta \Gamma ^ \mu _{ \rho \nu } $ 
	is a tensor since its the difference between two connections. This means that it is a well defined action 
	to take the covariant derivative of the objects
	In normal coordinates, at some point 
	\begin{align*}
		\delta \Gamma^ \rho _{ \mu \nu } &= \frac{1}{2 }  g ^{ \rho  \sigma }\left( 
		\partial  _ \mu \delta g _{ \sigma \nu } + \partial  _ \nu 
	g _{ \sigma \mu } - \partial  _ \sigma \delta g _{ \mu \nu } \right)   \\
						 &=  \frac{1}{2 } g ^{ \rho\sigma } \left(  
						 \nabla _ \mu \delta g _{ \sigma \nu } 
					 + \nabla _ \nu \delta g _{ \sigma \mu } 
				  - \nabla _{ \sigma } \delta g _{ \mu \nu }\right)  
	\end{align*}
	Since we're in normal coordinates, we can now replace our partial 
	derivatives $ \partial $ with covariant derivatives $ \nabla $. 
	This gives us a tensorial relation, and hence this 
	holds true in all coordinate frames. We can play the same 
	game with our Riemann tensor, 
	\[
	 R \indices{ ^ \sigma _{ \rho \mu \nu } }  = \partial  _ \mu \Gamma ^ \sigma _{ \nu \rho } 
	  - \partial  _ \nu \Gamma ^ \sigma _{ \mu \rho }
	\]  This implies that our small change 
	is given by 
	\begin{align*}
		\delta R \indices{ ^ \sigma _{ \rho \mu \nu } } &=  \partial  _ \mu  
		\delta \Gamma ^ \sigma _{ \mu \rho }  - \partial  _ \nu \delta \Gamma 
		^ \sigma _{ \mu \rho }\\
		&=  \nabla _ \mu \delta \Gamma ^ \sigma _{ \nu \rho } - 
		\nabla _ \nu \delta \Gamma ^ \sigma _{ \mu \rho } \\
	\end{align*} Since $ \Gamma  =0 $ in normal coordinates.  
\end{proof}
\end{claim}

Now we have that, after all the dust has settled, 
that 
\[
 \delta S = \int d ^ 4 x \, \sqrt{ - g }  \left[  
 R _{ \mu \nu }  - \frac{1}{2 } R g _{ \mu \nu } \right] \delta g ^{ \mu \nu } + \nabla _ \mu X ^ \mu  
\] Where we've written 
\[
 X ^ \mu = g ^{ \rho \nu } \delta \Gamma ^ \mu _{ \rho \nu }  - g ^{ \mu \nu } \delta \Gamma ^ \rho _{ \nu \rho }
\] When we impose that 
\[
 \delta S = 0 , \forall g ^{ \mu \nu } \implies 
 G _{ \mu \nu } = R _{ \mu \nu } - \frac{1}{2 } R g _{ \mu \nu } =0 
\] Multiplying this quantity by $ g ^{ \mu \nu } $, we have that 
$ R = 0 $, which implies the vacuum Einstein equations 
\[
 R_{ \mu \nu } = 0 
\]
\subsubsection{Dimensional analysis} 
We can use dimensional analysis to fill out the 
constants here. Our action $ S $ has dimension 
$ M L ^{ 2 } T ^{ - 1}  $, Our metric is dimensionless, and 
$ R $ has units of $ L ^{ - 2} $. This implies that 
our full action is 
\[
 S = \frac{ c ^ 3  }{  16 \pi G} \int d ^ 4 x \, \sqrt{ -g }  R
\]  Our Planck mass
is $ \mathcal{ M } _{ pl } ^ 2  = \frac{ \planck c }{8 \pi G }$ and 
this is approximately $ \mathcal{ M } _{ pl } \propto 10 ^{ 18 } G eV $. 
Work with units with $ c = 1 $, and $ \planck = 1 $. 

\subsubsection{The Cosmological Constant} 
We could add a further term to the action 
 \[
	 S = \frac{1}{2 } \mathcal{ M  }_{ pl } \int d ^ 4 x \, \sqrt{ - g}  ( R - 2 \Lambda ) 
\] We will study these. 
We have the equation 
\[
 R_{ \mu \nu } - \frac{1}{2 } R g _{ \mu \nu } = - \lambda g _{ \mu \nu } \implies 
 R = 4 \Lambda \implies R_{ \mu \nu }  = \Lambda g _{ \mu \nu }
\]

\subsection{Diffeomorphisms Revisited} 
Our metric has $ \frac{1}{2 } 4 \cdot  5  = 10 $ components. 
But two metrics related by $ x ^ \mu \to \tilde{x } ^ \mu ( x)  $ 
are physically equivalent. 
This means that actually we only have $ 10 - 4  = 6 $ degrees of freedom.
The change of coordinates can be viewed as a diffeomorphism 
\[
 \phi : \mathcal{ M } \to \mathcal{ M }
\] such 'diffeos' are the 'gauge symmetry' of general relativity.
Consider diffeomorphisms which take the form 
\[
	x ^ \mu \to \tilde{ x } ^ \mu ( x) = x ^ \mu + \delta x ^ \mu  
\] Moreover, we can 
view this as a diffeomorphism generated by a vector field 
$ X ^ \mu  = \delta x ^ \mu $. The metric 
transforms as $ g_{ \mu \nu } ( x) \to \tilde{ g } _{ \mu \nu } ( \tilde{ x  })   $
with $ \tilde{ g } _{ \mu \nu } ( \tilde{ x  } )  = \frac{\partial \tilde{ x } ^ \rho  }{\partial x ^ \mu }
\frac{\partial \tilde{ x } ^ \sigma  }{\partial  x ^ \nu }  g _{ \rho \sigma }  ( \tilde{ x  } )  $. 
From our definition that these coordinates are generated by a vector field, 
this is equal to 
\begin{align*}
	\dots &  = ( \delta \indices{ _ \mu ^ \rho } + \partial  _ \mu X ^ \rho ) 
	( \delta \indices{ _ \nu ^ \sigma  } + \partial  _ \nu X ^ \sigma ) 
	( g_{ \rho \sigma } ( x ) + X ^ \lambda \partial  _ \lambda  g_{ \rho \sigma } ( x) ) \\
	      &=  g_{ \mu \nu } ( x) + X ^ \lambda \partial  _ \lambda g _{ \mu \nu } ( x) 
	      + g _{ \mu \rho } ( x ) \partial  _ \nu X ^ \rho + 
	      g _{ \nu \rho } \partial  _ \mu X ^ \rho \\
\end{align*}  
This means that our infinitesimal change is 
\begin{align*}
	\delta g _{ \mu \nu }  & = X ^ \lambda \partial  _ \lambda g _{ \mu \nu } + 
	g _{ \mu \rho } \partial  _ \nu X ^ \rho + g _{ \nu \rho } \partial  _ \mu X ^ \rho \\
			       &= (  \mathcal{ L  } _ X g)_{ \mu \nu }  \\
\end{align*}
Alternatively, we can write this as 
\begin{align*}
	\delta g_{ \mu \nu } &=  \partial  _ \mu X _ \nu + \partial  _ \nu X _ \mu 
	+ X ^ \rho \left(  \partial  _ \rho g _{ \mu \nu } - 
	\partial  _ \mu g _{ \rho \nu } - \partial  _ \nu g _{ \mu \rho }\right) \\
	&=  \partial  _ \mu X _ \nu + \partial  _ \nu X _ \mu + 2 X ^ \rho g _{ \rho \sigma } T ^ \sigma _{ 
	\mu \nu } \\
	&=  \nabla _ \mu X _ \nu + \nabla _ \nu X _ \mu  \\
\end{align*}
Now, let's look back at the action
\[
 \sigma S = \int d ^ 4 \sqrt{ -g  }  G ^{ \mu \nu } \delta g _{ \mu \nu }
\] If we restrict to these changes of coordinates, 
the above 
\[
 \dots = 2 \int d ^ 4 x \, \sqrt{ - g }  G ^{ \mu \nu } \nabla _ \mu X _ \nu 
 =0 \forall X  =  -2 \int d ^ 4 x \, \sqrt{ - g }  ( \nabla _ \mu G ^{ \mu \nu } )  X^ \nu 
\]  this is because changing coordinates is a gauge symmetry. 
This means that our Einstein tensor obeys 
\[
 \nabla _ \mu G ^{ \mu \nu }  = 0
\] This means that diffeomorphisms are 
a symmetry implies the Bianchi identity. 
So, not all of the $ G ^{ \mu \nu } $ are independent. 
We have found four conditions, which means that the 
Einstein equations $ G _{ \mu \nu } $ is really only 
6 equations, which is the right number of components 
to determine our metric $ g _{ \mu \nu } $. 

\subsection{Some simple solutions} 
Let's look at what happens when we have a vanishing 
cosmological constant with $ \Lambda  =0 $. 
We need to solve  $ R_{ \mu \nu }  = 0$. 
It's tempting to write $ g _{ \mu \nu } = 0 $, 
but this isn't allowed because the metric requires an inverse! 
This makes gravity different from other field theories, 
we need to have the constraint that non of the eigenvalues are zero. 
This may suggest that the metric is not a fundamental field. There 
are lots of similarities to this an fluid mechanics. 
The simplest solution is Minkowski space time. 
\[
 ds ^ 2 = - d t ^ 2 + d \vec{x} ^ 2 
\] 
In the case, $ \Lambda > 0 $, look for solutions 
to $ R _{ \mu \nu }  = \Lambda g_{ \mu \nu }$ of the form 
\[
	d s^ 2 = - f (r ) ^ 2 d t ^ 2 + f ( r ) ^{ - 2} dr ^ 2 + r ^ 2 ( d \theta ^ 2 
	+ \sin^ 2 \theta d \psi ^ 2 ) 
\] 
We can compute our Ricci tensor 
to find 
\[
	R _{ t t }  =  - f 4 R_{ rr } f ^  3 ( f '' + \frac{2 f ' }{ r } + ( f ' ) ^2 f ), \quad 
	R_{ \psi \psi }  = \sin ^ 2 \theta R _{ \theta \theta  } = ( 1 - f ^ 2 - 2 f f ' r ) \sin ^ 2 \theta 
\] 
We have the system from $ R_{ \mu \nu } = \Lambda _{ \mu \nu } $. 
By setting $ \mu , \nu = t t , r r $ we get that 
 \[
	 f '' + \frac{2 f ' }{ r } + \frac{ ( f ' ) ^ 2 }{ f }  = - \frac{\Lambda }{ f }
\] 
Similarly, by setting $ \mu,  \nu = \theta \theta , \psi \psi  $, we have 
\[
 1 - 2 f f ' r - f ^ 2 = \Lambda r ^ 2 
\] This equation is solved by 
\[
 f ( r)  = \sqrt{ 1 - \frac{ r ^ 2 }{ R ^ 2 }} , \quad R ^ 2 = \frac{ 3 }{ \Lambda}
\] This is de Sitter space time (or alternatively called the static patch of 
d S ). Note that we have a valid range from $ r \in [ 0 , R ] $, but the 
metric appears to be singular at $ r = R $. 
We can look at geodesics. We have the action, where $ \sigma $ represents proper time 
\[
 S  = \int d \sigma \left[  
 - f ( r) ^ 2 \dot{t } ^ 2 + f ( r) ^{ - 2 } \dot{ r } ^ 2 
 + r ^ 2 \left(  \dot{ \theta } ^ 2 + \sin ^ 2 \theta \dot{ \phi  }  ^ 2  \right)  \right] 
\] Here, 
there are two conserved quantities. Nothing depends on $ \phi $, 
so we have that 
$ l = \frac{1}{2 } \frac{\partial  L }{ d \dot{\phi }  } = r ^ 2 \sin ^ 2 \theta \dot{\phi  } $. 
The other conserved quantity we get is the energy of the particle with 
\[
	E =  - \frac{1}{2 } \frac{\partial  L }{\partial  \dot{ t } } = f ( r) ^ 2 
	\dot{t }  
\] For a massive particle, 
we also require that the trajectory is timelike.
Since $ \sigma  $ represents proper time, this means 
that the Lagrangian itself is equal to 
\[
 - f ^ 2 \dot{ t } ^ 2 + f ^{ - 2 } \dot{ r } ^ 2 + r ^ 2 
 ( \dot{ \theta } ^ 2 + \sin ^ 2 \theta  \dot{ \phi } ^ 2  ) = - 1  
\]
Look for geodesics with $ \theta = \frac{\pi }{ 2 } $ and $ \dot{ \theta } ^ 2  $. 
This gives 
\[
	\dot{ r } ^ 2 + V_{ eff} ( r)  = E ^ 2  
\] with 
\[
	V_{ eff }( r) = ( 1 + \frac{l ^ 2 }{ r ^ 2 } ) ( 1 - \frac{ r ^  2 }{R ^ 2  } ) 
\] For $ l  = 0 $, we have that 
\[
	r ( \sigma)  = R \sqrt{ E ^ 2 - 1}  \sinh ( \frac{ \sigma }{ R } ) 
\]  This hits $ r = R $ in finite $ \sigma $. Meanwhile 
\[
	\frac{ d t }{ d \sigma } = E \left(  1 - \frac{ r ^ 2 }{ R ^ 2 }  \right)  ^{ - 1}
\] Solutions to this have $ t \to \infty $ as $ r \to R $. 
To see this, suppose $ r ( \sigma _ * )  = R $ and expand $ \sigma = \sigma_{ * }  - \epsilon $. 
We find that 
\[
 \frac{ d t }{ d \epsilon } \simeq  - \frac{ \alpha }{ \epsilon } 
\] this of course means that $ t \sim  - \alpha \log ( \frac{ \epsilon }{ R } ) $, 
and $ t \to \infty $ as $ `\epsilon \to 0 $. 
so, it takes a particle finite proper time  $ \sigma $ but infinite 
coordinate time $ t $. 
