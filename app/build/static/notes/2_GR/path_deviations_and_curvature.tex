\subsection{Path Deviations and Curvature} 

When we parallel transport a vector on a manifold along 
a curve, how the vector ends up
depends on the specific path taken. Specifically, 
on an infinitesimal 'square' path, this 
change in the vector is given by the Riemann 
curvature tensor. To illustrate this, 
we start by generating two curves by some vector 
fields $X, Y $, chosen such that the vector fields 
commute, hence $ [ X, Y ] = 0 $.  
We then move an infinitesimal amount in these two directions 
in a rectangle like fashion.

Let's take the path via $ q $ to  $ r$. 
The first thing we will do is to Taylor expand 
 $ Z $ from $ p $ to $q $ using $ \tau $ as a parameter. 
 This is a technique we shall repeatedly use, so take note!
 \[
  Z_ q ^ \mu = Z_ p ^ \mu + \delta \tau \left. \frac{ d Z ^ \mu }{ d \tau }
	  \right\vert_{ p } + \frac{1}{2 }( \delta \tau ) ^ 2 \left. \frac{ d ^ 2 Z ^ \mu }{ 
	  d \tau ^ 2 } \right\vert_{ p} + \dots 
 \] Now, by assumption of how we set up the 
 problem, we have that $ Z $ is parallel transported 
 along this integral curve of $ X  $, so obeys this equation 
 \[
  \frac{ d Z ^ \mu }{ d \tau } = - Z ^ \rho X ^ \nu \Gamma^ \mu _{ \nu \rho }
 \] If we evaluate this derivative at $ p $, it 
 vanishes since $ \Gamma ^ \mu _{ \nu \rho } ( p ) = 0 $ 
 since we're working in normal coordinates! 
 If we differentiate the above term to find our second order 
 term, then applying the product rule, we only have 
 one term which survives when we evaluate at $ p  $ since
 non-derivatives of $ \Gamma $ vanish. 
 \[
 \left.  \frac{ d ^ 2 Z ^ \mu }{ d \tau ^ 2} \right\vert_p = 
	  - Z^ \rho X ^ \nu \frac{d \Gamma^ \mu _{ \rho \nu } }{ d \tau }  = 
	  -  Z ^ \rho X ^ \nu X ^ \sigma \Gamma ^ \mu _{ \nu \rho , \sigma}
 \] Going into the last equality, 
 we used the chain rule. So, we now have an 
 expression for the Taylor expansion 
 \[
	 Z _ q ^ \mu = Z_ p ^ \mu - \frac{1}{2 } \left( X ^ \nu X ^ \sigma Z ^ \rho 
	 \Gamma ^ \mu _{ \rho \nu , \sigma  }\right)_p ( d \tau ) ^ 2 + \dots   
 \]  The next step to do is to 
 go from $ q $ to $ r $, repeating the same process. We 
 expand  $ Z_r ^ \mu $ about $ Z _ q ^ \mu $. Since 
 we are Taylor expanding along the integral curve 
 $ Y $ which is generated by the parameter $ \lambda $, 
 we have that 
 \begin{align*}
	 Z^\mu _ r &=  Z ^ \mu _ q +  \delta \lambda \left. \frac{d z ^ \mu }{ d \lambda } \right\vert_{ p } + \frac{1}{2 } ( \delta \lambda ) ^ 2 \left. \frac{d ^ 2 Z ^ \mu }{ d \lambda ^ 2 } \right\vert_{ p } \\
		   &=  Z ^ \mu _ q - \delta \lambda ( Y ^ \nu  Z^ \rho \Gamma ^ \mu _{ \rho \nu } )_ q  - \frac{1}{2 } ( \delta \lambda ) ^2 ( Y ^ \nu Y ^ \sigma Z ^ \rho \Gamma^ \mu 
		   _{ \rho \nu , \sigma } ) _ q + \dots\\
 \end{align*}
 Now don't despair. You can see we're in a bit of a bind here since 
 we have to evaluate the connection $ \Gamma $ at $ q $. Thus, 
 we can't use normal coordinates to make this term disappear. 
 We can however, do the next best thing and Taylor expand 
 this term about  $ p $, along the path of the integral curve 
 generated by $ X $
  \[
	  ( \Gamma^\mu_{ \rho \beta } ) _ q = ( \Gamma ^ \mu _{ \rho \beta } )_p + \delta \tau \left. \frac{d \Gamma ^ \mu _{ \rho \beta } }{  d \tau   } \right\vert_{ p} + \dots  = - ( Y ^ \nu Z ^ \rho X ^ \sigma \Gamma^\mu_{ \rho \nu , \sigma} ) d \tau 
	  \] Note that we used the
	  chain rule to rewrite $ d / d \tau $. Now, from normal coordinates the first term here vanishes. And, since we're substituting this into an expression which is first order in $ \delta \lambda $
	  anyway, we need only keep the derivative term.  By, the same 
	  logic, expanding out our second order derivative term 
	  and then Taylor expanding $ \Gamma $ around $ p $ gives
	  us 
\[
	\left. \frac{ d ^2 Z }{ d \lambda ^ 2  } \right\vert_{ q } = - ( Y ^ \nu Y ^ \sigma Z ^ \rho \Gamma^ \mu _{ \rho \nu, \sigma } )_p  
\] The upshot of this is that, 
we can now write the whole thing, from $ p $ to $ q $, 
as 
\[
	Z _ r ^\mu = Z _ p ^ \mu - \frac{1}{2 } ( \Gamma ^ \mu _{ \rho \nu , \sigma })_p \left[  X ^\nu X ^ \sigma Z ^ \rho d \tau ^ 2  + 2 Y ^ \nu Z ^ \rho 
	X ^ \sigma d \tau d \lambda + Y ^ \nu Y ^ \sigma Z ^ \rho d \lambda ^ 2\right] 
\] Now, if we were to go the other way and go through $s  $, 
we get exactly the same result but with $ X , Y $ switched 
to reflect the fact that we're changing the order of 
vectors we're travelling with, and also 
a switch in $ \lambda $ and $ \tau $ to reflect 
the change in the order of parametrisation.
This means that only the mixed term is changed. 
\[ 	Z _ r ^\mu = Z _ p ^ \mu - \frac{1}{2 } ( \Gamma ^ \mu _{ \rho \nu , \sigma })_p \left[  X ^\nu X ^ \sigma Z ^ \rho d \tau ^ 2  + 2 X ^ \nu Z ^ \rho 
	Y ^ \sigma d \tau d \lambda + Y ^ \nu Y ^ \sigma Z ^ \rho d \lambda ^ 2\right] 
\] Now, subtracting one from the other gives us our Riemann 
curvature tensor in normal coordinates, and hence since 
both sides are tensors, this holds in all coordinate systems. 
\begin{align*}
	\Delta Z ^ \mu _ r &=  Z _ r ^ \mu - ( Z _ r ' ) ^ \mu  \\
			   &=  \left(  \Gamma ^ \mu_{ \rho \nu , \sigma  }
			   - \Gamma ^ \mu_{ \rho \sigma , \nu } \right) ( Y^ \nu 
			   Z ^ \rho X ^ \sigma )_ p d \lambda d \tau  \\
			   &= ( R \indices{ ^ \mu _{ \rho \nu \sigma }} Y ^ \nu 
			   Z ^ \rho X ^ \sigma )_ p d \lambda  d \tau  \\
\end{align*} Going into the last line, 
we've once again used the fact that since our expression is a 
tensor in normal coordinates, this holds in all coordinate systems. 
