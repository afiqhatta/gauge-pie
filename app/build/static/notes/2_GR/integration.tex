\subsection{Integration} 
On a manifold, we integrate functions 
\[
	f : \mathcal{ M } \to \mathbb{ R } 
\] with the help of a special kind of 
a special kind of top form. 
The kind of form we need is called a volume form or orientation. 
This is a nowhere vanishing top form. Locally, 
it can be written as 
\[
v = v ( x) dx^ 1 \wedge  \dots \wedge  dx^ n , \quad v ( x) \neq 0 !
\] For some manifolds, globally we may not be able to glue
volume forms together over the whole manifold. 
If such a form exists, the manifold is said to be orientable. 
Not all manifolds are orientable, for example the Mobius strip. 
This says that $ v( x) $ must change direction and hence be zero. 
Or, $ \mathbb { RP } ^ n $. 
Given a volume form, we can integrate 
any function  $f: M \to \mathbb{ R} $ over  $ \mathcal{ M } $. 
In chart $ \mathcal{ O } \subset \mathcal{ M } $, we define
\[
\int_{ \mathcal{ O } } f v =  \ \int_{\mathcal{ U } } dx^ 1 \dots dx^ n f( x) v( x) 
\] Now, this tells us how to integrate a patch. 
Then, summing over patches gives us the whole integral. 
$ v ( x) $ can be thought of as our measure - ' the volume of some part of 
the manifold'. There is freedom in our choice of volume form here, 
we could've chosen lots of different volume forms which 
satisfy our condition above.

\subsubsection{Integrating over submanifolds} 

We haven't defined how to integrate, say a function 
over a $ p $-form on an 
$ n $ dimensional manifold, where $ p < n $ 
(since so far our definition of integration 
has only pertained to top forms. 

So to do things like this, we need to 
find a way to 'shift down' into a lower dimensional 
subspace and do things there. This is why we define the concept of a submanifold.

A new manifold $ \Sigma $ of dimension $ k < N $ is called a submanifold
of $ \mathcal{ M  }$ if there exists an injective map $ \phi : \Sigma \to \mathcal{ M }$ 
such that $ \phi^ * : T_p ( \Sigma ) \to T_p ( \mathcal{ M } ) $, the 
\textbf{pullback} of $ \phi $, is also 
injective. We require the condition of injectivity 
so that our submanifold doesn't intersect itself 
when we embedded it in our larger manifold.
The first condition is so that there are no crossings. 
The second condition is there so that we have no cusps in our tangent space.

We're now fully equipped to integrate over 
some portion of a submanifold of $ \mathcal{ M } $. 
You can think of this as a 'surface' or 'line' of 
some sort embedded in our manifold. 
We can integrate any $ \omega  \in \Lambda^ k ( \mathcal{ M } ) $ over  $\Sigma $ 
by first identifying it with the embedded portion of $ \Sigma $
in  $ \mathcal{ M } $, and then pulling it back into $\Sigma $
itself. Now, we're in a  $ p $ dimensional space, 
and we know how to integrate here since $ \phi ^ * \omega $ is 
now a top-form
\[
\int_{ \phi ( \Sigma ) } \omega = \int_{ \Sigma } \phi^ * \omega
\]  
Let's do an example where we integrate say over a line 
embedded in a bigger manifold. 
We define an injective map $ \sigma $ which 
takes our line $ C $ into our manifold. 
\[
\sigma : C\to \mathcal{ M }
\] defines a non intersecting curve in M. Then, for 
$ A \in \Lambda ^ 1 ( \mathcal{ M } ) $, we have, 
integrating over our embedding of our line in $\mathcal{ M } $, 
\[
\int_{ \sigma ( C) } A = \int_{ C } \sigma^ * A = \int d \tau A_{ \mu } ( x) \frac{ dx^ \mu }{d \tau }
\] 
The last equality comes from the fact that the 
components of a one-form pulled back transforms as
\[
 ( \sigma ^ * A )_\nu = A_\mu  \frac{ d x ^ \mu  }{ dt ^ \nu }
\] where in this case, since 
we're pulling back to a one dimensional 
manifold, we only have $ \nu = 0 $  (which we write
for brevity as $ \tau $). 

\subsubsection{Stokes' theorem}

\textit{How does this tie in to the bigger picture? }  
\textit{What use does Stoke's theorem have in general relativity?} 
\textit{How do we prove Stoke's theorem?} 


\begin{defn}{(Boundaries).} 
So far we've only considered manifolds which are smooth. 
However, we can 'chop off' a portion of 
our manifold to make a slightly different map that what we 
are used to, a new map 
\[
\phi_\alpha : \mathcal{ O }_\alpha \to \mathcal{ U }_\alpha \subset \frac{1}{2 } \mathbb{R}^n  = 
\left\{ (x_1, \dots x_ n ), \mid x_1 \geq 0  \right\} 
\] Our boundary of our manifold is the
set of points on $ \mathcal{ M }  $ which are mapped to 
$ ( 0 , x_2 , \dots x_ n ) $. 
Boundaries are $ n - 1 $ dimensional manifolds 
of our original manifold which has dimension $n$

\end{defn}

\begin{figure}[h]
	\centering
	\includegraphics[width=0.5\linewidth]{figures/submanifold.png}
	\caption{Here we have a manifold with boundary.}%
\end{figure}

\begin{thm}{Stokes' Theorem.}
Let $ \mathcal{ M } $ be a manifold with boundary. 
This is a manifold which just stops and gets cutoff somewhere. 
If we call the manifold $ \mathcal{ M } $, we call the boundary 
$ \partial  \mathcal{ M }$. 
If we take $ \omega  \in \Lambda^{ n - 1} ( \mathcal{ M } ). $
Our claim is that
\[
\int_{ \mathcal{ M } } d \omega = \int_{ \partial  \mathcal{ M }} \omega
\] This Stokes' theorem. It's 
presented in a more general form 
than what we're used to, but we'll see in the 
examples that this way of expressing this 
gives us both Stokes' theorem in three dimensions 
as well as Green's theorem. 

\end{thm}

\begin{example}{Stokes' theorem in one dimension. } 
Take $ \mathcal{ M } $ as the interval $ I $ with $ x \in [ a, b ] $. 
$ \omega ( x)  $ is a function and 
\[
d \omega = \frac{ d \omega }{ dx  } \cdot  dx
\] We have that 
\[
\int_{ \mathcal{ M } } d \omega = \int_a^ b \frac{ d \omega }{ dx } dx \quad \int_{ \partial  \mathcal{ M } } \omega = \omega( b ) - \omega ( a) 
\] In one dimension, we've
recovered integration by 
parts on a line. 

\end{example}

\begin{example}{Stokes' theorem in two dimensions. } 
In the second case, we have that 
\[
M \subset \mathbb{ R} ^ 2, \omega \in \Lambda^ 1 ( \mathcal{ M } ) 
\] This recovers 
\[
\int_{ \mathcal{ M } } d \omega = \int_{ \mathcal{ M } } \left( \frac{\partial \omega_2 }{\partial  x^ 1 }  - \frac{\partial  \omega_1 }{\partial  x^ 2}  \right) dx^ 1 \wedge  dx^2  
\] By Stokes theorem this is 
\[
\int_{ \partial  \mathcal{ M  } } \omega = \int_{ \partial  M } \omega_1 dx^ 1 + \omega_2 dx^ 2  
\] This equality is Green's theorem in a plane. 

\end{example}

\begin{example}{Stokes' theorem in three dimensions.} 
Finally, take $ \mathcal{ M } \subset \mathbb{ R}^ 3 $ and $ \omega \in \Lambda^ 2 ( \mathcal{ M } ) $
\begin{align*}
\int_{ \mathcal{ M } } d \omega & = \int dx^ 1 dx^ 2 dx^ 3 ( \partial_1 \omega_1 + \partial_2 \omega_2 + \partial  _ 3 \omega _ 3 ) \\
\int_{ \partial  M   } \omega  &= \int_{ \partial  M  } \omega_1 dx^ 2 dx^ 3 + \omega_ 2 dx^ 3 dx^ 1 + \omega_3 dx^ 1 dx^ 2 
\end{align*}

Equating the two expressions above 
gives us our usual notion of Stokes' theorem in three dimensions, which 
is disguised as Gauss' divergence theorem. 
\end{example}

\pagebreak 
