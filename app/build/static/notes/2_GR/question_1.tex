\subsection{Question 1} 
If we're given compoonents of a vector field and want to solve for its integral curve, then we need to solve the equation
\[ 
\left. \frac{ d x^\mu(t) }{ dt} \right\vert_{\phi(p)} = \left. X^\mu( x^\nu(t) ) \right\vert_{ \phi( p)}
\] 

So for the first integral curve, we need to solve the system 
\begin{align*} 
\frac{dx}{dt}  &= y \\
\frac{dy}{dt}  &= -x 
\end{align*} 
This is made a lot easier by writing out the system in polar coordinates (which is indeed a different cart for the manifold $\mathbb{R}^2$, and writing $(x , y)  = ( r \cos \theta, r \sin \theta)$ with the chain rule gives us 
\begin{align*} 
\dot{r} \cos \theta  - r \dot{\theta} \sin \theta  &= r \sin \theta \\
\dot{r} \sin \theta  + r \dot{\theta} \cos \theta &=  - r \cos \theta 
\end{align*} 
If we multiply the first equation by $\sin \theta$ and the second equation by $\cos \theta$, and then subtract the first equation from the second equation, we've eliminated the $\dot{ r}$ term. We're left wwith 
\[ 
\dot{\theta} =  -1 \implies \theta  =  - t + C
\] 
for some constant $C$. Substituting in $\dot{\theta} = - 1$ in our first equation gives the condition that $\dot{r} = 0 \implies r = R$ for $R$ constant. Hence our integral curves are merely circles of arbitrary radius about the origin. 

For our second vector field 
\[ 
X^\mu  = ( x - y, x +y ) 
\] 
we proceed exactly as before, with polar coordinates. One finds instead that $\dot{\theta} = 1$, and hence that $\dot{r}  = r$. Thus, we have that 
\[
\theta = t + A, \quad r = Be^t 
\] 
for arbitrary constants $A, B$. These curves are spirals. 

\pagebreak
