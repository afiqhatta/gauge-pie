\subsection{Taking the Lie Derivative of Covectors} 
We would like to repeat what we did for vectors, and take derivatives of covectors. 
To do this, we need to define pull-backs for covectors.
Suppose we had a covector $\omega$ living in the tangent space 
of some manifold $N$, in $T_p^*( N)$. 
We can then define the pull back of this vector field based on first pushing forward 
the vector field $X$. Thus, if $\phi : M \to  N$, then we define the pullback 
$\phi^ * \omega $ as the covector field in $T_p ( M ) $ as 
\[
( \phi^* \omega ) ( X)  = \omega ( \phi_* X ) 
\] What information can we glean from this? 
Well, we can try to figure out what the components of $\phi^ * \omega $ are. 
If we let $\{ y ^ \alpha \} $ to be coordinates on $N$, 
then we expand this covector as 
\[
\omega = \omega_\mu  dy^\mu  
\] Also recall that the components of a pushed forward vector field are 
\[
( \phi_* X )^ \mu   = X^\nu \frac{\partial y ^ \mu  }{ \partial x^\nu }
\] Now, if we take the equation 
\[
( \phi ^ * \omega )  = \omega ( \phi_* X ) 
\] Then, expanding out in terms of components, we have that 
\[
( \phi^* \omega ) ^\mu  dx_\mu  ( X^\nu  e_\nu  )  = w^\mu  dy_\mu  ( \phi_* X)^\nu  \frac{\partial }{\partial y ^\nu  }
\] Remember, we have to expand in the correct coordinates. The object $ \phi^* \omega $ lives in 
the space $M $ so we expand in the $ dx^ \mu  $ basis. On the other hand
we have that $\omega $ originally lives in $N $ so we expand in 
$ dy^ \mu  $. 
Substituting our expression for our push forward vector field, 
and we get that 
\[
( \phi^ * \omega )_\mu  X^\mu  = \omega_\nu  \frac{\partial y^\nu  }{\partial x^\mu }X^\mu  
\] This step requires a bit of explanation. 
After we substitute in the components for the pushed forward vector field, 
we then use the fact that on both manifolds, our basis vectors and 
our basis covectors contract to give a delta function 
\[
dx^\mu  \left(  \frac{\partial }{\partial x^ \nu } \right)= \delta \indices{^\mu  _\nu }, \quad dy^\mu  \left( \frac{\partial }{\partial y^\nu  }  \right) = \delta \indices{^\mu _\nu }						  
\] This implies that the components of our pulled back vector field are 
\[
( \phi^* \omega )_\mu  = \omega_\nu  \frac{\partial y^\nu }{\partial x^\mu  }
\] 
As in the case of vectors, we can also make rigourous the definition of a Lie derivative 
with respect to a covector field. 
This is denoted $\mathcal{L}_X \omega$, where $X$ is our underlying vector field
we're differentiating with. 
If our vector field $X$ imposes a flow map which we label as 
$\sigma_ t$, then our corresponding Lie derivative is derivative is defined as 
\[
\mathcal{ L }_X \omega  =\lim_{t \to 0} \frac{( \sigma_t^* \omega)   - \omega }{t }	 
\]There's an important point to be made here. 
In our previous definition of a Lie derivative for a vector field, 
we took the inverse diffeomorphism $\sigma_{ - t} $. 
But in this case, we need to take $t $ positive since 
we're doing a \textbf{pull-back} instead of a pushforward, like we did 
with vector fields. 

Let's go slow and try to compute 
the components of this derivative. 
Recall that for a flow map, we have that infinitesimally, 
\[
y^\nu  = x^\nu  + t X^\nu , \implies \delta \indices{^\nu _\mu  } + t \frac{dX^\nu }{dx^\mu  }	  
\] Thus for a general covector field, our components for the pull back 
are 
\[
( \sigma_t \omega )_\mu  = \omega_\mu  + t \omega_\nu \frac{dX^\nu }{dx^\mu  }
\] Hence, the components of a basis element under this flow becomes 
\[
( \sigma_t^* dx^\nu )  = dx^ \nu  + t dx^\mu  \frac{dX^\nu  }{dx^\mu  }
\] So, taking the limit, we have that our components of our 
Lie derivative are given by 
\[
\lim_{t \to 0  } \frac{\sigma^*_t dx^\nu   - dx^\nu  }{ t }  = dx^\mu  \frac{dX^\nu }{dx^\mu  }		 
\] Now, as before, we impose the 
Liebniz property of Lie derivatives, and expand out a general covector. 
Hence, we have that 
\begin{align*}
\mathcal{L}_X( \omega_\mu  dx^\mu  ) &= \omega_\mu  \mathcal{L}dx^\mu  + dx^\mu  \mathcal{L}_X( \omega_\mu  )  \\
 &= \omega_\mu  dx^ \nu  \frac{dX^\nu  }{dx^\mu  } + dx^ \nu  X^\mu   \frac{d \omega_\nu }{dx^ \mu  }
\end{align*} This implies that 
our components of our Lie derivative can be written nicely as 


\[
\mathcal{L }_X( \omega )_\mu   = ( X^\nu \partial_\nu \omega _\nu + \omega _\nu \partial_\mu X^\nu ) 
\]  
