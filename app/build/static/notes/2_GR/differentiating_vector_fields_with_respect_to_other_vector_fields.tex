\subsection{Differentiating vector fields with respect to other vector fields}

\subsubsection{Push-forwards and Pull-backs} 
A sensible question to now as is that, since we have these smooth vector fields, how do we differentiate a vector field with respect to another one? For example, if we have $X, Y \in \mathcal{M}$, what constitutes the notion of a change in $X$ with respect ot $Y$. The notion of derivatives on manifolds is difficult because we can't compare tangent spaces at different points in the manifold, for example $T_p(M )$ and $T_q(M) $ are tangent spaces at different points, and we could define different charts for each space, hence we have some degrees of freedom (and our derivative  wouldn't make sense).  To make sense of comparing different tangent spaces, we need to create way to compare the same functions, but on different manifolds. These are called push forwards and pull backs. 

Let's start by defining a smooth map between manifolds $\phi : M \rightarrow N$ ($\phi$ is not a chart here). We're not assuming that $M$ and $N$ are even the same dimension here, and so we can't assume $\phi^{ -1} $ doesn't even exists. 

Suppose we have a function $f: N \rightarrow \mathbb{R} $. How can we define a new function based on $f$ that makes sense, which goes from $M \rightarrow \mathbb{ R} $? We define the pull back of a function $f$, denoted $(\psi^* f) : M \rightarrow \mathbb{R} $ as 
\[
( \psi^* f ) ( p )  = f ( \psi ( p ) ), \quad p \in \mathbb{ M } 
\] So, we've converted this thing nicely. 

Our next question then is how, from a vector field $Y \in \mathcal{X} ( M )$, can we make a new vector field in $X \in \mathcal{X} ( M ) $? We can, and this is called the push-forward of a vector field, denoted $ \phi_* Y \in \mathcal{ X} ( N )$. We define that object as the vector field which takes 
\[ 
(\phi_*  Y ) ( f)  = Y ( \phi^* f ) 
\] 
This makes sense because $\phi^* f \in C^{\infty} ( M )$, so applying $Y$ makes sense. Now, to show $\phi_* Y \in \mathcal{X} ( N )$, we should verify that 
\[ 
\phi_* Y : C^\infty ( N ) \rightarrow C^\infty( N ), \quad f \mapsto C^{\infty} 
\] 
Well, this object philosophically maps 
\[ 
 f \mapsto \phi_* Y ( f)  = Y ( \phi^* f ) 
\]  But the object on the left hand side is a vector field ready to be turned into a tangent vector when we assign it to a point on the manifold: 
\[ 
p \mapsto Y_p ( \phi^* f ) 
\] Hence this object agrees with our definition. The fact that we have a map $\phi: M \rightarrow N$ and are pushing the vector field from $\mathcal{ X} ( M ) $ to $\mathcal{ X} ( N )$ is the reason why we call this new mapping a push forward. 

\subsubsection{Components for Push-forwards and Pull-backs} 
Now, since $\psi_* Y $ is a vector field, it's now in our interest to find out about what the components are for this thing. We want to find that the components $( \psi_*Y )^\nu$ such that 
\[ 
\psi_* Y  = ( \psi_* Y )^\nu \partial_\nu 
\]
We can work first by assigning coordinates to $\phi( x )$, which we denote by $y^\alpha  ( x) = \phi( x), \quad x\in M,  \alpha = 1, \dots dim(N)$. 
If we write out our vector field $Y$ as $Y  = Y^\mu\partial_\mu$, then our push-forward map in summation convention looks like 
\[ 
(\phi_* Y ) f = Y^\mu \frac{\partial f ( y ( x))}{ \partial x^\mu}  = Y^\mu \frac{ \partial y^\alpha }{ \partial x^\mu } \frac{ \partial f}{ \partial y^\alpha } 
\] 
In the second equality, we've applied the chain rule. Remember, $y$ pertains to coordinates in the manifold $N$, so on our push-forward, we have that our new components on the manifold $N$, we have that 
\[ 
(\phi_* Y )^\alpha  = Y^\mu \frac{ \partial y^\alpha }{ \partial x^\mu } 
\] 

\subsubsection{Introducing the Lie derivative} 
In what we've just presented, some objects are naturally pulled back and some are naturally pushed forward. However, things become when our map between manifolds is a diffeomorphism and hence invertible; which means we can pull back and push forward with whatever objects we want. 
We can use this idea to differentiate vector fields now. Recall that if we've given a vector field, $X \in \mathcal{X} (M )$, we can define a flow map $\sigma_t : M \rightarrow M$ This flow map diffeomorphism allows us to push vectors along flow lines in the manifolds, from the tangent spaces 
\[ 
T_p ( M ) \rightarrow T_{\sigma_t ( p )} ( M ) 
\] 

This is called the Lie derivative $\mathcal{L}_X$, a derivative which is induced by our flow map generated by $X$. For functions. we have that 
\[ 
\mathcal{L}_X f = \lim_{ t \rightarrow 0 } \frac{ f( \sigma_t ( x ))  - f( x) }{ t }  = \left. \frac{ df ( \sigma_t( x) ) }{ dt } \right\vert_{ t =0 } 
\] However, the effect of doing this is exactly the same as if we were to apply the vector field $X$ to the function: 
\[ 
\frac{ df}{ dx^\mu } \left. \frac{ dx^\mu } { dt } \right\vert_{ t = 0 }  = X^\mu \frac{ \partial f }{ \partial x^\mu }  = X( f)
\] Thus, a Lie derivative specialised to the case of functions just gives us $\mathcal{L}_X ( f)  = X( f) $. 
Now, the question is about how we can do this differentiation on vector fields. What we need to do is to 'flow' vectors at a point back in time to where they originally started, and look at this difference. 
\[ 
( \mathcal{L}_X Y )_p  = \lim_{ t \rightarrow 0 } \frac{ ( \sigma_{-t }^* (Y)_p - Y_p )  }{ t } 
\] 

Let's try and compute the most basic thing first, the Lie derivative of a basis element of the tangent space  $\partial_\mu$: 
\[
\sigma_{ -t}^* \partial_\mu = (\sigma_{ -t}^* \partial_\mu)^\nu \partial_\nu
\] Let's try and figure out what $ (\sigma_{ -t}^* \partial_\mu)^\nu$ is. Because of the fact that the diffeomorphism is induced by the vector field $X$, we have that 
\[
( \sigma_{ -t}^* (x))^\nu = x^\nu  - t X^\nu + \dots
\] Thus our components of a push-forward of an arbitrary vector field are given by 
\[ 
( \sigma_{ -t}^*  Y )^\nu  = Y^\sigma \frac{ \partial ( \sigma_{ -t } ( x) )^\nu}{ \partial x^\sigma } , \quad \text{ in our case }  Y^\sigma = \delta\indices{^\sigma_\mu} 
\] 
Substituting the expressions above with one another gives us that 
\[ 
( \sigma_{ -t} ( x) \partial_\mu )^\nu  = \delta\indices{^\nu_\mu}  - t \frac{ \partial X^\nu }{ \partial x^\mu } + 
\] Contracting this with $\partial_\nu$, and then subtracting off $\partial_\mu$, we have that
\[ 
\mathcal{L}_X \partial_\mu =  - \frac{ \partial X^\nu}{ \partial x^\nu} \partial_\nu 
\] We require that a Lie derivative obeys the Leibniz rule, so we have that applying on a general vector field $Y$, 
\begin{align*}
\mathcal{L}_X (Y)  & = \mathcal{L}_X( Y^\mu \partial_\mu ) \\
				&= \mathcal{L}_X ( Y^\mu ) \partial_\mu + Y^\mu \mathcal{L}_X ( \partial_\nu ) \\
				&= X^\nu \frac{ \partial Y^\mu }{ \partial x^\nu } \partial_\mu - Y^\mu \frac{ \partial X^\nu } { \partial x^\mu } \partial_\nu 
\end{align*} 
We realise that this however are just components of the commutator! So
\[
L_X ( Y ) = [ X, Y ] 
\] 

