\subsection{Construction of Lie algebras from Lie groups}
We show that every Lie group corresponds to a Lie algebra.

\subsubsection{Preliminaries} 
Let $ \mathcal{ M } $ be a smooth manifold of dimension D and $ p \in \mathcal{ M }$ a point. Introduce some coordinates $ \left\{  x^{ i  } \right\}, i = 1, \dots D $ in some region $ p \subset \mathcal{ M }$. Now, we choose coordinates such that our point $ p $ corresponds to origin $ x^{ i }  = 0 , i = 1, \dots D$. 

The tangent space $T_{ p } ( \mathcal{ M } ) $ to $ \mathcal{ M } $ at $ p $ is a D-dimensional vector space spanned by differential operators 
$ \frac{\partial  }{\partial x^{i }} , i = 1, \dots D $
which acts on functions $ f : \mathcal{ M  } \to  \mathbb{R} $. A tangent vector 
\[
	V = v^{i } \frac{\partial }{\partial x^{i  } } \in \mathcal{ T }_p ( \mathcal{ M }), \quad v^{ i } \in \mathbb{ R} 
\] This objects acts on functions $ f = f ( x) $ as 
\[
	V \cdot  f  = v^{ i } \frac{\partial  f ( x)  }{\partial x^{ i }} \mid_{ x = 0 } 
\] (Draw diagram of curve on manifold here. 
We now present a different way to think about tangent vectors. 
Consider  a smooth curve from $ \mathcal{ I } \subset \mathbb{ R} $. 
\[
 C : \mathcal{ I } \to  \mathcal{ M }
\] on $ \mathcal{ M } $ which passes through $ p \in \mathcal{ M }$; 
\[
	C : t \in \mathcal{ I } \subset \mathbb{ R} \mapsto x^{ i } ( t ) \in \mathbb{ R}, \quad i = 1, \dots, D 
\] We have that $ \left\{  x^{ i } ( t)  \right\} $ are continuous, differentiable. 
For now, just consider curves which are 
differentiable once. Also we insist that 
\[
	x^{ i } ( 0 )  = 0 \quad \forall i = 1, \dots, D 
\] ( Draw diagram of tangent here ) 
Our tangent vector to $ C $ at $ p $ is 
\[
	V_{ c }  = \dot { x}^{ i } ( 0 ) \frac{\partial }{\partial x^{ i }} \in \mathcal{ T }_p ( \mathcal{ M }), \quad \dot{ x}^{ i } (t ) = \frac{d x^{ i }( t) }{ dt }	\] The interpretation 
	of this tangent vector is nothing but the velocity of a 
	particle, say. 
Every smooth curve has a tangent vector at every point, even if 
the curve has closed endpoints, say 
\[
 J = [ 0 , L ] 
\] This tangent can be taken from derivatives in interior. 
Conversely, given an element of the tangent space. 
we can find a curve which yields that tangent vector 
by fixing the first term of its Taylor expansion. 

\pagebreak
