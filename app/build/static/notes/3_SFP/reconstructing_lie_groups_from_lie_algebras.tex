\subsection{Reconstructing Lie Groups from Lie Algebras} 

\subsubsection{Actions: a natural maps on Lie groups} 
So far, we've learnt about defining tangent spaces at $e \in G$, or if we're working in a $D$ dimensional Lie group, we can define a set of coordinates $\{ \theta_i \}_i$ for $i = 1, \dots, D$, we would identify this $e \in G$ with $\theta = 0$. But, what about tangent spaces at other points on the manifold?

For this reason, we define left and right actions. A Lie group is a very special type 
of manifold in that we
can define smooth maps which 
correspond to group mutliplication by an element. 
This is akin to having the Lie group $G$ act on itself by shuffling the group elements around by a fixed $h \in G$. This can be done either from the left or from the right. Define 
\begin{align*} 
	L_h: \quad & G \rightarrow G \\
	&	g \mapsto hg \\
	R_h: \quad & G \rightarrow G \\
	& g \mapsto gh
\end{align*}
We will now show that these maps are diffeomorphisms on our group manifold. 
First we show surjectivity. That is, we wish to show that 
\[
	\forall g' \in G, \exists g \in G , \text{such that } L_h ( g )  = g'  
\] Clearly, setting $ g = h ^ { -1 } g '$, satisfies this condition. 
These maps are also injective because
 \[
 \forall g , g' \in G, L_h( G ) = L _ h  ( g' ) \implies g = g' 
\] This is because $ L_h ( g) = L _ h ( g ' ) \implies hg = hg' \implies g = g $. 
Surjectivity and injectivity implies that this map is bijective. This means there's
an inverse map 
 \[
	 ( L _ h ) ^ { - 1}  = L _{ h ^ { - 1} }
\] which exists. Now, we appeal to the fact 
that group multiplication on a Lie group is a smooth operation 
to show that left actions are a diffeomorphism on $ \mathcal{ M }( G  ) $ . 
Now, this can be shown by introducing coordinates $ \left\{  \theta ^ i, \quad i = 1 , \dots D   \right\}  $ 
If we set $ g = g ( \theta ) \subset G $, and $ g ( 0 ) = e $, 
Let 
 \[
	 g ' = L_h ( g ) = h \cdot  g ( \theta ) = g ( \theta' ) 
\] assuming the same coordinate patch for $ g ' $. 
In coordinates,  $ L_ h$ is specified by
\[
	\theta^{ ' i  } = \theta ^{ 'i } ( \theta) , \quad i = 1, \dots D , \mathbb{ R} ^ D \to \mathbb{ R} ^  D
\] But this is just group multiplication 
on our Lie group, which was one of our starting 
axioms. Thus, we have a diffeomorphism. 
The fact that our left action has an inverse also implies 
that the Jacobian matrix 
\[
J \indices{ ^ i _ j } ( \theta)  = \frac{\partial \theta ^{ ' i } }{\partial \theta ^  j }  
\] is invertible. The inverse $ J ^ { -1 } $ exists, so $ det J \neq 0 $. 
There's more stuff which we an get out of this. 

\subsubsection{Actions move us between tangent spaces} 
This diffeomorphism 
also induces another map $ L^ * _{ h } $ from $ \mathcal{ T }_ g ( G ) $
to another tangent space $ \mathcal{ T }_{ hg } ( G ) $. So, 
we are 'moving' our tangent space along this line. 
This map is given by 
\begin{align*}
	L^*_{ h } & : \mathcal{T }_ g ( G ) \to \mathcal{ T }_{hg } ( G ) \\
	L^*_{ h } & : v = v^{ i } \frac{\partial }{\partial \theta ^ i } \in \mathcal{ T }_ g ( G ) \mapsto v' = v^{ ' i } \frac{\partial }{\partial \theta'^{i } } \in \mathcal{ T }_{ hg }( G ) 
\end{align*} In the above, we have from the chain rule that 
\[
	v^{' i }  = \mathcal{ J }( \theta )^ i _j v ^ j 
\]  We refer to the map $ L_ h ^ * $ as the differential map of 
$ L _ h $. We identify a left action diffeomorphism 
as a \textbf{change of coordinates}, and hence 
the motivation from this map is a chain rule transformation. 


From this, we define an object called a \textbf{vector field}. 
A vector field on $ G $ specifies a tangent vector
\[
	v ( g) \in \mathcal{ T }_g ( G ) , \quad g = g( \theta ) 
\] So, given a point on the manifold we get a tangent vector. 
In coordinates, we write $ v $ as an object parametrised with $ \theta $ 
so that 
\[
	v ( \theta) = v ^ i ( \theta) \frac{\partial }{\partial \theta^ i} \in \mathcal{ T }_{ g ( \theta) }( G )  
\] Now, $ v  $ is smooth if $ v^ i ( \theta) $ are continuous and differentiable: 
(when we move across our Lie group manifold, our 
tangent vector $ v $ doesn't change abruptly. 
We can take a vector $w$ in our Lie algebra, $ w  \in \mathcal{ T }_ e( G ) $, 
and from this we can define a vector field 
\[
	v ( g ) = L^ *_g ( w ) \in \mathcal{ T }_ g ( G ), \quad \forall g \in G 
\] As $ L ^ * _ g $ is smooth and invertible, we 
have that $ v ( g ) $ is smooth and invertible, and non vanishing. Our matrix
$ \mathcal{ J }  $ has no zero eigenvalues and so $ v ( g ) $ is non vanishing. 
Starting from a basis  $ \left\{  \omega_ a  \right\} , a = 1 , \dots D $ for $ \mathcal{ T }_ e ( G ) $, 
we have $ D $ independent, non vanishing vector fields 
\[
 v_a ( g)  = L ^ * _ g ( \omega_ a ) 
\] The Lefschetz fixed point theorem, or 'hairy ball theorem' 
states that any smooth vector field on $ S^ 2 $ has at least $  2$ zeros. 
Thus for our manifold, we have $ \mathcal{ M } ( G ) \neq S^ 2 $. In fact, 
out of two manifolds which are compact, the only manifold possible for a 
Lie group is that 
\[
	\mathcal{ M } ( G )  = T ^2, \quad G = U ( 1) \times U ( 1) 
\] In fact, this technology of left invariant of vector fields 
is enough technology to define Lie groups, without resorting to matrix groups 
as we have done previously.

\subsubsection{Left actions on Matrix Lie Algebras} 
Now let's think about a matrix Lie group, and apply this technology 
of left and right multiplication maps. Suppose we had a matrix 
Lie group $ G \subset Mat_n ( F) $ for  $ n \in \mathbb{ N } $ 
with our field $ F = \mathbb{ R} $ or  $ \mathbb{ C} $, 
then our left multiplcation map induces a differential map which is 
matrix multiplication. 
So,  $ \forall h \in G, X \in \mathcal{ L }( G ) $, we have that 
\[
	L^ * _ h  ( X ) = h X \in \mathcal{ T }_h ( G ) 
\]
Note that this only works in the language of matrix
Lie groups and Lie algebras because everything is a matrix, 
so $ h X $ makes sense. 
To show the above, we apply our usual recipe of finding a curve whose tangent
vector gives $ X $. So, take any curve 
\[
	 	C : t \in \mathcal{ J } \subset \mathbb{ R} \mapsto g ( t) \in G  , \quad g( t) = g( 0 ) + t \dot{g }( 0 ) + O ( t^ 2 )  
\] By definition, we have that 
\[
	\dot{ g }( 0 ) = \left. \frac{ dg ( t)  }{dt } \right\vert_{ t =0 } \in \mathcal{ T }_{ g ( 0 ) }( G )  
\]
We have by construction that $ g ( 0 ) = e = I $, and that $ \dot{g }( 0)  = X  $, with 
$ X \in \mathcal{ L } ( G ) \simeq \mathcal{ T }_e ( G )  $, since our Lie algebra is defined to be the set of tangent vectors at the origin. 
Now we can define a new curve, which is our original curve but 
with the left action $ L _ h  $  applied to it. 
\[
	C' : t \in \mathbb{ R} \mapsto h ( t) =  h \cdot g( t) \in G 
\] which is given by our left multiplication map. So, taking a Taylor 
expansion here we have that 
near $ t = 0$, 
\[
	h ( t) = h +  th X + O ( t^ 2) \implies h X \in \mathcal{ T }_ h ( G )  
\] This is the tangent space at $ h $ because our leading order 
term is $ h \in G$. Hence, since we've exhibited 
a curve whose vector is $ h X $ at $ h  $, we've shown that 
\[
	h X \in \mathcal{ T }_ h ( G) 
\] 
We can apply this idea to a smooth curve in general. 
Equivalently, we have that 
a smooth curve in general
\[
 C  : t \in \mathbb{ R} \mapsto g( t) \subset G 
\] for matrix lie groups we have that 
\[
	\dot{ g }( t) \in \mathcal{ T }_{ g ( t) } ( G ) \implies g ^{- 1} ( t) \dot{ g }( t)  = L ^ * _{ g ^{ -1 } ( t) } ( \dot{g }( t) ) \in \mathcal{ T }_{ g ^{ - 1  } ( t) g ( t) } =  \mathcal{ T } _ e ( G ) \simeq \mathcal{ L } ( G )   
\]
We use this fact to show that we 
can reconstruct Lie groups from Lie algebras. Working in the 
opposite direction, if we have a vector $ X \in \mathcal{ L } ( G ) $, 
we can reconstruct the curve 
\[
	C : \mathcal{ J } \subset \mathbb{ R} \rightarrow G 
\] by solving the ordinary differential equation that we derived earlier
\[
	g ^{ -1 } ( t) \frac{ d g( t) }{ dt } =  X, \quad g ( 0 ) = I  
\]  To solve this, we define the exponential of a matrix 
$ M \in Mat_ n ( F) $ as 
\[
	\text{Exp} M : = \sum_{n=0}^{\infty} \frac{1}{l ! } M ^ l \in Mat_n ( F) 
\] 
We solve our differential equation above by 
setting 
\[
	g ( t) = \text{Exp } ( tX) 
\] We check that this indeed solves the equation. 
We have that $ g ( 0 ) = \text{ Exp } ( 0 ) = 1$. 
\[
	\frac{dg ( t) }{ dt }  =  \sum_{n=1}^{\infty} \frac{1}{( l - 1 ) ! } t ^{ l - 1} X^ l = \text{ Exp } ( tX) \cdot X = g ( t) X 
\] Thus, it must be the case that 
$ \text{ Exp } ( tX ) \in G, \forall t \in \mathbb{ R} , \forall X \in \mathcal{ L } ( G )$. 
So we know a group determines a Lie algebra, but 
to what extent does a Lie algebra determine a Lie group? 
An exercise, we should check that 
if $ X \in \mathcal{ L } ( SU ( n ) ) $, we check that 
\[
	\text{ Exp } ( t X ) \in SU ( N ) \forall t \in \mathbb{ R} 	
\] 

\subsubsection{Reconstructing $ G $ from  $ \mathcal{ L } ( G ) $} 
If we set $ t = 1$, 
we have that 
 \[
	 \text{Exp } : \mathcal{ L  }( G ) \to G 
\]
This means that if we have a map $ X \in \mathcal{ L } ( G ) $ , 
exponentiating a this thing gives us an element in the Lie group, so $ \text{Exp} ( X) \in G $. 
If we think of exponential map as a map on the complex numbers, 
the inverse is the logarithm function which has branch points. 
So, we can't expect this map to be globally bijective! 
However, we can make a slightly weaker statement that the exponential map 
\[
	\text{Exp} L \mathcal{ L } ( G ) \to G 
\]  is bijective in some neighbourhood of the 
identity element $ e $ in our Lie group. 
The point is that given some knowledge of our Lie bracket, 
we can then locally reconstruct the Lie group. 
Hence, given $ X, Y \in \mathcal{ L } ( G)  $, we can 
construct 
\[
	g_{ X }  = \text{Exp} ( X) , g_Y  = \text{Exp} (Y) \in G 
\] We have that 
\[
	g_X g_Y  = g_Z = \text{Exp} ( Z ) \in G, \text{for some } Z \in \mathcal{L }(G ) 	 
\] near the identity. 
By the Baker-Campbell-Hausdorff formula, 
we have that $ Z $ is related to our vectors $ X, Y $ by 
\[
	Z = X + Y + \frac{1}{2 } [ X, Y ] + \frac{1}{12 } \left( [ X, [ X, Y ] ] - [ Y , [ X, Y ] ]  \right) 
\] Under a rescaling, we have that 
$ X \to \epsilon  X , Y \to \epsilon Y $  gives a small series in which 
our formula above is valid. 
Note that the formula is given purely in terms of sums of nested commutators, 
and the sum of these expressions. This means that $ Z $ \textbf{will be } in the 
Lie algebra. 
This reflects the fact that the formula is given in terms of the Lie algebra. 
This means we can apply this to all Lie algebras, 
not just matrix Lie algebras. 
Using the jacobi identity, we have that the BCH formula is 
not a unique expression. 

Now, what about away from the identity? 
Our conclusion from the previous part was that $ \mathcal{ L } ( G ) $ 
only determines $ G $ in the identity, in other words within the 
radius of convergence of our BCH formula. 
An example that the exponential map is not surjective is that 
 $ \text{Exp  } $ clearly is not surjective when $ G $ is not connected. 
 As an example, consider $ G = O ( n ) $. Our Lie algebra is 
 \[
	 \mathcal{ L } ( G )  = \left\{  X \in \text{Mat}_m ( \mathbb{ R }) \mid X^ T + X = 0  \right\} 
 \]   Now, $ X \in \mathcal{ L } O ( n ) \implies \tr X = 0 $. Now 
 examine $ \text{Exp} ( X)  $ for $ X \in \mathcal{ L } ( O ( n ) ) $. 
 We have that 
 \[
	 \det ( \text{Exp} ( X) ) = \text{exp} ( \tr X )  = + 1 
 \] This means that $ \text{Exp } ( X) \in SO ( n ) $. 
 In full generality, when $ G $ is compact, the image of 
 the lie algebra $ \mathcal{ L } ( G ) $ under the exponential map 
 is the \textbf{connected component} of the identity. 
This breaks down when $ G $ is not compact. 

Also, the exponential map is not 
injective either. This is always the case when $ G $ has a $ U ( 1) $ 
subgroup. Our trivial example is $ U ( 1 ) $ itself. 
Consider 
\[
	\mathcal{ L } ( U ( 1) )  = \left\{  X \in i x \in \mathbb{ C} \mid x \in \mathbb{ R}  \right\} 
\] We have that  $ g  = \text{Exp}( X )  = \text{exp}( i x ) \in U ( 1)  $. 
Our Lie algebra elements $ i x $ and $ i x + 2 \pi i $ yield 
the same group element. 

Let's explore the relationship between $ SU ( 2 )  $ and $ SO ( 3  ) $ 
some more. We've shown that their Lie algebras are isomorphic. 
\[
	\mathcal{ L } ( SU ( 2 ) ) \simeq \mathcal{ L } ( SO ( 3) ) 
\] Now, even though $ SU (2 )$ is not isomorphic to $ SO ( 3)$, 
We can construct a double-covering, or a globally 2: 1 map. 
We call this map $ d : SU ( 2 ) \to SO ( 3) $. 
It's constructed by 
 \[
 A \in SU ( 2) \mapsto d ( A ) \in SO ( 3 ) 
\] We represent $ A $ as a matrix so that 
 \[
	 d ( A)_{ ij }  = \frac{1}{2 } \tr_{2 } ( \sigma_i A \sigma_{ j } A^ \dagger 
 \]  We need to first establish that our right hand side is indeed in 
 $ SO ( 3)  $. We also need to establish that 
 $ d ( A )  = d ( - A ) $, therefore establishing that this is 
 indeed a double cover. 

 This map provides an isomorphism between $ SO ( 3)  $ 
 and our quotient 
 \[
  SO ( 3) \simeq SU ( 2 ) / \mathbb{ Z}_2 
 \] We quotient by the centre $ \mathbb { Z} _ 2  = \left\{  I_2 , - I_ 2  \right\}  $, 
 which is the set of matrices which commute with everything. 

 Let's think about this in terms of group manifolds. 
 We identify $\mathcal{ M } ( SU ( 2) ) \sim S^ 3 $, where we write this out as 
 \[S^ 3 \sim \left\{  \vec{x} \in \mathbb{ R}  ^ 4 \mid  \| x \| ^ 2 = 1 , x_1^ 2 + x_2 ^ 2 + x_3 ^ 2 + x_4 ^ 2 = 1  \right\} 
\] Our corresponding manifold $ \mathcal{ M } ( SO ( 3) ) $ is $ S^ 3 $ on the surface 
with antipodal points on the three sphere identified, $ \vec{x} \sim  - \vec{x}$. 
This is the same thing as $ S^ 3 _ +  $, which is our upper hemisphere with $ x_4 \geq 0 $, 
along with our equator with antipodal points in our equator. So, we have that
\[
	\mathcal{ M } ( SO ( 3) ) \simeq S^3 _ + \cup \text{Equator with antipodal points identitified} 
\] 
 However, this is the same thing as our three-ball with points on
the boundary identified.

( Insert figure here ) 





\pagebreak 
