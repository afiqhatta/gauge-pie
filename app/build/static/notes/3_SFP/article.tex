\documentclass[11pt, oneside]{article}   	% use "amsart" instead of "article" for AMSLaTeX format
\usepackage[margin = 1.1in]{geometry}            		% See geometry.pdf to learn the layout options. There are lots.
\geometry{letterpaper}                   		% ... or a4paper or a5paper or ... 
\usepackage[parfill]{parskip}    		% Activate to begin paragraphs with an empty line rather than an indent
\usepackage{graphicx}				% Use pdf, png, jpg, or eps§ with pdflatex; use eps in DVI mode
							% TeX will automatically convert eps --> pdf in pdflatex	
\usepackage{adjustbox}	
\usepackage[section]{placeins}



%% LaTeX Preamble - Common packages
\usepackage[utf8]{inputenc}
\usepackage[english]{babel}
\usepackage{textcomp} % provide lots of new symbols
\usepackage{graphicx}  % Add graphics capabilities
\usepackage{flafter}  % Don't place floats before their definition
\usepackage{amsmath,amssymb}  % Better maths support & more symbols
\usepackage[backend=biber]{biblatex}
\usepackage{amsthm}
\usepackage{bm}  % Define \bm{} to use bold math fontsx
\usepackage[pdftex,bookmarks,colorlinks,breaklinks]{hyperref}  % PDF hyperlinks, with coloured links
\usepackage{memhfixc}  % remove conflict between the memoir class & hyperref
\usepackage{mathtools}
\usepackage[T1]{fontenc}
\usepackage[scaled]{beramono}
\usepackage{listings}
\usepackage{physics}
\usepackage{tensor}
\usepackage{tikz}
\usepackage{subfiles}
\usetikzlibrary{decorations.markings}

\usepackage{helvet}

\newtheoremstyle{slanted}
{1em}%   Space above
{.8em}%   Space below
{}%  Body font
{}%          Indent amount (empty = no indent, \parindent = para indent)
{\bfseries}% Thm head font
{.}%         Punctuation after thm head
{0.5em}%     Space after thm head: " " = normal interword space;
{}%         \newline = linebreak
{}%          Thm head spec (can be left empty, meaning `normal')

%% Commands for typesetting theorems, claims and other things. 

\theoremstyle{slanted}
\newtheorem*{thm}{Theorem}
\newtheorem{theorem}{Theorem}
\newtheorem*{claim}{Claim}
\newtheorem*{example}{Example}
\newtheorem*{defn}{Definition}

\newcommand{\Lagr}{\mathcal{L}}
\newcommand{\vc}[1]{\mathbf{#1}}
\newcommand{\pdrv}[2]{\frac{\partial{#1}}{\partial{#2}}}
\newcommand{\thrint}[1]{\int d^3 \vc{x} \left( {#1} \right)}
\newcommand{\lalg}{ \mathcal{ G } } 
\newcommand{\rea}{ \mathbb{ R} } 
\newcommand{\com}{ \mathbb{ C} } 
\newcommand{\al}{ \alpha} 
\newcommand{\be}{\beta}
\newcommand{\ad}{\text{Ad}}
\newcommand{\subalg}{h} 

\title{Part III Symmetries, Fields and Particles}
\author{Notes by Afiq Hatta, based on lectures by Nick Dorey and notes by Nicholas Manton}
\begin{document}
\maketitle
\tableofcontents

\pagebreak 

\subfile{what_is_a_symmetry.tex}
\subfile{groups_to_represent_symmetries.tex}
\subfile{symmetries_correspond_to_conserved_quantities.tex}
\subfile{manifolds_and_tangent_spaces.tex}
\subfile{manifolds.tex}
\subfile{functions_on_manifolds.tex}
\subfile{tangent_vectors.tex}
\subfile{embedding_manifolds.tex}
\subfile{classifying_manifolds.tex}
\subfile{lie_groups.tex}
\subfile{what_is_a_lie_group.tex}
\subfile{matrix_groups.tex}
\subfile{lie_algebras.tex}
\subfile{construction_of_lie_algebras_from_lie_groups.tex}
\subfile{the_lie_algebra.tex}
\subfile{examples_of_derived_lie_algebras.tex}
\subfile{reconstructing_lie_groups_from_lie_algebras.tex}
\subfile{representations.tex}
\subfile{representations_of_lie_algebras.tex}
\subfile{a_canonical_representation_of_lie_groups_with_lie_algebras.tex}
\subfile{classifying_representations.tex}
\subfile{invariant_subspaces_and_reducibility.tex}
\subfile{a_canonical_representation_of_lie_algebras.tex}
\subfile{constructing_a_natural_map_of_between_representations_of_lie_groups_to_representations_of_lie_algebras.tex}
\subfile{_representations_of_su(2)_in_quantum_mechanics.tex}
\subfile{our_choice_of_basis.tex}
\subfile{roots_and_weights.tex}
\subfile{construction_of_an_irreducible_representation.tex}
\subfile{classifying_lie_algebras.tex}
\subfile{basic_representation_theory.tex}
\subfile{structure_of_the_representation.tex}
\subfile{killing_forms.tex}
\subfile{example_sheet_1.tex}
\subfile{question_1.tex}
\subfile{question_2.tex}
\subfile{question_3.tex}
\subfile{question_4.tex}
\subfile{question_5.tex}
\subfile{question_6.tex}
\subfile{question_7.tex}
\subfile{question_8.tex}
\subfile{question_9.tex}
\subfile{example_sheet_2.tex}
\subfile{question_2.tex}
\subfile{problems_left_to_contribute.tex}
\end{document} 
