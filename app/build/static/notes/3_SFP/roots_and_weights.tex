\subsection{Roots and Weights}
In representation theory, for our purposes, roots and weights will be
terms to describe the eigenvalues associated with a certain 
basis element in our Lie algebra and its corresponding
representation. 

\subsubsection{Roots} 
So, in our previous discussion we found that 
the Cartan basis diagonalised the map $\ad_ H $, 
with eigenvalues $0 , -2 , +2$. We shall refer 
to this as the roots of this system.

\subsubsection{Weights} 
Now, let's consider a finite representation of $\mathcal{ L } ( SU ( 2) ) $
with the Cartan basis as a backdrop. Recall that a finite 
representation is a representation whose vector space we're 
acting on is of finite dimension. 
For this analysis, we will make a big simplifying assumption that 
our Cartan element $ H $ has a \textbf{diagonalisable} representation. 
This means, when we map $ H $ into representation space as a 
$ \dim R \times \dim R $ matrix, then $ R ( H ) $ is diagonalisable. 

In other words, there's a basis set of eigenvectors such that 
\[
	R ( H ) \vec{v}_ \lambda = \lambda \vec{v}_{ \lambda } , \lambda \in \mathbb{ C}  
\]  We have that 
$ \lambda \in \mathbb{ C}  $  since we're working 
in complexified space. In addition, these set of eigenvectors $ \vec{v} _ \lambda$ should
also serve as a basis of our representation space $ V$.

Now, the fact that representations of Lie algebras
preserve the Lie bracket becomes very useful here. 
Since we have already worked out our commutation relations, 
we can deduce that 
\begin{align*}
	R ( H ) R ( E_{\pm } ) \vec{v}_{ \lambda } &=  
	\left( R( H ) R ( E_{\pm } ) + [ R ( H ) , R ( E_{ \pm } ) ] \right) \vec{v}_\lambda   \\
						   &=  ( \lambda \pm 2 ) \vec{v}_ \lambda  
\end{align*}
This means that given a weight, we can step up or down to get a new 
eigenvector $ R ( E_{ \pm } ) $ with a new weight 
$ \lambda \pm 2 $. 
