\subsection{Question 7} 

\subsubsection*{Showing that the Heisenberg matrices form a group} 
Group multiplication is an associative operation, so the associative property for this group holds by inheritance. To show closure, we multiply two arbitrary matrices $A, B \in G$, of this form in the group; 
\[ 
AB =  \begin{pmatrix} 1 & a & b \\ 0 & 1 & c \\ 0 & 0 & 1 \end{pmatrix} \begin{pmatrix} 1 & d & e \\ 0 & 1 & f \\ 0 & 0 & 1 \end{pmatrix}  = 
\begin{pmatrix} 1 & d + a & e + af + b \\ 0 & 1 & f + c \\ 0 & 0 & 1 \end{pmatrix} 
\] We observe that the resulting product is of the form required of group elements. By solving for the identity matrix, one can verify that the matrix inverse is given by 
\[ 
a= \begin{pmatrix} 1 & a & b \\ 0 & 1 & c \\ 0 & 0 & 1 \end{pmatrix}, \quad a^{ - 1}  = \begin{pmatrix} 1 & -a & ac - b \\ 0 & 1 & -c \\ 0 & 0 & 1 \end{pmatrix} 
\] Now, this is also in the form required of matrices in the group. Finally we have that $I \in G$, so we are done. To show that this is a Lie group, we argue that we can simply parametrise an element in $G$ as an element of $\mathbb{ R}^3 $, the underlying manifold. Moreover, as shown above, inversion and multiplication are smooth operations on this space. This group is not abelian. Consider the matrices
\[ 
A = \begin{pmatrix} 1 & 1 & 0 \\ 0 & 1 & 0 \\ 0 & 0 & 1 \end{pmatrix}, B = \begin{pmatrix} 1 & 0 & 0 \\ 0 & 1 & 1 \\ 0 & 0 & 1 \end{pmatrix}, \quad [ A , B ] = \begin{pmatrix} 0 & 0 & 1 \\ 0 & 0 & 0 \\ 0 & 0 & 0 \end{pmatrix} 
\] 
\subsubsection*{The corresponding Lie algebra} 
Suppose we can take a smooth curve parameterised by $ t \in \mathbb{ R} $, giving the group element 
\[ 
g ( t ) =  \begin{pmatrix} 1 & a( t ) & b( t )  \\ 0 & 1 & c( t )  \\ 0 & 0 & 1 \end{pmatrix}
\] Then, performing a Taylor expansion, we can write this group element infinitesimally as 
\[ 
g( t) = g( 0 )  + t X + \dots, \quad X = \begin{pmatrix} 0 & \dot{ a} ( 0 ) & \dot{b}( 0 )   \\ 0 & 0 & \dot{ c} ( 0 )   \\ 0 & 0 & 0 \end{pmatrix}
\] However, it's easy to see that we can pick an arbitrary curve to give rise to any strict upper triangle we want in the Lie algebra. Hence $L( G ) $ is the set of matrices generated by 
\[ 
 \mathcal{ B }  = \big \{ \begin{pmatrix} 0 & 1 & 0 \\ 0 & 0 & 0 \\ 0 & 0 & 0 \end{pmatrix}, \begin{pmatrix} 0 & 0 & 1 \\ 0 & 0 & 0 \\ 0 & 0 & 0 \end{pmatrix}, \begin{pmatrix} 0 & 0 & 0 \\ 0 & 1 & 0 \\ 0 & 0 & 0 \end{pmatrix} \big \} 
\]  We can calculate the bracket of two general elements in this Lie algebra as 
\[ 
 [ \begin{pmatrix} 0 & a_1 & b_1 \\ 0 & 0 & c_1 \\ 0 & 0 & 0 \end{pmatrix} , \begin{pmatrix} 0 & a_2 & b_2 \\ 0 & 0 & c_2 \\ 0 & 0 & 0 \end{pmatrix} ]   = \begin{pmatrix} 0 & 0 & a_1 c_2  - a_2 c_1 \\ 0 & 0 & 0 \\ 0 & 0 & 0 \end{pmatrix} 
\] We can generate a proper ideal from this. Consider the proper subspace of $ L ( G ) $  generated by matrices of the form 
\[ 
I = \bigg \{  \begin{pmatrix} 0  & 0 & a \\ 0 & 0 & 0 \\ 0 & 0 & 0 \end{pmatrix} \mid a \in \mathbb{ R} \bigg \} 
\] This is clearly a vector subspace under matrix addition, and is isomorphic 
to $ ( \mathbb { R } , + ) $ by identifying the top corner element as just a real number on its own
. From the multiplication formula above, we have that 
\[ [ X, Y ]  = \begin{pmatrix} 0 & 0 & 0 \\ 0 & 0 & 0 \\ 0 & 0 & 0 \end{pmatrix} \in I  \quad \forall X \in I, Y \in L ( G ) \] 
This subgroup is in the kernel of our Lie bracket, and hence is an ideal. But, clearly it is not the whole Lie algebra,
so it is a proper ideal. Hence, the Lie algebra is not simple. 
Also, a slicker way to say this
is that since elements of this form commute with $ \mathcal{ L } ( G ) $, 
its a non-trivial ideal. 

A simplified argument to show that 
its a Lie group is that our corresponding 
action is smooth, ie the map 
\[
 g \in G, g_L : G \times G, g_l ( h ) = gh \text{ is smooth, bijective with a smooth inverse }
\] 
\pagebreak 
