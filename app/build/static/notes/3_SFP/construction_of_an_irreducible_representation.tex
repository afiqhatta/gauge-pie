\subsubsection{Construction of an irreducible representation}

Our next thing to do in building a representation 
of $ \mathcal{ L } ( SU ( 2) ) $ is to then use 
our finite dimensionality constraint to construct 
our whole set of eigenvectors. A quick outline of these steps is 
\begin{enumerate}
	\item Declare a highest weight vector. 
	\item Step down with $ E_- $
	\item Show that we have no other vectors 
		the same eigenvalue 
\end{enumerate}

Since we have a finite dimensional representation, 
there must be a point where raising our vector 
with $ R ( E _ + ) $ terminates. This implies that 
there exists a vector $ \vec{v} _ \Lambda$ with 
$ \Lambda $ our highest weight, such that 
\[
 R ( H ) \vec{v} _ \Lambda = \Lambda \vec{v} _ \Lambda, \quad
 R ( E_ + ) \vec{v} _ \Lambda  = 0 
\] Now, from this object, let's see what we can do. 
Well, we can repeatedly apply a lowering 
operator on $ \vec{v} _ \Lambda$, and we use this 
procedure to build a set of eigenstates which we 
index as 
\[
	\vec{v}_{ \Lambda - 2k } = [ R ( E- ) ] ^ k  \vec{v} _ \Lambda 
\] This procedure is shown in the figure. 
\begin{figure}[h]
	\centering
	\begin{tikzpicture}[scale=0.7]
		\begin{scope}[very thick,decoration={
    				markings,
    				mark=at position 0.5 with {\arrow{>}}}] 
		\filldraw (0,0) circle (2pt) node[align=left,   below] {$ \vec{v}_\Lambda$}  ; 
		\filldraw (0, -2) circle (2pt) node[align=left, below] {$ \vec{v}_{ \Lambda - 2}$}; 
		\filldraw (0, -4) circle (2pt) node[align=left, below] {$ \vec{v}_{ \Lambda - 4}$}; 
		\filldraw (0, - 6) circle (2pt)  node[align=left, below] {$ \vec{v}_{ \Lambda - 4}$};

		\draw [postaction={decorate}] (0, 0) to [out=180,in=180] (0 , -2);
		\draw [postaction={decorate}] (0, -2) to [out=180,in=180] (0 , -4);
		\draw [postaction={decorate}] (0, -4) to [out=180,in=180] (0 , -6);

		\node[draw] at (-2.5,  -1) {$ R ( E_- ) $};
		\node[draw] at (-2.5,  -3) {$ R ( E_- ) $};
		\node[draw] at (-2.5,  -5) {$ R ( E_- ) $};
		\end{scope} 
	\end{tikzpicture}
\end{figure}

Our final goal for this section will be to show 
that these states are non-degenerate in their weights. 
This means that, for every given weight, we only have one associated vector. 
To show this, we use the assumption that our representation is 
\textbf{irreducible}. This assumption is important because 
it implies that every state in our representation space 
can be obtained by composing strings of 
$ R ( E_- ) , R ( E _ + ) $ and  $ R (H ) $. 
Using this assumption, if we prove that 
\[
 R ( E_ + ) \vec{v} _{ \Lambda - 2k } \propto \vec{v} _{ \Lambda - 2k + 2 }
\] then we are done, because this means 
that at we cannot access any other states other 
than the set $ \left\{  \vec{v}_{ \Lambda - 2 k }, k \in \mathbb{ Z}  \right\} $. 
We can prove this via induction. 
The base case is straight forward. If we assume that 
the above relation holds, then extending to the case 
$k \to k + 1 $, we get 
\begin{align*}
	R ( E_ + ) \vec{v}_{ \Lambda - 2 k - 2 } &=  R ( E_ + ) R( E _-) \vec{v}_{ \Lambda - 2k } \\
						 &=  \left( [ R ( E_ + ) , R ( E _ - ) ] + R ( E_-) R( E_ + ) 
						 \right) \vec{v} _{ \Lambda - 2k } \\
						 &=  \left(   R( H ) + R ( E_ - ) R ( E _ + )  \right) \vec{v}
						 _{ \Lambda - 2k }\\
						 &=  R( H ) \vec{v} _{ \Lambda - 2k } 
						 + R ( E_ - ) K \vec{v}_{ \Lambda - 2k + 2 }\\
						 &=  R( H ) \vec{v}_{ \Lambda - 2k } + K \vec{v}_{ \Lambda - 2k } \\
						 &=  \left(  \Lambda - 2k   + K \right)  \vec{v}_{ \Lambda - 2k } \\ 
\end{align*}
In this induction step we used $ K $ as the constant 
of proportionality here. We can do even better than this 
and determine this constant of proportionality.
Let's define our constant of proportionality 
recursively as 
\[
 \vec{v}_{ \Lambda - 2k } = r_ k \vec{v}_{ \Lambda - 2k + 2 }
\] Now, to get to the next term in the series, 
we apply our lowering operator to both sides. Once
we apply the appropriate commutation relations, we can 
obtain the relations as follows 
\begin{align*}
	R ( E_- ) R( E_ + ) \vec{v}_{ \Lambda - 2k } &=  r_ k R ( E _- ) \vec{v} _{ \Lambda - 2k + 2} \\
	\left(  [ R ( E_- ) ,R ( E _ + ) ] + R ( E _ + ) R ( E _ - )  \right) 
	\vec{v}_{ \Lambda - 2k } &=  r_ k \vec{v}_{ \Lambda - 2k } \\
	\left( - R ( H ) \vec{v}_{ \Lambda - 2k } + R ( E _ + ) \vec{v}_{ \Lambda - 2k - 2} \right) &=  
	r_ k \vec{v}_{ \Lambda - 2k }\\  
\end{align*} 
Now, substituting our expression for our recursion constant with $ k + 1 $, 
 \begin{align*}
	 - ( \Lambda - 2k ) \vec{v}_{ \Lambda - 2k } + r_{ k + 1 } \vec{v}_{ \Lambda - 2k } 
	 &=  r_ k \vec{v}_{ \Lambda - 2k } \\
\end{align*}
This yields the recursion relation that 
\[
 r_{ k +1 } = r_ k + \Lambda - 2k, \quad r_ 0 = 0  
\]  The boundary condition for $ r_0 $ can be 
obtained by just considering what happens at $ k = 0 $, 
we have that  $ R ( E_ + ) $ annihilates $\vec{v} _\Lambda$. 
Writing out the first few terms explicitly, 
one can easily be convinced that 
\[
 r_k = k ( \Lambda - k + 1 ) 
\] Now, our condition that  $ R $ is a 
finite dimensional representation gives us another 
condition that $ R ( E _) \vec{v}_{ \Lambda - 2N }  = 0 $
for some non-zero $ \vec{v}_{ \Lambda - 2 N } $. This implies that, going 
one step down, that we have 
\[
	0 = R ( E_ + ) \vec{v}_{ \Lambda - 2 (  N+ 1 ) } =  r_{ N + 1 } \vec{v}_{ \Lambda - 2N }
\] This implies that $ r_{N + 1 } = 0$. Using our previous 
recursion relation 
\[
	r_{ N + 1 } = ( N + 1 ) ( \Lambda - N - 1+ 1 ) = 0 \implies \Lambda = N, N \in \mathbb{ N } 
\] 
\pagebreak 
