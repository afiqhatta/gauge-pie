\section{The Killing Form}
We will now create a structure on 
our Lie algebra which is somewhat analogous to a 
metric on vectors (which we covered 
in the general relativity notes). 
Given a vector-space like Lie algebras, 
it is of interest for us to define a scalar product 
on the Lie algebra, which takes two vectors and returns a scalar. 

Let's first define an inner product first. 
\begin{defn}{(Inner product)} 
	Given a vector space $ V $ over a field $ F $, an inner product is 
	a bilinear map 
	\[
	 i : V \times V \to F 
	\] where this map is symmetric. This is notion is 
	analogous to the usual dot product we all know and love, 
	but we've made things a bit more general. 
\end{defn}
Right now, we'll define the inner product in this way. 
However, we'll want to make an extra definition, 
motivated by the fact that we don't care about inner 
products where, when we contract a specific element with 
all others in a lie algebra, we get zero. 
\begin{defn}{(Non-degeneracy)} 
	 We say that the inner product $ i $ is non-degenerate 
	 if for all $ v \in V ( v \neq  0 ) $, there 
	 is a $ w \in V $ such that $ i ( v, w ) \neq 0$. 
	 With this, our concept of an inner product looks more like a metric. 
\end{defn}
Now, a natural question 
to pose is if there is a natural inner product which we can 
write down for a Lie algebra? What does the term 'natural' 
even mean? We'll answer this question later. 
For the first question The answer is yes; our answer is the \textbf{killing form}. 
\begin{defn}{(Killing form)} 
	Our Killing form is a map from the Lie algebra $ \lalg $
	 \[
	 \kappa : \lalg \times \lalg \to F 
	\] defined as the map which takes two $ X , Y  \in \lalg $, 
	which takes 
	\[
	 K ( X, Y )  = \tr ( \ad_ X \circ \ad _ Y ) 
	\] So, we are taking 
	the trace of the composition of tow adjoint 
	representations. By the cyclic property of trace, we have that 
	this object is symmetric since we 
	can switch the adjoint maps around and still have the same map. Hence, 
	it's still and inner product. 
	In addition, since the adjoin maps are linear in both arguments, 
	we have that this map is linear. 
\end{defn}
Now, what does this map look like 
in terms of a basis of our Lie algebra? Let's look at the map 
\[
 ( \ad_ X \circ \ad_ Y ) : \lalg \to \lalg 
\] we write out the definition of the 
adjoint map explicitly, which is the commutator. 
Composing two commutator operations means that some $ Z $ is mapped 
as
\[
 Z \in \lalg \to [ X , [ Y , Z ] ] \in \lalg
\] What is the matrix representation of this map? 
To do this, we construct a basis $ \left\{  T ^ a : a = 1, \dots , D  \right\} $ 
for $ \lalg$. 
Writing out our components, and our structure constants explicitly, 
 \[
 X  = X_ a T ^ a , \quad 
 Y = Y  _a Y ^ a , \quad
 Z = Z_ a T ^ a, 
 \quad [ T ^ a , T ^ b ] = f ^{ ab } _ c T ^ c 
\] Multiplying components out, 
we find that 
\begin{align*}
	[ X. [ Y , Z ] ] &=  X_a Y _ b Z_ c [ T ^ a , [ T ^ b , T ^ c ]] \\
	&=  X_ a Y _ b Z _ c f ^{ ad } _ e f ^{ bc } _ d T ^ e  \\
	&=  M ( X, Y ) ^ c _ e  Z_ c T ^ e  \\
\end{align*} 
In this expression, we have that 
\[
 M ( X , Y ) ^ c _ e  = X_ a Y _ b f^{ ad } _ e f ^{ bc } _ d 
\] Taking the trace of this map, we 
find the components explicitly by 
taking the trace 
\[
 K ( X, Y ) = \tr _ D [ M ( X, Y ) ] = K ^{ ab } X_ a Y _ b 
\] where $ \kappa ^{ ab } = f ^{ ad } _ c f ^{ bc } _  d$. 

What does the term natural mean? 
It means that the map $ \kappa $ should be invariant under the 
adjoint action $ \lalg $. This action condition 
is given by 
 \[
 \kappa ( [ Z, X] , Y ) + \kappa ( X, [ Z, Y ] )  = 0, \forall X, Y , Z \in \lalg 
\]
We now prove this invariance condition. 
\begin{thm}{($ \kappa $ is invariant under the adjoint action ) } 
	Writing out the term on the left explicitly, 
	\[
		\kappa ( [ Z, X] , Y ] )  = \tr [ \ad _{ [ Z , X ] } \circ \ad_Y ] 
	\] the defining property of our adjoint 
	representation is that 
	\[
		\ad_{ [ Z, X ] }  = ( \ad _ Z \circ \ad_ X - \ad _ X \circ \ad _ Z ) 
	\] This means that the above reads 
	\[
	 \dots = \tr [ \ad_ Z \circ \ad_ X \circ \ad_ Y ] - \tr [ \ad_ X \circ 
	 \ad_ Z \circ \ad_ y ] 
	\] Similarly, applying this to the second term, 
	we find that 
	\[
	 \kappa ( X, [ Z, y ] )  = 
	 \tr [ \ad_ X \circ \ad _ Z \circ \ad_ Y ] - \tr [ \ad_ X \circ \ad_ Y \circ \ad_ Z ] 
	\] By cyclicity of the trace, this evaluates to zero. 

\end{thm}

\pagebreak 
