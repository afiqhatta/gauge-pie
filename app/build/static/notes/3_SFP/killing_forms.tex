\section{The Killing Form} 
Given a vector-space like Lie algebras, 
it is of interest for us to define a scalar product 
on the Lie algebra, which takes two vectors and returns a scalar. 

Let's first define an inner product first. 
\begin{defn}{(Inner product)} 
	Gven a vector space $ V $ over a field $ F $, an inner product is 
	a bilinear map 
	\[
	 i : V \times V \to F 
	\] where this map is symmetric. 
\end{defn}
Right now, we'll define the inner product in this way. 
However, we'll want to make an extra definition 
to constrain things a map. 
\begin{defn}{(Non-degeneracy)} 
	 We say that the inner product $ i $ is non-degenrate 
	 if for all $ v \in V ( v \neq  0 ) $, there 
	 is a $ w \in V $ such that $ i ( v, w ) \neq 0$. 
	 This is somewhat similar to our definition of a metric! 
\end{defn}
The question is, is there a natural inner product which we can 
write down for a Lie algebra? What does the term 'natural' 
even mean? The answer is yes; our answer is the \textbf{killing form}. 
\begin{defn}{(Killing form)} 
	Our Killing form is a map from the Lie algebra $ \lalg $
	 \[
	 \kappa : \lalg \times \lalg \to F 
	\] defined as the map which takes two $ X , Y  \in \lalg $, 
	which takes 
	\[
	 K ( X, Y )  = \tr ( \ad_ X \circ \ad _ Y ) 
	\] By the cyclicity of trace, we have that 
	this object is symmetric since we 
	can switch the ad maps around. 
	In addition, since the adjoin maps are linear in both arguments, 
	we have that this map is linear. 
\end{defn}
Since
\[
 ( \ad_ X \circ \ad_ Y ) : \lalg \to \lalg 
\] we map some $ Z $ like 
\[
 Z \in \lalg \to [ X , [ Y , Z ] ] \in \lalg
\] What is the matrix representation of this map? 
To do this, we construct a basis $ \left\{  T ^ a : a = 1, \dots , D  \right\} $ 
for $ \lalg$. 
Writing out our components explicitly, 
 \[
 X  = X_ a T ^ a , Y = Y  _a Y ^ a , Z = Z_ a T ^ a, [ T ^ a , T ^ b ] = f ^{ ab } _ c T ^ c 
\] Multiplying components out, 
we find that 
\begin{align*}
	[ X. [ Y , Z ] ] &=  X_a Y _ b Z_ c [ T ^ a , [ T ^ b , T ^ c ]] \\
	&=  X_ a Y _ b Z _ c f ^{ ad } _ e f ^{ bc } _ d T ^ e  \\
	&=  M ( X, Y ) ^ c _ e  Z_ c T ^ e  \\
\end{align*} 
In this expression, we have that 
\[
 M ( X , Y ) ^ c _ e  = X_ a Y _ b f^{ ad } _ e f ^{ bc } _ d 
\] Taking the trace of this map, we 
find the components explicitly by 
taking the trace 
\[
 K ( X, Y ) = \tr _ D [ M ( X, Y ) ] = K ^{ ab } X_ a Y _ b 
\] where $ \kappa ^{ ab } = f ^{ ad } _ c f ^{ bc } _  d$. 

What does the term natural mean? 
It means that the map $ \kappa $ should be invariant under the 
adjoint action $ \lalg $. This action condition 
is given by 
 \[
 \kappa ( [ Z, X] , Y ) + \kappa ( X, [ Z, Y ] )  = 0, \forall X, Y , Z \in \lalg 
\]
We now prove this invariance condition. 
\begin{thm}{($ \kappa $ is invariant under the adjoint action ) } 
	Writing out the term on the left explicitly, 
	\[
		\kappa ( [ Z, X] , Y ] )  = \tr [ \ad _{ [ Z , X ] } \circ \ad_Y ] 
	\] the defining property of our adjoint 
	representation is that 
	\[
		\ad_{ [ Z, X ] }  = ( \ad _ Z \circ \ad_ X - \ad _ X \circ \ad _ Z ) 
	\] This means that the above reads 
	\[
	 \dots = \tr [ \ad_ Z \circ \ad_ X \circ \ad_ Y ] - \tr [ \ad_ X \circ 
	 \ad_ Z \circ \ad_ y ] 
	\] Similarly, applying this to the second term, 
	we find that 
	\[
	 \kappa ( X, [ Z, y ] )  = 
	 \tr [ \ad_ X \circ \ad _ Z \circ \ad_ Y ] - \tr [ \ad_ X \circ \ad_ Y \circ \ad_ Z ] 
	\] By cyclicity of the trace, this evaluates to zero. 

\end{thm}

\pagebreak 
