\subsection{Question 1}

Our group axioms are identity, closure, inverses and associativity. 
We go about proving the first three, and in the case of matrix groups 
we can say that associativity is inherited from matrix multiplication. 


We check closure, since associativty is inherited from matrix multiplication.
\begin{align*}
C^T C = (AB)^T AB = B^T A^T A B = B^T B = I. 
\end{align*}
Finally, if $A \in O(n)$, then by inverting both sides, 
\[
A^T A = I \implies A^{-1} (A^T)^{-1} = A^{-1} (A^{-1})^T = I. 
\]
The case for the unitary group is entirely similar except that we replace transposition by Hermitian conjugation. 
To show that $O(n)$ is a subgroup of $U(n)$, we must first show that it's a subset of $U(n)$. Since $O(n)$ is defined for just matrices over the reals, Hermitian conjugation is equivalent to transposition over the reals. Thus, matrices in $O(n)$ are also unitary. Since $O(n)$ is a group in its own right, it's a subgroup. 

$SO(n)$ consists of matrices in $O(n)$ which have determinant 1. Thus, we have that $SO(n) \subset O(n) \subset U(n)$. Thus, $SO(n) \subset U(n)$. But since its matrices have determinant 1, $SO(n) \subset SU(n)$. 

\subsubsection*{Showing that $U(n)$ is a subgroup of $SO(2n)$} 
A complex vector $\mathbf{c} \in \mathbb{C}^n$ and be represented as the sum of a real and imaginary vector, where 
\[ 
\mathbf{c} = \mathbf{a} + i \mathbf{b}, \quad \mathbf{a}, \mathbf{b} \in \mathbb{R}^n 
\] 
Similarly, a unitary matrix $U \in U ( n) $ can be represented as the sum of a real and imaginary matrix. 
\[ 
U  = A + i B, \quad A, B \in Mat_n (\mathbb{R} ) 
\] 
Our condition of unitarity however imposes conditions on $A, B$. By comparting real and imaginary parts, we can extract these conditions from 
\[ 
U^\dagger U = (A^\dagger - i B^\dagger)( A + iB ) = (A^\dagger A  + B^\dagger B ) + i ( A^\dagger B - B^\dagger A ) = I \implies A^\dagger A + B^\dagger B = I, \quad A^\dagger B - B^\dagger A  = 0 
\] 
In addition, we can act on vectors in this notation, still treating the real and imaginary components separately. We find that 
\[ 
( A + iB)( \mathbf{a} + i \mathbf{b}) =  A\mathbf{a} - B\mathbf{b} + i (B \mathbf{a}  + A \mathbf{b} ) 
\] 
This whole system can be represented as a map on $\mathbb{R}^{2n}$ by stacking up $\mathbf{a}$ and $\mathbf{b}$. 
\[ 
\begin{bmatrix} \mathbf{a} \\ \mathbf{b} \end{bmatrix} \mapsto \begin{bmatrix} A & -B \\ B & A \end{bmatrix} \begin{bmatrix} \mathbf{a} \\ \mathbf{b} \end{bmatrix} 
\] 
In this step, we're representing the matrix $U$ as an operator on $\mathbb{R}^{ 2n}$ instead of the usual, fundamental representation. All that's left to show is that a matrix of this form is in $SO ( 2n)$. We compute the sum and use the conditions above to show that 
\[ 
\begin{pmatrix} A^T & B^T \\  - B^T & A^T \end{pmatrix} \begin{pmatrix} A & - B \\ B & A \end{pmatrix}  = \begin{pmatrix} A^T A + B^T B & B^T A - A^T B \\  - B^T  A + A^T B & A^T A + B^T B \end{pmatrix}  = I 
\] 
We've replaced the daggers in the above conditions with transposes since the matrices are real! 

More simply, we could argue from the fact that if $ v = ( v_1 , \dots, v_{ n  }) \in \mathbb{ C} ^ n$, 
then $ U ( n ) $ preserves our norm $ \| v \|^ 2   = \sum_i \| v_ i  \|^ 2  $. 
If we write $ v_{i } = x_ i + i y_ i $, then $ | v| ^ 2 = \sum_i x_i ^ 2 + y_i ^ 2 $, 
which is preserved by  $ O ( 2n ) $. Thus, $ U ( n ) \leq O ( 2n ) $. 
Thus, we've shown $ U ( n ) \leq O ( 2n ) $. 
Note that $ U ( n ) $ is connected, so it must be embedded in the 
connected component of the identity as $ I $ is sent to $ I $. 
$ U ( n ) $ is connected as we can diagonalise it as $ diag ( e^{ i \theta _ 1 }, \dots e^{ i \theta_ n } ) $, 
and take a path from any element to any element by parametrising $ \theta_ i $
with  $ t$. 

\pagebreak

