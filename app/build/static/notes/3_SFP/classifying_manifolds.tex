\subsection{Classifying manifolds} 
We can classify manifolds by some of their topological properties. We'll outline them here. 

\begin{itemize} 
	\item Connectedness is the property that we can construct a path from anywhere in the manifold to any other point in the manifold. For example, a sphere is connected, but a manifold whose set of points is two spheres is not. 
	\item Simply connectedness is subtly different from connectedness. It comes from the concept of a homotopy group. For a space to be simply connected, it means that any loop on the surface can be continuously deformed to a point. In the language of algebraic topology, we require that 
\[ 	
	\pi_1 ( \mathcal{M} ) \simeq \{ e \} 
\] 
Somthing that's not simply connected is a solid torus; since we can 'tie a string' around the centre donut. 
 	\item Compactness is when we can cover a space with a finite amount of subsets. This condition is equivalent to the condition, that if we're embedding in $\mathbb{R}^n$, then the subset is closed and bounded. 
For example, a quadratic curve on $\mathbb{R}$ is not a compact manifold since it's not bounded. 
\end{itemize}
 

\pagebreak
