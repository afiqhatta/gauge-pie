\subsection{The Lie Algebra $ \mathcal{ L } ( G ) $ } 

We stated earlier that Lie groups are manifolds in themselves. Well, on a manifold, we can define the tangent space at some point of reference (in this case we'll set to be $e \in G$) as $\mathcal{T}_{e}(G)$. We know that tangent spaces are vector spaces with the basis $\{ \partial_i\}_i $, so we could try making a Lie group out of this by trying to define the commutator. Let's 
proceed with constructing this relationship. 
Let $ G $ be a Lie group of dimension $ D $. Introduce 
coordinates $ \left\{  \theta ^{ i } \right\} , i  = 1, \dots D $ 
in some region $ \mathcal{ P } $ containing $ e $, the identity. 
\[
	\forall g \in \mathcal{ P }, \quad  g = g (\theta ) \in G, \quad g ( 0 ) =e  
\]  (Draw diagram of tangent space here) 
The tangent space $ \mathcal{ T }_{ e } ( G )  $ is a
D dimensional vector space. We can define a bracket 
\[
	[  \, , \, ] : \mathcal{ T }_e ( G) \times \mathcal{ T }_e ( G ) \to  \mathcal{ T }_ e ( G ) 
\] such that we have $ \mathcal{ L } ( G ) = ( \mathcal{ T}_e (  G ), [ \, , ] ) $
is a Lie algebra. 

\subsubsection{Constructing the Lie Algebra for Matrix Lie Groups} 
There are two ways to think about this. A tangent vector is a differential operator of a function, which can be written like 
\[ 
	V = v_i \partial_i \in \mathcal{T}_e(G). 
\] 
We can represent this as a matrix as follows. An element in an n-dimensional Lie group can be represented as $g(\theta) \in G$, where $\theta$ is an n-vector. Thus, we can map $V$ like 
\[
 	v_i \partial_i \mapsto \left. v_i \frac{\partial g(\theta)}{\partial \theta_i} \right\vert_{\theta = 0} \in Mat_n(F). 
\]

This construction is easiest to show for \textbf{Matrix Lie groups} $ G \subset Mat_n ( F) $ for either a real or $\mathbb{C} $ fields. We can 
map tangent vectors to matrices. 
\[
	\rho : \mathcal{ T}_{ e }( G ) \to Mat_n ( F)
\] given by the map 
\[
v^{ i }\frac{\partial }{\partial \theta^{ i  } } \in \mathcal{ T }_e ( G ) \mapsto v^{ i } \frac{\partial  g ( \theta ) }{\partial \theta^{ i }} \mid_{ \theta = 0 } \in Mat_n ( F)  
\] We see that this map $\rho $ is injective, but not necessarily 
surjective. 
This allows us to identify $ \mathcal{ T }_{ e }( G ) $ with 
the subspace of $ Mat_n ( F ) $ spanned by 
\[
\left\{ \left. \frac{\partial  g ( \theta ) }{\partial \theta^{ i } } \right \vert_{ \theta = 0 }   \right\} 
\] We have an obvious candidate for our Lie bracket 
\[
[ X, Y ] := XY - Y X, \quad X, Y \in \mathcal{ T }_e ( G ) 
\] We need to check that our properties for Lie 
algebras holds. This is clear since antisymmetry and Linearity are inherent 
straight from the definition of the matrix commutator. 
The Jacobi identity is easy to prove but requires a little bit of algebra. 
With this assignment, we are now free to define our commutator as the usual commutator for matrices. 
However, we need to show that the Lie algebra generated by the commutator is closed. 
There is a subtle point however, that we should check that the resulting object is still indeed in the tangent space. We do this by showing that there is a curve, that if we differentiate it at a point, we recover the commutator object itself. To do this, we reference what we said in our preliminaries about constructing curves that yield tangent vectors. 
In other words, we need to prove that, 
\[
[ X, Y ] \in \mathcal{ L } ( G ), \quad \forall X, Y \in \mathcal{ L }( G ) 
\] Let $ c $ be a smooth curve on $ G $ through the 
identity. Let 
\begin{align*}
	c : \mathcal{ J } \subset \mathbb{ R}  & \to G \\
	c : t  & \mapsto g ( t) \in G , \quad g ( 0 )  = I_n 
\end{align*} Now, if we consider 
the derivative we have 
\[
	\frac{ d g ( t) }{ dt } = \frac{ d \theta ^{ i ( t ) }}{d t } \cdot \frac{\partial  g ( \theta ) }{\partial d \theta^{ i }} 
\] If we take 
\[
\dot{ g } ( 0 )  = \left. \frac{ d g( t) }{ dt } \right \vert_{ t =0 } = \dot{ \theta}^{ i }( 0 )  = \left. \frac{\partial g ( \theta ) }{\partial \theta^{i }} \right \vert_{ \theta  =0 } \in \mathcal{ T }_e ( G )
\] This is a tangent vector to $c $ at $ e $. 
We have that 
\[
\dot{ g} ( 0 ) \in Mat_n ( F) \text{ is not generally in } G 
\] Observe that near to $ t  = 0 $, we have that 
\[	 g( t)  = 1 + tX + O ( t^2 ), \quad X = \dot{ g} ( 0 ) \in \mathcal{ L }( G )  
\] Given 2 elements $ X_1, X_2 \in \mathcal{ L } ( G ) $, since all 
we need to do is fix the first term of our Taylor expansion, 
it is simple to find curves $ c_1, c_2 $ such that 
\begin{align*}
c_1 &: t \mapsto g_1 ( t) \in G, \quad g_1 ( 0 )  = 1 \\
c_2 &: t \mapsto g_2( t) \in G, \quad g_2 ( 0 )  = 1
\end{align*} where we have $ \dot{ g}_1( 0 )  = X_1, \dot{ g}_2 ( 0 )  = X_2$. 
Near  $ t = 0 $, we have 
\begin{align*}
g_1 ( t ) & = 1 + X_1 t + W_1 t^2 + O ( t^3 ) \\
g_2 ( t) &= 1 + X_2 t + W_2 t^2 + O ( t^ 3 ) 
\end{align*}for some $ W_1, W_2 \in Mat_n ( F) $. 
We define a new curve 
\[
h ( T)  = g^{ - 1}_1( t)  g^{ -1 }_2 ( t ) g_1 ( t) g_2( t) \in G 
\] This curve is smooth since this is a composition of multiplication 
and inversion, which are smooth maps. This is equivalent to 
\[
g_1( t) g_2( t) = g_2 ( t ) g_1 ( t) h ( t) 
\] If we expand the curve about $ t = 0 $, we get 
that 
\begin{align*}
g_1( t)g_2 ( t) &=  1 + t ( X_1 + X_2 ) +t^2 ( X_1 X_2 + W_1 + W_2 ) + O ( t^3)  \\
g_2( t) g_1( t) &=  1 + t ( X_2 + X_1 ) + t^2 ( X_2 X_1 + W_1 + W_2 + O ( t^3 ) 
\end{align*}
Now, if we set 
\[
h ( t) = 1 + h_1 t + h_2 t^2 + O ( t^3 ) 
\], if we plug this into our relation we have that 
\[
h_1 = 0, \quad h_2  = ( X_1 X_2 - X_2 X_1 ) =  [ X_1 , X_2 ] 
\] Thus, we have that 
\[
h ( t)  = 1 + t^2 [ X_1 , X_2 ] + O ( t^3) 
\] We're not quite done yet! We need a curve whose 
linear part is the commutator. Define $ s = t^2 , s /geq 0 $ 
in other words a curve who's endpoint is at zero, then 
define a new curve 
\[
c_3 : s \mapsto g_3 ( s)  = h ( \sqrt{ s } ), \quad s \ge  0 
\] We have 
\[
g_3 (s)  = 1 + s [ X_1, X_2 ] + O ( s^{ 3 /2 } ) 
\] This is okay up to one derivative. If we have a second derivative
this term is not analytic or defined at 0! But we have a $ C^1 $ curve so we're 
okay. We have, from this curve, that 
\[
\frac{ d g_3 ( e) }{ ds } \mid_{ s  = 0}  = [ X_1 , X_2 ] 
\] This means that the commutator $ [ X_1 , X_2 ] $ is an element of our 
tangent space $ \mathcal{ L } ( G ) $. Thus, 
\[
\mathcal{ L } ( G ) = ( \mathcal{ T }_ e ( G ) , [ \, , \, ] ) 
\] defines a Lie algebra. 

\pagebreak
