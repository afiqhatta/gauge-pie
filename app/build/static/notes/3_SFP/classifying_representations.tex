\subsection{Classifying representations}
Representations of groups could have properaties of interest, for example unitarity or orthogonality. We say that an $n$ dimensional representation is orthogonal or unitary if the maps lie in $O(n)$ or $U(n) $ respectively. For example, our representation of the additive group on the reals has an orthogonal representation \[ \alpha \mapsto \begin{pmatrix} \cos \alpha & -\sin \alpha \\ \sin \alpha & \cos \alpha \end{pmatrix} \] 
We can also cook up a unitary 1 dimensional representation by sending \[ \alpha \rightarrow \exp ( i \alpha )\], for example. More strictly, we can have Hermitian or Anti-Hermitian representations if the corresponding matrices are Hermitian or Anti-Hermitian. 

