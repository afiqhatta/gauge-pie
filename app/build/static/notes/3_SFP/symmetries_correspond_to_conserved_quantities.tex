\subsection{Symmetries correspond to conserved quantities}  
Physics, the existence of symmetries lead to conserved quantities. In the case of rotations, in classical mechanics, rotational invariance leads to conservation of angular momentum by Noether's theorem. This is a conserved quantity derived from a symmetry acting on $\mathbb{R}^3 $. This manifests itself in the form of an angular momentum vector $\mathbf{L }  = (L_1, L_2, L_3 )$.  

In quantum mechanics, instead of phase space, we can work on Hilbert space $\mathcal{H}$, and do something similar. In Hilbert space, we work with state vectors $\ket{ \phi} \in \mathcal{H}$ and observable quantities are Hermitian operators $\hat{ \mathcal{ O }} $. Looking at generators of the rotation group, our analog of the angular momentum vector are angular momentum generators $\hat{ L}_i$ which we obey a 'spin' algebra 
\[
	 [\hat{L}_i, \hat{L}_j ] = i \epsilon_{ ijk} \mathcal{L}_k
\] 
which, we'll see later, that our commutator of operators in quantum mechanics is precisely our Lie algebra in the context of $\mathcal{ L }SO( 3)$. 

\pagebreak
  
