\section{Representations}

A representation is a map $ D $ from a group 
to our set of matrices, which satisfies the property that 
\[
	D : G \to  GL_ n ( F ) \subset \text{Mat}_n (  F), \quad n \in \mathbb{ N }
\] such that the image matrices 
are non singular. In addition, this map 
needs to be a homomorphism, so we have that 
\[
	D( g_1 )  D( g_2) = D ( g_1 g_2 ) , \quad \forall g_1, g_2 \in G 
\] A representation is \textbf{faithful} if $ D $ is injective. 
A Lie group $G$ representation $ D $ is a group representation 
but our map $ D $ \textbf{must be smooth!}.

The fact that this is a 
group homomorphism means that 
\[
	D ( g ) D ( e) =  D( g), \quad \forall g \in G  \quad \implies D ( e ) = I_ n 
\] In addition, 
the representation of a group inverse is thae 
same as the inverse of its representation. This is because 
\[
	\forall g \in G, D ( g ^{ -1  } ) D ( g) = I \implies D ( g ^{ -1  } )  = D ( g) ^{ -1 }
\]  
Let's go into a bit more detail about the representation
of a matrix Lie group. If $ G $ is a matrix Lie group, 
$ G \subset \text{Mat} _n ( F )  $, let $ D $ 
be a representation of $ G$, with  $ \dim ( D ) = n \neq m \neq \dim ( G ) $. 
Now, construct our representation of 
the Lie algebra. 
For each  $ X \in \mathcal{ L } ( G ) $, 
construct a curve $ C : t \in \mathcal{ I } \subset \mathbb{ R} \to g ( t) \in G $. 
We have the expansion 
\[
 g ( t) = I_m + Xt + O ( t ^ 2 ) 
\] Now, we can apply our
representation and construct an image curve
\[
	D ( g ( t ) ) \in GL( n, F ) \subset \text{ Mat }_n ( F) 
\] By Taylor's theorem, 
we have that 
\[
	D ( g ( t ) ) = D ( I _ m ) + \frac{d }{ dt } D ( g ( t) ) \mid_{ t = 0 } + O ( t^ 2 ) 
\] Where in this case, $ D ( I _ m )  = I _ n $ in
representation space. This motivates our definition 
for a representation of the Lie algebra of  $ G $.
We define 
\[
	d ( X) : = \frac{d }{ dt  }  D( g ( t ) ) \mid_{ t =0 }, \quad \forall X \in \mathcal{ L } ( G ) 
\] 
\begin{claim}
	This is a valid representation 
	of the Lie algebra. 

\begin{proof}
	For any $ X _ 1 , X _ 2 \in \mathcal{ L } ( G ) $, 
	construct curves 
	\begin{align*}
		c_1 & : t \to g_ 1 ( t ) \subset G \quad g_1 ( 0 )  = I_ m, \dot{g }_ 1  ( 0 ) = X_ 1   \\
		c_ 2 & : t \mapsto g _ 2 ( t) \subset G , \quad g _ 2 ( 0 ) = I _ m, \dot{ g } _ 2 ( 0 ) = X_ 2  
	\end{align*}
	We can then define a curve which we 
	considered previously. We have that 
	\[
		h ( t) = g_1 ^{ - 1} ( t) g _ 2 ^{ - 1} ( t) g_ 1 ( t) g _ 2 ( t ) \in G, \quad h ( t) = I _ n + t ^ 2 [ X_1, X_2 ] + O ( t^ 3 ) 
	\] Now, we need to use the fact 
	that $ D $ is a representation of $ G $, which means 
	we can factor things out smoothly. 
	So, we have that 
	\[ 
		D ( h ) = D ( g_1 ^{ - 1 } g_2  ^{ -1 } g_1 g_2 ) = D ( g_1 )  ^{ - 1} D ( g_ 2 ^{ - 1} D ( g_1 ) D( g_2) \forall t 
	\] We now use the representation expression which 
	we derived earlier, that 
	\[
		D ( g_1 ( t ) ) = D ( I _ m + t X_1 + \dots ) = D ( I _ m ) + t d ( X_1 ) + O ( t ^ 2 ) 
	\] We compare this with 
	\begin{align*}
		D ( h ( t ) ) & = D ( I _ m  + t ^ 2 [ X_1 , X_2 ] + O ( t ^ 3 ) ) \\
			      &=  D ( I _ m ) + t ^ 2 \frac{ d }{ dt ^ 2 } D ( h ( t ) ) \mid_{ t = 0 } \\
			      &=  I _ n + d ( [ X_1 , X_2 ] ) t ^ 2 + \dots  
	\end{align*}
	We have that since
	\[
		D ( h ) = D ( g_1 ) ^{ -1 } D ( g_2 ) ^{ - 1} D ( g_1) D ( g_2) 
	\] multiplying this term out, and comparing sides, we get that 
	\[
		d ( [ X_1 , X_2 ] ) = [ d ( X_1 ) , d ( X_2 ) ] 
	\] Also, we require that $ d  $ is linearity.  
\end{proof}
\end{claim}

As an exercise, given a representation $d $ of 
$ \mathcal{L  } ( G ) , \forall g \in G $  of
from 
\[
	g = \text{ Exp } ( X ), X \in \mathcal{ L } ( G ) 
\] If we define 
\[
	D ( g ) = D ( \text{Exp } X ) ;= \text{ Exp } ( d ( X) ) 
\] This is a valid representation. 


