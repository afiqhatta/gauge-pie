\documentclass[11pt, oneside]{article}   	% use "amsart" instead of "article" for AMSLaTeX format
\usepackage[margin = 1.1in]{geometry}            		% See geometry.pdf to learn the layout options. There are lots.
\geometry{letterpaper}                   		% ... or a4paper or a5paper or ... 
\usepackage[parfill]{parskip}    		% Activate to begin paragraphs with an empty line rather than an indent
\usepackage{graphicx}				% Use pdf, png, jpg, or eps§ with pdflatex; use eps in DVI mode
								% TeX will automatically convert eps --> pdf in pdflatex	
\usepackage{adjustbox}	
\usepackage[section]{placeins}


%% LaTeX Preamble - Common packages
\usepackage[utf8]{inputenc}
\usepackage[english]{babel}
\usepackage{textcomp} % provide lots of new symbols
\usepackage{graphicx}  % Add graphics capabilities
\usepackage{flafter}  % Don't place floats before their definition
\usepackage{amsmath,amssymb}  % Better maths support & more symbols
\usepackage[backend=biber]{biblatex}
\usepackage{amsthm}
\usepackage{bm}  % Define \bm{} to use bold math fontsx
\usepackage[pdftex,bookmarks,colorlinks,breaklinks]{hyperref}  % PDF hyperlinks, with coloured links
\usepackage{memhfixc}  % remove conflict between the memoir class & hyperref
\usepackage{mathtools}
\usepackage[T1]{fontenc}
\usepackage[scaled]{beramono}
\usepackage{listings}
\usepackage{physics}
\usepackage{tensor}
\usepackage{simplewick} 
\usepackage{tikz} 
\usepackage{import}
\usepackage{xifthen}
\usepackage{pdfpages}
\usepackage{transparent}
\usepackage{pgfplots}
\usepackage[compat=1.1.0]{tikz-feynman}
\usepackage{subfiles}
\usepackage{simpler-wick}
\usepackage{slashed}
\usepackage{fancyhdr}

\pagestyle{fancy}
\fancyhf{}
\rhead{Notes by Afiq Hatta}
\lhead{Quantum Field Theory II}
\rfoot{Page \thepage}

%% Commands for typesetting theorems, claims and other things.
\newtheoremstyle{slanted}
{1em}%   Space above
{.8em}%   Space below
{}%  Body font
{}%          Indent amount (empty = no indent, \parindent = para indent)
{\bfseries}% Thm head font
{.}%         Punctuation after thm head
{0.5em}%     Space after thm head: " " = normal interword space;
{}%         \newline = linebreak
{}%          Thm head spec (can be left empty, meaning `normal')

%% Commands for typesetting theorems, claims and other things. 

\theoremstyle{slanted}
\newtheorem{theorem}{Theorem}
\newtheorem*{thm}{Theorem}
\newtheorem*{claim}{Claim}
\newtheorem{example}{Example}
\newtheorem*{defn}{Definition}

\newcommand{\Lagr}{\mathcal{L}} 
\newcommand{\vc}[1]{\mathbf{#1}}
\newcommand{\pdrv}[2]{\frac{\partial{#1}}{\partial{#2}}}
\newcommand{\thrint}[1]{\int d^3 \vc{x} \left( {#1} \right)}

%% QFT specific macros 
\newcommand{\intp}{ \int \frac{ d^3 p }{ (2 \pi)^3 } \, }
\newcommand{\ann}[1]{a_{ \mathbf{ #1 }}}
\newcommand{\crea}[1]{a^\dagger_{ \mathbf{ #1 }}}
\newcommand{\ve}[1]{ \mathbf{ #1 } } 
\newcommand{\mode}[ 1]{ e^{ i \mathbf{ #1 } \cdot \mathbf{x} }}
\newcommand{\nmode}[1]{ e^{  - i \mathbf{ #1 } \cdot \mathbf{x} }}
\newcommand{\freq}[1]{\omega_\mathbf{ #1} } 
\newcommand{\scal}[1]{\phi ( \mathbf{ #1 })} 
\newcommand{\mom}[1]{ \pi (\mathbf{ #1 })} 
\newcommand{\arr}{\rightarrow} 
\newcommand{\planck}{\hbar}

\newcommand{\incfig}[1]{%
\def\svgwidth{\columnwidth}
\resizebox{0.75\textwidth}{!}{\input{./figures/#1.pdf_tex}}
}

\newcommand{\anop}[2]{ #1_\mathbf{#2}}
\newcommand{\crop}[2]{#1_\mathbf{#2}^\dagger}
\renewcommand{\op}[1]{\hat{\mathbf{#1}}}

\usepackage{helvet} 

%tikz decoration commands 
\usetikzlibrary{decorations.pathmorphing}


\title{Notes on Quantum Field Theory II}
\author{Afiq Hatta} 
\begin{document} 
\maketitle
\tableofcontents

\pagebreak

\section{Path Integrals}%
\label{sec:path_integrals}

In this section, we'll introduce the path integral in QM, look at some methods with integrals, and then explore Feynman rules. 
Throughout these notes, we'll leave $ \hbar$,
but feel free to set this to $ 1 $ throughout 
the course. 
Let's introduce the path integral from 
the standpoint of quantum mechanics. 
The goal here is to take Schrödinger's equation 
and reformulate it into a "path integral", 
which is roughly speaking, a weighted integral 
summing over all probable paths. 
Let's consider a Hamiltonian 
in just one dimension, which as usual we 
can decompose into a kinetic and potential term
\[
	\hat{H }  = H \left( \hat{ x  } , \hat{ p } \right) = \frac{\hat{p } ^ 2 }{ 2m } 
	+ V\left( \hat{x} \right), \quad  \text{ with} \left[  
	\hat{x } , \hat{ p } \right]  = i \planck 
\] 
Schrödinger's equation says thst 
if we have a state, its time evolution is governed the equation below, which we write as its formal 
solution by 
\[
	i \planck \frac{\partial  }{\partial  t } \ket{ \psi ( t ) } 
	= \overline{ H } \ket{ \psi ( t ) } 
\] Our formal solution 
is given by mutliplying by the time evolution 
operator 
\[
	\ket{ \psi ( t )  } = e ^{  - i H \frac{t}{\planck } } \ket{ \psi (  0 )  }
\] 
In the Schrodinger picture, we have that 
\begin{itemize}
	\item States evolve in time 
	\item Operators and their eigenstates 
		are constant in time (fixed). 
\end{itemize}
Wavefunctions in position space are denoted
\[
	\psi ( x , t ) = \bra{ x  }\ket{ \psi ( t ) } 
\] This gives Schrodingers equation as 
\[
	\bra{ \hat{ x }} \overline{ H }\ket{ \psi  ( t ) }  =
	\left(  - \frac{\planck ^ 2 }{ 2 m } \frac{\partial  ^ 2 }{\partial  x ^2 }  + V ( x )  \right) \psi ( x , t )  
\] How do we convert this differential equation 
into an integral equation? 
We have that 
\begin{align*}
	\psi ( x , t )  & = \bra{ x }e ^{  - i H \frac{t}{\planck } }\ket{ \psi ( 0 ) } \\
	&=  \int_{ - \infty } ^{ \infty } \bra{ x } 
	e ^{ \frac{ - i H t }{ \planck } }\ket{x_0 } \bra{ x_0}\ket{ \psi ( 0 )} \\
	&=  \int_{ - \infty }^{ \infty } dx_0 K ( x , x_0 ; t ) \psi( x_0 , 0 )    
\end{align*} We can introduce an integral quite straightforwardly 
by introducing a projection operator 
onto initial positions. 
We insert a complete set of states 
\[
 I = \int dx_0 \ket{ x_0 }\bra{ x_0 } 
\] 
We call $ K ( x, x_0; t ) $ the Kernel. 
Repeat this procedure for $ n $ intermediate 
times and positions. 
Let $ 0 = t_0 < t_1 < \dots < t_ n < t _{ n + 1 }  = T $, 
and we factor 
\[
	e^{ -\frac{i H T }{ \planck } }  = e ^{  - \frac{i}{\planck } \overline{ H } \left( 
	t _{ n +1 }  -  t_ n \right)  } \dots e^{  - \frac{ i }{ \planck } \overline{ H } \left( 
	t_1 - t_0 \right)  }
\] Then, 
\[
	K ( x, x_0 ,T ) = \int_{  - \infty } ^{ \infty } \left[  
	\prod_{ r = 1 }^ n dx _ r \bra{ x_{ r + 1 } } 
e ^{ - \frac{ i \overline{ H } }{ \planck } \left( t _{ r + 1 }  - t_ r  \right)  } 
\ket{ x _ r } \right]  \bra{ x _1} e ^{   - \frac{ i \overline{ H } }{ \planck} \left( 
t_1 - t_0 \right)  }\ket{ x_0 } 
\] Integrals are over all possible 
position eigenstates at times 
$ t_ r , r = 1 , \dots n $. 
Consider a free theory, with $ V ( \overline{ x } ) = 0 $. 
Let's define a corresponding free kernel
\[
	K _ 0 \left( x, x' ; t  \right)  = \bra{ x }\exp\left( \frac{i \hat{p }^ 2   }{2m }t \right)
	\ket{ x' } 
\]  Insert, on the right side, the completeness relation 
for the identity. 
\[
 I  = \int_{  - \infty }^{ \infty } \frac{ dp }{ 2 \pi \planck } \ket{ p }\bra{ p }, 
 \bra{ x }\ket{ p }  = \frac{1}{ \sqrt{ 2 \pi \planck }  } e ^{ \frac{ i p x }{ \planck }}
\] This gives 
\[
	K_0 \left( x, x ' ; t  \right)   = 
	\int_{ \infty } ^{ \infty } \frac{ dp }{ 2 \pi \planck } e ^{  - i p ^ 2 t / 
	2m \planck  } e ^{ i p ( x - x ') / \planck}  = 
	e ^{ \frac{ im ( x - x ' ) ^ 2 }{ 2 \planck t } } \sqrt{ \frac{ m }{ 2 \pi i 
	\planck t}} 
\] Note, 
\[
	\lim_{ t \to 0 } K_ 0 \left( x, x ' ; t  \right)   =\delta ( x   - x ' ) 
\] which is as expected from $ \bra{ x }\ket{ x ' }  = \delta ( x   - x ' ) $. 
From the Baker-Campbell-Haudorf formula, 
we have that 
\[
 e ^{ \epsilon \hat{A } }e^{ \epsilon \hat{B } } = 
 \exp\left( \epsilon \overline{ A } + \epsilon \overline{ B }  + \frac{ \epsilon ^ 2 }{2 } 
 \left[  \overline{ A}, \hat{ B } \right] + \dots \right) \neq e ^{ \epsilon \left( 
 \hat{ A } + \hat{ B }\right) } 
\] For small $ \epsilon $, we have that 
\[
	e ^{ \epsilon ( \hat{ A  } + \hat{ B } ) } = e^{ \epsilon \hat{ A   } }e ^{ \epsilon \hat{ B } } \left(  1 + O ( \epsilon ^ 2 )  \right)
\] Now let $ \epsilon = \frac{1}{n } $, raise the above to the $ n $ power, 
so we have the result that 
\[
	e ^{ \hat{ A } + \hat{ B }  } = \lim_{ n \to \infty } \left( 
	e ^{ \hat{ A } / n } e ^{ \hat{B } / n  }\right) ^ n 
\] Take $ t _{  r + 1 }  - t_ r  = \delta t $, with $ \delta t \ll T $. 
Also take $ n $ large such that $ n \delta t  = T $. 
Then we can write that 
 \[
	 e ^{ \frac{ - i \hat{ H } \delta t }{ \planck } }  = \exp\left( \frac{ 
	 - i \hat{ p } ^ 2 \delta t }{ 2 m \planck }  \right)  \exp\left( 
 - \frac{ i V ( \hat{ x } ) \delta t }{ \planck }\right) \left[  
 1 + O ( \delta t ) ^ 2  \right]  
\] 
Writing out the above, this gives us 
\[
	\bra{ x _{ r + 1 } } \exp\left(   - \frac{ i \hat{ H } \delta t  }{ \planck } \right)
	\ket{ x _ r }  = e ^{  - i V ( \hat{ x } ) \delta t  / \planck } K _ 0 ( 
	x _{ r + 1 }, x _ r ; \delta t )  =
	\sqrt{ \frac{m}{2 \pi i \planck  \delta t }} \exp
	\left[  \frac{ i m }{ 2 \planck} \left( x_{ r + 1 }  - x _ r  / \delta t  \right)^ 2 \delta t 
	- \frac{ i  }{\planck} V ( x _r ) \delta t \right] 
\] with $ T = n \delta t $. 
This gives our final expression as 
\[
 K \left( x, x_0 ; T  \right)   = 
 \int \left[  \prod_{r = 1 }^ r d x _r  \right]  
 \left( \frac{m}{ 2 \pi i \planck \delta t }  \right)  ^{ n +1  / 2 } 
 \exp \left[  i \sum_{ r = 0 } ^ n \left[  
 \frac{m}{ 2} \left( \frac{x_{ r + 1 }  - x _ r }{ \delta t }  \right) ^ 2  - \frac{1}{ \planck}
 V ( x _ r ) \right]   \delta t  \right] 
\] In the limit $ n \to \infty $, $ \delta t \to 0 $, 
with $ n \delta t  = T $, the exponent becomes 
 \[
 \frac{1}{ \planck} \int _ 0 ^ T \left[  
 \frac{1}{2 } m \dot{  X  } ^ 2  - V ( x )  \right] = 
 \int_ 0 ^ T dt L \left( x , \dot{ x }   \right) 
\] where $ L $ is our classical Lagrangian. 
The classical action $S  = \int dt L ( x , \dot{ x   } )  $. 
The path integral 
\[
 K \left( x, x_0 ; t  \right)   = \bra{x }e ^{  - \frac{ i \hat{ H } t }{ \planck}} \ket{ x_0 } 
 \int \mathcal{ D } x e ^{  \frac{i}{\planck} S }
\] The functional integral $ \mathcal{ D } x  = \lim_{ \delta t \to 0 , n \delta T 
\text{ fixed}} \left( ... \right) \prod_{ r = 1 }^{ n } \left( ... dx_r  \right) $. 
we wont need to care about normalisation factors. 

\subsection*{Summary}

\subsubsection{Path Integral Derivations}
\begin{itemize}
	\item You get path integrals from repeatedly inserting 
		the completeness relation 
		\[
		 I = \int dx_0 \, \ket{x_0 }\bra{ x_0 } 
		\]
	\item The kernel is 
		\[
		 K \left( x, x_0 ,t   \right)   = \bra{x}e^{ -\frac{i \hat{H } t }{ \planck}}\ket{ x_0}
		\] 
	\item Our action is defined as 
		\[
			S  = \int_{ 0 } ^ T dt L \left(  x , \dot{ x }  \right) 
		\] 
	\item Our measure is the two-way limit 
		\[
			\mathcal{ D } x  = \lim_{ \delta t \to 0 , n \delta \text{ fixed}} 
			\sqrt{\frac{m}{ 2 \pi i \planck \delta t } }  \prod_{ r  = 1 } ^ n 
			\left( \sqrt{  \frac{m  }{2 \pi i \planck \delta t  }} dx _r  \right) 
		\]  
\subsubsection{Feynman Diagrams}
\begin{itemize}
	\item For each graph with $ n $ vertices, we add a combinatoric factor of 
		\[
		 \frac{| D _ n |  }{ | G _ n  | }  = \sum \frac{1}{ | \text{Aut } \Gamma | }
		\] 
\end{itemize}

\section{Regularisation and Renormalisation}%
\label{sec:regularisation_and_renormalisation}

\section{Gauge Theories}%
\label{sec:gauge_theories}


\section{Formulating the Path Integral} 

In this section, we'll be moving on from 
our standard procedure of quantising a given Hamiltonian 
in quantum mechanics. We'll be introducing 
the concept of a path integral. The path integral 
is a 'functional integral' where we integrate over 
all possible paths with a Gaussian probability factor.

\subsection{Classical and Quantum Mechanics}
In classical mechanics, we use the Lagrangian 
as a conduit to encode the information about our 
physical system. The Lagrangian is given by a function 
of position and velocity, with 
\[
	\mathcal{ L }  = \mathcal{ L } \left( q _ a , \dot {q} _ a  \right) 
\] where $ a = 1 , \dots  , N $ is an index for each particle 
in our system. We can convert this to the Hamiltonian formalism 
where we put position and momentum on the same pedestal and 
define our conjugate momenta
\[
 p _ a = \frac{\partial  \mathcal{ L }}{\partial  \dot{q } ^ a  } 
\] We then work in terms of the Hamiltonian 
which is the Legendre transformation of the 
Lagrangian, where we eliminate $ \dot{ q }^ a   $
everywhere in the Lagrangian in favour of $ p^ a$ as follows 
\[
	H ( q_a , p _ a  ) = \sum _{ a } \dot{ q }_ a p ^ a  - \mathcal{ L } \left( q _ a , \dot{ q } _ a  \right)   
\] 
The quantum mechanical analog of this is the same. 
However, $ p_ i $ and $ q _ i $ are \textbf{promoted} to 
what we call operators, and obey commutation relations 
which as we know, eventually lead to discrete energy levels 
in the Hamiltonian. In quantum mechanics,
we write the position and momentum operators 
as $ \vec{q}^ i  $ and $ \vec{p} ^ i $ for position  
and momentum respectively. 
In the Heisenberg picture of quantum mechanics, 
operators (and not states), depend on time. 
So, we impose the commutation relations 
for some fixed coordinate time $ t \in \mathbb{ R } $, where 
\[
 \left[  \vec{q}^ i , \vec{p} _ j  \right]   = i  \delta \indices{ ^ i _ j }  
\]  
In classical field theory, we 
promote operators 
to fields instead. If $ \phi ( \vec{x} , t ) $  
represents a classical scalar field at some point in time $ t $, 
then the field as well as it's conjugate momentum $ \pi \left( \vec{x}, t   \right) $ 
obey the commutation relations
\[
	\left[  \phi ( \vec{x},  t ) , \pi ( \vec{y}, t  )  \right]  
	= i \delta ^ 3 \left( \vec{x} - \vec{y} \right) 
\] There is however a caveat in performing these 
approaches to quantisation. 
The theory is not manifestly Lorentz invariant. This is because 
when we imposed the equal time commutation 
relations above, we had to pick a preferred coordinate 
time $ t $. 

\subsection{Formulating the Path Integral} 
We use the Hamiltonian 
as a starting point. 
\[
	H ( q, p ) = \frac{p ^ 2 }{ 2m } + V ( q ) 
\] The Schrödinger equation for 
a state $ \ket{ \psi ( t ) } $ which is time dependent 
is given by 
\[
	i \frac{ d }{ dt } \ket{ \psi ( t ) }  = \op{H }  \ket{ \psi ( t ) } 
\] Now, this is 
is a first order differential equation 
and can be solved provided that 
we have the right initial conditions. 
For now, let's just write down the solution 
in a 'formal' sense, where we 'exponentiate' 
the Hamiltonian whilst being vague 
about what this actually means. 
We write the solution as 
\[
	\ket{ \psi ( t ) } = \exp\left(  - i \op{ H } t  \right)  
	\ket{ \psi ( 0 ) } 
\] To do calculations however, we 
need construct an appropriate basis of states.
For this section, we'll use 
the position basis $ \ket{ q , t } $, for 
$ q, t  \in \mathbb{ R } $. These states are 
are defined to be the eigenstates of 
the position operator $ \op{q} \left( t  \right)  $, 
so that 
\[
	\op{q } \left( t   \right) \ket{ q , t }  = q \ket { q , t }   
\] We'll impose the condition 
that these states are normalised 
so that for a fixed time, we have 
\[
 \bra{q , t}\ket{ q ' , t } = \delta \left( q - q '  \right) 
\] We impose the analogous conditions as 
well for momentum eigenstates. For now though, we'll work in 
the Schrodinger picture so that $ \op{q } $ is fixed
and hence we have that the eigenstates $ \ket{ q } $ 
are time-independent. Since these states form 
a basis, we have that they obey the completeness 
relation 
\[
	1 = \int d ^ 3 q \, \ket{ q }\bra{ q } 
\] We also label the time-independent 
momentum eigenstates as $ \ket{ p } $, 
and impose the completeness relation 
\[
	1 = \frac{ d ^ 3 p }{ ( 2 \pi )^{ 3 }  } \ket{ p } \bra{ p } 
\] Note the factor of $ 2 \pi $ that we divide by. 
Other literature doesn't include this. 
With this set of basis states, we can 
now write the abstract state $ \ket{ \psi( t ) } $ 
in terms of the position basis, where we denote 
\[
	\psi ( q, t ) = \bra{ q}\ket{ \psi  ( t ) } 
	=  \bra{q } \exp\left(  - i \op{H }t  \right)\ket{ \psi ( 0 ) } 
\] We will put this into an integral form 
for reasons we will discuss later. 
To put any equation in integral 
form, the rule of thumb is 
to employ the completeness relations 
for either the position or momentum basis.
We get that 
\begin{align*}
	\bra{ q } \exp\left( - i \op{ H } t  \right) \ket{ \psi ( 0 ) } &=  \int d ^ 3 q '\,  \bra{ q }\exp \left(  - i \op{ H } t  \right)\ket{ q ' } \bra{ q ' }\ket{\psi ( 0 )}    \\ 
									&=  \int d ^ 3 q '\,  \bra{q}\exp \left(  - i \op{H } t  \right) \ket{ q' }\psi ( q' , 0 )  \\ 							&=  \int  d ^ 3 q' \, K \left( q, q' ; t  \right)  \psi ( q' , 0 )  \\
									&=  \int d ^ 3  q ' K ( q , q ' ; t ) \psi ( q ' , 0 )  
\end{align*}  
Here we've defined $ K ( q , q '; t )  = \bra{q}\exp\left(  - i \op{H } t  \right) \ket{ q ' } $. 
Now to make progress, we need to find
a meaningful expression for what $ K \left( q, q ; T \right)  $ 
actually is. First, 'split up' 
our $ \exp (  - i \op{H } T)$ term into smaller pieces - 
that is, partition $ T $ as 
\begin{equation}
	\exp\left(  - i \op{H } T  \right)  = 
	\exp \left(  - i \op{H } \left( t_{ n + 1 }  - t _ n  \right)   \right) \exp \left(  - i \op{H } \left( t_n  - t_{ n - 1 }  \right)   \right)  \dots \exp\left(  - i \op{H } \left( t _ 1  - t _ 0  \right)   \right)  
\end{equation} here, we set by definition that 
$ t_{ n + 1 }  =T  > t _ n > t _{ n - 1 } > \dots > t _ 1 > t _ 0 = 0 $. For example, setting $ n = 1$ and inserting one 
integral as part of the completeness relation, 
we get that 
\begin{align*}
	K \left( q, q' ; T  \right)  &=  
	\bra{ q }\exp\left(  - i \op{H } \left( t_2 - t_1  \right)  \right) \exp\left(  - i \op{ H } \left( t_1  - t_0   \right)   \right) \ket{q ' }  \\
	&=  \int d q_1 \,  
	\bra{ q } \exp\left(  - i \op{H } \left( t_2  - t_1  \right) \right)\ket{q_1 } \bra{q_1 } \exp\left( - i \op{H } \left( t_1  - t_0  \right)   \right)\ket{q ' } 
\end{align*}
where we've set $ t_2 = T $. We can generalise this 
to the case where we have $ n $ time slices. We 
have that 
\begin{equation}
	K\left( q, q' , r  \right)   = 
	\int \prod_{ i = 1 } ^ n \left( d q _  r \, 
	\bra{q _{ r +  1} } \exp \left(  - i \op{ H } 
\left(  t _{ r + 1 }  - t _ r  \right)  \right) \ket{q _ r }\right)\bra{q _  1}\exp\left(  - i \op{H } \left( t_1 - t_0  \right)   \right) \ket{q ' } 
\end{equation}

\pagebreak
\section*{Example Sheet 1}

\subsection*{Question 1 (2018)}
We expand the exponential 
involving $ \lambda $ as 
\begin{align*}
	\mathcal{ Z } \left( \lambda  \right)   & = \frac{1}{\sqrt{ 2 \pi }  } 
	\int dx e ^{  - \frac{1}{2 } x ^ 2 } \sum_{ l = 0 } ^ n 
	\left(  - \lambda \frac{x ^ 4 }{ 4 ! }  \right)  ^ l \frac{1}{l ! } \\
						&=  \sum_{l = 0 }^{ n }  \frac{1}{\sqrt{ 2 \pi }  } \left(  - \frac{\lambda}{4 ! }  \right) ^{ l } \frac{1}{l ! } \int_{ \mathbb{ R } } dx \, 
						e ^{  - \frac{1}{2 } x ^ 2 } x ^{ 4l } 
\end{align*}
Now, we evaluate the integral using a trick. 
We arbitrarily set 
\[
 I ( \alpha )  = \int dx \, e ^{ \frac{1}{2 } \alpha x ^ 2 } 
\] Differentiating this integral with respect to $ \alpha $, 
we have that 
\[
	\frac{d ^{ 2l } I }{ d \alpha ^{ 2l } }  = \int_{ \mathbb{ R } } 
	dx \, \left( \frac{1}{2 }  \right)^{ 2l } x ^{ 4l } e ^{  - \frac{\alpha}{2 } x ^ 2 }
	= \sqrt{ 2 \pi }  \left( \frac{1}{2 }  \right)  ^{ 2l } 1 \left( 3  \right)  \dots 
	\left( 4l - 1  \right)  
\] Cancelling out factors and using the 
standard formula for odd factorials, 
we get that 
\[
	\int dx \, x ^{ 4l } e ^{  - \frac{\alpha}{2 } x ^ 2 }  = \sqrt{ 2 \pi }  \frac{\left( 4l  \right)  ! }{ 4 ^ l \left( 2l  \right)  ! }
\] Substituting this in 
means that we get our expression for our 
partition function as 
\[
	\mathcal{ Z } _ n \left( \lambda  \right)   = 
	\sum_{ l = 0 } ^ n \left( - \frac{ \lambda }{ 4 ! }  \right)^{ l } 
	\frac{\left( 4l  \right)  ! }{  4 ^ l \left( 2l  \right)  ! }
\]
Our contributing Feynman diagrams 
at $ l \ll 3 $ are shown in the figure.
At $ l  = 1$,  $a_ l = \frac{1}{8 } $, 
which is in agreement with a figure of $ 8 $ 
diagram. At $l  =2  $, $ a _ l  = \frac{35}{384}$, 
which agrees with the sum of the automorphism factors 
at 2 loops.

At  $l = 3 $, $ a _ l = \frac{385}{ 3072}$. 

\begin{figure}[htpb]
	\centering
	\input{e1_q1.pdf_tex}
	\caption{Feynman diagrams and their automorphism 
	factors}%
	\label{fig:}
\end{figure}
We need to sum multiple diagrams, 
which are connected with $ n $ loops to get terms in the 
expansion. 
There are two ways to get terms in the expansion. 
One is to sum all possible diagrams, 
the other is to sum connected diagrams with a certain number of loops! 

What are the possible 3 loop diagrams? 
What are the automorphism factors? 

\end{document} 
