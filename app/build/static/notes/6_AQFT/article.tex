\documentclass[11pt, oneside]{article}   	% use "amsart" instead of "article" for AMSLaTeX format
\usepackage[margin = 1.1in]{geometry}            		% See geometry.pdf to learn the layout options. There are lots.
\geometry{letterpaper}                   		% ... or a4paper or a5paper or ... 
\usepackage[parfill]{parskip}    		% Activate to begin paragraphs with an empty line rather than an indent
\usepackage{graphicx}				% Use pdf, png, jpg, or eps§ with pdflatex; use eps in DVI mode
								% TeX will automatically convert eps --> pdf in pdflatex	
\usepackage{adjustbox}	
\usepackage[section]{placeins}


%% LaTeX Preamble - Common packages
\usepackage[utf8]{inputenc}
\usepackage[english]{babel}
\usepackage{textcomp} % provide lots of new symbols
\usepackage{graphicx}  % Add graphics capabilities
\usepackage{flafter}  % Don't place floats before their definition
\usepackage{amsmath,amssymb}  % Better maths support & more symbols
\usepackage[backend=biber]{biblatex}
\usepackage{amsthm}
\usepackage{bm}  % Define \bm{} to use bold math fontsx
\usepackage[pdftex,bookmarks,colorlinks,breaklinks]{hyperref}  % PDF hyperlinks, with coloured links
\usepackage{memhfixc}  % remove conflict between the memoir class & hyperref
\usepackage{mathtools}
\usepackage[T1]{fontenc}
\usepackage[scaled]{beramono}
\usepackage{listings}
\usepackage{physics}
\usepackage{tensor}
\usepackage{simplewick} 
\usepackage{tikz} 
\usepackage{import}
\usepackage{xifthen}
\usepackage{pdfpages}
\usepackage{transparent}
\usepackage{pgfplots}
\usepackage[compat=1.1.0]{tikz-feynman}
\usepackage{subfiles}
\usepackage{simpler-wick}
\usepackage{slashed}
\usepackage{fancyhdr}

\pagestyle{fancy}
\fancyhf{}
\rhead{Notes by Afiq Hatta}
\lhead{Quantum Field Theory II}
\rfoot{Page \thepage}

%% Commands for typesetting theorems, claims and other things.
\newtheoremstyle{slanted}
{1em}%   Space above
{.8em}%   Space below
{}%  Body font
{}%          Indent amount (empty = no indent, \parindent = para indent)
{\bfseries}% Thm head font
{.}%         Punctuation after thm head
{0.5em}%     Space after thm head: " " = normal interword space;
{}%         \newline = linebreak
{}%          Thm head spec (can be left empty, meaning `normal')

%% Commands for typesetting theorems, claims and other things. 

\theoremstyle{slanted}
\newtheorem{theorem}{Theorem}
\newtheorem*{thm}{Theorem}
\newtheorem*{claim}{Claim}
\newtheorem{example}{Example}
\newtheorem*{defn}{Definition}

\newcommand{\Lagr}{\mathcal{L}} 
\newcommand{\vc}[1]{\mathbf{#1}}
\newcommand{\pdrv}[2]{\frac{\partial{#1}}{\partial{#2}}}
\newcommand{\thrint}[1]{\int d^3 \vc{x} \left( {#1} \right)}

%% QFT specific macros 
\newcommand{\intp}{ \int \frac{ d^3 p }{ (2 \pi)^3 } \, }
\newcommand{\ann}[1]{a_{ \mathbf{ #1 }}}
\newcommand{\crea}[1]{a^\dagger_{ \mathbf{ #1 }}}
\newcommand{\ve}[1]{ \mathbf{ #1 } } 
\newcommand{\mode}[ 1]{ e^{ i \mathbf{ #1 } \cdot \mathbf{x} }}
\newcommand{\nmode}[1]{ e^{  - i \mathbf{ #1 } \cdot \mathbf{x} }}
\newcommand{\freq}[1]{\omega_\mathbf{ #1} } 
\newcommand{\scal}[1]{\phi ( \mathbf{ #1 })} 
\newcommand{\mom}[1]{ \pi (\mathbf{ #1 })} 
\newcommand{\arr}{\rightarrow} 

\newcommand{\incfig}[1]{%
\def\svgwidth{\columnwidth}
\resizebox{0.75\textwidth}{!}{\input{./figures/#1.pdf_tex}}
}

\newcommand{\anop}[2]{ #1_\mathbf{#2}}
\newcommand{\crop}[2]{#1_\mathbf{#2}^\dagger}
\renewcommand{\op}[1]{\hat{\mathbf{#1}}}

\usepackage{helvet} 

%tikz decoration commands 
\usetikzlibrary{decorations.pathmorphing}


\title{Notes on Quantum Field Theory II}
\author{Afiq Hatta} 
\begin{document} 
\maketitle
\tableofcontents

\pagebreak 

\section{Formulating the Path Integral} 

In this section, we'll be moving on from 
our standard procedure of quantising a given Hamiltonian 
in quantum mechanics. We'll be introducing 
the concept of a path integral. The path integral 
is a 'functional integral' where we integrate over 
all possible paths with a Gaussian probability factor.

\subsection{Classical and Quantum Mechanics}
In classical mechanics, we use the Lagrangian 
as a conduit to encode the information about our 
physical system. The Lagrangian is given by a function 
of position and velocity, with 
\[
	\mathcal{ L }  = \mathcal{ L } \left( q _ a , \dot {q} _ a  \right) 
\] where $ a = 1 , \dots  , N $ is an index for each particle 
in our system. We can convert this to the Hamiltonian formalism 
where we put position and momentum on the same pedestal and 
define our conjugate momenta
\[
 p _ a = \frac{\partial  \mathcal{ L }}{\partial  \dot{q } ^ a  } 
\] We then work in terms of the Hamiltonian 
which is the Legendre transformation of the 
Lagrangian, where we eliminate $ \dot{ q }^ a   $
everywhere in the Lagrangian in favour of $ p^ a$ as follows 
\[
	H ( q_a , p _ a  ) = \sum _{ a } \dot{ q }_ a p ^ a  - \mathcal{ L } \left( q _ a , \dot{ q } _ a  \right)   
\] 
The quantum mechanical analog of this is the same. 
However, $ p_ i $ and $ q _ i $ are \textbf{promoted} to 
what we call operators, and obey commutation relations 
which as we know, eventually lead to discrete energy levels 
in the Hamiltonian. In quantum mechanics,
we write the position and momentum operators 
as $ \vec{q}^ i  $ and $ \vec{p} ^ i $ for position  
and momentum respectively. 
In the Heisenberg picture of quantum mechanics, 
operators (and not states), depend on time. 
So, we impose the commutation relations 
for some fixed coordinate time $ t \in \mathbb{ R } $, where 
\[
 \left[  \vec{q}^ i , \vec{p} _ j  \right]   = i  \delta \indices{ ^ i _ j }  
\]  
In classical field theory, we 
promote operators 
to fields instead. If $ \phi ( \vec{x} , t ) $  
represents a classical scalar field at some point in time $ t $, 
then the field as well as it's conjugate momentum $ \pi \left( \vec{x}, t   \right) $ 
obey the commutation relations
\[
	\left[  \phi ( \vec{x},  t ) , \pi ( \vec{y}, t  )  \right]  
	= i \delta ^ 3 \left( \vec{x} - \vec{y} \right) 
\] There is however a caveat in performing these 
approaches to quantisation. 
The theory is not manifestly Lorentz invariant. This is because 
when we imposed the equal time commutation 
relations above, we had to pick a preferred coordinate 
time $ t $. 

\subsection{Formulating the Path Integral} 
We use the Hamiltonian 
as a starting point. 
\[
	H ( q, p ) = \frac{p ^ 2 }{ 2m } + V ( q ) 
\] The Schrödinger equation for 
a state $ \ket{ \psi ( t ) } $ which is time dependent 
is given by 
\[
	i \frac{ d }{ dt } \ket{ \psi ( t ) }  = \op{H }  \ket{ \psi ( t ) } 
\] Now, this is 
is a first order differential equation 
and can be solved provided that 
we have the right initial conditions. 
For now, let's just write down the solution 
in a 'formal' sense, where we 'exponentiate' 
the Hamiltonian whilst being vague 
about what this actually means. 
We write the solution as 
\[
	\ket{ \psi ( t ) } = \exp\left(  - i \op{ H } t  \right)  
	\ket{ \psi ( 0 ) } 
\] To do calculations however, we 
need construct an appropriate basis of states.
For this section, we'll use 
the position basis $ \ket{ q , t } $, for 
$ q, t  \in \mathbb{ R } $. These states are 
are defined to be the eigenstates of 
the position operator $ \op{q} \left( t  \right)  $, 
so that 
\[
	\op{q } \left( t   \right) \ket{ q , t }  = q \ket { q , t }   
\] For now though, we'll work in 
the Schrodinger picture so that $ \op{q } $ is fixed
and hence we have that the eigenstates $ \ket{ q } $ 
are time-independent.
With this set of basis states, we can 
now write the abstract state $ \ket{ \psi( t ) } $ 
in terms of the position basis, where we denote 
\[
	\psi ( q, t ) = \bra{ q}\ket{ \psi  ( t ) } 
	=  \bra{q } \exp\left(  - i \op{H }t  \right)\ket{ \psi ( 0 ) } 
\]  
\end{document} 
