\section{The formation of black holes} 
In this section, we'll cover the formation of black holes. 
To start this discussion, we'll need to discuss 
the idea of symmetry on manifolds and in metrics. 
The word 'black hole' should be a big enough hint 
that the kinds of metrics we'll be considering exhibit 
spherical symmetry. But, since general relativity is done
in the frame-work of both \textbf{space} and  \textbf{time}, 
we need to make clear what 'spherical symmetry' actually means. 

First, lets discuss how to obtain the metric for a standard $2 $-sphere. 
Working in signature $\left(  - , + , + , +  \right)  $, 
we have that the metric on the $ 2$ -sphere in Cartesian 
coordinates is given by the following, with the constraint
\[
 ds ^ 2  = dx ^ 2 + dy ^ 2 + dz ^ 2, \quad x ^ 2 + y ^ 2 + z ^ 2  = 1
\] If we reparametrize the coordinates as follows, using 
our standard spherical coordinates with $ r = 1 $, 
we embed the $2 $ -sphere in $ \mathbb{ R } ^ 3 $
\begin{align*}
	x &=  \cos \phi \sin \theta  \\ 
	y &=  \sin \phi \sin \theta  \\ 
	z &=   \cos \theta
\end{align*}
Applying a coordinate transformation for the one forms $ dx, dy, d z$, 
we get that
\begin{align*}
	dx &=  \cos \phi \cos \theta d \theta  - \sin \phi \sin \theta d \phi  \\
	dy &=  \sin \phi \cos \theta d \theta + \cos \phi \sin \theta d \phi  \\
	dz &=   - \sin d \theta  
\end{align*} 
Substituting into the above, we can read off that 
the components of the metric are given by 
\[
 d s ^ 2 =  d \theta ^ 2 + \sin ^ 2 \phi d \phi ^ 2 
\] This 
metric comes from considering the 2-sphere manifold. 
Now, we know that the symmetry group of a 2-sphere is 
$ O( 3 ) $ if we consider reflections, and just $ SO ( 3 ) $ 
if we consider only rotations. Hence, we 
say that the metric admits an isometry group of $ SO ( 3 ) $. 

\begin{defn}{(Isometries on a metric)}
	An  \textbf{isometry} is a transformation 
	on a metric space which leaves 
	distances between points invariant. 
	The image one has in their mind immediately 
	might be a rotation or a reflection on a two dimensional 
	plane. 

\end{defn}
