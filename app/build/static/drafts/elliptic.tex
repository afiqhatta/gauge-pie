
\pagebreak 
 
\section{Elliptic Partial Differential Equations}
The most relevant text for learning this kind of 
material is the book by Gilbert and Trudinger, called 
Elliptic Partial Differential Equtions of 2nd order. 

The key theory of this course is a theory 
called Schrouder theory, but the treatment of 
elliptic PDEs will be different from that of the textbook. 

We will build a background on linear theory and then 
move onto non-linear theory. 
So what will we be talking about? Let's give 
some backgorund theory and motivation for the things we will 
be doing. 

In this course, we will study second order linear PDEs of the 
form 
\[
 Lu  = a ^{ ij } D _{ ij } u + b ^ i D _ i u + c u = f 
\] We will be working in Euclidean space throughout. 
$ L $ is our linear operator. 
We will be working in a bounded domain (which is an open, connected set)
$ \Omega \subset \mathbb{ R } ^ n $. 
Here, we have that 
\[
	D _ i = \frac{\partial  }{\partial  x _ i }  , \quad D_{ ij }  = 
	\frac{\partial  ^ 2 }{\partial  x _ i \partial  x _ j } 
\] Throughout this course, 
we will also keep summation convention implicit. 
So we have that $ a ^{ ij }  D_{ ij } u $ means that 
we sum over both the $ i , j $ indices to get 
\[
 \sum _{ i , j = 1 , \dots n } ^{ n } a ^{ ij } D _{ ij } u 
\]  And similarly we have that $ b ^ i D _ i $ 
means $ \sum b^ i D _ i u  $. 
We have that $ a ^{ ij } , b ^ i , c , f $ are given functions on 
$ \Omega $. 

We also assume symmetry. $ a ^{ ij } = a ^{ ji } $ for each $ i , j $. 
This will be without loss of generality of $ u $ is a classical solution, 
in otherwords the second derivative is continuous. $ u \in  C^ 2 \left(  \Omega  \right)  $. 
This is since we can replace $ a ^{ ij } $ with $ \overline{ a } ^{ ij }  = 
\frac{ a ^{ ij }  + a ^{ ji } }{ 2 } $. 

Under this symmetric assumption, 
the matrix of top order coefficients $ A \left(  x  \right)   = a _{ ij } \left(  x  \right)  $ 
is an $ n $ by $ n $ real symmetric matrix $ \forall x \in \Omega $. 
This means you can find real eigenvalues. 
If $ \lambda \left(  x  \right)  , \Lambda \left(  x  \right)  $ are 
the smallest and largest eigenvalues ie $ \lambda ( x) | \xi | ^ 2 \leq a^{ ij } \xi _ i \xi_ j 
\leq \Lambda \left(  x  \right)  | \xi | ^ 2  $ for any arbitrary vector 
\[
	\xi  = ( \xi_1 , \dots , \xi_ n ) \in \mathbb{ R } ^{ n } 
\] 
The operator  $ L  $ is said to be
\begin{itemize}
	\item Elliptic in $ \Omega $ if $ \lambda (x ) >  0 $  $ \forall x \in \Omega $./ 
	\item Strictly elliptic in $ \Omega $ if $ \lambda (x  ) \geq \lambda _ 0 > 0 $
	for some constant  $ \lambda_0 $ and all $ x \in \Omega $. 
\item Is uniformly elliptic in  $ \omega $ if $ \lambda ( x) > 0 $ and 
	\[
		\sup_{ \Omega } \frac{\Lambda \left( x  \right)  }{ \lambda (x )} < \infty 
	\] 
\end{itemize}
A main part of the course (Schauder theory) is to establish, 
under suitable assumptions on $ L $, (including ellipticity), 
and existence and regularity theory for the Dirichlet problem associated with $  L $. 
What does this mean? 

We would like to solve 
\[
 \begin{cases}
	 L u = f & \text{ in } \Omega \\ 
	 u = g \text{ on } \partial  \Omega \\ 
 \end{cases}
\] where $ f , g $ are given functions, and we seek $ u \in C ^ 2 \left(  \Omega  \right) $.
We can solve this equation without integrating! 
The main idea is to deform the Laplacian to the form above. 

By solving we mean we prove existence of solutions. 
We hope to squeeze some non-linear theory in, and will mention some non-linear theory.
A key motivation for 
developing such a linear theory is its applications 
to certain natural non-linear PDEs elliptic PDEs, 
arising in Geometry, Physics, 
Calculus of Variations and other areas. 

In the linear theory, 
we'll be proving statements such as 
\begin{itemize}
	\item If $ a ^{ ij }, b ^ i, c, ,f   \in C ^{ 0 , \alpha } \left( \Omega  \right)  $, 
		and that $ u \in C ^ \left( 2, \alpha  \right)  \left( \Omega \right) $
\end{itemize}


