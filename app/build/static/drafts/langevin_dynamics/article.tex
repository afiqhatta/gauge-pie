\documentclass[11pt, a4paper]{article}   	% use "amsart" instead of "article" for AMSLaTeX format
 
\usepackage[margin = 1.0in]{geometry}            		% See geometry.pdf to learn the layout options. There are lots.
\geometry{letterpaper}                   		% ... or a4paper or a5paper or ... 
\usepackage[parfill]{parskip}    		% Activate to begin paragraphs with an empty line rather than an indent
\usepackage{graphicx}				% Use pdf, png, jpg, or eps§ with pdflatex; use eps in DVI mode
							% TeX will automatically convert eps --> pdf in pdflatex	
\usepackage{adjustbox}	
\usepackage[section]{placeins}


%% LaTeX Preamble - Common packages
\usepackage[utf8]{inputenc}
\usepackage[english]{babel}
\usepackage{textcomp} % provide lots of new symbols
\usepackage{graphicx}  % Add graphics capabilities
\usepackage{flafter}  % Don't place floats before their definition
\usepackage{amsmath,amssymb}  % Better maths support & more symbols
\usepackage[backend=biber]{biblatex}
\usepackage{amsthm}
\usepackage{bm}  % Define \bm{} to use bold math fontsx
\usepackage[pdftex,bookmarks,colorlinks,breaklinks]{hyperref}  % PDF hyperlinks, with coloured links
\usepackage{memhfixc}  % remove conflict between the memoir class & hyperref
\usepackage{mathtools}
\usepackage[T1]{fontenc}
\usepackage[scaled]{beramono}
\usepackage{listings}
\usepackage{physics}
\usepackage{tensor}
\usepackage{tikz}
\usepackage{multicol}
\usepackage{subfiles}



%% Commands for typesetting theorems, claims and other things.
\newtheoremstyle{slplain}% name
  {.5em} % Space above
  {.8em} % Space below
  {} % Body font
  {} % Indent amount
  {\bfseries} % Theorem head font
  {.} % Punctuation after theorem head
  {.1em} % Space after theorem head
  {} % Theorem head spec (can be left empty, meaning `normal')

\theoremstyle{slplain}
\newtheorem{theorem}{Theorem}
\newtheorem*{thm}{Theorem}
\newtheorem*{claim}{Claim}
\newtheorem*{example}{Example}
\newtheorem*{defn}{Definition}
\newtheorem*{tech}{Technique}
\newtheorem*{question}{Questions} 

\newcommand{\Lagr}{\mathcal{L}}
\newcommand{\vc}[1]{\mathbf{#1}}
\newcommand{\pdrv}[2]{\frac{\partial{#1}}{\partial{#2}}}
\newcommand{\thrint}[1]{\int d^3 \vc{x} \left( {#1} \right)}
\newcommand{\R}{ \mathbb{R}}
\newcommand{\planck}{\hbar}
\newcommand{\peninf}{\mathscr{i}} 
\newcommand{\scri}{\mathscr{J}}


\author{Afiq Hatta} 
\title{Langevin Dynamics}
\begin{document}

\title 
\tableofcontents

\section{Introduction} 
Speaker: Bidisha from ICTS Bangalore 

This talk is about non-linear Langevin dynamics in 
Holography. 

\section{Papers} 
Effective quantum field theories for the Thermodynamic 
Bethe Ansatz - Ivan Kostov 

Confinement as Analytic Continuation Beyond Inifinty 
 - Masahito Yamazaki 
 
Renormalon problems? 

Techniques
 - Place things on a cylinder 
  - when coupling constants switch sign

Partition function on different kinds of worldsheets. 

\section{Papers} 
Loganajan, Chakravanty, Sivakuman

\section{Motivation} 
In non equilibrium processes, how do we compute 
real time correlation functions? 
In the path integral framework, 
we usually study equilibrium processes, 
but we should extend this to non-equilibrium states. 

This framework is known as the
Schwinger-Keldysh path integral framework. 
(We draw up the time contour) looks like a hook. 
We initialise the density matrix $ \rho_ i  = \ket{ \psi } \bra{ \psi } $ 
on the time contour. 
We're splitting the parts of the Lagrangian by 
doubling the degrees of freedom along $ Q_ 1 $ and $ Q _  2 $ 
as 
\[
	\mathcal{ L } _{ SK }  = \mathcal{ L } _ 1 Q _ 1 - \mathcal{ L } _ 2 ( Q _ 2 ) 
\] 
Suppose we have a quantum brownian particle 
with degrees of freedom denoted by $ q $, with 
interacting degrees of freedom $ X $. 
Our density matrix is given by 
\[
	\rho_ 0 = \rho _{ O P } \otimes \rho _{ B O }
\] 
Let's initialise our density matrix 
at $ t_0 $ as $ \rho_0 $. 
We double the degrees of freedom on legs of the contour 
as $ q_ i $  and $ X_ i $. 
Our microscopic lagrangian is given by 
\[
 \mathcal{ L } _{ \text{micro } }  = \mathcal{ L } _ P + \mathcal{ L } _ B 
 + \lambda q X , \quad \mathcal{ L } _{ SK }  = \mathcal{ L } _ 1 - \mathcal{ L } _ 2 
\] Our evolution is given 
by 
\[
	\rho_{ f} = \left(  Q _{ 1 f } , Q_{ 2 f}  \right)   = 
	\int d Q _{ 10 } d Q _{ 2 0 } \rho_0 ( Q_{ 10 } Q _{ 2 0 } ) \int 
	[ \mathcal{ D } Q_ 1 ] [ \mathcal{ D } Q _ 2 ] \exp \left(  
	i \int_{ Q_ 1 ( t_1 )  = Q_{ 10 } } ^{ Q_{ 1f } ) = Q_{ 1 f }} dt \left(  L _ 1 ( Q_ 1 )  - L _ 2 ( Q_ 2 )  \right) \right) 
\]
What is the reduced density matrix of the particles? 
To do this, we need to integrate out some things. 
Let's say that $ X_{ 1f } = X _{ 2f }  = X _  f $. 
representing the bath degrees of freedom 
are integrated out. 
Then, computing the reduced density matrix 
we get that 
\[
	\rho_{ p f }  ( q_{ 1f } , q_{ 2f } ) 
	= \int d q_{ 10 } \int d q_{ 20} \rho_{ p_0 } ( q_{ 10 } , q_{ 20} ) 
	\int_{ q_1 ( t_0 )  = q_{ 10 } , q_ 2 ( t_0 ) =  q_{ 20 } }
	^{ q_1 ( t_ f )  = q _{ 1f }, q_2 (t_ f ) = q_{ 2f }  } 
	\left[  \mathcal{ D } q_1  \right]  \left[  
	\mathcal{ D } q_ 2 \right]  
	e^{ i \left[  \frac{1}{2 } mp_0 \int_{ t_0 } ^{ t_ f }
	dt \left[ ( \dot{ q } _ 1 ^ 2  - \mu _ 0 ^ 2 q_1 ^ 2 ) 
	- ( \dot{ q } ^ 2 _ 2 - \mu 0 ^ 2 q_ 2 ^ 2 ) + W_{ SK } \right] \right] }
\] The part $ W _{ SK } $ is known as the influence 
phase of the particle. 
We expand $ W_{ SK} $ as 
\[
	W _{ SK } = \sum_{n=1}^{\infty } \lambda ^ n W _{ SK } ^{ ( n ) } 
\] with each of the components being 
\[
	W _{ SK } ^{ ( n ) }  = i ^{ n -1  } \sum_{ i_1  \dots i_n = 1 } ^ 2 
	\int _{ t_1 } ^{ t_  f } dt_1 \dots \int_{ t_1 } ^{ t _{ n- 1 } } 
	dt_ n \left< T _ c O_{ i_1 } ( t_1 ) \dots O _{ i _ n  } ( t_ n )   \right> q _{ i_1 } ( t _ 1 ) q _{ i_2 } ( t_ 2 ) \dots q_{ i _ n } ( t _ n ) 
\] Here, we observe that 
the effective action is non-local in time. 
For a class of baths, the bath correlations decay exponentially fast.
For example, we have that 
\[
	W ^{ ( 2) }  = \int dt_1 dt_2 
	\left< T _c O _{ i_1 } ( t_1 ) O_{ i_2 } ( t_2 )  \right> 
	q_{ i_1 } ( t_1 ) q_{ i_2 } ( t_2 ) 
\] Our 
correlated part enclosed in angle brackets exponentially 
decays as $ e ^{ - \alpha t _{ 12 } } $. 
We also can Taylor expand out $ q _{ i_2 } ( t_2) $  in 
from $ t_1 $. 
Referencing Feynman-Vernon. 
\subsection{Quadratic Quantum Effective Theory of $ q $} 
If we focus on the saddle point of this
integral, then we get an equation of motion 
\[
 \frac{d ^ 2 q}{  d t ^  2 } + \gamma \frac{ dq }{ dt } + 
 \overline{ \mu }^ 2 q = \left< f ^ 2  \right> \mathcal{ N } 
\]  
We have that our noise has distribution 
\[
	\rho ( \mathcal{ N } ) = \exp \left[  
	 - \int dt \frac{\left< f ^ 2  \right> }{ 2 } \mathcal{ N } ^ 2 \right] 
\] Our equation is called the Linear Langevin equation. 
We have that 
\[
 \left< f ^ 2  \right> = \frac{2}{\beta } \gamma \sim \left< 
 \left\{  O ( t_1 ) , O ( t_2 )  \right\} \right>
\]  The above is the statement of equilibrium.
This is how we derive 
the equation of motion from the 
path integral. 
We define the helper variables
\[
 q_1 , q_2 \to q_ a = \frac{ q_1+ q_2 }{ 2 } ; \quad q _{ d } 
 = q_1 - q_2
\] 
Consider the following stochastic path integral 
\[
 \int \left[  dq_ a  \right]  \left[  d \mathcal{ N }  \right]

 P \left[  N  \right]  \delta ( \left< f ^ 2  \right> \mathcal{ N } 
 - E ( q_ a , \mathcal{ N } ) ) 
\] this can be written as 
\[
 \int \mathcal{ D } q_ a \mathcal{ D } \mathcal{ N } 
 e ^{  - i \int dt q_ d ( E ( q_ a , \mathcal{ N } )  - \left< f ^ 2  \right> 
 \mathcal{ N } )} P ( \mathcal{ N } ) 
\] This can 
be rewritten as 
a path integral as 
\[
 \int \mathcal{ D } q_{ a } \mathcal{D} q_{ d } e ^{ i \int dt \mathcal{ L }_{ \text{ eff } }  } 
\] Our Lagrangian for the effective action is 
\[
 \mathcal{ L } _{ \text{ eff } }  = 
 \frac{ i \left< f ^ 2  \right> }{ 2  } q_ d ^ 2 + 
 \dot{ q } _ d \dot{ q } _ a  - \gamma q _ d \dot{ q } _ a  - 
 \overline{ \mu } ^ 2 q_d  q_ a 
\] Thus, 
we've computed non linear corrections to the Linear langevin equation 
How to get this 
big effective action $ \mathcal{ L } _{ \text{ eff } }  $ 
Our most general Lagrangian for the effective action 
is 
\[
 \mathcal{ L } _{ \text{ eff } }  = \frac{i \left< f ^ 2  \right> }{2 } 
 ( \text{ insert most general terms here })
\]
If we take the classical limit 
using saddle points of our effective Lagrangian, 
we get that 

\subsection{Non-Linear FDR}
We have that 
\[
 \zeta _N =  - \frac{ 12 }{ \beta  } \zeta _{ \gamma } 
\]  from OTO extension of effective theory. 
$ \zeta _ N $ comes from time-reversal invariance and thermality of 
bath. 


\section{Holographic problem} 
We'll now talk about the Holographic SK prescription. 
The system we'll talk about 
is a quark moving in a thermal CFT d-dimensional plasma. 
In AdS / CFT, this quark is dual to an open dtring from the boundary 
of an AdS d + 1 dimensional black brane.

The generating function for real time CFT correlators when initial 
state was thermal. It has also been done that 
we verified non-linear FDR for quark using holographic SK.
Check out Hong Liu, et al.

The string is in the black brane, the black brane has Hawking radiation. 
Modes captured by diffferent Green's function. 

Fluctutation and dissipation are 
measurable quantities. This can be tested! 
\end{document} 
